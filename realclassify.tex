\documentclass{amsproc}
\linespread{1.25}
\usepackage{amsthm,amsmath,amsfonts,mathrsfs,amssymb}

\newcommand{\Singular}{\textsc{Singular}}
\newcommand{\realclassify}{\texttt{realclassify.lib}}

\DeclareMathOperator{\ord}{ord}
\DeclareMathOperator{\requiv}{\overset{r}{\sim}}
\DeclareMathOperator{\m}{\mathfrak{m}}
\DeclareMathOperator{\jt}{jet}
\DeclareMathOperator{\supp}{supp}
\DeclareMathOperator{\sign}{sign}
\DeclareMathOperator{\R}{\mathbb{R}}
\DeclareMathOperator{\C}{\mathbb{C}}

\title{The classification of real singularities using \textsc{Singular}}

\author{Magdaleen S. Marais}
\address{Magdaleen S. Marais\\
African Institute for Mathematical Sciences and Stellenbosch University\\
6 Melrose Rd\\
Muizenberg 7945, Cape Town\\
South Africa}
\email{magdaleen@aims.ac.za}

\author{Andreas Steenpa\ss}
\address{Andreas Steenpa\ss\\
Department of Mathematics\\
University of Kaiserslautern\\
Erwin-Schr\"odinger-Str.\\
67663 Kaiserslautern\\
Germany}
\email{steenpass@mathematik.uni-kl.de}

\thanks{ }
\subjclass[2000]{}
\keywords{}
\begin{document}
\begin{abstract}
The algorithms implemented in the library ``realclassify.lib" in \textsc{singular} are discussed in this paper. The purpose of this library is to classify the~$0$ and $1$ modal isolated hypersurface singularities at $0$ of corank $0$, $1$ and $2$ over the real numbers as computed by V.I.~Arnold in \cite{AVG1985}. 
\end{abstract}
\maketitle
\section{Introduction}
The goal of this paper is to present the algorithms that were implemented in a library, ``realclassify.lib'' \cite{realclassify} in \textsc{singular}, which classifies the $0$ and $1$ modal isolated hypersurface singularities at $0$ of corank $0$, $1$ and $2$ over the real numbers $\mathbb R$, using right-equivalence, as was computed by V.I.~Arnold in \cite{AVG1985}. Hence we consider smooth real functions with critical point $0$ and critical value $0$, i.e.~functions in $\m^2$, where $\m$ denotes the ideal of function germs vanishing at the origin. Two function germs $f, g\in\mathbb R[[x_1,\ldots, x_n]]$, $f,g\in \m^2$, where $\mathbb R[[x_1,\ldots,x_n]]$ is the power series ring in coordinates $x_1,\ldots,x_n$ over $\mathbb R$, are considered as right-equivalent, denoted by $f\requiv g$, if there exists an $\mathbb R$-algebra automorphism $\phi$ of $\mathbb R[[x_1,\ldots, x_n]]$ such that $\phi(f)=g$. Because the only possible non-approximated input power series over the real numbers in computer systems, and thus in particularly in \textsc{singular}, are polynomials with rational coefficients, some of the algorithms we used apply only to such input polynomials.  There already is a \textsc{Singular} library ``classify.lib'' \cite{classify}, which for a given polynomial computes its type in Arnold's classification over the complex numbers. We used this library as a basis to build ``realclassify.lib'' on. The methods used in ``classify.lib'' will not be discussed in this paper. For more information in this regard \cite{Kruger} can be studied.

Using the Splitting Lemma (Lemma~\ref{SplittingLemma}) any function germ $f$ over the real numbers with an isolated singularity at $0$, using a suitable coordinate system, can be written as the sum of two functions of which the variables do not coincide. One of the  functions, called the nondegenerate part of $f$, is a nondegenerate quadratic form and the other function, called the residual part of $f$, is an element of $\m^3$. The number of variables in the residual part is equal to the corank, denoted by $c$, of $f$. In this paper we will only consider germs with corank $0$, $1$ and $2$.

The algorithm that we used to implement the Splitting Lemma is discussed in section \ref{TheSplittingLemma}.

In \cite{AVG1985} Arnold divided the real singularities of modality $0$ and $1$ up to stable equivalence into main types
which split up into more subtypes by changing the sign in front of certain
terms. Two functions are stably equivalent if they are right-equivalent after the direct addition of nondegenerte quadratic forms. Hence, for all nonquadratic forms $f$, after applying the Splitting Lemma we only need to consider the residual part of $f$ to complete the classification of $f$.   It can be easily seen that the subtypes are complex equivalent to a
complex singularity type of the same name as its corresponding real main
singularity type (see Table \ref{normal forms}). In fact there is a bijection between the complex singularity
types of modality $0$ and $1$ and the main real singularity types. Thus, if we
can determine the complex singularity type of a function germ in these cases, we only need to consider the subtypes of the
corresponding
main real singularity type to determine the real type of the function germ.

We denote the $k$-jet of a function germ $f$ by $\jt(f,k)$. In section \ref{ResultsRegardingTheFactorizationOfHomogeneousPolynomialsOverRAndQ} we discuss the factorization of the lowest nonzero jets of two right-equivalent functions. These results are used in sections \ref{RealSingularitiesOfZeroModality} to \ref{ExceptionalSingularities}, where the algorithms are given to classify germs in $\m^3$. Firstly, we show that the $k$-jets of two right-equivalent functions, where $k$ is the order of the functions, factorize in the same way over $\mathbb R$ (Lemma~\ref{kjet}). Since the real numbers are floating numbers in \textsc{Singular}, it is not always possible to factorize over the real numbers using \textsc{Singular}. In some cases we prove that these lowest jets will always factorize in linear terms over the rational numbers and thus in \textsc{Singular}, for example when the lowest jet is of degree $k$ and has $k$ similar factors (Lemma \ref{x^3}). In some other cases this unfortunately is not true. Since we can distinguish between some singularities, by only considering the number of real roots of its lowest nonzero jet, the similar factorization of the lowest nonzero jets is still useful in these cases. Because the lowest nonzero jet of a function germ is a homogeneous polynomial, we use dehomogenization and the library ``rootsur.lib" \cite{roots}, a library counting the real roots of a univariate polynomial in \textsc{Singular}, to count the roots. We for instance use this method in the classification of the $D_4$ (Algorithm~\ref{D[4]}) and $E_6$ (Algorithm \ref{E[6]}) cases. 

Sometimes other methods are used in addition to the above methods, for example the normal form of singularities of type $E_{14}$ is determined by systematically computing higher jets of the $\mathbb R$-algebra automorphism between the normal form of the function and the function itself considering the terms of these functions, after the lowest jet is determined using the above methods (Algorithm \ref{E[14]}).

In the cases $J_{10}$ and $J_{10+k}$, $k>0$ we used different normal forms than Arnold. These forms are more suitable to work with over the real numbers and specifically simplify the computations to classify these cases over the real numbers. We include a list of the normal forms we used at the beginning of section \ref{TheRealClassificationOfTheResidualPart}.

Lastly, we used the method of blowing up the origin (Algorithm \ref{BlowingUp}), which simplifies singularities. In the case $Y_{r,s}$, $r,s>4$ (Algorithm \ref{Y[r,s]}) enough information can be subtracted from the resulting, simplified, singularity such that the case can be solved, in some specific cases, unfortunately using approximations.  

\section{The Splitting Lemma}\label{TheSplittingLemma}
The following well-known theorem, called the Splitting Lemma, allows us to
reduce the classification to germs of corank $c$ or, equivalently, to germs
in $\m^3$ contained in $\mathbb R[[x_1,\ldots,x_c]]$.
\newtheorem{SplittingLemma}{Theorem}[section]
\newtheorem{AlgorithmSplittingLemma}[SplittingLemma]{Algorithm}
\begin{SplittingLemma}\label{SplittingLemma}
If $f\in \m^2\subset \mathbb R[[x_1,\ldots,x_n]]$ has corank $c$, then
\[ f\requiv g-\sum_{i=c+1}^{c+\lambda} x_i^2+\sum_{i=c+\lambda+1}^nx_i^2,\]
with $g\in \m^3$. $g$ is called the residual part of $f$ and $\lambda$
is called
the inertia index of the quadratic form of $f$. $g$ and $\lambda$ are
uniquely determined up to right equivalence.
\end{SplittingLemma}
We implemented Theorem \ref{SplittingLemma} in \textsc{Singular},
using the following algorithm. Since $\textsc{Singular}$ use real floating numbers, which do not represent the real numbers, instead of real numbers, rounding errors can occor working in this setting. On the other hand, Algorithm \ref{AlgorithmSplittingLemma} also function correctly working over the rational numbers, with a rational input polynomial. In this case the output polynomial $g$ will also be rational and calculations will be exact.

We denote the set of all monomials with nonzero coefficients that appear in a power series $f$, called the support of $f$, by $\supp(f)$.

\begin{AlgorithmSplittingLemma}(Splitting Lemma) \label{AlgorithmSplittingLemma}
\end{AlgorithmSplittingLemma}
\noindent\textnormal{\bf Input:} $f\in \m^2\subset\mathbb
R[x_1,\ldots,x_n]$ and $k\in\mathbb N$ such that $f$ is
$k$-determined.\newline
\textnormal{\bf Output:} $c:=\textnormal{corank}(f)$, $\lambda\in\mathbb
N$ and
$g\in \m^3\cap\mathbb R[x_1,\ldots,x_c]$ with \[\displaystyle
f\requiv g-\sum_{i=c+1}^{c+\lambda} x_i^2+\sum_{i=c+\lambda+1}^nx_i^2.\]
\begin{itemize}
\item $S:=\emptyset;$
\item for($i=1;\ i\le n;\  i++$)\newline
\phantom{}\quad if($\textnormal{jet}(f,2)\in \mathbb R[x_1,\ldots,\hat
x_i,\ldots,x_n]$);\newline
\phantom{}\quad\quad $S:=S\cup\{x_i\}$;\newline
\phantom{}\quad else\newline
\phantom{}\quad\quad $f:=\textnormal{jet}(f,k);$\newline
\phantom{}\quad\quad write $f$ as $f=ax_i^2+px_i+r$ with $p,r\in\mathbb
R[x_1,\ldots,\hat x_i,\ldots,x_n]$\newline
\phantom{}\quad\quad and $a\in\mathbb R[x_1,\ldots,x_n];$\newline
\phantom{}\quad\quad if($a(0)=0$)\newline
\phantom{}\quad\quad\quad choose $j\in\{i+1,\ldots,n\}$ with
$x_ix_j\in\textnormal{supp}(f)$;\newline
\phantom{}\quad\quad\quad apply $x_j\mapsto x_j+x_i$;\newline
\phantom{}\quad\quad\quad write $f$ as $f=ax_i^2+px_i+r$ with $p,r\in\mathbb
R[x_1,\ldots,\hat x_i,\ldots,x_n]$\newline
\phantom{}\quad\quad\quad and $a\in\mathbb R[x_1,\ldots,x_n];$\newline
\phantom{}\quad\quad\quad\quad while($p\neq 0$)\newline
\phantom{}\quad\quad\quad\quad\quad apply $x_i\mapsto
x_i-\frac{p}{2a(0)}$;\newline
\phantom{}\quad\quad\quad\quad\quad $f:=\textnormal{jet}(f,k);$\newline
\phantom{}\quad\quad\quad\quad\quad write $f$ as $f=ax_i^2+px_i+r$ with
$p,r\in\mathbb R[x_1,\ldots,\hat x_i,\ldots,x_n]$\newline
\phantom{}\quad\quad\quad\quad\quad and $a\in\mathbb
R[x_1,\ldots,x_n];$\newline
\item $c =\# S$;
\item change the order of variables such that $\displaystyle
f=g+\sum_{i=c+1}^na_ix_i^2$;
\item $\lambda:=\#(\{a_i:i=c+1,\ldots,n\}\cap\mathbb R^-)$;
\item change the order of variables such that
\[f=g-\sum_{i=c+1}^{c+\lambda}a_ix_i^2+\sum_{i=c+\lambda+1}^na_ix_i^2,\
a_i>0;\]
\item apply $x_i\mapsto 0, i=c+1,\ldots,n$ to $f$;
\item $g:=f$;
\item return $c, \lambda, g$;
\end{itemize}

\begin{proof}
If we can prove that we can write $f$ as \begin{equation}\label{eq4}
f=a+\sum_{j=1}^i a_jx_j^2+r
\end{equation}
where
$r\in\mathbb R[x_{i+1},\ldots,x_n],\quad a_j\in\mathbb R,\quad a\in \m^3$ and
the degree of $x_j$ in $a$ is greater than $1$, for all $j\in\{1,\ldots,i\}$,
after $i$ applications of the for-loop, then we have the desired result after
$n$ applications of the for-loop. We will prove this by induction. After $0$
applications it is trivial to write $f$ in the form of (\ref{eq4}) for $i=0$.
Suppose we can write $f$ in this form after $i-1$ applications. If $x_i\in
S$, then it is also trivial to write $f$ in the form of (\ref{eq4}). Suppose
$x_i\not\in S$. It follows from the induction hypotheses that $tx_j\not\in
\textnormal{supp}(f)$ for all monomials $t\in\mathbb R[x_1,\ldots,\hat
x_j,\ldots,x_n]$ and $j\in\{1,\ldots,i-1\}$. Then the $k$-jet of $f$ can
be written as
\[f=ax_i^2+px_i+r\]
with $p,r\in\mathbb R[x_1,\ldots,\hat x_i,\ldots,x_n]$ and $a\in\mathbb
R[x_1,\ldots,x_n]$. If $a(0)\neq 0$, then $x_i^2\in\textnormal{supp}(f)$. If
$a(0)=0$, then $x_i^2\not\in\textnormal{supp}(f)$. Since $x_i$ appears in some
term of the $2$-jet of $f$ and since it follows from the induction that
$x_ix_j\not\in\textnormal{supp}(f)$ for $j\in\{1,\ldots,i-1\}$, it follows
that
there exists a $j\in\{i+1,\ldots,n\}$ such that
$x_ix_j\in\textnormal{supp}(f)$.
If we choose such an index $j$ and apply the transformation
\begin{equation}\label{eq2}
x_j\mapsto x_j+x_i,\quad x_ix_j\mapsto x_ix_j+x_i^2,
\end{equation}
then $x_i^2\in\textnormal{supp}(f)$. Writing $f$ in the form $ax_i^2+px_i+r$,
we now have that $a(0)\neq0$.

Now suppose that $p\neq 0$, i.e.~there appear terms in $f$ where the degree
of $x_i$ is one. Then the while-loop applies and the following transformation
is made
\begin{equation}\label{eq3}
x_i\mapsto x_i-\frac{p}{2a(0)}\qquad
(\textnormal{this is allowed since }a(0)\neq 0).
\end{equation}
Then
\begin{eqnarray*}
f&=&a(x_i-\frac{p}{2a(0)})^2+p(x_i-\frac{p}{2a(0)})+r\\
&=&ax_i^2-\frac{apx_i}{a(0)}+\frac{ap^2}{4a^2(0)}+px_i-\frac{p^2}{2a(0)}+r\\
&=&ax_i^2+(1-\frac{a}{a(0)})px_i+\frac{ap^2}{4a^2(0)}-\frac{p^2}{2a(0)}+r\\
&=&ax_i^2+bpx_i+\frac{a'p^2}{4a^2(0)}+r',
\end{eqnarray*}
where
\begin{equation*}
b=1-\frac{a}{a(0)}\in\m,\quad a'=a-a(0)\in\m,\quad
r'=\frac{p^2}{4a(0)}-\frac{p^2}{2a(0)}+r\in\mathbb R[x_1,\ldots,\hat
x_i,\ldots,x_n].
\end{equation*}
Since
\begin{equation*}
b=b_2x_i^2+b_1x_i+b_0,
\end{equation*}
where
\begin{equation*}
b_2\in\mathbb R[x_1,\ldots,x_n] \quad\textnormal{and}\quad b_1,b_0\in\mathbb
R[x_1,\ldots,\hat x_i,\ldots,x_n],
\end{equation*}
we have that
\begin{eqnarray*}
bpx_i&=&b_2px_i^3+b_1x_i^2+b_0px_i\\
&=&(b_2px_i+b_1p)x_i^2+b_0px_i\\
&=&b'x_i^2+b_0px_i,
\end{eqnarray*}
where
\begin{equation*}
b'\in\m\in\mathbb R[x_1,\ldots,x_n]\quad\textnormal{and}\quad b_0\in\mathbb
R[x_1,\ldots,\hat x_i,\ldots,x_n].
\end{equation*}
Also
\begin{equation*}
\frac{a'p^2}{4a^2(0)}=c_2x_i^2+c_1x_i+c_0
\end{equation*}
where
\begin{equation*}
c_1,c_0\in\mathbb R[x_1,\ldots,\hat x_i,\ldots,x_n]\quad\textnormal{and}\quad
c_2\in\mathbb R[x_1,\ldots,x_n].
\end{equation*}
Since $a'\in\m$ and $p^2\in\m^2$ it follows that $\frac{a'p^2}{4a(0)}\in\m^3$
and hence $c_2\in\m$. Because $a(0)\neq 0$ and $b'(0)=0$ and $c_2(0)=0$, it
follows that $(a+b'+c_2)(0)\neq 0$. Furthermore,
since $b\in\m$, it follows that $b_0\in\m$ and hence
$\textnormal{order}(b_0p)>\textnormal{order}(p)$. Similarly since
$\textnormal{order}(\frac{a'p^2}{4d^2(0)})>\textnormal{order}(p)$ it
follows that $\textnormal{order}(c_1)>\textnormal{order}(p)$ and therefore
$\textnormal{order}(b_0p+c_1)>\textnormal{order}(p)$. Thus we conclude that
\begin{eqnarray}
f&=&(a+b'+c_2)x_i^2+(b_0p+c_1)x_i+r'+c_0\nonumber\\
&=&a''x_i^2+p''x_i+r''\label{eq1}
\end{eqnarray}
where
\begin{eqnarray*}
&&a'' = (a+b'+c_2)\in\mathbb R[x_1,\ldots,x_n],\quad
p'',r''\in\mathbb R[x_1,,\ldots,\hat x_i,\ldots,x_n],\\
&&a''(0)\neq0\quad\textnormal{and}\quad\textnormal{order}(p'')>\textnormal{order}(p).
\end{eqnarray*}
If $\textnormal{jet}(k,p''x_i)\neq 0$, then the while-loop will be applied
again
and we will get as output a polynomial of the form (\ref{eq1}). Since
the degree
of $p''$ strictly increases after each application of the while-loop,
$\textnormal{jet}(k,p''x_i)$ is equal to zero after finitely many applications
and the while-loop terminates. Considering the transformations  (\ref{eq2})
and
(\ref{eq3}), it follows that after each application of the while-loop,
it holds
that $tx_j\not\in\textnormal{supp}(f)$ for $j\in\{1,\ldots,i-1\}$. Hence
after $i$ applications of the for-loop, $f$ is in the form (\ref{eq4}).

Now, $x_i^2\in\textnormal{supp}(f)$ after $n$ applications of the for-loop
if and only if $x_i$ appears in some term of $\textnormal{jet}(2,f)$ after
$i-1$ applications of the for-loop, i.e.~$x_i$ will be in the Jacobian of
$f$ if and only if $x_i\in S$. Hence the corank of $f$ is $\#S$.
\end{proof}

\section{The real classification of the residual part of $f$}\label{TheRealClassificationOfTheResidualPart}

In \cite{AVG1985} Arnold divided the real singularities of modality $0$ and $1$, using stable equivalence, into main types which split up into more subtypes by changing the sign in front of certain terms. Each of these subtypes is complex equivalent to the complex singularity of the same name as its corresponding real main singularity type. We include the complex transformations for the corank two singularities in the table below. Since there is a real main singularity type of modality $0$ or $1$ for each complex singularity type of modality $0$ or $1$, respectively, there is a bijection between the main real singularity types of modality $0$ and $1$ and the complex singularity types of modality $0$ and $1$, respectively. It is not known yet whether the complex forms of higher modal cases  split up into corresponding real forms. The fact is that it is not known whether modality is preserved in these cases. In the next table we list the normal forms that are used in this article. From here onwards we will be working with stable equivalence. For all nonquadratic forms $f$ it is thus only necessary, after applying the Splitting Lemma to consider the residual part of function germs, i.e.~germs in $\m^3$.

\begin{table}[!hbp]
\centering
\caption{Real Normal Forms of modality $0$ and $1$.}
\label{normal forms}
\begin{tabular}{|c|c|c|c|c|}
\hline
&Complex NF & NF of real subtypes& Transformation&Restrictions\\\hline
$A_k$&$x^{k+1}$&$x^{k+1}$&$x\mapsto x$, $y\mapsto y$&$k\ge 1$\\
&&$-x^{k+1}$&$x\mapsto {i}^{\frac{2}{k+1}}x$, $y\mapsto y$&$k\ge 1$\\
\hline
$D_k$&$x^2y+y^{k-1}$&$x^2y+y^{k-1}$&$x\mapsto x$, $y\mapsto y$&$k\ge 4$\\
&&$x^2y-y^{k-1}$&$x\mapsto i^{\frac{2k-3}{k-1}}x$, $y\mapsto {i}^{\frac{2}{k-1}}y$&$k\ge 4$\\
\hline
$E_6$&$x^3+y^4$&$x^3+y^4$&$x\mapsto x$, $y\mapsto y$&-\\
&&$x^3-y^4$&$x\mapsto x$, $y\mapsto\sqrt{i} y$&-\\
\hline
$E_7$&$x^3+xy^3$&$x^3+xy^3$&$x\mapsto x$, $y\mapsto y$&-\\
\hline
$E_8$&$x^3+y^5$&$x^3+y^5$&$x\mapsto x$, $y\mapsto y$&-\\
\hline
$X_9$&$x^4+ax^2y^2+y^4$&$x^4+ax^2y^2+y^4$&$x\mapsto x$, $y\mapsto y$&$a^2\neq 4$\\
&&$x^4+ax^2y^2-y^4$&$x\mapsto x$, $y\mapsto \sqrt{i}y$&-\\
&&$-x^4+ax^2y^2+y^4$&$x\mapsto\sqrt{i}x$, $y\mapsto y$&-\\
&&$-x^4+ax^2y^2-y^4$&$x\mapsto\sqrt{i}x$, $y\mapsto \sqrt{i}y$&$a^2\neq 4$\\
\hline
$J_{10}$&$x^3+ax^2y^2+y^6$&$x^3+ax^2y^2+y^6$&$x\mapsto x$, $y\mapsto y$& $4a^3+27\neq0$\\
&&$x^3+ax^2y^2-y^6$&$x\mapsto x$, $y\mapsto \sqrt[3]{i}y$&$-4a^3+27\neq0$\\
\hline
$J_{10+k}$&$x^3+xy^4+ay^6$&$x^3+xy^4+ay^6$&$x\mapsto x$, $y\mapsto y$&$a\neq 0$, $k>0$\\
&&$x^3-xy^4+ay^6$&$x\mapsto x$, $y\mapsto \sqrt{i}y$&$a\neq 0$, $k>0$\\
\hline
$X_{9+k}$&$x^4+x^2y^2+ay^{4+k}$&$x^4+x^2y^2+ay^{4+k}$&$x\mapsto x$, $y\mapsto y$&$a\neq 0$, $k>0$\\
&&$x^4-x^2y^2+ay^{4+k}$&$x\mapsto x$, $y\mapsto iy$&$a\neq 0$, $k>0$\\
&&$-x^4+x^2y^2+ay^{4+k}$&$x\mapsto \sqrt{i}x$, $y\mapsto i^{\frac{3}{2}}y$&$a\neq 0$, $k>0$\\
&&$-x^4-x^2y^2+ay^{4+k}$&$x\mapsto \sqrt{i}x$, $y\mapsto \sqrt{i}y$&$a\neq 0$, $k>0$\\
\hline
$Y_{r,s}$&$x^2y^2+x^r+ay^s$&$x^2y^2+x^r+ay^s$&$x\mapsto x$, $y\mapsto y$&$a\neq 0$, $r,s>4$\\
&&$x^2y^2-x^r+ay^s$&$x\mapsto i^{\frac{2}{r}}x$, $y\mapsto i^{\frac{2r-2}{r}}y$&$a\neq 0$, $r,s>4$\\
&&$-x^2y^2+x^r+ay^s$&$x\mapsto x$, $y\mapsto iy$&$a\neq 0$, $r,s>4$\\
&&$-x^2y^2-x^r+ay^s$&$x\mapsto i^{\frac{2}{r}}x$, $y\mapsto i^{\frac{r-2}{r}}y$&$a\neq 0$, $r,s>4$\\
\hline
$E_{12}$&$x^3+y^7+axy^5$&$x^3+y^7+axy^5$&$x\mapsto x$, $y\mapsto y$&-\\
\hline
$E_{13}$&$x^3+xy^5+ay^8$&$x^3+xy^5+ay^8$&$x\mapsto x$, $y\mapsto y$&-\\
\hline
$E_{14}$&$x^3+y^8+axy^6$&$x^3+y^8+axy^6$&$x\mapsto x$, $y\mapsto y$&-\\
&&$x^3-y^8+axy^6$&$x\mapsto x$, $y\mapsto \sqrt[4]iy$&-\\
\hline
$Z_{11}$&$x^3y+y^5+axy^4$&$x^3+y^5+axy^4$&$x\mapsto x$, $y\mapsto y$&-\\
\hline
$Z_{12}$&$x^3y+xy^4+ax^2y^3$&$x^3y+xy^4+ax^2y^3$&$x\mapsto x$, $y\mapsto y$&-\\
\hline
$Z_{13}$&$x^3y+y^6+axy^5$&$x^3y+y^6+axy^5$&$x\mapsto x$, $y\mapsto y$& -\\
&&$x^3y-y^6+axy^5$&$x\mapsto i^{\frac{11}{9}}x$, $y\mapsto \sqrt[3]i y$& -\\
\hline
$W_{12}$&$x^4+y^5+ax^2y^3$&$x^4+y^5+ax^2y^3$&$x\mapsto x$, $y\mapsto y$&-\\
&&$-x^4+y^5+ax^2y^3$&$x\mapsto\sqrt{i} x$, $y\mapsto y$&-\\
\hline
$W_{13}$&$x^4+xy^4+ay^6$&$x^4+xy^4+ay^6$&$x\mapsto x$, $y\mapsto y$&-\\
&&$-x^4+xy^4+ay^6$&$x\mapsto \sqrt{i}x$, $y\mapsto i^{\frac{7}{8}}y$&-\\
\hline
\end{tabular}
\end{table}

It is known that the milnor number of a function germ $g$, which we denote by $\mu(g)$, is the same regardless whether we work over the complex- or real numbers. Therefore a singularity over the complex numbers is nondegenerate if and only if the singularity is nondegenerate over the real numbers. The restrictions to normal forms in the complex case are thus the same in the real case. Furthermore we know that we can transform the normal forms in the real case that differ from those we use in the complex case by a complex transformation to a complex normal form. Since the milnor number is invariant under complex transformations, we can then determine the restrictions. Let us take the $X_9$ case as an example and consider the real normal form $-x^4+ax^2y^2+y^4$. This normal form is complex equivalent to $x^4+aix^2y^2+y^4$, which corresponds to the given corresponding complex normal form. Since $a\in\mathbb R$, it follows that $(ia)^2=-a^2\neq4$ and therefore this singularity is nondegenerate for all values of $a$.

Using the $1-1$ correspondence between the main real singularity types and complex singularity types, and a \textsc{Singular} library ``classify.lib" \cite{classify}, that classifies complex singularities, classifying a real germ boils down to determining to which of the corresponding subtypes the germ is equivalent.

\subsection{Results regarding the factorization of homogeneous polynomials
over $\mathbb R$ and $\mathbb Q$}\label{ResultsRegardingTheFactorizationOfHomogeneousPolynomialsOverRAndQ}

The results in this subsection are preliminary results that will be used in sections \ref{RealSingularitiesOfZeroModality} to \ref{ExceptionalSingularities}.

Let $f,g\in \m^2\subset\mathbb R[[x_1,\ldots,x_n]]$ such that
$f\requiv g$, i.e.~$\phi(f)=g$, where $\phi$ is an $\mathbb R$-algebra
automorphism of  $\mathbb R[[x_1,\ldots,x_n]]$. Since $\phi$ is an automorphism it is defined by the images $\phi(x_1),\ldots,\phi(x_n)$. We define the $j$-jet of $\phi$, denoted by $\phi_j$, to be the automorphism defined by $\phi_j(x_1):=\jt(\phi(x_1),j),\ldots,\phi_j(x_n):=\jt(\phi(x_n),j)$. By using the next result, given $f$ and $g$, we can
determine $\phi_1$.

\newtheorem{kjet}{Lemma}[section]
\begin{kjet}\label{kjet}
Let $f,g\in \m^2\subset\mathbb R[[x_1,\ldots,x_n]]$. Suppose $f\requiv g$
and $k$
is the lowest degree of $f$. If jet$(f,k)$ factorizes as
\[f_1^{s_1}(x_1,\ldots,x_n)\cdots f_t^{s_t}(x_1,\ldots,x_n)\]
over $\mathbb R$, then jet$(g,k)$ will factorize
as \[f_1^{s_1} (\phi_1(x_1),\ldots,\phi_1(x_n)),\cdots
f_t^{s_t}(\phi_1(x_1)),\ldots,(\phi_1(x_n)))\] over $\mathbb R$, where $\phi$
is an $\mathbb R$-algebra automorphism of $\mathbb R[[x_1,\ldots,x_n]]$
such that $\phi(f)=g$.
\end{kjet}
\begin{proof}
Since $f\requiv g$, there exists a ring automorphism $\phi\in\mathbb
R[[x_1,\ldots,x_n]]$ such that  $\phi(f)=g$. Now
assume that the monomials of lowest degree $k$ of $f$ factorize
as \[f_1^{s_1}(x_1,\ldots,x_n)\cdots f_t^{s_t}(x_1,\ldots,x_n)\] over
$\mathbb R$, i.e.~$f=f_1^{s_1}\cdots f_t^{s_t}+f'$, where the degree and order of
$f_1^{s_1},\ldots,f_t^{s_t}$ is $k$ and the order of $f'$ is greater than $k$.
Now, for all $j\in\{1,\ldots,n\}$, \[\phi(x_j)=\phi_1(x_j)+\textnormal{terms of
degree higher than $1$}.\] Since $\phi$ is a homomorphism, it follows that

\begin{eqnarray*}
\phi(f)&=&f_1^{s_1}\big(\phi(x_1),\ldots,\phi(x_n)\big)\cdots
f_t^{s_t}\big(\phi(x_1),\ldots,\phi(x_n)\big)+\phi(f')\\&=&f_1^{s_1}\big(\phi_1(x_1),\ldots,\phi_1(x_n)\big)\cdots
f_t^{s_t}\big(\phi_1(x_1),\ldots,\phi_1(x_n)\big)\\&&+\big(\phi-\phi_1\big)(f_1^{s_1}\cdots f_t^{s_t})+\phi(f').\\\end{eqnarray*}
Since the $\deg\big(\phi_1(x_j))=1$ it follows that
\[\deg\big(f_1^{s_1}\big(\phi_1(x_1),\ldots,\phi_1(x_n)\big)\cdots
f_t^{s_t}\big(\phi_1(x_1),\ldots,\phi_1(x_n)\big)\big)=k.\] Because
\[\textnormal{order}\big(\big(\phi-\phi_1\big)(f_1^{s_1}\cdots f_t^{s_t})\big)>k\quad\textnormal{and}\quad\textnormal{order}\big(\phi(f')\big)>k\]
it follows that
\begin{eqnarray*}
\textnormal{jet}(\phi(f),k)=\textnormal{jet}(g,k)=f_1^{s_1}\big(\phi_1(x_1),\ldots,\phi_1(x_n)\big)\cdots
f_t^{s_t}\big(\phi_1(x_1),\ldots,\phi_1(x_n)\big).
\end{eqnarray*}
\end{proof}

Since \textsc{Singular} use real floating point numbers instead of real numbers, rounding errors in computations in this setting can
occur. Therefore, in this paper, the above result would not be of much help without the
following result.
\newtheorem{x^3}[kjet]{Lemma}
\begin{x^3}\label{x^3}
If $f\in\mathbb Q[x,y]$ is homogeneous and factorizes as (i) $g_1^d$, (ii)
$g_1\cdot g_2^{d}$ or as  (iii) $g_1^2(g_1-g_2)(g_1+g_2)$, $d>1$, over
$\mathbb
R$, where $g_1,g_2$ are polynomials of degree $1$, then $f$ will factorize as
(i) $ag_1'^d$, (ii) $ag_1'\cdot g_2'^d$, (iii) $ag_1'^2(g_1'-g_2')(g_1'+g_2')$ respectively,
where $g_1', g_2'$ are polynomials of degree $1$ over $\mathbb Q$ and
$a\in\mathbb Q$.
\end{x^3}
\begin{proof}

(i) Let $f=(a_1x+a_2y)^d$, $a_1,a_2\in\mathbb R$. Without loss of generality,
suppose $a_1\neq 0$. Then $f=a_1^d(x+\frac{a_2}{a_1}y)^d$. Since $f=
a_1^dx^d+da_1^{d-1}a_2x^{d-1}y+\cdots+a_2^dy\in\mathbb Q[x,y]$, it follows
that $a_1^d\in\mathbb Q$. Hence $(x+\frac{a_2}{a_1}y)^d\in\mathbb Q[x,y]$
from which it follows that $(x+\frac{a_2}{a_1})^d\in\mathbb Q[x]$. Since
$\mathbb Q$ is a perfect field it follows that $\frac{a_2}{a_1}\in\mathbb
Q$. Thus $f=a{g'}_1^d$, where $a:=a_1^d\in\mathbb Q$ and
$g_1=x+\frac{a_2}{a_1}y\in\mathbb Q[x,y]$.

(ii) Let $f=(a_1x+a_2y)(a_3x+a_4y)^{d}$, $a_1,\ldots,a_4\in\mathbb
R$. Suppose $a_1,a_3\neq 0$.
For the cases $a_1,a_4\neq 0$, $a_2,a_3\neq 0$ and $a_2,a_4\neq 0$ the
proofs are similar.
Similar as above, it follows that $a_1a_3^{d}\in\mathbb Q$. Hence
$(x+\frac{a_2}{a_1}y)(x+\frac{a_4}{a_3}y)^d\in\mathbb Q[x,y]$ from which
it follows that $(x+\frac{a_2}{a_1})(x+\frac{a_4}{a_3})^d\in\mathbb
Q[x]$. Since $\mathbb Q$ is a perfect field, it follows that
$(x+\frac{a_2}{a_1})(x+\frac{a_4}{a_3})\in\mathbb Q[x]$ and
hence that $(x+\frac{a_4}{a_3})^{d-1}\in\mathbb Q[x]$. Therefore
also $(x+\frac{a_4}{a_3})\in\mathbb Q[x]$ which implies that
$(x+\frac{a_2}{a_1})\in\mathbb Q[x]$.
Thus $f=ag'_1g'^d_2$ with $a:=a_1a_3^d\in\mathbb Q$,
$g'_1:=(x+\frac{a_2}{a_1}y)\in\mathbb Q[x,y]$, and
$g'_2:=(x+\frac{a_2}{a_1}y)\in\mathbb Q[x,y]$.

(iii) If $f$ factorizes as $g_1^2(g_1-g_2)(g_1+g_2)$ over $\mathbb R$, it
follows similarly as above that $g_1\in\mathbb Q[x,y]$. Therefore
$(g_1-g_2)(g_1+g_2)=g_1^2-g_2^2\in\mathbb Q[x,y]$ and hence $g_2^2\in\mathbb
Q[x,y]$. Using (i) $g_2\in\mathbb Q[x,y]$ and the result follows.
\end{proof}

\subsection{Real $0$-modal singularities of corank $0$ and $1$}\label{RealSingularitiesOfZeroModality}

If the corank of a singularity is $c$ then the output polynomial $g$, i.e.~the residual part of the input polynomial $f$, in
Algorithm \ref{AlgorithmSplittingLemma} will be a polynomial in $c$ variables. Troughout the rest of the paper we assume that $f$, and thus $g$, is a polynomial over $\mathbb Q$. 

If the output is $0$ then it follows  that $f$ is of type
$A_1^+$ or of type $A_1^-$ depending on the inertia index $\lambda$ of $f$. If the inertia index of $f$ is nonzero and less than the number of variables in the base ring, then $f$ is both of type $A_1^+$ and $A_1^-$, depending how one chooses to order the variables. If the inertia index is equal to the number of variables in the base ring, $f$ is of type $A_1^-$. Lastly, if the inertia index is $0$ then $f$ is of type $A_1^+$. 

If $c=1$, then the singularity is of type
$A_k^+$ or of type $A_k^-$ for some $k>1$. Furthermore $g$ is a univariate polynomial in this case, say $g\in\mathbb Q[x]$. Note that if $k$ is even then
$A_k^+\requiv A_k^-$. The value of $k$ is given by the order of $g$ minus $1$. This follows since $\pm x^{k+1}$ and $g$ are right-equivalent and thus have the same order. The sign of the singularity type is determined by the sign of the coefficient of $x^{k+1}$. This follows, since it follows from Lemma \ref{kjet}  that $\jt(g,1)=\pm(\phi_1(x))^{k+1}=\pm(\alpha x)^{k+1}$, where $\phi(\pm x^{k+1})=g$, $\alpha\in\mathbb R$ and the sign depends on the singularity type. Since $k+1$ is even and $\alpha\in\mathbb R$, $\phi$ does not change the sign of the coefficient of $x^{k+1}$. We use the following algorithm, after applying the
Splitting Lemma in case $c=0$ or $c=1$.

\newtheorem{A[k]}[kjet]{Algorithm}
\begin{A[k]}(Algorithm for the case $A_k$)
\end{A[k]}
\noindent\textnormal{\bf Input:} $f\in \mathbb Q[x_1,\ldots,x_n]$ of
complex singularity type $A_k$, the output polynomial $g$ after
applying Algorithm \ref{AlgorithmSplittingLemma}, the corank $c$ of $f$
and the inertia index $\lambda$ of $f$.\newline
\textnormal{\bf Output:} the real singularity type of $f$, i.e.~$A_k^-$
or $A_k^+$, $k\in\mathbb N$.
\begin{itemize}
\item if($c=0$)\newline
\phantom{}\quad if($\lambda<n$)\newline
\phantom{}\quad\quad type of singularity $:=A_1^+$;\newline
\phantom{}\quad else\newline
\phantom{}\quad\quad type of singularity $:=A_1^-$;
\item if($c=1$)\newline
\phantom{}\quad $k:= \ord(g)-1$;\newline
\phantom{}\quad if($k$ is even)\newline
\phantom{}\quad\quad type of singularity $:= A_k^+$;\newline
\phantom{}\quad else\newline
\phantom{}\quad\quad $s:=$ coefficient of $x^{k+1}$;\newline
\phantom{}\quad\quad if($s>0$)\newline
\phantom{}\quad\quad\quad type of singularity $:= A_k^+$;\newline
\phantom{}\quad\quad else\newline
\phantom{}\quad\quad\quad type of singularity $:= A_k^-;$
\item return type of singularity;
\end{itemize}

\subsection{Real $0$-modal singularities of corank $2$}

The goal of the rest of the paper is to classify singularities of corank $2$. In these cases $0\neq g\in\m^3$ is a polynomial in two variables, say $g\in\mathbb Q[x,y]$. Using the \textsc{Singular} library ``classify.lib'' we determine the complex singularity type and thus the main real singularity type of $g$, or equivalently $f$. The purpose of the remaining algorithms in the paper is to classify the correct real subtype of $g$, or equivalently $f$. We now consider each complex type, or equivalently every real main type, seperately.

If $g$ is of complex singularity type $D_4$, we only need
to consider the real singularity types $D_4^-$ and $D_4^+$ to determine the
real singularity type of $g$. Because $g\in \m^3$, $\ord(g)\ge3$,
and therefore $\jt(g,3)$ is homogeneous. We now consider
$\jt(g,3)$ and the way it factorizes.

Let $s_1$, $s_2$, $t_1$ and $t_2$  respectively be the coefficients of the monomials $x^3$, $y^3$, $x^2y$ and $y^2x$ in $g$. If, $s_1\neq 0$ and $s_2=0$ we ensure that the coefficient of $y^3$ is
nonzero by swapping the variables. If $s_1=0$ and $s_2=0$, the coefficient
of $y^3$ is nonzero after applying the transformation defined by $x\mapsto x$, $y\mapsto 2y$ and the transformation defined by $x\mapsto
x+y$, $y\mapsto y$ if $t_1=t_2$, and applying only the transformation defined by $x\mapsto x+y$, $y\mapsto
y$ if $t_1\neq -t_2$, to $g$. We now can write $g$ in the form
\begin{equation*}
g=a_0y^3+a_1x^2y+a_2xy^2+a_3x^3,\quad a_0,\ldots,a_3\in\mathbb Q,\quad a_0\neq 0.
\end{equation*}
Using Lemma \ref{kjet}, $\jt(g,3)$ factorizes into three linear factors over
$\mathbb R$ if and only if $g$ is of type $D_4^-$. Furthermore, since
$x\nmid
\jt(g,3)$ and $y\nmid \jt(g,3)$, $\jt(g,3)$ factorizes into three linear factors if and only if
the dehomogenization $\jt(g,3)(x,1)$ factorizes into three factors of degree $1$, i.e.~if and only if $\jt(g,3)$ has three real roots.

Here follows the algorithm we used.

\newtheorem{D[4]}[kjet]{Algorithm}
\begin{D[4]}(Algorithm for the case $D_4$)\label{D[4]}
\end{D[4]}
\noindent\textnormal{\bf Input:} $g\in \m^3\subset\mathbb Q[x,y]$ of complex
singularity type $D_4$.\newline
\textnormal{\bf Output:} the real singularity type of $g$, i.e.~$D_4^-$
or $D_4^+$.
\begin{itemize}
\item $g := \jt(g,3)$;
\item $s_1:=$ coefficient of ${x^3}$;
\item $s_2 :=$ coefficient of ${y^3}$;
\item if$(s_2=0$ and $s_1\neq0)$\newline
\phantom{}\quad Swap the variables $x$ and $y$;
\item   if$(s_2=0$ and $s_1=0)$\newline
\phantom{}\quad $t_1:=$ coefficient of ${x^2y}$;\newline
\phantom{}\quad $t_2:=$ coefficient of ${xy^2}$;\newline
\phantom{}\quad if$(t_1+t_2=0)$\newline
\phantom{}\quad\quad Apply $x\mapsto x$, $y\mapsto 2y$ (to $g$);\newline
\phantom{}\quad\quad Write $g$ as $g=t'_1x^2y+t'_2xy^2$, $t'_1, t'_2\in\mathbb
Q, t'_1\neq t'_2$;\newline
\phantom{}\quad Apply $x\mapsto x+y$, $y\mapsto y$ (to $g$);\newline
\phantom{}\quad Write $g$ as $g=a_0y^3+a_1x^2y+a_2xy^2+a_3x^3$, $a_0,\ldots,a_3\in\mathbb
Q$, $a_0\neq 0$;\newline
\phantom{}\quad Apply $x\mapsto 1$, $y\mapsto y$;\newline
\phantom{}\quad Write $g$ as $g=a_0y^3+a_1y^2+a_2y+a_3$;\newline
\phantom{}\quad $n:= \#$ real roots of $g$;\newline
\phantom{}\quad if$(n=3)$\newline
\phantom{}\quad\quad type of singularity := $D_4^-$;\newline
\phantom{}\quad if$(n\neq 3)$\newline
\phantom{}\quad\quad type of singularity := $D_4^+$;
\item return type of singularity;
\end{itemize}

Implementing  the Algorithms in this paper in \textsc{Singular}, we used the library ``rootsur.lib" \cite{roots} to count the number of real roots of a univariate polynomial.

The following two results are proved in \cite{Siersma}.

\newtheorem{kDeterminacyD[k]k>4}[kjet]{Lemma}
\begin{kDeterminacyD[k]k>4}\label{kDeterminacyD[k]k>4}
A singularity of type $D_k^+$ or $D_k^-$ is $k$-determined.
\end{kDeterminacyD[k]k>4}
\newtheorem{transformationD[k]}[kjet]{Lemma}
\begin{transformationD[k]}\label{transformationD[k]}
Let $j\ge 4$. Then
\[x^2y+a_0x^j+a_1x^{j-1}y+\cdots+a_jy^j\requiv x^2y+a_jx^j,\quad a_0,\ldots,a_j\in\mathbb R,\]
using the $\mathbb R$-algebra automorphisms
\begin{eqnarray*}
x&\mapsto&x+p_1,\textnormal{ where }
p_1=-\frac{1}{2}(a_1x^{j-2}+\cdots+a_{j-1}y^{j-2})\\
y&\mapsto&y+p_2,\textnormal{ where } p_2=-a_0x^{j-2} \,.
\end{eqnarray*}
\end{transformationD[k]}

By Lemma \ref{kDeterminacyD[k]k>4} the determinacy of a singularity of  complex type
$D_k$ is $k-1$. Therefore we only need to consider the
$(k-1)$-jet of $g$ in this case. Using Lemma~\ref{kjet} and Lemma~\ref{x^3}, we transform $g$ into a polynomial of the form
\[x^2y+\textnormal{terms of degree higher than $3$,}\]
by factorizing the $3$-jet of $g$ as $g_1^2g_2$, $g_1$ and $g_2$ of
degree $1$,
and then applying the automorphism defined by $g_1\mapsto x$, $g_2\mapsto y$ to $g$. We
now systematically consider the terms of each degree $3<j<k$. By applying the
transformations in Lemma~\ref{transformationD[k]}, for each $j$, only the term
$a_jy^j$ will each time possibly not vanish. If $j<k-1$, $a_j=0$, and if $j=k-1$, $a_j\neq 0$, otherwise $g$ is not of complex type
$D_k$. Thus, after applying these transformations,
we can write $g$ as $g=x^2y+\alpha y^{k-1}, a\neq0$. Clearly if $\alpha>0$ then
$x^2y+\alpha y^{k-1}\requiv x^2y+y^{k-1}$ and if $\alpha<0$ then $x^2y+\alpha y^{k-1}\requiv
x^2y-y^{k-1}$.

\newtheorem{D[k]k>4}[kjet]{Algorithm}
\begin{D[k]k>4}(Algorithm for the case $D_k$, $k>4$)
\end{D[k]k>4}
\noindent\textnormal{\bf Input:} $g\in \m^3\subset\mathbb Q[x,y]$ of complex
singularity type $D[k]$, $k\in\mathbb N$, $k>4$.\newline
\textnormal{\bf Output:} the real singularity type of $g$, i.e.~$D_k^-$
or $D_k^+$.
\begin{itemize}
\item $k:= \mu(g)$;
\item $d:=k-1;$
\item $g:=\jt(g,d);$
\item Factorize $\jt(g,3)$ as $g_1^2g_2$, $g_1$ and $g_2$ linear;
\item Apply $g_1\mapsto x$, $g_2\mapsto y$ (to $g$);
\item $\textnormal{for}(j=4; j<k; j++)$\newline
\phantom{}\quad $\textnormal{if}(\jt(g,j)-x^2y\neq0)$\newline
\phantom{}\quad\quad Write $\jt(g,j)-x^2y$ as
$a_0x^j+a_1x^{j-1}y+\cdots +a_jy^j,\quad a_0,\ldots a_j\in\mathbb Q$;\newline
\phantom{}\quad\quad Apply $x\mapsto x-\frac{1}{2}(a_1x^{j-2}+\cdots
+a_{j-1}y^{j-2})$, $y\mapsto y-a_0x^{j-2}$ (to $g$);\newline
\phantom{}\quad\quad $g:=\jt(g,d)$;\newline
\item Write $g$ as $g=x^2y+\alpha y^{k-1}$, $0\neq\alpha\in\mathbb Q$;
\item $\textnormal{if}(\alpha>0)$\newline
\phantom{}\quad type of singularity := $D_k^+$;\newline
else\newline
\phantom{}\quad type of singularity := $D_k^-$;
\item return type of singularity;
\end{itemize}

If $g$ is of complex type $E_6$ then $g\requiv x^3+y^4$ or $g\requiv x^3-y^4$. Therefore there exists an $\mathbb
R$-algebra automorphism $\phi$ of $\mathbb R[x,y]$ such that
$\phi(g)=(\phi(x))^3+(\phi(y))^4$ or such that
$\phi(g)=(\phi(x))^3-(\phi(y))^4$ . We ensure that the
coefficient of
$x^3$ is nonzero by swapping the variables if necessary. Now, using Lemma
\ref{kjet} and Lemma~\ref{x^3}, $\jt(g,3)$ factorizes as
$c(g_1)^3$, $c\in\mathbb Q$ and $g_1=b_0x+b_1y\in\mathbb
Q[x,y]$. Again,
using Lemma \ref{kjet}, it follows that by applying $x\mapsto\frac{x-b_1y}{b_0},\
y\mapsto y$ to $g$, $\phi$ is transformed such that $\phi_1(x)=c'x$, $c'\in\mathbb
R$. Since $\phi$ is an automorphism, $\phi_1(y)=d_0x+d_1y$, $d_0,d_1\in\mathbb R$,
with $d_1\neq 0$. Hence
\begin{equation*}
(\phi(y))^4=d_1^4y^4+\textnormal{terms of degree 4 and higher, not of the
form $\alpha y^4$, $\alpha\in\mathbb R$.}
\end{equation*}
If we can show that $(\phi(x))^3$ does not contain a term of the form
$\alpha y^4$, $\alpha\in\mathbb R$, then we can determine whether $g$
is of type $E_6^-$ or $E_6^+$ by considering the the sign of the coefficient of the monomial $y^4$. Now
\begin{eqnarray*}
\textnormal{jet}((\phi(x))^3,4)-\textnormal{jet}((\phi(x))^3,3)&=&3(\phi_1(x)^2)(\phi_2(x)-\phi_1(x))\\&=&3(c'x)^2(\phi_2(x)-\phi_1(x))
\end{eqnarray*}
 which means that $(\phi(x))^3$ does not have terms of the form $\alpha y^4$,
 $\alpha\in\mathbb R$ .
\newtheorem{E[6]}[kjet]{Algorithm}
\begin{E[6]}(Algorithm for the case $E_6$)\label{E[6]}
\end{E[6]}
\noindent\textnormal{\bf Input:} $g\in \m^3\subset\mathbb Q[x,y]$ of complex
singularity type $E[6]$.\newline
\textnormal{\bf Output:} the real singularity type of $g$, i.e.~$E_6^-$
or $E_6^+$.
\begin{itemize}
\item $h:= \jt(g,3);$
\item $s:=$ coefficient of ${x^3};$
\item $\textnormal{if}(s=0)$\newline
\phantom{}\quad Swap the variables $x$ and $y$;\\
\phantom{}\quad $s:=$ coefficient of ${x^3}$;
\item Factorize $h$ in linear factors over $\mathbb Q[x,y]$, with a factor
$g_1=b_0x+b_1y$;
\item Apply (to $g$) $x\mapsto \frac{x-b_1y}{b_0}$, $y\mapsto y$;
\item Write $g$ as $g=cx^3+d y^4+$ terms of degree $4$
and higher, not of the form $\alpha y^4$, $c,d\in\mathbb Q, \alpha\in\mathbb R$;
\item \textnormal{if}$(d>0)$\newline
\phantom{}\quad type of singularity := $E_6^+$;\newline
\phantom{} else\newline
\phantom{}\quad type of singularity := $E_6^-$;
\item return type of singularity;
\end{itemize}

\subsection{Real parabolic $1$-modal singularities of corank $2$}
Given a vector $w=(w_1,w_2)$ of integers we define the weighted degree of $x^\alpha y^\beta$ by \[w-\deg(x^\alpha y^\beta):=w_1\alpha+w_2\beta.\] 

\subsubsection{$X_9$}
For the case $X_9$, the normal form given by Arnold in \cite{AVG1985} is
\[
\pm x^4 +a x^2 y^2 \pm y^4 \,, a \in \R \,,
\]
with $a^2 \neq 4$ if the signs of $x^4$ and $y^4$ are equal.

The structure of the real equivalence classes of this main type is rather
involved in comparison to most other cases, e.g.~we have that
$x^4 +3 x^2 y^2 +y^4$ can be transformed into $x^4+\frac{6}{5} x^2 y^2 +y^4$
via $x \mapsto \sqrt[4]{\frac{1}{5}}(x+y)$,
$y \mapsto \sqrt[4]{\frac{1}{5}}(x-y)$ and that $-x^4 +4x^2 y^2 -y^4$ can be
transformed into $x^4 -10 x^2 y^2 +y^4$ via
$x \mapsto \sqrt[4]{\frac{1}{2}}(x+y)$ and
$y \mapsto \sqrt[4]{\frac{1}{2}}(x-y)$. We will first examine this structure in
detail and then give an algorithm to determine the equivalence class of a given
singularity of real main type $X_9$.

If we were in the complex case, the normal form would be
\begin{equation}\label{eqn:X9_cnf}
x^4 +ax^2y^2 +y^4
\end{equation}
with $a \in \C$ and $a^2 \neq 4$. In order to investigate for which values
$a' \in \C$ this can be transformed, for a given $a$, into
$x^4 +a'x^2y^2 +y^4$, we may proceed as follows, e.g. using \Singular:
We first apply a generic coordinate transformation
\[
\varphi: \; x \mapsto rx+sy, \; y \mapsto tx+uy, \; r,s,t,u \in \C
\]
to (\ref{eqn:X9_cnf}). Note that it suffices to consider the 1-jet of $\varphi$
because $X_9$ is 4-determined as can be shown using the highest corner method
(cf. TODO). We can then set up the ideal of transformations which take $a$ to
$a'$. By eliminating $r$, $s$, $t$ and $u$ from this ideal, we get a polynomial
of degree 6 in both $a$ and $a'$ which does not involve any other variables.
Factorizing this polynomial gives the following solutions:
\begin{align*}
a'_1 &= a  & a'_3 &= \frac{2a+12}{a-2} & a'_5 &= \frac{-2a+12}{a+2} \\
a'_2 &= -a & a'_4 &= \frac{2a-12}{a+2} & a'_6 &= \frac{2a+12}{-a+2}
\end{align*}
Hence the equivalence class of a generic complex singularity of main type $X_9$
can be represented by $x^4 +a'x^2y^2 +y^4$ for six different values of
$a' \in \C$. The non-generic cases are precisely those with
$a' \in \{-6, 0, 6\}$ where some of the above solutions coincide.

TODO: Draw a picture here?

If, as the next step, we restrict ourselves to singularities of main type $X_9$
given by polynomials with real coefficients, but still allow complex
transformations, the picture changes. In this case we have to distinguish
between the subtypes $X_9^{++}$, $X_9^{--}$, $X_9^{+-}$, and $X_9^{-+}$,
referring to the signs of $x^4$ and $y^4$ in the normal form, respectively. By
simply interchanging the variables $x$ and $y$, the last two subtypes can be
considered as one.
The question now is for which $i \in \{1, \ldots, 6\}$ there is a (possibly
complex) transformation that takes
$\pm x^4 +ax^2y^2 \pm y^4$ to $\pm x^4 +a'_i x^2y^2 \pm y^4$, and we may
ask this for each combination of signs.
It turns out that for transformations from $X_9^{++}$ to itself, from
$X_9^{--}$ to itself or from $X_9^{++}$ to $X_9^{--}$ and vice-versa, all the
solutions $a'_1, \ldots, a'_6$ hold valid. For those from $X_9^{+-}$ to
itself (and thus also from $X_9^{-+}$ to itself, from $X_9^{+-}$ to
$X_9^{-+}$
and vice-versa), only $a'_1$ and $a'_2$ remain. For any of the cases from
$X_9^{++}$ or $X_9^{++}$ to $X_9^{+-}$ or $X_9^{+-}$, there is a valid
transformation if and only if the parameter $a$ is in $\{-6, 0, 6\}$, and it
then becomes~$0$.

TODO: Draw a picture here?

If we allow only for real transformations, i.e. automophisms of $\R[[x,y]]$,
we get even more equivalence classes. Then, besides the identity map taking $a$
to $a'_1=a$ and identifying $X_9^{+-}$ with $X_9^{-+}$,
only the following equivalences hold:
\begin{align}
X_9^{++,a} \; &\requiv \; X_9^{++,a'_5} \; \text{ for } a > -2 \\
X_9^{--,a} \; &\requiv \; X_9^{--,a'_3} \; \text{ for } a < 2 \\
X_9^{++,a} \; &\requiv \; X_9^{--,a'_4} \; \text{ for } a < -2 \\
X_9^{--,a} \; &\requiv \; X_9^{++,a'_6} \; \text{ for } a > 2
\end{align}

TODO: Explain how this can be computed?
TODO: Draw a picture here.

The main tool for determining the subtype of a given real singularity of main
type $X_9$ is blowing-up. Again, it suffices to consider the 4-jet of the
polynomial $f = kx^4 + lx^3y + mx^2y^2 + nxy^3 + oy^4 \in \R[x, y]$ defining
the given singularity. Before the blowing-up, we have to make sure that the
coefficient of $x^4$ is non-zero. If this is not
the case, this can be easily achieved by switching the variables if
$o \neq 0$, or otherwise by applying either $y \mapsto x+y$, $y \mapsto 2x+y$,
or $y \mapsto 3x+y$. These transformations yield $l+m+n$, $2l+4m+8n$, and
$3l+9m+27n$, respectively, as coefficients of $x^4$, which cannot all at once
be zero.

TODO: Say what WE take as normal forms.

Applying the blowing-up map $x \mapsto x \,, y \mapsto 1$ results in a
polynomial $f_{\text{blowup}} \in \R[x]$ of degree 4 which has either four, two
or no real roots. This number is invariant with respect to any transformation
applied to $f$ before the blowing-up. Blowing-up the normal forms shows the
following picture:

\begin{tabular}{l|c}
Subtype & Number of real roots after the blowing-up \\ \hline
$X_9^{++,a} \,, a < -2$ & 4 \\
$X_9^{+-}$ & 2 \\
$X_9^{++,a} \,, a > -2$ & 0 \\
$X_9^{--,a} \,, a < 2$ & 0
\end{tabular}

To distinguish the two cases where the number of real roots is zero, it suffices
to look at the sign of the (now for sure non-zero) coefficient of $x^4$: It
stays invariant under any real transformation whatsoever.

\realclassify{} also tries to determine the parameter $a$ as far as possible,
but in some cases it is not determined uniquely, and several possibilities are
given. The method we use proceeds as follows: Very similar to the complex case,
eliminating the coefficients of a generic coordinate transformation from the
ideal of transformations which take the given polynomial $f$ to its normal form
yields an univariate polynomial $p_f(a)$ of degree 6 in the parameter $a$. The
possible values of $a$ are among the real roots of this polynomial.

In order to
test these roots separately, we try to factorize $p_f$, but there is no
guaranty that it splits into linear factors and there are indeed cases where it
does not. We then employ different means to exclude as many
factors as possible. One method is to verify, for each factor, that it has at
least one real root in the interval which contains $a$ according to the
subtype. Finally, another check can be done by switching over to the
quotient ring defined by each factor. If the value of $a$ is indeed one of the
real roots of a certain factor of $p_f$, then the ideal of generic
transformations which take the given polynomial to its normal form has at least
one real root in this ring, which can in many cases be checked by using the
command \texttt{nrRootsDeterm()} from \Singular{}'s \texttt{rootsmr.lib}.

The complete algorithm to determine the subtype and the parameter of a given
real singularity of main type $X_9$ now reads as follows:


\subsubsection{$J_{10}$}
For the main real singularity type $J_{10}$ we use the complex normal form $x^3+axy^4+ y^6$, which splits up into $x^3+axy^4\pm y^6$ in the real case (see Table \ref{normal forms}),  instead of the complex normal form $x^3+ax^2y^2+xy^4$, which splits up into $x^3+ax^2y^2\pm xy^4$, used by Arnold in \cite{AVG1985}. Using this different normal form makes calculations in \textsc{Singular}, determining the real singularity type easier. Furthermore, note that, in mapping $x\mapsto -x$, in the real normal forms we used, does not change the sign of $y^6$, while in Arnold's normal forms it changes the sign of the term $xy^4$, which makes the normal forms we used in general more suitable for computations in  the real case.

Now, having a polynomial $g$ over $\mathbb Q$ of complex type $J_{10}$, using Lemma \ref{kjet} and Lemma \ref{x^3}, we transform $g$ to a polynomial of the form \[b''x^3+\textnormal{ terms of higher degree,}\] where $b''\in\mathbb Q$.  Since $g$ is right-equivalent to one of the germs $x^3+axy^4\pm y^6$, for some $a\in\mathbb R$, there exists an $\mathbb R$-algebra automorphism $\phi$ such that $\phi(x^3+axy^4+y^6)=g$ or such that $\phi( x^3+axy^4-y^6)=g$. Clearly, $\jt(\phi(x),1)=\sqrt[3]{b''}x$. Hence \[g=b'' x^3+c''x^2y^2+d''xy^4+e''y^6+\textnormal{ terms of higher $w$-degree,}\] where $b'',c'',d'',e''\in\mathbb{Q}$ and $w=(\frac{1}{3},\frac{1}{6})$. Applying the map defined by $x\mapsto \frac{1}{\sqrt[3]{b''}}x$ and $y\mapsto\frac{1}{\sqrt[6]{b''}}y$ transforms $g$ to\[g=x^3+c'x^2y^2+d'xy^4+e'y^6+\textnormal{ terms of higher $w$-degree,}\] where $c'=\frac{c''}{b''}\in\mathbb{Q}$, $d'=\frac{d''}{b''}\in\mathbb{Q}$ and $e'=\frac{e''}{b''}\in\mathbb{Q}$. (Since it is only necessary to consider the weighted 1-jet of $g$ from here onwards, as is clear from the rest of the arguments, we may instead of the above transformation, transform $g$ by multiplication with the scalar $\frac{1}{b''}$). We get rid of the term $c'x^2y^2$ by applying te map defined by $x\mapsto x-\frac{c'}{3}y^2$ and $y\mapsto y$, i.e.
\[g=x^3+dxy^4+ey^6+g_1,\]where $d,e\in\mathbb Q$ and $w-\deg(g_1)>1$. Note that after the above transformations $\phi_1(x)=x$. Hence it follows that $w-\deg(\phi(g_1))>1$. Therefore the singularity type of $g$ is determined by the sign of $e$.

\newtheorem{J[10]}[kjet]{Algorithm}
\begin{J[10]}(Algorithm for the case $J_{10}$)
\end{J[10]}
\noindent\textnormal{\bf Input:} $g\in \m^3\subset\mathbb Q[x,y]$ of complex singularity type $J_{10}$.\newline
\textnormal{\bf Output:} the real singularity type of $g$, i.e.~$J_{10}^-$ or $J_{10}^+$.
\begin{itemize}
\item $h:= \jt(g,3)$;
\item $s_1:=$ coefficient of $x^3$;
\item $s_2:=$ coefficient of $y^3$;
\item if ($s_1=0$)\newline
\phantom{}\quad\quad Swap the variables $x$ and $y$;
\item factorize $h$ over $\mathbb Q[x,y]$, with a factor $t_1x+t_2y$;
\item apply (to $g$) $x\mapsto\frac{x-t_2y}{t_1}$, $y\mapsto y$;
\item write $g$ as $b''x^3+\textnormal{terms of higher degree},\quad b''\in\mathbb Q$;
\item if $(b'' <0)$\newline
\phantom{}\quad $x\mapsto -x$, $y\mapsto y$;
\item $g:= \frac{1}{\mid b''\mid} g$;
\item write $g$ as $x^3+c'x^2y^2+d'xy^4+e'y^6+$ terms of higher $w$-degree, where $w=(\frac{1}{3},\frac{1}{6})$ and $c',d',e'\in\mathbb Q$;
\item Apply (to $g$) $x\mapsto x-\frac{c'}{3}y^2$, $y\mapsto y$;
\item write $g$ as $x^3+dxy^4+ey^6+$ terms of higher $w$-degree',\quad $c,e\in\mathbb Q$.
\item if $(e>0)$\newline
\phantom{}\quad  type of singularity $:=J_{10}+$;\newline
\phantom{}else\newline
\phantom{}\quad type of singularity $:=J_{10}-$;
\item return type of singularity;
\end{itemize}

\subsection{Real hyperbolic $1$-modal singularities of corank $2$}
In the $J_{10+k}$ $k>0$ cases, similar to the $J_{10}$ case, we have used different normal forms than Arnold in \cite{AVG1985}.
Here we have used the complex normal form $x^3+xy^4+a y^{6+k}$, instead of the normal form $x^3+x^2y^2+ay^{6+k}$. Since the normal form $x^3+xy^4+a y^{6+k}$ does not split up in the real case, it is much more suitable than the normal form $x^3+x^2y^2+ay^{6+k}$ that splits up in $x^3\pm x^2y^2+ay^{6+k}$ in the real case. Using this normal form we thus only need the complex classification in this case.

Considering the case when $g$ is of complex type $X_{9+k}$, we know that $g$ is of the
form \[g=a_0x^4+a_1x^3y+a_2x^2y^2+a_3xy^3+a_4y^4+\textnormal{ terms of higher degree,}\]
$a_0,\ldots,a_4\in\mathbb Q$.
If $a_0= 0$, we first transform $g$ to a polynomial where the coefficient of $x^4$ is nonzero. If $a_4\neq
0$ we can achieve this by the transformation defined by $x\mapsto y$, $y\mapsto x$.

If $a_4=0$, then at least one of $a_1$, $a_2$ or $a_3$ is not $0$, otherwise the
singularity  is not of complex type $X_{9+k}$.  If we apply the transformation defined by $x\mapsto x+y$ and
$y\mapsto y$ in this situation, we will transform $g$ so that the new
coefficient of $x^4$ is $a_1+a_2+a_3$. Thus if we can ensure that $a_1+a_2+a_3\neq 0$, we
can use this map and have the desired result. Now, since $v_1=a_1+a_2+a_3$,
$v_2=2a_1+4a_2+8a_3$ and $v_3=3a_1+6a_2+9a_3$ is linear independent in the
three dimensional
$\mathbb R$-vector space with formal basis $(a_1, a_2, a_3)$, it follows that all
three vectors are zero if and only if $a_1,a_2$ and $a_3$ is zero. Knowing $a_1$, $a_2$
and $a_3$ cannot all be zero, at least one of $v_1$, $v_2$ or $v_3$ is not
zero. If $v_1=0$ and $v_2\neq 0$, then the map defined by $x\mapsto x$ and $y\mapsto 2y$
will transform $g$ to \[g=a_0'x^4+a_1'x^3y+a_2'x^2y^2+a_3'xy^3+a_4'y^4+\textnormal{
terms of higher degree,}\] where $a_1'+a_2'+a_3'=2a_1+4a_2+8a_3\neq 0$ and $a_0',\ldots,a_4'\in\mathbb Q$. If $v_1=0$ and
$v_3\neq 0$, then the map defined by $x\mapsto x$ and $y\mapsto 3y$ will transform $g$
to  \[g=a_0'x^4+a_1'x^3y+a_2'x^2y^2+a_3'xy^3+a_4'y^4+\textnormal{ terms of higher
degree,}\]where $a_1'+a_2'+a_3'=3a_1+6a_2+9a_3\neq0$. 

Now, $g\requiv h$, where $h=x^4+x^2y^2+ay^{4+k}$ if $g$ is of type
$X_{9+k}^{++}$, $h=-x^4-x^2y^2+ay^{4+k}$ if $g$ is of type $X_{9+k}^{--}$,
$h=-x^4+x^2y^2+ay^{4+k}$ if $g$ is of type $X_{9+k}^{-+}$ and
$h=x^4-x^2y^2+ay^{4+k}$ if $g$ is of type $X_{9+k}^{+-}$, for some $0\neq a\in\mathbb R$.  It follows
from Lemma~\ref{kjet} that $\textnormal{jet}(g,4)$ will factor as
$\pm\phi_1(x)^2(\phi_1(x)^2+\phi_1(y)^2)$ in the first two cases and as
$\pm\phi_1(x)^2(\phi_1(x)-\phi_1(y))(\phi_1(x)+\phi_1(y))$ in the second two
cases, where $\phi$ is the $\mathbb R$-algebra automorphism such that
$\phi(g)=h$. Since $a'_0\neq 0$,  the dehomogenization $\jt(g,1)(x,1)$ has one root in the first two
cases and
three roots in the last two cases.

Now, since the sign of the terms in the $4$-jet of $h$ in the first two cases
does not differ and the power of both $x$ and $y$ in all the terms is even,
we only need to consider the sign of the $x^4$-term to distinguish between
the $X_{9+k}^{++}$ and $X_{9+k}^{--}$ cases.

If $\jt(g,1)$ has three roots, we transform $g$ by using Lemma \ref{kjet}
and Lemma~\ref{x^3} such that $\phi_1(x)=\alpha x$, $\alpha\in\mathbb R$. Hence we
can now
write $\jt(g,1)$ in the form $\jt(g,1)=a_0x^4+a_1x^3y+a_2x^2y^2=\pm(\alpha x)^2((\alpha
x)^2\mp\phi_1(y)^2)=\pm(\alpha x)^4\mp (\alpha x)^2\phi_1(y)^2$, $a_0,a_1,a_2\in\mathbb Q$. Since
$\phi$ is
an automorphism, $\phi_1(y)=\beta x+\gamma y$ with $\gamma \neq 0$,
and since the sign of $\gamma$ does not influence the sign of $(\alpha
x)^2\phi_1(y)^2$, and we know that the sign in front of $(\alpha x)^4$
and $(\alpha x)^2\phi_1(y)^2$ differ, we decide between the cases
$X_{9+k}^{+-}$ and $X_{9+k}^{-+}$ by considering the sign of the term $x^2y^2$.

\newtheorem{X[9+k]}[kjet]{Algorithm}
\begin{X[9+k]}(Algorithm for the case $X_{9+k}$)\label{X[9+k]}
\end{X[9+k]}
\noindent\textnormal{\bf Input:} $g\in \m^3\subset\mathbb Q[x,y]$ of complex
singularity type $X_{9+k}$.\newline
\textnormal{\bf Output:} the real singularity type of $g$, i.e.~$X_{9+k}^-$
or $X_{9+k}^+$.
\begin{itemize}
\item $g:=\jt(g,4)$;
\item $s_1:=$ coefficient of ${x^4}$;
\item $s_2:=$ coefficient of ${y^4}$;
\item if$(s_2\neq0$ and $s_1=0)$\newline
\phantom{}\quad Swap the variables $x$ and $y$;\newline
\item if$(s_2\neq0$ and $s_1=0)$\newline
\phantom{}\quad $t_1:=$ coefficient of ${x^3y}$;\newline
\phantom{}\quad $t_2:=$ coefficient of ${x^2y^2}$;\newline
\phantom{}\quad $t_3:=$ coefficient of ${xy^3}$;\newline
\phantom{}\quad if$(t_1+t_2+t_3=0)$\newline
\phantom{}\quad\quad if$(2t_1+4t_2+8t_3\neq0)$\newline
\phantom{}\quad\quad\quad Apply $x\mapsto x$, $y\mapsto 2y$ (to $g$);\newline
\phantom{}\quad\quad else\newline
\phantom{}\quad\quad\quad Apply $x\mapsto x$, $y\mapsto 3y$ (to $g$);\newline
\phantom{}\quad Apply $x\mapsto x$, $y\mapsto x+y$ (to $g$);
\item Write $g$ as $g=a_0x^4+a_1x^3y+a_2x^2y^2+a_3xy^3+a_4y^4,\quad a_0,\ldots,a_4\in\mathbb Q$;
\item $g':=g(x,1)=a_0x^4+a_1x^3+a_2x^2+a_3x+a_4;$
\item $n:=\#$ real roots of $g'$;
\item if$(n=1)$\newline
\phantom{}\quad if$(a_0>0)$\newline
\phantom{}\quad\quad type of singularity$ :=X_{9+k}^{++}$;\newline
\phantom{}\quad else\newline
\phantom{}\quad\quad type of singularity$ := X_{9+k}^{--}$;
\item if$(n=3)$\newline
\phantom{}\quad Factorize $g$ in linear factors over $\mathbb Q[x,y]$,
with $g_1:=t_1x+t_2y$ the factor\newline
\phantom{}\quad with multiplicity $2$;\newline
\phantom{}\quad Apply $x\mapsto\frac{x+t_2y}{t_1}$, $y\mapsto y$ (to $g$);\newline
\phantom{}\quad Write $g$ as $g=a_0x^4+a_1x^3y+a_2x^2y^2,\quad a_0,a_1,a_2\in\mathbb Q$;\newline
\phantom{}\quad if$(a_2>0)$\newline
\phantom{}\quad\quad type of singularity$ := X_{9+k}^{-+}$;\newline
\phantom{}\quad else\newline
\phantom{}\quad\quad type of singularity$ := X_{9+k}^{+-}$;
\item return type of singularity;
\end{itemize}

Next we consider the case when $g$ is of one of the following complex cases: $Y_{r,s}$ and $\widetilde Y_r$, $r,s>4$. Note that in the complex case, normal forms of type $\widetilde Y_r$ are right-equivalent to normal forms of type $Y_{r,s}$. This is unfortunately not true in the real case and we hence have to treat these cases separately.  Similar to the $D_4$ case, we distinguish between the two cases, using Lemma \ref{kjet} and the library ``rootsur.lib", noticing that the $4$-jet of polynomials of type $Y_{r,s}$ factorize into four linear factors, while the $4$-jet of polynomials of type $\widetilde Y_r$ do not have any linear factors.

For the case when $g$ is of type $\widetilde Y_r$, for some $r$ we have to distinguish between the following normal forms in the real case $\pm(x^2+y^2)^2+ax^r$ and determine the value of $r$. Using the fact that the $4$-jet $(x^2+y^2)^2$ is always positive, substituting any real numbers for $x$ and $y$,  we distinguish between the two cases by considering the sign of $a_0$, after transforming the input polynomial to a polynomial which $4$-jet is of the form $a_0x^4+a_1x^3y+a_2x^2y^2+a_3x^3y+a_4y^4$. Such a transformation is used in the calculations of the classification of the $X_{9+k}$ case. It is already known (see~\cite{AVG1985}) that $r=\frac{\mu-9}{2}+4$.

Considering the cases when $g$ is of type $Y_{r,s}$, for some $r$ and $s$, $r,s>4$, we have to distinguish between the normal forms $\pm x^2y^2\pm x^r+ay^s$ and determine the values of $r$ and $s$.  We determine the sign in front of the monomial $x^2y^2$ in the same way as determining the sign in front of the monomial $(x^2+y^2)^2$ in the case $\widetilde Y_r$.

For the determining of $r$ and $s$ we use the method of blowing up the origin. By blowing up the origin once, polynomials right-equivalent to one of the normal forms $\pm x^2y^2\pm x^r+ay^s$ transform to two different germs of which one is right-equivalent to one of the germs $\pm x^2\pm x^{r-4}+ax^{s-4}y^s$ and one is right-equivalent to one of the germs  $\pm x^2\pm x^{s-4}+ax^{r-4}y^r$, for some $0\neq a\in \mathbb R$, if $r$ and $s$ are different, and to one germ which is right-equivalent to one of the germs $\pm x^2\pm x^{r-4}+ax^{r-4}y^r$, for some $0\neq a\in\mathbb R$, if  $r=s$. Since the milnor number of $f_1=\pm x^2\pm x^{s-4}+ax^{r-4}y^r$ is $s-5$ and the milnor number of $f_2=\pm x^2\pm x^{r-4}+ax^{s-4}y^s$  is $r-5$, if $r,s\neq 4$, we determine $r$ and $s$ by calculating the milnor numbers of the resulting germs. Now, if necessary swap $x$ and $y$ such that $r\le s$. If $r$ is uneven, then $x^2y^2+x^{r}+ay^s\requiv x^2y^2- x^{r}+ay^s$ and $-x^2y^2+x^{r}+ay^s\requiv -x^2y^2- x^{r}+ay^s$. When $r=6$ the sign of the monomial $x^r$ in the normal form of $g$ is negative if the inertia index of $f_1$, after applying the Splitting Lemma, is greater than zero, if the sign of $y^2$ in $f_1$ is positive, and is greater than one if the sign of $y^2$ in $f_1$ is negative. Otherwise the sign of $x^r$ in the normal form of $g$ is positive. If $r$ is even and $r\neq 6$ we determine the sign in front of the monomial $x^r$, similar to previous calculations, considering the sign of $a_0$ after transforming the residual part to a germ of which the $r-4$-jet is of the form $a_0x^{r-4}+a_1x^{r-5}y+\cdots+a_{r-5}x^{r-4}y^{r-5}+a_{r-4}y^{r-4}$.  

We implemented Algorithm \ref{BlowingUp} that blows an input polynomial $g$ in $\mathbb Q[x_1,x_2]$ of complex type $Y_{r,s}$ up at the origin and gives as output a list $L$, containing the following entries. $L[1]$ and $L[2]$: the values of $r$ and $s$, $r\le s$, respectively; $L[3]$: the inertia index of the resulting germ, after blowing up $g$, with lowest milnor number, if this resulting germ have a singularity on the exceptional divisor, i.e. $L[1]\neq 5$, and $0$ otherwise; $L[4]$: the sign of the monomial $x_1^r$ or $x_2^r$ in the normal form of the resulting singularity with lowest milnor number; $L[5]$ the value $1$ if the resulting germs have singularities that occur at irrational points on the exceptional divisor, and $0$ otherwise. If the value of $L[5]$ is $1$, the values of $L[3]$ and $L[4]$ is determined using floating real numbers and thus may be wrong. 

In Algorithm \ref{BlowingUp}, we consider the different blowing up charts seperately. Without loss of generality, let us consider the chart where $x_i$ is the local equation for the exceptional divisor. After determining the strict transform $p$, we determine whether $p$ has singularities on the exceptional divisor, and if that is the case, where on the exceptional divisor do the singularities occur. We do this by determining a minimal prime decomposition $P_1\cap\cdots\cap P_s$, of the radical of the ideal $P=\langle \textnormal{jacob}(p),p,x_i\rangle$ over $\mathbb Q$, using the singular library ``primdec.lib" \cite{primdec.lib} (If $p$ has zero or one singularity on the exceptional divisor, $s=1$, otherwise, if $p$ has two singularities on the exceptional divisor, $s=1$ or $s=2$, depending whether all the $P_j$'s are prime over $\mathbb R$).  If $P_j$ is not prime over $\mathbb R$ for any $j$, then $p$ has singularities at irrational points on the exceptional divisor. This can be determined by considering the degrees of a reduced standard basis, using lexicographical ordering, for every $P_j$. If the degree of any basis element of $P_j$, for some $j$, is of degree $2$ then $P_j$ is not prime over $\mathbb R$. 

As we will see later, if $p$ intersect the exceptional divisor smoothly or the singularities of $p$ occur at rational points on the exceptional divisor in one chart, this will be the case in both charts. Let us first consider this case. In this case we firstly determine a list $L_1$ which contains the resulting singularities at $0$, or $x_1$ in case $p$ intersects the exceptional divisor smoothly, in both charts, after $g$ is blown up. Considering the chart with exceptional divisor $x_i$, if $P=\langle 1\rangle$, then $p$ intersect the exceptional divisor smoothly and $x_1$ is added to the list $L_1$. When $p$ does not intersect the exceptional divisor smoothly, we can determine the rational points on the exceptional divisor where singularities occur by determining the zero set of $P$. We do this by determining the zero point $(s_{1j},s_{2j})$ of each $P_j$, using a reduced basis determined by using lexicographical ordering. Transforming $p$ by the transformation defined by $x_1\mapsto x_1-s_{1j}$, $x_2\mapsto x_2-s_{2j}$ we map the singularity that occur at  $(s_{1j},s_{2j})$ to $0$. We add these resulting singularities to $L_1$. Repeating the process for both charts, $L_1$ contains all the resulting singularities, possibly repeated, after blowing up $g$ at the origin, and $x_1$ for each resulting germ that intersects the exceptional divisor smoothly. We simplify $L_1$ by deleting repeating singularities and ordering the entries according to the milnor number of the entries. Having the list $L_1$, the methods discussed above are used to determine the values of the list $L$. 

If there occur a singularity of $p$ at an irrational point in some chart on the exceptional divisor both charts will contain two singularities of the strict transform at irrational points on the exceptional divisor. Thus it is only necessary to consider one chart. This can be seen as follows. Since $g$ is of type $Y_{r,s}$, $g$ is of the form \[g=(a_0x_1+a_1x_2)^2(b_0x_1+b_1x_2)^2+\textnormal{ higher terms in $x_1$ and $x_2$, $a_0,a_1,b_0,b_1\in\mathbb R$.}\] Without loss of generality we consider the chart where $x_1$ is the local equation for the exceptional divisor. Then \[p=(a_0+a_1x_2)^2(b_0+b_1x_2)^2+\textnormal{ terms that are divisible by $x_1$.}\] Taking $p$ and $x_1$ in $P$ into account the only possible zero points of $P$ are the roots of the rational polynomial $p_1 = (a_0+a_1x_2)^2(b_0+b_1x_2)^2$. If $a_1=0$ or $b_1=0$ it follows from the fact that $\mathbb Q$ is a perfect field that $p_1$ only has a rational root. Hence $a_1\neq 0$ and $b_1\neq 0$. Therefore   $(0,-\frac{a_0}{a_1})$ and $(0,-\frac{b_0}{b_1})$ are the only possible zero points of $P$. Because one of these points are irrational $a_0$ and $b_0$ are nonzero. Since $\mathbb Q$ is a perfect field and $(a_0+a_1x_2)^2(b_0+b_1x_2)^2$ is  a rational polynomial it follows that $(a_0+a_1x_2)(b_0+b_1x_2)$ is also a rational polynomial. Therefore both roots $-\frac{a_0}{a_1}$ and $-\frac{b_0}{b_1}$ are irrational if any one of the roots is irrational. Furthermore, $(a_0+a_1t)(b_0+b_1t)\in\mathbb Q(t)$ is a minimal polynomial for $-\frac{a_0}{a_1}$ and $-\frac{b_0}{b_1}$ over $\mathbb Q$. Hence $P$ has zero points at $-\frac{a_0}{a_1}$ and $-\frac{b_0}{b_1}$ or no zero points at all. Since every chart contain the whole exceptional divisor, except one point, each chart will contain at least one singular point of $p$, and hence in this case two singular points of $p$, on the exceptional divisor. Similarly, the singularities of $p$ on the exceptional divisor in the chart with exceptional divisor $x_2$, will occur at the irrational points $(-\frac{a_1}{a_0},0)$ and $(-\frac{b_1}{b_0},0)$. Furthermore, as we will see in the next paragraph, in such cases the milnor numbers of the two resulting singularities will be equal. Hence such cases will only occur if the normal form of $g$ is of either one of the forms $\pm x_1^2x_2^2\pm x_1^r+ax_2^r$.

In such a case, we first ensure that the singularities of $p$ occur at irrational points $(0,e)$ and $(0,-e)$, for some $e\in\mathbb R$, on the exceptional divisor. This is done by applying the rational transformation $x_1\mapsto x_1$, $x_2\mapsto x_2-(\frac{a_0b_1+b_0a_1}{a_1b_1})x_1$ to $g$. Now $g$ will be of the form
\[g=c(x_1+ex_2)^2(x_1-ex_2)^2+\textnormal{ higher terms in $x_1$ and $x_2$, $c,e\in\mathbb R$}.\]
Determining $p$, again, we have
\[p=c(1+ex_2)^2(1-ex_2)^2+\textnormal{ terms that are divisible by $x_1$.}\]
 Working over the ring $\mathbb Q(t)\cong \mathbb Q(e)\cong \mathbb Q (-e)$, with minimum polynomial $(t-e)(t+e)$, we shift the singularities of $p$ to $0$ by the transformations defined by $x_1\mapsto x_1$, $x_2\mapsto x_2-t$. Since $t$ can represent either $e$ or $-e$, the resulting polynomial represent both singularities working over $\mathbb Q(t)$. The milnor numbers for the resulting singularities, and thus L[1] and L[2], can be detemined working over the ring $\mathbb Q(t)$. Since both resulting singularities are represented by one polynomial over $\mathbb Q(t)$, the milnor number will be the same in both cases. Since $t$ represents $e$ or $-e$, $t$ does not have a sign. Therefore it is not enough working over $\mathbb Q(t)$ when determining the inertia index of the resulting singularities or the sign of the monomial $x_1^r$ or the monomial $x_2^r$ in  any one of the resulting normal forms. The resulting singularities over $\mathbb R$ can be determined by replacing $t$ with $e$ and $-e$, respectively. Since it is only possible to work with floating real numbers in \textsc{Singular} approximation errors may occur. We determine the approximated values of $e$ and $-e$ using the library ``solve.lib" \cite{solve.lib} in \textsc{Singular}. Using the previously discussed methods the approximated values of $L[3]$ and $L[4]$ can now be determined.

\newtheorem{BlowUpO}[kjet]{Algorithm}
\begin{BlowUpO}(Algorithm for blowing up singularities of main type $Y_{r,s}$)\label{BlowingUp}
\end{BlowUpO}
\noindent\textnormal{\bf Input:} $g\in\m^3\subset\mathbb Q[x_1,x_2]$ of complex singularity type $Y_{r,s}$.\newline
\textnormal{\bf Output:} a List $L$ with entries $L[1]$ and $L[2]$: the values of $r$ and $s$, $r\le s$, respectively; $L[3]$: the inertia index of the resulting germ, after blowing up $g$, with lowest milnor number if $L[1]\neq 5$, and $0$ otherwise; $L[4]$: the sign of the monomial $x_1^r$ or $x_2^r$ in the normal form of the resulting singularity with lowest milnor number; $L[5]$ the value $1$ if the resulting germs do not intersect the exceptional divisor smoothly, i.e.~$L[1]\neq 5$, and occur at irrational points on the exceptional divisor and $0$ otherwise. If the value of $L[5]$ is $1$, the values of $L[3]$ and $L[4]$ are determined using floating real numbers and thus may be wrong. 
\begin{itemize}
\item $L_1:=\emptyset$;
\item $L:=\emptyset$;
\item for ($i=1,2$)\newline
\phantom{}\quad $\phi_1:$ $x_i\mapsto x_i$, $x_j\mapsto x_ix_j$, $j\neq i$;\newline
\phantom{}\quad $p:=\phi_1(g)$;\newline
\phantom{}\quad while($p$ is divisible by $x_i$)\newline
\phantom{}\quad\quad $p=p/x_i$;\newline
\phantom{}\quad Let $P_1,\ldots,P_s$ be the prime ideals in the prime decomposition\newline
\phantom{}\quad of the radical of the ideal $\langle\textnormal{jacob}(p),p,x_1\rangle$ over $\mathbb Q$ and let\newline
\phantom{}\quad $f_{1j}, f_{2j}$ be a reduced standard basis for $P_j$, using lexicographical \newline
\phantom{}\quad ordering;\newline
\phantom{}\quad\quad if($\deg(f_{1j})=1$, i.e.~$P_j$ is prime, for all $j$)\newline
\phantom{}\quad\quad\quad $m=0$;\newline
\phantom{}\quad\quad else\newline
\phantom{}\quad\quad\quad $m=1$;\newline
\phantom{}\quad\quad If($m=0$)\newline
\phantom{}\quad\quad\quad for$(j=1; j\le s;j++)$\newline
\phantom{}\quad\quad\quad\quad if$(P_j=\langle 1\rangle$)\newline
\phantom{}\quad\quad\quad\quad\quad add $x_1$ to $L_1$;\newline
\phantom{}\quad\quad\quad\quad if$(P_j\neq\langle 1\rangle$)\newline
\phantom{}\quad\quad\quad\quad\quad for$(i=1,2)$\newline
\phantom{}\quad\quad\quad\quad\quad\quad $\phi_2$: $x_1\mapsto x_1-s_1$, $x_2\mapsto x_2-s_2$, where $s_1$ is the solution of  \newline
\phantom{}\quad\quad\quad\quad\quad\quad $x_1$ in $f_{2j}$ and $s_2$ is the solution of $x_2$ in $f_{1j}$;\newline
\phantom{}\quad\quad\quad\quad\quad\quad Add $\phi_2(p)$ to $L_1$;\newline
\phantom{}\quad\quad else\newline
\phantom{}\quad\quad\quad $\phi_3$: $x_1\mapsto x_1$, $x_2\mapsto x_2 -\frac{a_0b_1+a_1b_0}{a_1b_1}x_1$, where\newline
\phantom{}\quad\quad\quad $g=(a_0x_1+a_1x_2)^2(b_0x_1+b_1x_2)^2+$ higher terms in $x_1$ and $x_2$,\newline
\phantom{}\quad\quad\quad $a_0,a_1,b_0,b_1\in\mathbb R$;\newline
\phantom{}\quad\quad\quad $g :=\phi_3(g)=c(x_1-ex_2)(x_1+ex_2)+$ higher terms in $x_1$ and $x_2$,\newline
\phantom{}\quad\quad\quad $c,e\in\mathbb R$;\newline
\phantom{}\quad\quad\quad $p:=\phi_1(g)$;\newline
\phantom{}\quad\quad\quad while($p$ is divisible by $x_i$)\newline
\phantom{}\quad\quad\quad\quad $p=p/x_i$;\newline
\phantom{}\quad\quad\quad $\phi_4: x_1\mapsto x_1, x_2\mapsto x_2-t$;\newline
\phantom{}\quad\quad\quad $h_1=\phi_4(p)\in\mathbb Q(t)[x_1,x_2];$\newline
\phantom{}\quad\quad\quad $\mu_1 = $ milnor number of $h_1$;\newline
\phantom{}\quad\quad\quad Determine the real value $e$ and replace $t$ by $e$ in $h_1$,\newline
\phantom{}\quad\quad\quad i.e~now, $h_1\in\mathbb R[x_1,x_2]$  ;\newline
\phantom{}\quad\quad\quad Apply the Splitting Lemma to $h_1$;\newline
\phantom{}\quad\quad\quad $f_1:= $ the residual part of $h_1$;\newline
\phantom{}\quad\quad\quad $\lambda:= $ the inertia index of $h_1$;\newline
\phantom{}\quad\quad\quad $f'_1:=\jt(f_1,\mu_1-3)$;\newline
\phantom{}\quad\quad\quad $s_1:=$ coefficient of $x_1^{\mu_1-3}$;\newline
\phantom{}\quad\quad\quad $s_2:=$ coefficient of $x_2^{\mu_1-3}$;\newline
\phantom{}\quad\quad\quad if$(s_1=0$ and $s_2\neq 0)$\newline
\phantom{}\quad\quad\quad\quad Apply $x_1\mapsto x_2, x_2\mapsto x_1$;\newline
\phantom{}\quad\quad\quad sign = coefficient of $x_1^{\mu_1-3}$;\newline
\item if$( m = 0 )$\newline
\phantom{}\quad $l:=$ the size of $L_1$;\newline
\phantom{}\quad for($i = 1, i\le l, i++$)\newline
\phantom{}\quad \quad if($L_1[i]=x_1$);\newline
\phantom{}\quad\quad\quad $\mu_i=4;$\newline
\phantom{}\quad\quad else\newline
\phantom{}\quad\quad\quad $\mu_i=$ milnor number of $L_1[i]+4$;\newline
\phantom{}\quad Delete repeating entries of $L_1$ and order the entries of $L_1$ from small to\newline
\phantom{}\quad big with regard to its milnor number; \newline
\phantom{}\quad if$(L_1[1]\neq x_1)$\newline
\phantom{}\quad\quad Apply the Splitting Lemma to $L_1[1]$;\newline
\phantom{}\quad\quad $f_1:=$ the residual part of $L_1[1]$;\newline
\phantom{}\quad\quad $\lambda:=$ inertia index of $L_1[1]$\newline
\phantom{}\quad\quad $f_1':=\jt(f_1,\mu_1-3)$;\newline
\phantom{}\quad\quad $s_1:=$ coefficient of $x_1^{\mu_1-3}$;\newline
\phantom{}\quad\quad $s_2:=$ coefficient of $x_2^{\mu_1-3}$;\newline
\phantom{}\quad\quad if($s_1=0$ and $s_2\neq 0$)\newline
\phantom{}\quad\quad\quad Apply $x_1\mapsto x_2$, $x_2\mapsto x_1$;\newline
\phantom{}\quad\quad sign $=$ coefficient of $x_1^{\mu_1-3}$;\newline
\phantom{}\quad else\newline
\phantom{}\quad\quad $\lambda = 0$;\newline
\phantom{}\quad\quad sign $=0$;\newline
\phantom{}\quad $L[1]=\mu_1$;\newline
\phantom{}\quad $L[2]=\mu_2$;\newline
\phantom{}\quad $L[3]=\lambda$;\newline
\phantom{}\quad $L[4]=$ sign;\newline
\phantom{}\quad $L[5]=m$;\newline
\item return $L$; 
\end{itemize}

Here follows the algorithm we implemented to classify the singularities, using the output list $L$ in Algorithm \ref{BlowingUp}, of type $\widetilde Y_r$ and $Y_{r,s}$.

\newtheorem{Y[1}[kjet]{Algorithm}
\begin{Y[1}(Algorithm for the cases $\widetilde Y_r$ and $Y_{r,s}$)\label{Y[r,s]}
\end{Y[1}
\noindent\textnormal{\bf Input:} $f\in \m^3\subset\mathbb Q[x,y]$ of complex singularity type $\widetilde Y_r$ or $Y_{r,s}$.\newline
\textnormal{\bf Output:} the real singularity type of $g$, i.e.~$\widetilde Y_r^+$, $\widetilde Y_r^-$, $Y_{r,s}^{++}$, $Y_{r,s}^{+-}$, $Y_{r,s}^{--}$, $Y_{r,s}^{-+}$.
\begin{itemize}
\item $\mu :=$ milnor number of $f$; 
\item $h :=  \jt(g,4);$
\item $s_1 := \textnormal{coefficient of }x^4$;
\item $s_2 := \textnormal{coefficient of }y^4$;
\item if($s_1=0$ and $s_2\neq 0$)\newline
\phantom{}\quad Swap the variables $x$ and $y$ in $h$;
\item if($s_1=0$ and $s_2=0$)\newline
\phantom{}\quad $t_1=\textnormal{coefficient of }x^3y\ (\textnormal{in } h);$\newline
\phantom{}\quad $t_2=\textnormal{coefficient of }x^2y^2\ (\textnormal{in } h);$\newline
\phantom{}\quad $t_3=\textnormal{coefficient of }xy^3\ (\textnormal{in } h);$\newline
\phantom{}\quad if($t_1+t_2+t_3=0$)\newline
\phantom{}\quad\quad if($2t_1+4t_2+8t_3\neq 0$)\newline
\phantom{}\quad\quad\quad Apply (to $h$) $x\mapsto x$, $y\mapsto y$;\newline
\phantom{}\quad\quad else\newline
\phantom{}\quad\quad\quad Apply (to $h$) $x\mapsto x$, $y\mapsto 3y$;\newline
\phantom{}\quad\quad Apply (to $h$) $x\mapsto x$, $y\mapsto x+y$;
\item Write $h$ as $a_0x^4+a_1x^3y+a_2x^2y^2+a_3xy^3+a_4y^4$, $a_0,\ldots,a_4\in\mathbb Q, a_0\neq 0$;
\item sign = coefficient of $x^4$ (in $h$);
\item Apply $x\mapsto x$, $y\mapsto 1$;
\item Write $h$ as $a_0x^4+a_1x^3+a_2x^2+a_3x+a_4$;
\item $r := \#$ real roots of $h$;
\item if($r=0$)\newline
\phantom{}$r'=\frac{\mu-9}{2}+4$;\newline
\phantom\quad sign$>0$;\newline
\phantom{}\quad\quad type of singularity := $\widetilde Y_{r'}^+$;\newline
\phantom{}\quad else\newline
\phantom{}\quad\quad type of singularity := $\widetilde Y_{r'}^-$;
\item if($r\neq 0$)\newline
\phantom{}\quad Blowing up the origin of $g$ and let $L$ be a list with entries $L[1]$ and $L[2]$: the values of $r$ and $s$, $r\le s$, respectively; $L[3]$: the inertia index of the resulting singularity, after blowing up $g$, with lowest milnor number if the resulting germs do not cut the exceptional divisor smoothly, i.e~$L[1]\neq 5$, and $0$ otherwise; $L[4]$: the sign of the monomial $x_1^r$ or $x_2^r$ in the normal form of the resulting singularity with lowest milnor number; $L[5]$ the value $1$ if the resulting germs intersects the exceptional divisor smoothly and occur at irrational points on the exceptional divisor and $0$ otherwise.\newline
\phantom{}\quad $\mu_1:=L[1]$;\newline
\phantom{}\quad $\mu_2:=L[2]$;\newline
\phantom{}\quad $\lambda:=L[3]$;\newline
\phantom{}\quad $\sign_1:= L[4]$;\newline
\phantom{}\quad $m:=L[5]$;\newline
\phantom{}\quad if(($\mu_1+1$) mod $2=1$)\newline
\phantom{}\quad\quad if(sign$>0$)\newline
\phantom{}\quad\quad\quad type of singularity $:= Y_{\mu_1+1,\mu_2+1}^{++}$;\newline
\phantom{}\quad\quad else\newline
\phantom{}\quad\quad\quad type of singularity $:=Y_{\mu_1+1,\mu_2+1}^{+-}$;\newline
\phantom{}\quad else\newline
\phantom{}\quad\quad if($\mu_1=5$ )\newline
\phantom{}\quad\quad\quad\quad if(sign$<0$)\newline
\phantom{}\quad\quad\quad\quad\quad redefine $\sign_1$ as $\sign_1:=\lambda-1$;\newline
\phantom{}\quad\quad\quad\quad else\newline
\phantom{}\quad\quad\quad\quad\quad  redefine $\sign_1$ as $\sign_1:=\lambda$;\newline
\phantom{}\quad\quad\quad\quad if($\sign_1>0$ and sign$>0$)\newline
\phantom{}\quad\quad\quad\quad\quad type of singularity $:= Y_{\mu_1+1,\mu_2+1}^{+-}$;\newline
\phantom{}\quad\quad\quad\quad if($\sign_1>0$ and sign$<0$)\newline
\phantom{}\quad\quad\quad \quad\quad type of singularity $:= Y_{\mu_1+1,\mu_2+1}^{--}$;\newline
\phantom{}\quad\quad\quad\quad if($\sign_1=0$ and sign$>0$)\newline
\phantom{}\quad\quad\quad\quad\quad type of singularity $:= Y_{\mu_1+1,\mu_2+1}^{++}$;\newline
\phantom{}\quad\quad\quad\quad if($\sign_1=0$ and sign$<0$)\newline
\phantom{}\quad\quad\quad\quad\quad type of singularity $:= Y_{\mu_1+1,\mu_2+1}^{-+}$;\newline
\phantom{}\quad\quad\quad else\newline
\phantom{}\quad\quad\quad\quad $\sign_1=$ coefficient of $x^{\mu_1-3}$\newline
\phantom{}\quad\quad\quad \quad if(sign$>0$ and $\sign_1>0$)\newline
\phantom{}\quad\quad\quad\quad type of singularity $:= Y_{\mu_1+1,\mu_2+1}^{++}$;\newline
\phantom{}\quad\quad\quad \quad if(sign$>0$ and $\sign_1<0$)\newline
\phantom{}\quad\quad\quad\quad type of singularity $:= Y_{\mu_1+1,\mu_2+1}^{+-}$;\newline
\phantom{}\quad\quad\quad \quad if(sign$<0$ and $\sign_1>0$)\newline
\phantom{}\quad\quad\quad\quad type of singularity $:= Y_{\mu_1+1,\mu_2+1}^{--}$;\newline
\phantom{}\quad\quad\quad \quad if(sign$<0$ and $\sign_1<0$)\newline
\phantom{}\quad\quad\quad\quad type of singularity $:= Y_{\mu_1+1,\mu_2+1}^{-+}$;\newline
\item return type of singularity;
\end{itemize}

\subsection{Real exceptional $1$-modal singularities of corank $2$}\label{ExceptionalSingularities}
Lastly we treat the exceptional singularities. The cases $E_{12}$, $E_{13}$, $Z_{11}$ and $Z_{12}$ are solved only using the complex classification of $g$. Singularities of complextype $Z_{13}$ are solved, similar to case $D_{4}$, using Lemma \ref{kjet} and Lemma \ref{x^3} to transform the $4$-jet of $g$, to the form $x^3y$, after which the sign of the term $y^6$ is considered. 

To classify the singularities of complextype $W_{12}$ and $W_{13}$ we only need to consider the signs of the coefficients of the monomials of the form $x^4$ in the $4$-jet of~$g$.

For the case $E_{14}$ we use Lemma \ref{kjet} and Lemma \ref{x^3} to write the $3$-jet of $g$ in the form $sx^3$, $s\in\mathbb Q$. To simplify calculations we divide $g$ by $s$ such that $\jt(g,3)=x^3$ and after the following transformations multiply $g$ by $s$ again. We sistematically transform $g$ such that the $4$-jet is $a_0x^3$ and then such that the $5$-jet is $a_0x^3$, $a_0=\pm 1$.  We now, similarly to the case $D_4$, after multiplying $g$ with $s$ again, consider the sign of the $y^8$-term to determine the real singularity type. We only include the algorithm for the case $E_{14}$, since the algorithms for the other exceptional cases are trivial.

\newtheorem{E[14]}[kjet]{Algorithm}
\begin{E[14]}(Algorithm for the case $E_{14}$)\label{E[14]}
\end{E[14]}
\noindent\textnormal{\bf Input:} $f\in \m^3$, $f\in\mathbb Q[x,y]$ of complex singularity type $E_{14}$.\newline
\textnormal{\bf Output:} the real singularity type of $f$, i.e.~$E_{14}^+$ or $E_{14}^-$.
\begin{itemize}
\item $h_1 :=\jt(g,3);$
\item $s :=$ coefficient of $h_1$;
\item if($s=0$)\newline
\phantom{}\quad swap variables $x$ and $y$ in $g$;
\item $g := \frac{g}{s}$;
\item $h_2 :=\jt(g,3)$;
\item Factors $:=$ the list of factors of $h_1$ over $\mathbb Q$;
\item write $g_1$ as $t_1x+t_2y$, $t_1,t_2\in\mathbb Q$, $t_1 := \pm-1$, where $g_1$ is the first entry in the list $F$;
\item Apply (to $g$): $x\mapsto \frac{x-t_2y}{t_1}$, $y\mapsto y$;
\item Write $g$ as $a_0x^3+ $terms of higher degree,\quad$a_0\in\mathbb Q$;
\item $g_3:=\frac{\jt(f,4)-a_0x^3}{3x^2}$;
\item Apply (to $g$): $x\mapsto g_3$, $y\mapsto y$;
\item $g_4:=\frac{\jt(f,5)-a_0x^3}{3x^2}$;
\item Apply (to $g$): $x\mapsto x-g_4$, $y\mapsto y$;
\item $g := g*s$;
\item $w:=$ the coefficient of $y^8$ in $g$;
\item if($w>0$)\newline
\phantom{}\quad type of singularity $:= E_{14}^+$;\newline
 else\newline
\phantom{}\quad type of singularity $:= E_{14}^-$;
\item return type of singularity;
\end{itemize}
 \begin{thebibliography}{99} 
\bibitem{AVG1985} Arnold, V.I.; Gusein-Zade, S.M.; Varchenko, A.N.: Singularities of Differential Maps. Vol.~I, Birkh\"auser (1985).
\bibitem{A1975} Arnold, V.I.: \textit{Normal form of functions near degenerate critical points.}, Russian Mth. Surveys 29 ii (1975), 10-50.
\bibitem{DGPS}
Decker, W.; Greuel, G.-M.; Pfister, G.; Sch{\"o}nemann, H.: 
\newblock {\sc Singular} {3-1-5} --- {A} computer algebra system for polynomial computations.
\newblock {http://www.singular.uni-kl.de} (2012).
\bibitem{Kruger} Kr\"uger, K.: Klassifikation von Hyperfl\"agensingularit\"aten, Diploma Thesis (1997).
\bibitem{GLS2007}Greuel, G.-M.; Lossen, C.; Shustin E.: Introduction to Singularities and Deformations, Springer, Berlin (2007).
\bibitem{GP2008} Greuel G.-M.; Pfister G.; A Singular introduction to Commutative Algebra, 2nd Ed., Springer, Berlin (2008).
\bibitem{classify} 
Kr\"uger, K.: 
{\tt classify.lib}. {A} {\sc Singular} {3-1-5} library for classifying isolated hypersurface singularities w.r.t. right equivalence, based on the determinator of singularities by V.I. Arnold (2012).
\bibitem{realclassify} 
Marais, M. and Steenpass, A.: 
{\tt realclassify.lib}. {A} {\sc Singular} {3-1-5} library for classifying isolated hypersurface singularities over the reals w.r.t. right equivalence, based on the determinator of singularities by V.I. Arnold. This library is based on classify.lib by Kai Kr\"uger, but handles the real case, while classify.lib does the complex classification (2012).
\bibitem{primdec.lib} Pfister, G.; Decker, W.;  Schoenemann, H.; Laplagne, S.: {\tt primdec.lib}. {A} {\sc Singular} {3-1-5} library for Primary Decomposition and Radical of Ideals (2012).
\bibitem{Siersma} Siersma D.: Classification and deformation of Singularities, disertation, University of Amsterdam (1974).
\bibitem{roots} 
Tobis, A.: 
{\tt rootsur.lib}. {A} {\sc Singular} {3-1-5} library for Counting number of real roots of univariate polynomial (2012).
\bibitem{solve.lib} Wenk, M.: {\tt solve.lib}. Pohl, W.: {A} {\sc Singular} {3-1-5} library for Complex Solving of Polynomial Systems (2012).

\end{thebibliography}

\end{document}
