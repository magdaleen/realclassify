\documentclass{amsproc}
\linespread{1.25}
\usepackage{amsthm,amsmath,amsfonts,mathrsfs,amssymb}

\DeclareMathOperator{\ord}{ord}
\DeclareMathOperator{\requiv}{\overset{r}{\sim}}
\DeclareMathOperator{\m}{\mathfrak{m}}

\title{The classification of real singularities using \textsc{Singular}}

\author{Magdaleen S. Marais}
\address{Magdaleen S. Marais\\
African Institute for Mathematical Sciences\\
6 Melrose Rd\\
Muizenberg 7945, Cape Town\\
South Africa}
\email{magdaleen@aims.ac.za}

\author{Andreas Steenpa\ss}
\address{Andreas Steenpa\ss\\
Department of Mathematics\\
University of Kaiserslautern\\
Erwin-Schr\"odinger-Str.\\
67663 Kaiserslautern\\
Germany}
\email{steenpass@mathematik.uni-kl.de}

\thanks{ }
\subjclass[2000]{}
\keywords{}
\begin{document}
\begin{abstract}
\end{abstract}
\maketitle
\section{Prelimanary definitions and results}
Consider the space $\mathbb R^n$ with fixed coordinate system
$x_1,\ldots,x_n$. We denote the ring of power series over $\mathbb R$
by $\mathbb R[[x_1,\ldots,x_n]]$.
\newtheorem{Monomial}{Definition}[section]
\begin{Monomial}\label{Monomial}
A monomial in $n$ variables (or indeterminates) $x_1,\ldots,x_n$ is a
power product
$x^{\alpha}=x_1^{\alpha_1}\cdots
x_n^{\alpha_n},\quad\alpha=(\alpha_1,\ldots,\alpha_n)\in\mathbb N^n.$ We call
$|\alpha|=\alpha_1+\cdots+\alpha_n$ the total degree of $x$ and $\alpha_i$ the
degree of $x_i$. Given a polynomial $f$ the degree of $f$, written as
$\deg(f)$,
is the maximum degree of its monomials. The degree of $x_i$ in $f$ is defined
similarly. The order of $f$, written as $\ord(f)$, is the minimum degree
of its monomials.
\end{Monomial}
\newtheorem{Support}[Monomial]{Definition}
\begin{Support}
The Support of a power series $f:=\sum f_\alpha  x^{\alpha}$ is the set
of all monomials that appear in this series with nonzero coefficients. We
denote the support of a power series $f$ by $\textnormal{supp}(f)$.
\end{Support}
\newtheorem{r-equiv}[Monomial]{Definition}
\begin{r-equiv}\label{r-equiv}
Let $f,g\in R[[x_1,\ldots,x_n]]$. Then $f$ is called right-equivalent to
$g$, $f
\requiv g$, if there exists an automorphism  $\phi$ of $R[[x_1,\ldots,x_n]]$
such that $\phi(f)=g$.
\end{r-equiv}
\newtheorem{jet}[Monomial]{Definition}
\begin{jet}\label{jet}
The $k$-jet, $k\in\mathbb N$ of a power series $f:=\sum f_\alpha  x^{\alpha}$
is
defined by $\displaystyle\textnormal{jet}(f,k):=\sum_{|\alpha|\le
k}f_{\alpha}x^{\alpha}$, the sum of terms of total degree less or equal
to $k$.
We say that $f$ is finitely determined if there exists $k>0$ such that every
$g\in R[[x_1,\ldots,x_n]]$ with the same $k$-jet as $f$ is $r$-equivalent
to $f$.  We say that $f$ is $k$-determined in this situation and the minimal
such $k$ is called the $k$-determinancy of $f$.
\end{jet}
\newtheorem{LocalAlgebra}{Definition}
\begin{LocalAlgebra}\label{LocalAlgebra}
Let $f \in R[[x_1, \ldots, x_n]]$ be a power series. Then
the local algebra $Q_f$ of $f$ is defined to be the quotient of the algebra
of function-germs by the gradient ideal $I_{\Delta f}$ of $f$:
\[Q_f:=\frac{\mathbb R[[x_1,\ldots,x_n]]}{I_{\Delta f}}.\]
\end{LocalAlgebra}

TODO: add definition of quasihomogeneous functions.

\newtheorem{UpperDiagonalMonomial}{Definition}
\begin{UpperDiagonalMonomial}\label{UpperDiagonalMonomial}
Let $f$ be a quasihomogeneous function of weighted degree $d$.
A monomial is called an upper monomial (reps. lower monomial, a diagonal
monomial) or is said to lie above (reps. below, on) the diagonal if its
weighted degree is larger than $d$ (reps. smaller than $d$, equal to $d$).
\end{UpperDiagonalMonomial}
\newtheorem{SemiQuasiHomogeneousFunction}{Theorem}
\begin{SemiQuasiHomogeneousFunction}\label{SemiQuasiHomogeneousFunction}
Let $f$ be a semi-quasihomogeneous function with quasihomogeneous part $f_0$.
Fix a monomial basis of local algebra of $f_0$ and
let $e_1,\ldots, e_s$ be the set of all upper monomials therein.
Then $f$ is equivalent to a function of the form $f_0+\sum c_ke_k$, where
the $c_k$ are constants.
\end{SemiQuasiHomogeneousFunction}
\section{The Splitting Lemma}
We denote the ideal of function-germs vanishing at the origin by $\m$.

TODO: define corank

The following well-known theorem, called the Splitting Lemma, allows us to
reduce the classification to germs of corank $n$ or, equivalently, to germs
in $m^3$.
\newtheorem{SplittingLemma}{Theorem}[section]
\newtheorem{AlgorithmSplittingLemma}[SplittingLemma]{Algorithm}
\begin{SplittingLemma}\label{SplittingLemma}
If $f\in m^2\subset \mathbb R[[x_1,\ldots,x_n]]$ has corank $c$, then
\[ f\sim^rg-\sum_{i=c+1}^{c+\lambda} x_i^2+\sum_{i=c+\lambda+1}^nx_i^2,\]
with $g\in m^3$. $g$ is called the residual part of $f$ and $\lambda$
is called
the inertia index of the quadratic form of $f$. $g$ and $\lambda$ are
uniquely determined up to right equivalence.
\end{SplittingLemma}
We implemented Theorem \ref{SplittingLemma} for polynomials in $\mathbb
Q[x_1,\ldots,x_n]\subset\mathbb R[x_1,\ldots,x_n]$, in \textsc{Singular},
using the following algorithm.

TODO: add a reference for this algorithm

\begin{AlgorithmSplittingLemma} \label{AlgorithmSplittingLemma}
\end{AlgorithmSplittingLemma}
\noindent\textnormal{\bf Input:} $f\in m^2\subset\mathbb
R[x_1,x_2,\ldots,x_n]$ and $k\in\mathbb N$ such that $f$ is
$k$-determined.\newline
\textnormal{\bf Output:} $c:=\textnormal{corank}(f)$, $\lambda\in\mathbb
N$ and
$g\in m^3\cap\mathbb R[x_1,\ldots,x_c]$ with \[\displaystyle
f\sim^rg-\sum_{i=c+1}^{c+\lambda} x_i^2+\sum_{i=c+\lambda+1}^nx_i^2.\]
\begin{itemize}
\item $S:=\emptyset;$
\item for($i=1;\ i\le n;\  i++$)\newline
\phantom{}\quad if($\textnormal{jet}(f,2)\in \mathbb R[x_1,\ldots,\hat
x_i,\ldots,x_n]$);\newline
\phantom{}\quad\quad $S:=S\cup\{x_i\}$;\newline
\phantom{}\quad else\newline
\phantom{}\quad\quad $f:=\textnormal{jet}(f,k);$\newline
\phantom{}\quad\quad write $f$ as $f=ax_i^2+px_i+r$ with $p,r\in\mathbb
R[x_1,\ldots,\hat x_i,\ldots,x_n]$\newline
\phantom{}\quad\quad and $a\in\mathbb R[x_1,\ldots,x_n];$\newline
\phantom{}\quad\quad if($a(0)=0$)\newline
\phantom{}\quad\quad\quad choose $j\in\{i+1,\ldots,n\}$ with
$x_ix_j\in\textnormal{supp}(f)$;\newline
\phantom{}\quad\quad\quad apply $x_j\mapsto x_j+x_i$;\newline
\phantom{}\quad\quad\quad write $f$ as $f=ax_i^2+px_i+r$ with $p,r\in\mathbb
R[x_1,\ldots,\hat x_i,\ldots,x_n]$\newline
\phantom{}\quad\quad\quad and $a\in\mathbb R[x_1,\ldots,x_n];$\newline
\phantom{}\quad\quad\quad\quad while($p\neq 0$)\newline
\phantom{}\quad\quad\quad\quad\quad apply $x_i\mapsto
x_i-\frac{p}{2a(0)}$;\newline
\phantom{}\quad\quad\quad\quad\quad $f:=\textnormal{jet}(f,k);$\newline
\phantom{}\quad\quad\quad\quad\quad write $f$ as $f=ax_i^2+px_i+r$ with
$p,r\in\mathbb R[x_1,\ldots,\hat x_i,\ldots,x_n]$\newline
\phantom{}\quad\quad\quad\quad\quad and $a\in\mathbb
R[x_1,\ldots,x_n];$\newline
\item $c =\# S$;
\item change the order of variables such that $\displaystyle
f=g+\sum_{i=c+1}^na_ix_i^2$;
\item $\lambda:=\#(\{a_i:i=c+1,\ldots,n\}\cap\mathbb R^-)$;
\item change the order of variables such that
\[f=g-\sum_{i=c+1}^{c+\lambda}a_ix_i^2+\sum_{i=c+\lambda+1}^na_ix_i^2,\
a_i>0;\]
\item apply $x_i\mapsto 0, i=c+1,\ldots,n$ to $f$;
\item $g:=f$;
\item return $c, \lambda, g$;
\end{itemize}

\begin{proof}
If we can prove that we can write $f$ as \begin{equation}\label{eq4}
f=a+\sum_{j=1}^i a_jx_j^2+r
\end{equation}
where
$r\in\mathbb R[x_{i+1},\ldots,x_n],\quad a_j\in\mathbb R,\quad a\in m^3$ and
the degree of $x_j$ in $a$ is greater than $1$, for all $j\in\{1,\ldots,i\}$,
after $i$ applications of the for-loop, then we have the desired result after
$n$ applications of the for-loop. We will prove this by induction. After $0$
applications it is trivial to write $f$ in the form of (\ref{eq4}) for $i=0$.
Suppose we can write $f$ in this form after $i-1$ applications. If $x_i\in
S$, then it is also trivial to write $f$ in the form of (\ref{eq4}). Suppose
$x_i\not\in S$. It follows from the induction hypotheses that $tx_j\not\in
\textnormal{supp}(f)$ for all monomials $t\in\mathbb R[x_1,\ldots,\hat
x_j,\ldots,x_n]$ and $j\in\{1,\ldots,i-1\}$. Then the $k$-jet of $f$ can
be written as
\[f=ax_i^2+px_i+r\]
with $p,r\in\mathbb R[x_1,\ldots,\hat x_i,\ldots,x_n]$ and $a\in\mathbb
R[x_1,\ldots,x_n]$. If $a(0)\neq 0$,, then $x_i^2\in\textnormal{supp}(f)$. If
$a(0)=0$, then $x_i^2\not\in\textnormal{supp}(f)$. Since $x_i$ appear in some
term of the $2$-jet of $f$ and since it follows from the induction that
$x_ix_j\not\in\textnormal{supp}(f)$ for $j\in\{1,\ldots,i-1\}$, it follows
that
there exists a $j\in\{i+1,\ldots,n\}$ such that
$x_ix_j\in\textnormal{supp}(f)$.
If we choose such an index $j$ and apply the transformation
\begin{equation}\label{eq2}
x_j\mapsto x_j+x_i,\quad x_ix_j\mapsto x_ix_j+x_i^2,
\end{equation}
then $x_i^2\in\textnormal{supp}(f)$. Writing $f$ in the form $ax_i^2+px_i+r$,
we now have that $a(0)\neq0$.

Now suppose that $p\neq 0$, i.e.~there appear terms in $f$ where the degree
of $x_i$ is one. Then the while-loop applies and the following transformation
is made
\begin{equation}\label{eq3}
x_i\mapsto x_i-\frac{p}{2a(0)}\qquad
(\textnormal{this is allowed since }a(0)\neq 0)
\end{equation}
Then
\begin{eqnarray*}
f&=&a(x_i-\frac{p}{2a(0)})^2+p(x_i-\frac{p}{2a(0)})+r\\
&=&ax_i^2-\frac{apx_i}{a(0)}+\frac{ap^2}{4a^2(0)}+px_i-\frac{p^2}{2a(0)}+r\\
&=&ax_i^2+(1-\frac{a}{a(0)})px_i+\frac{ap^2}{4a^2(0)}-\frac{p^2}{2a(0)}+r\\
&=&ax_i^2+bpx_i+\frac{a'p^2}{4a^2(0)}+r',
\end{eqnarray*}
where
\begin{equation*}
b=1-\frac{a}{a(0)}\in m,\quad a'=a-a(0)\in m,\quad
r'=\frac{p^2}{4a(0)}-\frac{p^2}{2a(0)}+r\in\mathbb R[x_1,\ldots,\hat
x_i,\ldots,x_n].
\end{equation*}
Since
\begin{equation*}
b=b_2x_i^2+b_1x_i+b_0,
\end{equation*}
where
\begin{equation*}
b_2\in\mathbb R[x_1,\ldots,x_n] \quad\textnormal{and}\quad b_1,b_0\in\mathbb
R[x_1,\ldots,\hat x_i,\ldots,x_n],
\end{equation*}
we have that
\begin{eqnarray*}
bpx_i&=&b_2px_i^3+b_1x_i^2+b_0px_i\\
&=&(b_2px_i+b_1p)x_i^2+b_0px_i\\
&=&b'x_i^2+b_0px_i,
\end{eqnarray*}
where
\begin{equation*}
b'\in m\in\mathbb R[x_1,\ldots,x_n]\quad\textnormal{and}\quad b_0\in\mathbb
R[x_1,\ldots,\hat x_i,\ldots,x_n].
\end{equation*}
Also
\begin{equation*}
\frac{a'p^2}{4a^2(0)}=c_2x_i^2+c_1x_i+c_0
\end{equation*}
where
\begin{equation*}
c_1,c_0\in\mathbb R[x_1,\ldots,\hat x_i,\ldots,x_n]\quad\textnormal{and}\quad
c_2\in\mathbb R[x_1,\ldots,x_n].
\end{equation*}
Since $a'\in m$ and $p^2\in m^2$ it follows that $\frac{a'p^2}{4a(0)}\in m^3$
and hence $c_2\in m$. Because $a(0)\neq 0$ and $b'(0)=0$ and $c_2(0)=0$, it
follows that $(a+b'+c_2)(0)\neq 0$. Furthermore,
since $b\in m$, it follows that $b_0\in m$ and hence
$\textnormal{order}(b_0p)>\textnormal{order}(p)$. Similarly since
$\textnormal{order}(\frac{a'p^2}{4d^2(0)})>\textnormal{order}(p)$ it
follows that $\textnormal{order}(c_1)>\textnormal{order}(p)$ and therefore
$\textnormal{order}(b_0p+c_1)>\textnormal{order}(p)$. Thus we conclude that
\begin{eqnarray}
f&=&(a+b'+c_2)x_i^2+(b_0p+c_1)x_i+r'+c_0\nonumber\\
&=&a''x_i^2+p''x_i+r''\label{eq1}
\end{eqnarray}
where
\begin{eqnarray*}
&&a'' = (a+b'+c_2)\in\mathbb R[x_1,\ldots,x_n],\quad
p'',r''\in\mathbb R[x_1,,\ldots,\hat x_i,\ldots,x_n],\\
&&a''(0)\neq0\quad\textnormal{and}\quad\textnormal{order}(p'')>\textnormal{order}(p).
\end{eqnarray*}
If $\textnormal{jet}(k,p''x_i)\neq 0$, then the while-loop will be applied
again
and we will get as output a polynomial of the form (\ref{eq1}). Since
the degree
of $p''$ strictly increases after each application of the while-loop,
$\textnormal{jet}(k,p''x_i)$ is equal to zero after finitely many applications
and the while-loop terminates. Considering the transformations  (\ref{eq2})
and
(\ref{eq3}), it follows that after each application of the while-loop,
it holds
that $tx_j\not\in\textnormal{supp}(f)$ for $j\in\{1,\ldots,i-1\}$. Hence
after $i$ applications of the for-loop, $f$ is in the form (\ref{eq4}).

Now, $x_i^2\in\textnormal{supp}(f)$ after $n$ applications of the for-loop
if and only if $x_i$ appears in some term of $\textnormal{jet}(2,f)$ after
$i-1$ applications of the for-loop, i.e.~$x_i$ will be in the Jacobian of
$f$ if and only if $x_i\in S$. Hence the corank of $f$ is $\#S$.
\end{proof}

\section{The real classification of the residual part of $f$}
TODO: explain modality

TODO: define normal form (cf. Arnold p. 242)

TODO: add a table of the singularities

TODO: add the proof that complex types correspond to real types

TODO: add a comment that this is not known to be true in higher modality

Arnold divided the real singularities of modality $0$ and $1$ into main types
which "split up" into more subtypes by changing the sign in front of certain
terms. It can be easily seen that the subtypes are complex equivalent to a
complex singularity type of the same name as its corresponding real main
singularity type. In fact there is a bijection between the complex singularity
types of modality $0$ and $1$ and the main real singularity types. Thus, if we
can determine the complex singularity type of a function germ in $\mathbb
R[[x_1,\ldots,x_n]]$, we only need to consider the subtypes of the
corresponding
main real singularity type to determine the real type of the function germ.
After applying the Splitting Lemma, we only need to consider function germs
in $m^3$.

\subsection{Results regarding the factorization of homogeneous polynomials
over $\mathbb R$ and $\mathbb Q$}
\paragraph{}Let $f,g\in m^2\subset\mathbb R[[x_1,\ldots,x_n]]$ such that
$f\requiv g$, i.e.~$\phi(f)=g$, where $\phi$ is an $\mathbb R$-algebra
automorphism of  $\mathbb R[[x_1,\ldots,x_n]]$. We denote the $i$-jet of
$\phi$ by $\phi_i$. By using the next result, given $f$ and $g$, we can
determine $\phi_1$.

TODO: explain jet of a map

TODO: In the following lemma and it's proof, rewrite $f(\phi)$ as $\phi(f)$

\newtheorem{kjet}{Lemma}[section]
\begin{kjet}\label{kjet}
Let $f,g\in m^2\subset\mathbb R[[x_1,\ldots,x_n]]$. Suppose $f\requiv g$
and $k$
is the lowest degree of $f$. If jet$(f,k)$ factorizes as
\[f_1^{s_1}(x_1,\ldots,x_n)\cdots f_t^{s_t}(x_1,\ldots,x_n)\]
over $\mathbb R$, then jet$(g,k)$ will factorize
as \[f_1^{s_1} (\phi_1(x_1),\ldots,\phi_1(x_n)),\cdots
f_t^{s_t}(\phi_1(x_1)),\ldots,(\phi_1(x_n)))\] over $\mathbb R$, where $\phi$
is an $\mathbb R$-algebra automorphism of $\mathbb R[[x_1,\ldots,x_n]]$
such that $\phi(f)=g$.
\end{kjet}
\begin{proof}
Since $f\requiv g$, there exists a ring automorphism $\phi\in\mathbb
R[[x_1,\ldots,x_n]]$ such that  $\phi(f)=f\circ\phi=g$. Now
assume that the monomials of lowest degree $k$ of $f$ factorize
as $f_1^{s_1}(x_1,\ldots,x_n)\cdots f_t^{s_t}(x_1,\ldots,x_n)$ over
$\mathbb R$, i.e.~$f=f_1^{s_1}\cdots f_n^{s_t}+f'$, where the order of
$f_1^{s_1},\ldots,f_t^{s_t}$ is $k$ and the order of $f'$ is greater than $k$.
Now, for all $j\in\{1,\ldots,n\}$, $\phi(x_j)=\phi_1(x_j)+\textnormal{terms of
degree higher than $1$}$. Since $\phi$ is a homomorphism, it follows that
\begin{eqnarray*}\phi(f)&=&f\circ\phi\\&=&f_1^{s_1}\big(\phi(x_1),\ldots,\phi(x_n)\big)\cdots
f_t^{s_t}\big(\phi(x_1),\ldots,\phi(x_n)\big)+\phi(f')\\&=&f_1^{s_1}\big(\phi_1(x_1),\ldots,\phi_1(x_n)\big)\cdots
f_t^{s_t}\big(\phi_1(x_1),\ldots,\phi_1(x_n)\big)\\&&+\big(\phi-\phi_1\big)(f)+\phi(f').\\\end{eqnarray*}
Since the $\deg\big(\phi_1(x_j))=1$ it follows that
\[\deg\big(f_1^{s_1}\big(\phi_1(x_1),\ldots,\phi_1(x_n)\big)\cdots
f_t^{s_t}\big(\phi_1(x_1),\ldots,\phi_1(x_n)\big)\big)=k.\] Because
\[\textnormal{order}\big(\big(\phi-\phi_1\big)(f)\big)>k\quad\textnormal{and}\quad\textnormal{order}\big(\phi(f')\big)>k\]
it follows that
\begin{eqnarray*}
&&\textnormal{jet}(\phi(f),k)=\textnormal{jet}(g,k)\\&&=f_1^{s_1}\big(\phi_1(x_1),\ldots,\phi_1(x_n)\big)\cdots
f_t^{s_t}\big(\phi_1(x_1),\ldots,\phi_1(x_n)\big).
\end{eqnarray*}
\end{proof}

Since \textsc{Singular} use real floating point numbers, which do not
represent
the real numbers, instead of real numbers, rounding errors in computations can
occur. Therefore the above result would not be of much help without the
following result.
\newtheorem{x^3}[kjet]{Lemma}
\begin{x^3}\label{x^3}
If $f\in\mathbb Q[x,y]$ is homogeneous and factorizes as (i) $g_1^d$, (ii)
$g_1\cdot g_2^{d}$ or as  (iii) $g_1^2(g_1-g_2)(g_1+g_2)$, $d>1$, over
$\mathbb
R$, where $g_1,g_2$ are polynomial of degree $1$, then $f$ will factorize as
$ag_1'^d$, $ag_1'\cdot g_2'^d$, $ag_1'^2(g_1'-g_2')(g_1'+g_2')$ respectively,
where $g_1', g_2'$ are polynomials of degree $1$ over $\mathbb Q$ and
$a\in\mathbb Q$.
\end{x^3}
%If $f\in\mathbb Q[x_1,\ldots,x_n]$ is homogeneous of degree $d$ and can
factorize as $(g)^d$ over $\mathbb R$, where $g$ is a polynomial of degree
$1$, then $f$ will factorize as $a(g')^d$, where $g'$ is a polynomial of
degree $1$ over $\mathbb Q$ and $a\in\mathbb Q$.
\begin{proof}

(i) Let $f=(a_1x+a_2y)^d$, $a_1,a_2\in\mathbb R$. Without loss of generality,
suppose $a_1\neq 0$. Then $f=a_1^d(x+\frac{a_2}{a_1}y)^d$. Since $f=
a_1^dx^d+da_1^{d-1}a_2x^{d-1}y+\cdots+a_2^dy\in\mathbb Q[x,y]$, it follows
that $a_1^d\in\mathbb Q$. Hence $(x+\frac{a_2}{a_1}y)^d\in\mathbb Q[x,y]$
from which it follows that $(x+\frac{a_2}{a_1})^d\in\mathbb Q[x]$. Since
$\mathbb Q$ is a perfect field it follows that $\frac{a_2}{a_1}\in\mathbb
Q$. Thus $f=a{g'}_1^d$, where $a:=a_1^d\in\mathbb Q$ and
$g_1=x+\frac{a_2}{a_1}y\in\mathbb Q[x,y]$.

(ii) Let $f=(a_1x+a_2y)(a_3x+a_4y)^{d}$, $a_1,a_2,a_3,a_4\in\mathbb
R$. Suppose $a_1,a_3\neq 0$.
For the cases $a_1,a_4\neq 0$, $a_2,a_3\neq 0$ and $a_2,a_4\neq 0$ the
proofs are similar.
Similar as above, it follows that $a_1a_3^{d}\in\mathbb Q$. Hence
$(x+\frac{a_2}{a_1}y)(x+\frac{a_4}{a_3}y)^d\in\mathbb Q[x,y]$ from which
it follows that $(x+\frac{a_2}{a_1})(x+\frac{a_4}{a_3})^d\in\mathbb
Q[x]$. Since $\mathbb Q$ is a perfect field, it follows that
$(x+\frac{a_2}{a_1})(x+\frac{a_4}{a_3})\in\mathbb Q[x]$ and
hence that $(x+\frac{a_4}{a_3})^{d-1}\in\mathbb Q[x]$. Therefore
also $(x+\frac{a_4}{a_3})\in\mathbb Q[x]$ which implies that
$(x+\frac{a_2}{a_1})\in\mathbb Q[x]$.
Thus $f=ag'_1g'_2$ with $a:=a_1a_3^d\in\mathbb Q$,
$g'_1:=(x+\frac{a_2}{a_1}y)\in\mathbb Q[x,y]$, and
$g'_2:=(x+\frac{a_2}{a_1}y)\in\mathbb Q[x,y]$.

(iii) If $f$ factorizes as $g_1^2(g_1-g_2)(g_1+g_2)$ over $\mathbb R$, it
follows similarly as above that $g_1\in\mathbb Q[x,y]$. Therefore
$(g_1-g_2)(g_1+g_2)=g_1^2-g_2^2\in\mathbb Q[x,y]$ and hence $g_2^2\in\mathbb
Q[x,y]$. Using (i) $g_2\in\mathbb Q[x,y]$ and the result follows.
\end{proof}
%\newtheorem{x^3}[kjet]{Lemma}
%\begin{x^3}\label{x^3}
%If $f\in\mathbb Q[x_1,\ldots,x_n]$ is homogeneous of degree $d$ and can
factorize as $(g)^d$ over $\mathbb R$, where $g$ is a polynomial of degree
$1$, then $f$ will factorize as $a(g')^d$, where $g'$ is a polynomial of
degree $1$ over $\mathbb Q$ and $a\in\mathbb Q$.
%\end{x^3}
%\begin{proof}
%If $f=0$ then the answer is trivial. Suppose
$f=(a_1x_1+\cdots+a_nx_n)^d$, $a_1,\ldots,a_n\in\mathbb
Q$. Without loss of generality, suppose $a_1\neq 0$. Then
$f=a_1^d(x_1+\frac{a_2}{a_1}x_2+\cdots+\frac{a_n}{a_1}x_n)^d=a_1^dx_1^d+f_{d-1}\cdot
x_1^{d-1}+\cdots+f_1\cdot x_1$, where $f_{d-1},\ldots,f_1\in\mathbb
Q[x_2,\ldots,x_n]$. Hence $a_1^d\in\mathbb Q$ and
$(x_1+\frac{a_2}{a_1}x_2+\cdots+\frac{a_n}{a_1}x_n)^d=(\frac{a_2}{a_1}x_2+\frac{a_3}{a_1}x_3+\cdots+\frac{a_n}{a_1}x_n)x_1^{d-1}+f'_2\cdot
x_1^{d-2}+\cdots+f'_d\cdot x_1\in\mathbb Q[x_1,\ldots,x_n]$,
where $f'_2,\ldots,f'_n\in\mathbb Q[x_2,\ldots,x_n].$ Therefore
$\frac{a_2}{a_1},\ldots,\frac{a_n}{a_1}\in\mathbb Q$. Thus by putting
$a:=a_1^d$ and $g':=x_1+\frac{a_2}{a_1}x_2+\cdots+\frac{a_n}{a_1}x_n$
the result follows.
%\end{proof}
%\newtheorem{x^ny}[kjet]{Lemma}
%\begin{x^ny}\label{x^ny}
%If $f\in\mathbb Q[x_1,\ldots,x_n]$ is homogeneous of degree $d$ and can
factorize as $g_1^{d-1}\cdot g_2$ over $\mathbb R$, where $g_1$ and $g_2$ are
polynomials of degree $1$, then $f$ will factorize as $a{g'}_1^{d-1}\cdot
g_2$, where $g'_1,g'_2$ are polynomials of degree $1$ over $\mathbb Q$
and $a\in\mathbb Q$.
%\end{x^ny}
%\begin{proof}
%\end{proof}
\subsection{Real singularities of $0$-modality}

If the corank of a singularity is $c$ then the output polynomial $g$ in
Algorithm \ref{AlgorithmSplittingLemma} will be a polynomial in $c$ variables.
If the output is $0$ then it follows  that the singularity is either of type
$A[1]^+$ or type $A[1]^-$. If $c=1$, then the singularity is either of type
$A[k]^+$ or of type $A[k]^-$ for some $k>1$. Note that if $k$ is even then
$A[k]^+\requiv A[k]^-$. We use the following algorithm, after applying the
Splitting Lemma in case $c=0$ or $c=1$.

\newtheorem{A[k]}[kjet]{Algorithm}
\begin{A[k]}(Algorithm for the case $A[k]$)
\end{A[k]}
\noindent\textnormal{\bf Input:} $f\in \mathbb R[x_1,\ldots,x_n]$ of
complex singularity type $A[k]$, the output polynomial $g\in m^3$ after
applying Algorithm \ref{AlgorithmSplittingLemma}, the corank $c$ of $f$
and the inertia index $\lambda$ of $f$.\newline
\textnormal{\bf Output:} the real singularity type of $f$, i.e.~$A[k]^-$
or $A[k]^+$, $k\in\mathbb N$.
\begin{itemize}
\item if($c=0$)\newline
\phantom{}\quad if($\lambda<n$)\newline
\phantom{}\quad\quad typeofsing$:="A[1]^+"$;\newline
\phantom{}\quad else\newline
\phantom{}\quad\quad typeofsing$:="A[1]^-"$;
\item if($c=1$)\newline
\phantom{}\quad int $k:= $deg(leadmon($g$))$ -1$;\newline
\phantom{}\quad if($k$ is even)\newline
\phantom{}\quad\quad nf:=$x^{k+1}$;\newline
\phantom{}\quad\quad typeofsing$:= A[k]$;\newline
\phantom{}\quad else\newline
\phantom{}\quad\quad nf:= $-x^{k+1}$;\newline
\phantom{}\quad\quad typeofsing$:= A[k]^-;$
\item return typeofsing;
\end{itemize}

In the following algorithm we use the fact that the normal form of the type
$D[4]^+$ factorizes into two factors and the normal form of the type $D[4]^-$
factorizes into three factors.

\newtheorem{D[4]}[kjet]{Algorithm}
\begin{D[4]}(Algorithm for the case $D[4]$)
\end{D[4]}
\noindent\textnormal{\bf Input:} $f\in m^3\subset\mathbb R[x,y]$ of complex
singularity type $D[4]$.\newline
\textnormal{\bf Output:} the real singularity type of $f$, i.e.~$D[4]^-$
or $D[4]^+$.
\begin{itemize}
\item $f := \textnormal{jet}(f,3)$;
\item $s_1:= \frac{f}{x^3}$;
\item $s_2 := \frac{f}{y^3}$;
\item if$(s_2=0$ and $s_1\neq0)$\newline
\phantom{}\quad Swap the variables $x$ and $y$;
\item   if$(s_2=0$ and $s_1=0)$\newline
\phantom{}\quad $t_1:=\frac{f}{x^2y}$;\newline
\phantom{}\quad $t_2:=\frac{f}{x^2y}$;\newline
\phantom{}\quad if$(t_1+t_2=0)$\newline
\phantom{}\quad\quad Apply $x\mapsto x$, $y\mapsto 2y$;\newline
\phantom{}\quad\quad Write $f$ as $f=t'_1x^2y+t'_2xy^2$, $t'_1, t'_2\in\mathbb
R, t'_1\neq t'_2$;\newline
\phantom{}\quad Apply $x\mapsto x+y$, $y\mapsto y$;\newline
\phantom{}\quad Write $f$ as $f=ay^3+bx^2y+cxy^2+dx^3$, $a,b,c,d\in\mathbb
R$, $a\neq 0$;\newline
\phantom{}\quad Apply $x\mapsto 1$, $y\mapsto y$;\newline
\phantom{}\quad Write $f$ as $f=ay^3+by^2+cy+d$;\newline
\phantom{}\quad $n:= \#$ real roots of $f$;\newline
\phantom{}\quad if$(n=3)$\newline
\phantom{}\quad\quad typeofsing:="$D[4]^-$";\newline
\phantom{}\quad if$(n\neq 3)$\newline
\phantom{}\quad\quad typeofsing:="$D[4]^+$";
\item return typeofsing;
\end{itemize}
\begin{proof}
Since the input polynomial is of complex singularity type $D[4]$, we only need
to consider the real singularity types $D[4]^-$ and $D[4]^+$ to determine the
real singularity type of $f$. Because $f\in m^3$, $\textnormal{order}(f)\ge3$,
and therefore $\textnormal{jet}(f,3)$ is homogeneous. We now consider
$\textnormal{jet}(f,3)$ and the way how it factorizes.

If $s_1\neq 0$ and $s_2=0$, we can ensure that the coefficient of $y^3$ is
nonzero by swapping the variables. If $s_1=0$ and $s_2=0$, the coefficient
of $y^3$ is nonzero after applying $x\mapsto x$, $y\mapsto 2y$ and $x\mapsto
x+y$, $y\mapsto y$ if $t_1=t_2$, and applying only $x\mapsto x+y$, $y\mapsto
y$ if $t_1\neq -t_2$. We now can write $f$ in the form
\begin{equation*}
f=ay^3+bx^2y+cxy^2+dx^3,\quad a,b,c,d\in\mathbb R, a\neq 0.
\end{equation*}
Using Lemma \ref{kjet}, $f$ factorizes into three linear factors over
$\mathbb R$ if $f$ is of type $D[4]^-$ and does not factorize into three
linear
factors over $\mathbb R$ if $f$ is of type $D[4]^+$. Furthermore, since
$x\nmid
f$ and $y\nmid f$, $f$ factorizes into three linear factors if and only if
$f(x,1)$ factorizes into three factors of degree $1$. Since $f(x,1)$ is of
degree $3$, it follows that $f(x,1)$ factorizes into three factors of
degree $1$
if and only if $f$ has three real roots.

%We will now show that we can determine whether $f$ is of type $D[4]^-$
or $D[4]^+$ by counting the number $n$ of real roots of $f':=ay^3+by^2+cy+d$.
%Now suppose $f$ factorize in $3$ linear factors, i.e.
%\[f=(ey+fx)(gy+hx)(iy+jx),\quad e,f,g,h,i,j\in\mathbb R\]
%where
%\begin{eqnarray*}
%egi&=&a\neq0,\quad \textnormal{i.e.}e,g,i\neq0\\
%ehj+gfj+ifh&=&b\\
%fgi+hei+jeg&=&c\\
%fhj&=&d.
%\end{eqnarray*}
%Then
%\begin{equation*}
%f'=ay^3+by^2+cy+d
%\end{equation*}
%will factorize as
%\begin{equation*}
%f'=(ey+f)(gy+h)(iy+j).
%\end{equation*}
%Conversely, suppose
%\begin{equation*}
%f'=ay^3+by^2+cy+d,\quad a,b,c,d\in\mathbb R, a\neq0
%\end{equation*}
%factorize in $3$ liner factors, i.e.
%\begin{equation*}
%f'=(ey+f)(gy+h)(iy+j)
%\end{equation*}
%where
%\begin{eqnarray*}
%egi&=&a\neq0,\quad \textnormal{i.e.}e,g,i\neq0\\
%ehj+gfj+ifh&=&b\\
%fgi+hei+jeg&=&c\\
%fhj&=&d.
%\end{eqnarray*}
%Then
%\begin{equation*}
%(ey+fx)(gy+hx)(iy+jx)=ay^3+bx^2y+cxy^2+dx^3=f
%\end{equation*}
%and hence $f$ will factorize in $3$ linear factors over $\mathbb R$.
Therefore $f$ is of type $D[4]^-$ if and only if $f'=ay^3+by^2+cy+d$
has $3$ real roots.
\end{proof}
\newtheorem{kDeterminacyD[k]k>4}[kjet]{Lemma}
\begin{kDeterminacyD[k]k>4}\label{kDeterminacyD[k]k>4}
A singularity of type $D[k]^+$ or $D[k]^-$ is $k$-determined.
\end{kDeterminacyD[k]k>4}
\newtheorem{transformationD[k]}[kjet]{Lemma}
\begin{transformationD[k]}\label{transformationD[k]}
Let $i\ge 4$. Then
\[x^2y+a_0x^i+a_1x^{i-1}y+\cdots+a_iy^i\sim^rx^2y+a_ix^i\]
using the $\mathbb R$-algebra automorphisms
\begin{eqnarray*}
x&\mapsto&x+p_1,\textnormal{ where }
p_1=-\frac{1}{2}(a_1x^{i-2}+\cdots+a_{i-1}y^{i-2})\\
y&\mapsto&y+p_2,\textnormal{ where } p_2=-a_0x^{i-2} \,.
\end{eqnarray*}
\end{transformationD[k]}
\newtheorem{D[k]k>4}[kjet]{Algorithm}
\begin{D[k]k>4}(Algorithm for the case $D[k]$, $k>4$)
\end{D[k]k>4}
\noindent\textnormal{\bf Input:} $f\in m^3\subset\mathbb R[x,y]$ of complex
singularity type $D[k]$, $k\in\mathbb N$, $k>4$.\newline
\textnormal{\bf Output:} the real singularity type of $f$, i.e.~$D[k]^-$
or $D[k]^+$.
\begin{itemize}
\item $k:= \textnormal{milnornumber}(f)$;
\item $d:=k-1;$
\item $f:=\textnormal{jet}(f,d);$
\item Factorize $\textnormal{jet}(f,3)$ as $g_1^2g_2$, $g_1$ and $g_2$ linear;
\item Apply $g_1\mapsto x$, $g_2\mapsto y$;
\item $\textnormal{for}(i=4;i<k;i++)$\newline
\phantom{}\quad $f:=\textnormal{jet}(f,d)$;\newline
\phantom{}\quad $\textnormal{if}(\textnormal{jet}(f,i)-x^2y\neq0)$\newline
\phantom{}\quad\quad Write $\textnormal{jet}(f,i)-x^2y$ as
$a_0x^i+a_1x^{i-1}y+\cdots +a_iy^i$;\newline
\phantom{}\quad\quad Apply $x\mapsto x-\frac{1}{2}(a_1x^{i-2}+\cdots
+a_{i-1}y^{i-2})$, $y\mapsto y-a_0x^{i-2}$;\newline
\phantom{}\quad $f:=\textnormal{jet}(f,d)$;\newline
\phantom{}\quad Write $f$ as $f=x^2y+ay^{k-1}$, $a\neq 0$;
\item $\textnormal{if}(a>0)$\newline
\phantom{}\quad typeofsing:=$D[k]^+$;\newline
else\newline
\phantom{}\quad typeofsing:=$D[k]^-$;
\item return typeofsing;
\end{itemize}
\begin{proof}
By Lemma \ref{kDeterminacyD[k]k>4} the determinacy of a singularity of type
$D[k]^-$ or $D[k]^+$ is $k-1$. Therefore we only need to consider the
$(k-1)$-jet of the input polynomial~$f$. Using Lemma \ref{kjet} and Lemma
\ref{x^3} we transform $f$ into a polynomial of the form
\[x^2y+\textnormal{terms of degree higher than $3$}\]
by factorizing the $3$-jet of $f$ as $g_1^2g_2$, $g_1$ and $g_2$ of
degree $1$,
and then applying the automorphisms $g_1\mapsto x$, $g_2\mapsto y$ to $f$. We
now systematically consider the terms of each degree $3<i<k$. By applying the
transformations in Lemma \ref{transformationD[k]}, for each $i$, only the term
$a_iy^i$ will possibly not vanish. If $i<k-1$, $a_i=0$, otherwise it follows
from Theorem \ref{SemiQuasiHomogeneousFunction} that $f$ is not of complex
type
$D[k]$. If $i=k-1$, $a_i\neq 0$, otherwise, it follows again from Theorem
\ref{SemiQuasiHomogeneousFunction} that $f$ is not of complex type
$D[k]$. Thus
we can write $f$ as $x^2y+ay^{k-1}, a\neq0$. Clearly if $a>0$ then
$x^2y+ay^{k-1}\requiv x^2y+y^{k-1}$ and if $a<0$ then $x^2y+ay^{k-1}\requiv
x^2y-y^{k-1}$.
\end{proof}
\newtheorem{E[6]}[kjet]{Algorithm}
\begin{E[6]}(Algorithm for the case $E[6]$)
\end{E[6]}
\noindent\textnormal{\bf Input:} $f\in m^3\subset\mathbb R[x,y]$ of complex
singularity type $E[6]$.\newline
\textnormal{\bf Output:} the real singularity type of $f$, i.e.~$E[6]^-$
or $E[6]^+$.
\begin{itemize}
\item $g:= \textnormal{jet}(f,3);$
\item $s:= \frac{g}{x^3};$
\item $\textnormal{if}(s=0)$\newline
\phantom{}\quad Swap the variables $x$ and $y$;\\
\phantom{}\quad $s:=\frac{g}{x^3}$;
\item Factorize $g$ in linear factors over $\mathbb Q[x,y]$, with a factor
$g_1=ax+by$;
\item Apply (to $f$) $x\mapsto \frac{x-by}{a}$, $y\mapsto y$;
\item Write $f$ as $f=cx^3+\sigma y^4+\textnormal{terms of degree $4$
and higher, not of the form $\alpha y^4$, $\alpha\in\mathbb R$}$;
\item \textnormal{if}$(\sigma>0)$\newline
\phantom{}\quad typeofsing:= "$E[6]^+$";\newline
\phantom{} else\newline
\phantom{}\quad typeofsing:="$E[6]^-$";
\end{itemize}
\begin{proof}
Since $f\requiv x^3+y^4$ or $f\requiv x^3-y^4$, there exists an $\mathbb
R$-algebra automorphism $\psi$ of $\mathbb R[[x_1,\ldots,x_n]]$ such that
$\psi(f)=f\circ\psi=(\psi(x))^3+(\psi(y))^4$ or such that
$\psi(f)=f\circ\psi=(\psi(x))^3-(\psi(y))^4$ . We ensure that the
coefficient of
$x^3$ is nonzero by swapping the variables if necessary. Now, using Lemma
\ref{kjet} and Lemma~\ref{x^3}, $g:=\textnormal{jet}(f,3)$ factorizes as
$c(g_1)^3$, $c\in\mathbb Q$ and $g_1=ax+by\in\mathbb
Q[x_1,\ldots,x_n]$. Again,
using Lemma \ref{kjet}, it follows that by applying $x\mapsto\frac{x-by}{a},\
y\mapsto y$, there exists a $\phi$ such that $\phi_1(x)=c'x$, $c'\in\mathbb
R$. Since $\phi$ is an automorphism, $\phi_1(y)=dx+ey$, $d,e\in\mathbb R$,
with $e\neq 0$. Hence
\begin{equation*}
(\phi(y))^4=e^4y^4+\textnormal{terms of degree 4 and higher, not of the
form $\alpha y^4$, $\alpha\in\mathbb R$.}
\end{equation*}
If we can show that $(\phi(x))^3$ does not contain a term of the form
$\alpha y^4$, $\alpha\in\mathbb R$, then we can determine whether $f$
is of type $E[6]^-$ or $E[6]^+$ by looking at the sign of $y^4$. Now
\begin{eqnarray*}
\textnormal{jet}((\phi(x))^3,4)-\textnormal{jet}((\phi(x))^3,3)&=&3(\phi_1(x)^2)(\phi_2(x)-\phi_1(x))\\&=&3(c'x)^2(\phi_2(x)-\phi_1(x))
\end{eqnarray*}
 which means that $(\phi(x))^3$ does not have terms of the form $\alpha y^4$,
 $\alpha\in\mathbb R$ .
\end{proof}
\newtheorem{X[9+k]}[kjet]{Algorithm}
\begin{X[9+k]}(Algorithm for the case $X[9+k]$)
\end{X[9+k]}
\noindent\textnormal{\bf Input:} $f\in m^3\subset\mathbb Q[x,y]$ of complex
singularity type $X[9+k]$.\newline
\textnormal{\bf Output:} the real singularity type of $f$, i.e.~$X[9+k]^-$
or $X[9+k]^+$.
\begin{itemize}
\item $f:=\textnormal{jet}(f,4)$;
\item $s_1:=\frac{f}{x^4}$;
\item $s_2:=\frac{f}{y^4}$;
\item if$(s_2\neq0$ and $s_1=0)$\newline
\phantom{}\quad Swap the variables $x$ and $y$;\newline
\item if$(s_2\neq0$ and $s_1=0)$\newline
\phantom{}\quad $t_1:=\frac{f}{x^3y}$;\newline
\phantom{}\quad $t_2:=\frac{f}{x^2y^2}$;\newline
\phantom{}\quad $t_3:=\frac{f}{xy^3}$;\newline
\phantom{}\quad if$(t_1+t_2+t_3=0)$\newline
\phantom{}\quad\quad if$(2t_1+4t_2+8t_3\neq0)$\newline
\phantom{}\quad\quad\quad Apply $x\mapsto x$, $y\mapsto 2y$;\newline
\phantom{}\quad\quad else\newline
\phantom{}\quad\quad\quad Apply $x\mapsto x$, $y\mapsto 3y$;\newline
\phantom{}\quad Apply $x\mapsto x$, $y\mapsto x+y$;
\item Write $f$ as $f=ax^4+bx^3y+cx^2y^2+dxy^3+ey^4$;
\item Apply $y\mapsto 1$, $x\mapsto x$, get $f'=ax^4+bx^3+cx^2+dx+e;$
\item $n:=\#$ real roots of $f'$;
\item if$(n=1)$\newline
\phantom{}\quad if$(a>0)$\newline
\phantom{}\quad\quad typeofsing$:=X[9+k]^{++}$;\newline
\phantom{}\quad else\newline
\phantom{}\quad\quad typeofsing$:=X[9+k]^{--}$;
\item if$(n=3)$\newline
\phantom{}\quad Factorize $f$ in linear factors over $\mathbb Q[x,y]$,
with $g_1:=ax+by$ the factor\newline
\phantom{}\quad with multiplicity $2$;\newline
\phantom{}\quad Apply $x\mapsto\frac{x+by}{a}$, $y\mapsto y$;\newline
\phantom{}\quad Write $f$ as $f=ax^4+bx^3y+cx^2y^2$;\newline
\phantom{}\quad Apply $x\mapsto 1$, $y\mapsto y$;\newline
\phantom{}\quad Write $f$ as $f=a+by+cy^2$;\newline
\phantom{}\quad if$(c>0)$\newline
\phantom{}\quad\quad typeofsing$:= X[9+k]^{-+}$;\newline
\phantom{}\quad else\newline
\phantom{}\quad\quad typeofsing$:=X[9+k]^{+-}$;
\end{itemize}
\begin{proof}
Since input polynomial is of complex type $X_{9+k}$, we know that it is of the
form \[f=ax^4+bx^3y+cx^2y^2+dxy^3+ey^4+\textnormal{ terms of higher degree.}\]
If $a= 0$, we first transform $f$ to a polynomial where $a\neq 0$. If $e\neq
0$ we can achieve this by the transformation $x\mapsto y$, $y\mapsto x$.

If $e=0$, then at least one of $b$, $c$ or $d$ is not $0$, otherwise the
singularity  is not of complex type $X_{9+k}$.  If we apply $x\mapsto x+y$ and
$y\mapsto y$ in this situation, we will transform $f$ so that the new
coefficient of $x^4$ is $b+c+d$. Thus if we can ensure that $a+b+c\neq 0$, we
can use this map and have the desired result. Now, since $v_1=b+c+d$,
$v_2=2b+4c+8d$ and $v_3=3b+6c+9d$ is linear independent in the
threedimensional
$\mathbb R$-vector space with formal basis $(b, c, d)$, it follows that all
three vectors are zero if and only if $b,c$ and $d$ is zero. Knowing $b$, $c$
and $d$ cannot all be zero, at least one of $v_1$, $v_2$ or $v_3$ is not
zero. If $v_1=0$ and $v_2\neq 0$, then the map $x\mapsto x$ and $y\mapsto 2y$
will transform $f$ to \[f=a'x^4+b'x^3y+c'x^2y^2+d'xy^3+e'y^4+\textnormal{
terms of higher degree,}\] where $b'+c'+d'=2b+4c+8d\neq 0$. If $v_1=0$ and
$v_3\neq 0$, then the map $x\mapsto x$ and $y\mapsto 3y$ will transform $f$
to  \[f=a'x^4+b'x^3y+c'x^2y^2+d'xy^3+e'y^4+\textnormal{ terms of higher
degree,}\]where $b'+c'+d'=3b+6c+9d\neq0$.

Now, $f\requiv g$, where $g=x^4+x^2y^2+ay^{4+k}$ if $f$ is of type
$X[9+k]^{++}$, $g=-x^4-x^2y^2+ay^{4+k}$ if $f$ is of type $X[9+k]^{--}$,
$g=-x^4+x^2y^2+ay^{4+k}$ if $f$ is of type $X[9+k]^{-+}$ and
$g=x^4-x^2y^2+ay^{4+k}$ if $f$ is of type $X[9+k]^{+-}$.  It follows
from Lemma
\ref{kjet} that $\textnormal{jet}(f,4)$ will factor as
$\pm\phi_1(x)^2(\phi_1(x)^2+\phi_1(y)^2)$ in the first two cases and as
$\pm\phi_1(x)^2(\phi_1(x)-\phi_1(y))(\phi_1(x)+\phi_1(y))$ in the second two
cases, where $\phi$ is the $\mathbb R$-algebra automorphism such that
$\phi(f)=g$. Since $a\neq 0$,  $f(x,1)$ has one root in the first two
cases and
three roots in the last two cases.

%Since the coefficient of $x^4$ in $f$ is nonzero, it follows that
$\phi_1(x)=\alpha x+\beta y$ and $\phi(y)=\gamma x+\delta y$, where
$\alpha,\delta\neq 0$.
Now, since the sign of the terms in the $4$-jet of $g$ in the first two cases
do not differ and the power of both $x$ and $y$ in all the terms are even,
we only need to look at the sign of the $x^4$-term to distinguish between
the $X[9+k]^{++}$ and $X[9+k]^{--}$ cases.

If $f$ has three roots, we can transform $f$ by using Lemma \ref{kjet}
and Lemma
\ref{x^3} such that $\phi_1(x)=\alpha x$, $\alpha\in\mathbb R$. Hence we
can now
write $f$ in the form $f=ax^4+bx^3y+cx^2y^2=\pm(\alpha x)^2((\alpha
x)^2\mp\phi_1(y)^2)=\pm(\alpha x)^4\mp (\alpha x)^2\phi_1(y)^2$. Since
$\phi$ is
an automorphism, $\phi_1(y)=\alpha x+\beta y$ with $\beta \neq 0$,
and since the sign of $\beta$ does not influence the sign of $(\alpha
x)^2\phi_1(y)^2$, and we know that the sign in front of $(\alpha x)^4$
and $(\alpha x)^2\phi_1(y)^2$ differ, we can decide between the cases
$X[9+k]^{+-}$ and $X[9+k]^{-+}$ by looking at the sign of the term $x^2y^2$.
\end{proof}
\end{document}
