\documentclass[noend]{amsproc}
\linespread{1}
\usepackage{amsthm,amsmath,amsfonts,mathrsfs,amssymb}
\usepackage{algorithm}
\usepackage{algorithmic}
\usepackage[T1]{fontenc}   % for bold \Singular

\newtheorem{theorem}{Theorem}
\newtheorem{lemma}[theorem]{Lemma}

% ALGORITHM style
%%%%%%%%%%%%%%%%%%%%%%%%%%%%%%%%%%%
\renewcommand{\algorithmicrequire}{\textbf{Input:}}
\renewcommand{\algorithmicensure}{\textbf{Output:}}

\newcommand{\Singular}{\textsc{Singular}}
\newcommand{\realclassify}{\texttt{realclassify.lib}}
\newcommand{\classify}{\texttt{classify.lib}}

\DeclareMathOperator{\ord}{ord}
\DeclareMathOperator{\requiv}{\overset{r}{\sim}}
\DeclareMathOperator{\m}{\mathfrak{m}}
\DeclareMathOperator{\jt}{jet}
\DeclareMathOperator{\corank}{corank}
\DeclareMathOperator{\supp}{supp}
\DeclareMathOperator{\sign}{sign}
\DeclareMathOperator{\diag}{diag}
\DeclareMathOperator{\N}{\mathbb{N}}
\DeclareMathOperator{\Q}{\mathbb{Q}}
\DeclareMathOperator{\R}{\mathbb{R}}
\DeclareMathOperator{\C}{\mathbb{C}}

\title[The classification of real singularities using \textsc{Singular}, %
Part I]%
{The classification of real singularities using \textsc{Singular}\\
Part I: Splitting Lemma and Simple Singularities}

\author{Magdaleen S. Marais}
\address{Magdaleen S. Marais\\
African Institute for Mathematical Sciences and Stellenbosch University\\
6 Melrose Rd\\
Muizenberg 7945, Cape Town\\
South Africa}
\email{magdaleen@aims.ac.za}

\author{Andreas Steenpa\ss}
\address{Andreas Steenpa\ss\\
Department of Mathematics\\
University of Kaiserslautern\\
Erwin-Schr\"odinger-Str.\\
67663 Kaiserslautern\\
Germany}
\email{steenpass@mathematik.uni-kl.de}

\thanks{ }
\subjclass[2000]{}
\keywords{}
\begin{document}
\begin{abstract}
The algorithms implemented in the library ``realclassify.lib" in
\textsc{singular} are discussed in this paper. The purpose of this library is
to classify the~$0$ and $1$ modal isolated hypersurface singularities at $0$ of
corank $0$, $1$ and $2$ over the real numbers as computed by V.I.~Arnold in
\cite{AVG1985}.
\end{abstract}
\maketitle
\section{Introduction}
The goal of this paper is to present the algorithms that were implemented in a
library, ``realclassify.lib'' \cite{realclassify} in \textsc{singular}, which
classifies the $0$ and $1$ modal isolated hypersurface singularities at $0$ of
corank $0$, $1$ and $2$ over the real numbers $\mathbb R$, using
right-equivalence, as was computed by V.I.~Arnold in \cite{AVG1985}. Hence we
consider smooth real functions with critical point $0$ and critical value $0$,
i.e.~functions in $\m^2$, where $\m$ denotes the ideal of function germs
vanishing at the origin. Two function germs $f, g\in\mathbb R[[x_1,\ldots,
x_n]]$, $f,g\in \m^2$, where $\mathbb R[[x_1,\ldots,x_n]]$ is the power series
ring in coordinates $x_1,\ldots,x_n$ over $\mathbb R$, are considered as
right-equivalent, denoted by $f\requiv g$, if there exists an $\mathbb
R$-algebra automorphism $\phi$ of $\mathbb R[[x_1,\ldots, x_n]]$ such that
$\phi(f)=g$. Because the only possible non-approximated input power series over
the real numbers in computer systems, and thus in particularly in
\textsc{singular}, are polynomials with rational coefficients, some of the
algorithms we used apply only to such input polynomials.  There already is a
\textsc{Singular} library ``classify.lib'' \cite{classify}, which for a given
polynomial computes its type in Arnold's classification over the complex
numbers. We used this library as a basis to build ``realclassify.lib'' on. The
methods used in ``classify.lib'' will not be discussed in this paper. For more
information in this regard \cite{Kruger} can be studied.

Using the Splitting Lemma (Lemma~\ref{SplittingLemma}) any function germ $f$
over the real numbers with an isolated singularity at $0$, using a suitable
coordinate system, can be written as the sum of two functions of which the
variables do not coincide. One of the  functions, called the nondegenerate part
of $f$, is a nondegenerate quadratic form and the other function, called the
residual part of $f$, is an element of $\m^3$. The number of variables in the
residual part is equal to the corank, denoted by $c$, of $f$. In this paper we
will only consider germs with corank $0$, $1$ and $2$.

The algorithm that we used to implement the Splitting Lemma is discussed in
section \ref{TheSplittingLemma}.

In \cite{AVG1985} Arnold divided the real singularities of modality $0$ and $1$
up to stable equivalence into main types
which split up into more subtypes by changing the sign in front of certain
terms. Two functions are stably equivalent if they are right-equivalent after
the direct addition of nondegenerte quadratic forms. Hence, for all
nonquadratic forms $f$, after applying the Splitting Lemma we only need to
consider the residual part of $f$ to complete the classification of $f$.   It
can be easily seen that the subtypes are complex equivalent to a
complex singularity type of the same name as its corresponding real main
singularity type (see Table \ref{normal forms}). In fact there is a bijection
between the complex singularity
types of modality $0$ and $1$ and the main real singularity types. Thus, if we
can determine the complex singularity type of a function germ in these cases,
we only need to consider the subtypes of the
corresponding
main real singularity type to determine the real type of the function germ.

We denote the $k$-jet of a function germ $f$ by $\jt(f,k)$. In section
\ref{ResultsRegardingTheFactorizationOfHomogeneousPolynomialsOverRAndQ} we
discuss the factorization of the lowest nonzero jets of two right-equivalent
functions. These results are used in sections
\ref{RealSingularitiesOfZeroModality} to \ref{ExceptionalSingularities}, where
the algorithms are given to classify germs in $\m^3$. Firstly, we show that the
$k$-jets of two right-equivalent functions, where $k$ is the order of the
functions, factorize in the same way over $\mathbb R$ (Lemma~\ref{kjet}). Since
the real numbers are floating numbers in \textsc{Singular}, it is not always
possible to factorize over the real numbers using \textsc{Singular}. In some
cases we prove that these lowest jets will always factorize in linear terms
over the rational numbers and thus in \textsc{Singular}, for example when the
lowest jet is of degree $k$ and has $k$ similar factors (Lemma \ref{x^3}). In
some other cases this unfortunately is not true. Since we can distinguish
between some singularities, by only considering the number of real roots of its
lowest nonzero jet, the similar factorization of the lowest nonzero jets is
still useful in these cases. Because the lowest nonzero jet of a function germ
is a homogeneous polynomial, we use dehomogenization and the library
``rootsur.lib" \cite{roots}, a library counting the real roots of a univariate
polynomial in \textsc{Singular}, to count the roots. We for instance use this
method in the classification of the $D_4$ (Algorithm~\ref{D[4]}) and $E_6$
(Algorithm \ref{E[6]}) cases.

Sometimes other methods are used in addition to the above methods, for example
the normal form of singularities of type $E_{14}$ is determined by
systematically computing higher jets of the $\mathbb R$-algebra automorphism
between the normal form of the function and the function itself considering the
terms of these functions, after the lowest jet is determined using the above
methods (Algorithm \ref{E[14]}).

In the cases $J_{10}$ and $J_{10+k}$, $k>0$ we used different normal forms than
Arnold. These forms are more suitable to work with over the real numbers and
specifically simplify the computations to classify these cases over the real
numbers. We include a list of the normal forms we used at the beginning of
section \ref{TheRealClassificationOfTheResidualPart}.

Lastly, we used the method of blowing up the origin (Algorithm
\ref{BlowingUp}), which simplifies singularities. In the case $Y_{r,s}$,
$r,s>4$ (Algorithm \ref{Y[r,s]}) enough information can be subtracted from the
resulting, simplified, singularity such that the case can be solved, in some
specific cases, unfortunately using approximations.

\section{The Splitting Lemma}\label{TheSplittingLemma}
The following well-known theorem, called the Splitting Lemma, allows us to
reduce the classification to germs of corank $c$ or, equivalently, to germs
in $\m^3$ contained in $\mathbb R[[x_1,\ldots,x_c]]$.


\begin{theorem}\label{SplittingLemma}
If $f\in \m^2\subset \mathbb R[[x_1,\ldots,x_n]]$ has corank $c$, then
\[ f\requiv g-\sum_{i=c+1}^{c+\lambda} x_i^2+\sum_{i=c+\lambda+1}^nx_i^2,\]
with $g\in \m^3$. $g$ is called the residual part of $f$ and $\lambda$
is called
the inertia index of the quadratic form of $f$. $g$ and $\lambda$ are
uniquely determined up to right equivalence.
\end{theorem}

We implemented Theorem \ref{SplittingLemma} in \textsc{Singular},
using the following algorithm. Since $\textsc{Singular}$ use real floating
numbers, which do not represent the real numbers, instead of real numbers,
rounding errors can occor working in this setting. On the other hand, Algorithm
\ref{AlgorithmSplittingLemma} also function correctly working over the rational
numbers, with a rational input polynomial. In this case the output polynomial
$g$ will also be rational and calculations will be exact.

We denote the set of all monomials with nonzero coefficients that appear in a
power series $f$, called the support of $f$, by $\supp(f)$.

\begin{algorithm}[h]
\caption{\label{AlgorithmSplittingLemma} Algorithm for the Splitting Lemma}
\begin{algorithmic}[1]
\REQUIRE{$f\in \m^2\subset\mathbb
R[x_1,\ldots,x_n]$ and $k\in\mathbb N$ such that $f$ is
$k$-determined.}

\ENSURE{$c:=\textnormal{corank}(f)$, $\lambda\in\mathbb
N$ and
$g\in \m^3\cap\mathbb R[x_1,\ldots,x_c]$ with \[\displaystyle
f\requiv g-\sum_{i=c+1}^{c+\lambda} x_i^2+\sum_{i=c+\lambda+1}^nx_i^2.\]}

\STATE $S:=\emptyset$
\FOR{$i=1;\ i\le n;\  i++$}
\IF{$\jt(f,2)\in \mathbb R[x_1,\ldots,\hat
x_i,\ldots,x_n]$}
\STATE $S:=S\cup\{x_i\}$
\ELSE
\STATE $f:=\jt(f,k)$
\STATE write $f$ as $f=ax_i^2+px_i+r$ with $p,r\in\mathbb
R[x_1,\ldots,\hat x_i,\ldots,x_n]$ and $a\in\mathbb R[x_1,\ldots,x_n]$
\IF{$a(0)=0$}
\STATE choose $j\in\{i+1,\ldots,n\}$ with $x_ix_j\in\supp(f)$
\STATE apply $x_j\mapsto x_j+x_i$
\STATE write $f$ as $f=ax_i^2+px_i+r$ with $p,r\in\mathbb
R[x_1,\ldots,\hat x_i,\ldots,x_n]$ and $a\in\mathbb R[x_1,\ldots,x_n]$
\WHILE{$p\neq 0$}
\STATE apply $x_i\mapsto x_i-\frac{p}{2a(0)}$
\STATE $f:=\jt(f,k)$
\STATE write $f$ as $f=ax_i^2+px_i+r$ with $p,r\in\mathbb R[x_1,\ldots,\hat x_i,\ldots,x_n]$ and $a\in\mathbb R[x_1,\ldots,x_n]$
\ENDWHILE
\ENDIF
\ENDIF
\ENDFOR
\STATE $c =\# S$
\STATE change the order of variables such that $f=g+\sum_{i=c+1}^na_ix_i^2$
\STATE $\lambda:=\#(\{a_i:i=c+1,\ldots,n\}\cap\mathbb R^-)$
\STATE change the order of variables such that
\[f=g-\sum_{i=c+1}^{c+\lambda}a_ix_i^2+\sum_{i=c+\lambda+1}^na_ix_i^2,\
a_i>0\]
\STATE apply $x_i\mapsto 0, i=c+1,\ldots,n$ to $f$
\STATE $g:=f$
\RETURN{$c, \lambda, g$}
\end{algorithmic}
\end{algorithm}

\begin{proof}
If we can prove that we can write $f$ as \begin{equation}\label{eq4}
f=a+\sum_{j=1}^i a_jx_j^2+r
\end{equation}
where
$r\in\mathbb R[x_{i+1},\ldots,x_n],\quad a_j\in\mathbb R,\quad a\in \m^3$ and
the degree of $x_j$ in $a$ is greater than $1$, for all $j\in\{1,\ldots,i\}$,
after $i$ applications of the for-loop, then we have the desired result after
$n$ applications of the for-loop. We will prove this by induction. After $0$
applications it is trivial to write $f$ in the form of (\ref{eq4}) for $i=0$.
Suppose we can write $f$ in this form after $i-1$ applications. If $x_i\in
S$, then it is also trivial to write $f$ in the form of (\ref{eq4}). Suppose
$x_i\not\in S$. It follows from the induction hypotheses that $tx_j\not\in
\textnormal{supp}(f)$ for all monomials $t\in\mathbb R[x_1,\ldots,\hat
x_j,\ldots,x_n]$ and $j\in\{1,\ldots,i-1\}$. Then the $k$-jet of $f$ can
be written as
\[f=ax_i^2+px_i+r\]
with $p,r\in\mathbb R[x_1,\ldots,\hat x_i,\ldots,x_n]$ and $a\in\mathbb
R[x_1,\ldots,x_n]$. If $a(0)\neq 0$, then $x_i^2\in\textnormal{supp}(f)$. If
$a(0)=0$, then $x_i^2\not\in\textnormal{supp}(f)$. Since $x_i$ appears in some
term of the $2$-jet of $f$ and since it follows from the induction that
$x_ix_j\not\in\textnormal{supp}(f)$ for $j\in\{1,\ldots,i-1\}$, it follows
that
there exists a $j\in\{i+1,\ldots,n\}$ such that
$x_ix_j\in\textnormal{supp}(f)$.
If we choose such an index $j$ and apply the transformation
\begin{equation}\label{eq2}
x_j\mapsto x_j+x_i,\quad x_ix_j\mapsto x_ix_j+x_i^2,
\end{equation}
then $x_i^2\in\textnormal{supp}(f)$. Writing $f$ in the form $ax_i^2+px_i+r$,
we now have that $a(0)\neq0$.

Now suppose that $p\neq 0$, i.e.~there appear terms in $f$ where the degree
of $x_i$ is one. Then the while-loop applies and the following transformation
is made
\begin{equation}\label{eq3}
x_i\mapsto x_i-\frac{p}{2a(0)}\qquad
(\textnormal{this is allowed since }a(0)\neq 0).
\end{equation}
Then
\begin{eqnarray*}
f&=&a(x_i-\frac{p}{2a(0)})^2+p(x_i-\frac{p}{2a(0)})+r\\
&=&ax_i^2-\frac{apx_i}{a(0)}+\frac{ap^2}{4a^2(0)}+px_i-\frac{p^2}{2a(0)}+r\\
&=&ax_i^2+(1-\frac{a}{a(0)})px_i+\frac{ap^2}{4a^2(0)}-\frac{p^2}{2a(0)}+r\\
&=&ax_i^2+bpx_i+\frac{a'p^2}{4a^2(0)}+r',
\end{eqnarray*}
where
\begin{equation*}
b=1-\frac{a}{a(0)}\in\m,\quad a'=a-a(0)\in\m,\quad
r'=\frac{p^2}{4a(0)}-\frac{p^2}{2a(0)}+r\in\mathbb R[x_1,\ldots,\hat
x_i,\ldots,x_n].
\end{equation*}
Since
\begin{equation*}
b=b_2x_i^2+b_1x_i+b_0,
\end{equation*}
where
\begin{equation*}
b_2\in\mathbb R[x_1,\ldots,x_n] \quad\textnormal{and}\quad b_1,b_0\in\mathbb
R[x_1,\ldots,\hat x_i,\ldots,x_n],
\end{equation*}
we have that
\begin{eqnarray*}
bpx_i&=&b_2px_i^3+b_1x_i^2+b_0px_i\\
&=&(b_2px_i+b_1p)x_i^2+b_0px_i\\
&=&b'x_i^2+b_0px_i,
\end{eqnarray*}
where
\begin{equation*}
b'\in\m\in\mathbb R[x_1,\ldots,x_n]\quad\textnormal{and}\quad b_0\in\mathbb
R[x_1,\ldots,\hat x_i,\ldots,x_n].
\end{equation*}
Also
\begin{equation*}
\frac{a'p^2}{4a^2(0)}=c_2x_i^2+c_1x_i+c_0
\end{equation*}
where
\begin{equation*}
c_1,c_0\in\mathbb R[x_1,\ldots,\hat x_i,\ldots,x_n]\quad\textnormal{and}\quad
c_2\in\mathbb R[x_1,\ldots,x_n].
\end{equation*}
Since $a'\in\m$ and $p^2\in\m^2$ it follows that $\frac{a'p^2}{4a(0)}\in\m^3$
and hence $c_2\in\m$. Because $a(0)\neq 0$ and $b'(0)=0$ and $c_2(0)=0$, it
follows that $(a+b'+c_2)(0)\neq 0$. Furthermore,
since $b\in\m$, it follows that $b_0\in\m$ and hence
$\textnormal{order}(b_0p)>\textnormal{order}(p)$. Similarly since
$\textnormal{order}(\frac{a'p^2}{4d^2(0)})>\textnormal{order}(p)$ it
follows that $\textnormal{order}(c_1)>\textnormal{order}(p)$ and therefore
$\textnormal{order}(b_0p+c_1)>\textnormal{order}(p)$. Thus we conclude that
\begin{eqnarray}
f&=&(a+b'+c_2)x_i^2+(b_0p+c_1)x_i+r'+c_0\nonumber\\
&=&a''x_i^2+p''x_i+r''\label{eq1}
\end{eqnarray}
where
\begin{eqnarray*}
&&a'' = (a+b'+c_2)\in\mathbb R[x_1,\ldots,x_n],\quad
p'',r''\in\mathbb R[x_1,,\ldots,\hat x_i,\ldots,x_n],\\
&&a''(0)\neq0\quad\textnormal{and}\quad\textnormal{order}(p'')
>\textnormal{order}(p).
\end{eqnarray*}
If $\textnormal{jet}(k,p''x_i)\neq 0$, then the while-loop will be applied
again
and we will get as output a polynomial of the form (\ref{eq1}). Since
the degree
of $p''$ strictly increases after each application of the while-loop,
$\textnormal{jet}(k,p''x_i)$ is equal to zero after finitely many applications
and the while-loop terminates. Considering the transformations  (\ref{eq2})
and
(\ref{eq3}), it follows that after each application of the while-loop,
it holds
that $tx_j\not\in\textnormal{supp}(f)$ for $j\in\{1,\ldots,i-1\}$. Hence
after $i$ applications of the for-loop, $f$ is in the form (\ref{eq4}).

Now, $x_i^2\in\textnormal{supp}(f)$ after $n$ applications of the for-loop
if and only if $x_i$ appears in some term of $\textnormal{jet}(2,f)$ after
$i-1$ applications of the for-loop, i.e.~$x_i$ will be in the Jacobian of
$f$ if and only if $x_i\in S$. Hence the corank of $f$ is $\#S$.
\end{proof}

\section{The real classification of the residual part of $f$}%
\label{TheRealClassificationOfTheResidualPart}

In \cite{AVG1985} Arnold divided the real singularities of modality $0$ and
$1$, using stable equivalence, into main types which split up into more
subtypes by changing the sign in front of certain terms. Each of these subtypes
is complex equivalent to the complex singularity of the same name as its
corresponding real main singularity type. We include the complex
transformations for the corank two singularities in the table below. Since
there is a real main singularity type of modality $0$ or $1$ for each complex
singularity type of modality $0$ or $1$, respectively, there is a bijection
between the main real singularity types of modality $0$ and $1$ and the complex
singularity types of modality $0$ and $1$, respectively. It is not known yet
whether the complex forms of higher modal cases  split up into corresponding
real forms. The fact is that it is not known whether modality is preserved in
these cases. In the next table we list the normal forms that are used in this
article. From here onwards we will be working with stable equivalence. For all
nonquadratic forms $f$ it is thus only necessary, after applying the Splitting
Lemma to consider the residual part of function germs, i.e.~germs in $\m^3$.

\begin{table}[!hbp]
\centering
\caption{Real Normal Forms of modality $0$ and $1$.}
\label{normal forms}
\begin{tabular}{|c|c|c|c|c|}
\hline
&Complex NF & NF of real subtypes& Transformation&Restrictions\\\hline
$A_k$&$x^{k+1}$&$x^{k+1}$&$x\mapsto x$, $y\mapsto y$&$k\ge 1$\\
&&$-x^{k+1}$&$x\mapsto {i}^{\frac{2}{k+1}}x$, $y\mapsto y$&$k\ge 1$\\
\hline
$D_k$&$x^2y+y^{k-1}$&$x^2y+y^{k-1}$&$x\mapsto x$, $y\mapsto y$&$k\ge 4$\\
&&$x^2y-y^{k-1}$&$x\mapsto i^{\frac{2k-3}{k-1}}x$,
$y\mapsto {i}^{\frac{2}{k-1}}y$&$k\ge 4$\\
\hline
$E_6$&$x^3+y^4$&$x^3+y^4$&$x\mapsto x$, $y\mapsto y$&-\\
&&$x^3-y^4$&$x\mapsto x$, $y\mapsto\sqrt{i} y$&-\\
\hline
$E_7$&$x^3+xy^3$&$x^3+xy^3$&$x\mapsto x$, $y\mapsto y$&-\\
\hline
$E_8$&$x^3+y^5$&$x^3+y^5$&$x\mapsto x$, $y\mapsto y$&-\\
\hline
$X_9$&$x^4+ax^2y^2+y^4$&$x^4+ax^2y^2+y^4$&$x\mapsto x$,
$y\mapsto y$&$a^2\neq 4$\\
&&$x^4+ax^2y^2-y^4$&$x\mapsto x$, $y\mapsto \sqrt{i}y$&-\\
&&$-x^4+ax^2y^2+y^4$&$x\mapsto\sqrt{i}x$, $y\mapsto y$&-\\
&&$-x^4+ax^2y^2-y^4$&$x\mapsto\sqrt{i}x$, $y\mapsto \sqrt{i}y$&$a^2\neq 4$\\
\hline
$J_{10}$&$x^3+ax^2y^2+y^6$&$x^3+ax^2y^2+y^6$&$x\mapsto x$,
$y\mapsto y$& $4a^3+27\neq0$\\
&&$x^3+ax^2y^2-y^6$&$x\mapsto x$, $y\mapsto \sqrt[3]{i}y$&$-4a^3+27\neq0$\\
\hline
$J_{10+k}$&$x^3+xy^4+ay^6$&$x^3+xy^4+ay^6$&$x\mapsto x$,
$y\mapsto y$&$a\neq 0$, $k>0$\\
&&$x^3-xy^4+ay^6$&$x\mapsto x$, $y\mapsto \sqrt{i}y$&$a\neq 0$, $k>0$\\
\hline
$X_{9+k}$&$x^4+x^2y^2+ay^{4+k}$&$x^4+x^2y^2+ay^{4+k}$&$x\mapsto x$,
$y\mapsto y$&$a\neq 0$, $k>0$\\
&&$x^4-x^2y^2+ay^{4+k}$&$x\mapsto x$, $y\mapsto iy$&$a\neq 0$, $k>0$\\
&&$-x^4+x^2y^2+ay^{4+k}$&$x\mapsto \sqrt{i}x$, $y\mapsto i^{\frac{3}{2}}y$
&$a\neq 0$, $k>0$\\
&&$-x^4-x^2y^2+ay^{4+k}$&$x\mapsto \sqrt{i}x$, $y\mapsto \sqrt{i}y$&$a\neq 0$,
$k>0$\\
\hline
$Y_{r,s}$&$x^2y^2+x^r+ay^s$&$x^2y^2+x^r+ay^s$&$x\mapsto x$, $y\mapsto y$
&$a\neq 0$, $r,s>4$\\
&&$x^2y^2-x^r+ay^s$&$x\mapsto i^{\frac{2}{r}}x$, $y\mapsto i^{\frac{2r-2}{r}}y$
&$a\neq 0$, $r,s>4$\\
&&$-x^2y^2+x^r+ay^s$&$x\mapsto x$, $y\mapsto iy$&$a\neq 0$, $r,s>4$\\
&&$-x^2y^2-x^r+ay^s$&$x\mapsto i^{\frac{2}{r}}x$, $y\mapsto i^{\frac{r-2}{r}}y$
&$a\neq 0$, $r,s>4$\\
\hline
$E_{12}$&$x^3+y^7+axy^5$&$x^3+y^7+axy^5$&$x\mapsto x$, $y\mapsto y$&-\\
\hline
$E_{13}$&$x^3+xy^5+ay^8$&$x^3+xy^5+ay^8$&$x\mapsto x$, $y\mapsto y$&-\\
\hline
$E_{14}$&$x^3+y^8+axy^6$&$x^3+y^8+axy^6$&$x\mapsto x$, $y\mapsto y$&-\\
&&$x^3-y^8+axy^6$&$x\mapsto x$, $y\mapsto \sqrt[4]iy$&-\\
\hline
$Z_{11}$&$x^3y+y^5+axy^4$&$x^3+y^5+axy^4$&$x\mapsto x$, $y\mapsto y$&-\\
\hline
$Z_{12}$&$x^3y+xy^4+ax^2y^3$&$x^3y+xy^4+ax^2y^3$&$x\mapsto x$, $y\mapsto y$&-\\
\hline
$Z_{13}$&$x^3y+y^6+axy^5$&$x^3y+y^6+axy^5$&$x\mapsto x$, $y\mapsto y$& -\\
&&$x^3y-y^6+axy^5$&$x\mapsto i^{\frac{11}{9}}x$, $y\mapsto \sqrt[3]i y$& -\\
\hline
$W_{12}$&$x^4+y^5+ax^2y^3$&$x^4+y^5+ax^2y^3$&$x\mapsto x$, $y\mapsto y$&-\\
&&$-x^4+y^5+ax^2y^3$&$x\mapsto\sqrt{i} x$, $y\mapsto y$&-\\
\hline
$W_{13}$&$x^4+xy^4+ay^6$&$x^4+xy^4+ay^6$&$x\mapsto x$, $y\mapsto y$&-\\
&&$-x^4+xy^4+ay^6$&$x\mapsto \sqrt{i}x$, $y\mapsto i^{\frac{7}{8}}y$&-\\
\hline
\end{tabular}
\end{table}

It is known that the milnor number of a function germ $g$, which we denote by
$\mu(g)$, is the same regardless whether we work over the complex- or real
numbers. Therefore a singularity over the complex numbers is nondegenerate if
and only if the singularity is nondegenerate over the real numbers. The
restrictions to normal forms in the complex case are thus the same in the real
case. Furthermore we know that we can transform the normal forms in the real
case that differ from those we use in the complex case by a complex
transformation to a complex normal form. Since the milnor number is invariant
under complex transformations, we can then determine the restrictions. Let us
take the $X_9$ case as an example and consider the real normal form
$-x^4+ax^2y^2+y^4$. This normal form is complex equivalent to
$x^4+aix^2y^2+y^4$, which corresponds to the given corresponding complex normal
form. Since $a\in\mathbb R$, it follows that $(ia)^2=-a^2\neq4$ and therefore
this singularity is nondegenerate for all values of $a$.

Using the $1-1$ correspondence between the main real singularity types and
complex singularity types, and a \textsc{Singular} library ``classify.lib"
\cite{classify}, that classifies complex singularities, classifying a real germ
boils down to determining to which of the corresponding subtypes the germ is
equivalent.

\subsection{Results regarding the factorization of homogeneous polynomials
over $\mathbb R$ and $\mathbb Q$}%
\label{ResultsRegardingTheFactorizationOfHomogeneousPolynomialsOverRAndQ}

The results in this subsection are preliminary results that will be used in
sections \ref{RealSingularitiesOfZeroModality} to
\ref{ExceptionalSingularities}.

Let $f,g\in \m^2\subset\mathbb R[[x_1,\ldots,x_n]]$ such that
$f\requiv g$, i.e.~$\phi(f)=g$, where $\phi$ is an $\mathbb R$-algebra
automorphism of  $\mathbb R[[x_1,\ldots,x_n]]$. Since $\phi$ is an automorphism
it is defined by the images $\phi(x_1),\ldots,\phi(x_n)$. We define the $j$-jet
of $\phi$, denoted by $\phi_j$, to be the automorphism defined by
$\phi_j(x_1):=\jt(\phi(x_1),j),\ldots,\phi_j(x_n):=\jt(\phi(x_n),j)$. By using
the next result, given $f$ and $g$, we can
determine $\phi_1$.

\begin{theorem}\label{kjet}
Let $f,g\in \m^2\subset\mathbb R[[x_1,\ldots,x_n]]$. Suppose $f\requiv g$
and $k$
is the lowest degree of $f$. If jet$(f,k)$ factorizes as
\[f_1^{s_1}(x_1,\ldots,x_n)\cdots f_t^{s_t}(x_1,\ldots,x_n)\]
over $\mathbb R$, then jet$(g,k)$ will factorize
as \[f_1^{s_1} (\phi_1(x_1),\ldots,\phi_1(x_n)),\cdots
f_t^{s_t}(\phi_1(x_1)),\ldots,(\phi_1(x_n)))\] over $\mathbb R$, where $\phi$
is an $\mathbb R$-algebra automorphism of $\mathbb R[[x_1,\ldots,x_n]]$
such that $\phi(f)=g$.
\end{theorem}
\begin{proof}
Since $f\requiv g$, there exists a ring automorphism $\phi\in\mathbb
R[[x_1,\ldots,x_n]]$ such that  $\phi(f)=g$. Now
assume that the monomials of lowest degree $k$ of $f$ factorize
as \[f_1^{s_1}(x_1,\ldots,x_n)\cdots f_t^{s_t}(x_1,\ldots,x_n)\] over
$\mathbb R$, i.e.~$f=f_1^{s_1}\cdots f_t^{s_t}+f'$, where the degree and order
of
$f_1^{s_1},\ldots,f_t^{s_t}$ is $k$ and the order of $f'$ is greater than $k$.
Now, for all $j\in\{1,\ldots,n\}$, \[\phi(x_j)=\phi_1(x_j)+\textnormal{terms of
degree higher than $1$}.\] Since $\phi$ is a homomorphism, it follows that

\begin{eqnarray*}
\phi(f)&=&f_1^{s_1}\big(\phi(x_1),\ldots,\phi(x_n)\big)\cdots
f_t^{s_t}\big(\phi(x_1),\ldots,\phi(x_n)\big)+\phi(f')\\
&=&f_1^{s_1}\big(\phi_1(x_1),\ldots,\phi_1(x_n)\big)\cdots
f_t^{s_t}\big(\phi_1(x_1),\ldots,\phi_1(x_n)\big)\\
&&+\big(\phi-\phi_1\big)(f_1^{s_1}\cdots f_t^{s_t})+\phi(f').\\
\end{eqnarray*}
Since the $\deg\big(\phi_1(x_j))=1$ it follows that
\[\deg\big(f_1^{s_1}\big(\phi_1(x_1),\ldots,\phi_1(x_n)\big)\cdots
f_t^{s_t}\big(\phi_1(x_1),\ldots,\phi_1(x_n)\big)\big)=k.\] Because
\[
\textnormal{order}\big(\big(\phi-\phi_1\big)(f_1^{s_1}\cdots f_t^{s_t})\big)>k
\quad\textnormal{and}\quad\textnormal{order}\big(\phi(f')\big)>k
\]
it follows that
\begin{eqnarray*}
\textnormal{jet}(\phi(f),k)=\textnormal{jet}(g,k)
=f_1^{s_1}\big(\phi_1(x_1),\ldots,\phi_1(x_n)\big)\cdots
f_t^{s_t}\big(\phi_1(x_1),\ldots,\phi_1(x_n)\big).
\end{eqnarray*}
\end{proof}

Since \textsc{Singular} use real floating point numbers instead of real
numbers, rounding errors in computations in this setting can
occur. Therefore, in this paper, the above result would not be of much help
without the
following result.

\begin{lemma}\label{x^3}
If $f\in\mathbb Q[x,y]$ is homogeneous and factorizes as (i) $g_1^d$, (ii)
$g_1\cdot g_2^{d}$ or as  (iii) $g_1^2(g_1-g_2)(g_1+g_2)$, $d>1$, over
$\mathbb
R$, where $g_1,g_2$ are polynomials of degree $1$, then $f$ will factorize as
(i) $ag_1'^d$, (ii) $ag_1'\cdot g_2'^d$, (iii) $ag_1'^2(g_1'-g_2')(g_1'+g_2')$
respectively,
where $g_1', g_2'$ are polynomials of degree $1$ over $\mathbb Q$ and
$a\in\mathbb Q$.
\end{lemma}
\begin{proof}

(i) Let $f=(a_1x+a_2y)^d$, $a_1,a_2\in\mathbb R$. Without loss of generality,
suppose $a_1\neq 0$. Then $f=a_1^d(x+\frac{a_2}{a_1}y)^d$. Since $f=
a_1^dx^d+da_1^{d-1}a_2x^{d-1}y+\cdots+a_2^dy\in\mathbb Q[x,y]$, it follows
that $a_1^d\in\mathbb Q$. Hence $(x+\frac{a_2}{a_1}y)^d\in\mathbb Q[x,y]$
from which it follows that $(x+\frac{a_2}{a_1})^d\in\mathbb Q[x]$. Since
$\mathbb Q$ is a perfect field it follows that $\frac{a_2}{a_1}\in\mathbb
Q$. Thus $f=a{g'}_1^d$, where $a:=a_1^d\in\mathbb Q$ and
$g_1=x+\frac{a_2}{a_1}y\in\mathbb Q[x,y]$.

(ii) Let $f=(a_1x+a_2y)(a_3x+a_4y)^{d}$, $a_1,\ldots,a_4\in\mathbb
R$. Suppose $a_1,a_3\neq 0$.
For the cases $a_1,a_4\neq 0$, $a_2,a_3\neq 0$ and $a_2,a_4\neq 0$ the
proofs are similar.
Similar as above, it follows that $a_1a_3^{d}\in\mathbb Q$. Hence
$(x+\frac{a_2}{a_1}y)(x+\frac{a_4}{a_3}y)^d\in\mathbb Q[x,y]$ from which
it follows that $(x+\frac{a_2}{a_1})(x+\frac{a_4}{a_3})^d\in\mathbb
Q[x]$. Since $\mathbb Q$ is a perfect field, it follows that
$(x+\frac{a_2}{a_1})(x+\frac{a_4}{a_3})\in\mathbb Q[x]$ and
hence that $(x+\frac{a_4}{a_3})^{d-1}\in\mathbb Q[x]$. Therefore
also $(x+\frac{a_4}{a_3})\in\mathbb Q[x]$ which implies that
$(x+\frac{a_2}{a_1})\in\mathbb Q[x]$.
Thus $f=ag'_1g'^d_2$ with $a:=a_1a_3^d\in\mathbb Q$,
$g'_1:=(x+\frac{a_2}{a_1}y)\in\mathbb Q[x,y]$, and
$g'_2:=(x+\frac{a_2}{a_1}y)\in\mathbb Q[x,y]$.

(iii) If $f$ factorizes as $g_1^2(g_1-g_2)(g_1+g_2)$ over $\mathbb R$, it
follows similarly as above that $g_1\in\mathbb Q[x,y]$. Therefore
$(g_1-g_2)(g_1+g_2)=g_1^2-g_2^2\in\mathbb Q[x,y]$ and hence $g_2^2\in\mathbb
Q[x,y]$. Using (i) $g_2\in\mathbb Q[x,y]$ and the result follows.
\end{proof}

\subsection{Real $\mathbf{0}$-modal singularities of corank $\mathbf{0}$ and
$\mathbf{1}$}%
\label{RealSingularitiesOfZeroModality}

Throughout the rest of the paper we write $f$ for the given input polynomial,
$g$ for its residual part which can be obtained by applying the Splitting
Lemma~\ref{AlgorithmSplittingLemma}, and $c$ for the corank of $f$.
We also assume that $f$, and thus $g$, is a polynomial over $\mathbb Q$.
With these notations, $g$ is a polynomial in $c$ variables.

If $c = 0$, then it follows  that $f$ is of type
$A_1^+$ or of type $A_1^-$ depending on the inertia index $\lambda$ of $f$. If
the inertia index of $f$ is nonzero and less than the number of variables in
the base ring, then $f$ is both of type $A_1^+$ and $A_1^-$, depending how one
chooses to order the variables. If the inertia index is equal to the number of
variables in the base ring, $f$ is of type $A_1^-$. Lastly, if the inertia
index is $0$, then $f$ is of type $A_1^+$.

If $c=1$, then the singularity is of type
$A_k^+$ or of type $A_k^-$ for some $k>1$. Furthermore $g$ is a univariate
polynomial in this case, say $g\in\mathbb Q[x]$. Note that if $k$ is even, then
$A_k^+\requiv A_k^-$. The value of $k$ is given by the order of $g$ minus $1$.
This follows since $\pm x^{k+1}$ and $g$ are right-equivalent and thus have the
same order. The sign of the singularity type is determined by the sign of the
coefficient of $x^{k+1}$. This follows since it follows from Lemma \ref{kjet}
that $\jt(g,1)=\pm(\phi_1(x))^{k+1}=\pm(\alpha x)^{k+1}$, where $\phi(\pm
x^{k+1})=g$, $\alpha\in\mathbb R$, and the sign depends on the singularity
type.  Since $k+1$ is even and $\alpha\in\mathbb R$, $\phi$ does not change the
sign of the coefficient of $x^{k+1}$. We use algorithm~\ref{alg:A_k}, after
applying the Splitting Lemma in case $c=0$ or $c=1$.

\begin{algorithm}[h]
\caption{\label{alg:A_k} Algorithm for the case $A_k$}
\begin{algorithmic}[1]

\REQUIRE{$f\in \mathbb Q[x_1,\ldots,x_n]$ of complex singularity type $A_k$,
the output polynomial $g$ after applying
Algorithm \ref{AlgorithmSplittingLemma}, the corank $c$ of $f$ and the inertia
index $\lambda$ of $f$}

\ENSURE{the real singularity type of $f$, i.e.~$A_k^-$ or $A_k^+$,
$k\in\mathbb N$}

\IF{$c=0$}
\IF{$\lambda<n$}
\STATE type $:=A_1^+$
\ELSE
\STATE type $:=A_1^-$
\ENDIF
\ENDIF
\IF{$c=1$}
\STATE $k:= \ord(g)-1$
\IF{$k$ is even}
\STATE type $:=A_k^+$
\ELSE
\STATE $s:=$ coefficient of $x^{k+1}$
\IF{$s > 0$}
\STATE type $:=A_k^+$
\ELSE
\STATE type $:=A_k^-$
\ENDIF
\ENDIF
\ENDIF
\RETURN{type}

\end{algorithmic}
\end{algorithm}

\subsection{Real $\mathbf 0$-modal singularities of corank $\mathbf 2$}
The goal of the rest of the paper is to classify singularities of corank $2$.
In these cases $0\neq g\in\m^3$ is a polynomial in two variables, say
$g\in\mathbb Q[x,y]$. Using the \textsc{Singular} library {\tt classify.lib} we
determine the complex singularity type and thus the main real singularity type
of $g$, or equivalently $f$. The purpose of the remaining algorithms in the
paper is to classify the correct real subtype of $g$, or equivalently $f$. We
now consider each complex type, or equivalently every real main type,
seperately.

\subsubsection{$D_4$}

The normal form of the complex singularity type $D_4$ is $x^2y+y^3$, which
splits into $x^2y+y^3$ ($D_4^+$) and $x^2y-y^3$ ($D_4^-$) in the real case.
The two cases can be distinguished by blowing-up; the details are carried out
in algorithm~\ref{alg:D_4}. Since the determinacy of
$D_4$ is $3$, it suffices to look at the $3$-jet. If we want to use
$x \mapsto x$, $y \mapsto 1$ as blowing-up map, we first have to make sure that
the coefficient of $x^3$ is non-zero. It is easy to check that this is achieved
by lines 2 to 13 of algorithm~\ref{alg:D_4}. Finally, the number of real roots
after blowing up the 3-jet is an invariant of the real subtype which is 1 in
the case $D_4^+$ and 3 for $D_4^-$.

\begin{algorithm}[h]
\caption{\label{alg:D_4}\label{D[4]} Algorithm for the case $D_4$}
\begin{algorithmic}[1]

\REQUIRE{$g\in \m^3\subset\mathbb Q[x,y]$ of complex singularity type $D_4$}
\ENSURE{the real singularity type of $g$, i.e.~$D_4^+$ or $D_4^-$}
\STATE $h := \jt(g,3)$
\STATE $s_1:=$ coefficient of ${x^3}$ in $h$
\STATE $s_2 :=$ coefficient of ${y^3}$ in $h$
\IF{$(s_1 = 0)$}
\IF{$(s_2 \neq 0)$}
\STATE swap the variables $x$ and $y$ in $h$
\ELSE
\STATE $t_1:=$ coefficient of ${x^2y}$ in $h$
\STATE $t_2:=$ coefficient of ${xy^2}$ in $h$
\IF{$(t_1+t_2 \neq 0)$}
\STATE apply $x\mapsto x$, $y\mapsto x+y$ to $h$
\ELSE
\STATE apply $x\mapsto x$, $y\mapsto 2x+y$ to $h$
\ENDIF
\ENDIF
\ENDIF
\STATE apply $x\mapsto x$, $y\mapsto 1$ to $h$
\STATE $n:= \#$ real roots of $h$
\IF{$(n<3)$}
\RETURN $D_4^+$
\ELSE
\RETURN $D_4^-$
\ENDIF

\end{algorithmic}
\end{algorithm}

Implementing the algorithms in this paper in \textsc{Singular}, we used the
library \texttt{rootsur.lib} (\cite{roots}) to count the number of real roots
of a univariate polynomial.

\subsubsection{$D_k, k > 4$}

For the cases $D_k$ with $k > 4$, the complex normal form is $x^2y+y^{k-1}$. It
splits up into $x^2y+y^{k-1}$ ($D_k^+$) and $x^2y-y^{k-1}$ ($D_k^-$) for each
$k$ over the reals. We use the following two results from \cite{Siersma} to
distinguish between the two cases:

\begin{lemma}\label{kDeterminacyD[k]k>4}
A singularity of type $D_k^+$ or $D_k^-$ is $(k-1)$-determined.
\end{lemma}

\begin{lemma}\label{transformationD[k]}
Let $j\ge 4$. Then
\[
x^2y+a_0x^j+a_1x^{j-1}y+\cdots+a_jy^j\requiv x^2y+a_jx^j,
\quad a_0,\ldots,a_j\in\mathbb R,
\]
using the $\mathbb R$-algebra automorphisms
\begin{eqnarray*}
x&\mapsto&x+p_1,\textnormal{ where }
p_1=-\frac{1}{2}(a_1x^{j-2}+\cdots+a_{j-1}y^{j-2})\,,\\
y&\mapsto&y+p_2,\textnormal{ where } p_2=-a_0x^{j-2} \,.
\end{eqnarray*}
\end{lemma}

By Lemma \ref{kDeterminacyD[k]k>4} the determinacy of a singularity of  complex
type
$D_k$ is $k-1$. Therefore we only need to consider the
$(k-1)$-jet of $g$ in this case. Using Lemma~\ref{kjet} and Lemma~\ref{x^3}, we
transform $g$ into a polynomial of the form
\[x^2y+\textnormal{terms of degree higher than $3$}\]
by factorizing the $3$-jet of $g$ as $g_1^2g_2$, $g_1$ and $g_2$ of
degree $1$,
and then applying the automorphism defined by $g_1\mapsto x$, $g_2\mapsto y$ to
$g$. We
now systematically consider the terms of each degree $3<j<k$. By applying the
transformations in Lemma~\ref{transformationD[k]}, for each $j$, the only term
of total degree $j$ which possibly remains is $a_jy^j$. This term vanishes for
$j<k-1$ and it does not vanish for $j=k-1$, otherwise $g$ is not of complex
type $D_k$. Thus, after applying these transformations,
we can write $g$ as $g=x^2y+\alpha y^{k-1}$ with $a\neq0$. Clearly if $\alpha>0$ then
$x^2y+\alpha y^{k-1}\requiv x^2y+y^{k-1}$ and if $\alpha<0$ then
$x^2y+\alpha y^{k-1}\requiv x^2y-y^{k-1}$.

\begin{algorithm}[h]
\caption{\label{alg:D_k} Algorithm for the case $D_k$, $k > 4$}
\begin{algorithmic}[1]

\REQUIRE{$g \in \m^3\subset\mathbb Q[x,y]$ of complex singularity type $D_k$,
$k\in\mathbb N$, $k>4$}

\ENSURE{the real singularity type of $g$, i.e.~$D_k^-$ or $D_k^+$}

\STATE $k:= \mu(g)$
\STATE $h:=\jt(g,k-1)$
\STATE factorize $\jt(h,3)$ as $h_1^2h_2$, where $h_1$ and $h_2$ are linear
\STATE apply $h_1\mapsto x$, $h_2\mapsto y$ to $h$
\FOR{$(j = 4, \ldots, k-1)$}
\IF{$(\jt(h,j)-x^2y\neq0)$}
\STATE write $\jt(h,j)-x^2y$ as
$a_0x^j+a_1x^{j-1}y+\cdots +a_jy^j,\quad a_0,\ldots a_j\in\mathbb Q$
\STATE apply $x\mapsto x-\frac{1}{2}(a_1x^{j-2}+\cdots
+a_{j-1}y^{j-2})$, $y\mapsto y-a_0x^{j-2}$ to $h$
\STATE $h:=\jt(h,k-1)$
\ENDIF
\ENDFOR
\STATE write $h$ as $h=x^2y+\alpha y^{k-1}$, $0\neq\alpha\in\mathbb Q$
\IF{$(\alpha>0)$}
\RETURN $D_k^+$
\ELSE
\RETURN $D_k^-$
\ENDIF

\end{algorithmic}
\end{algorithm}

\subsubsection{$E_6$}

In this case, whose complex normal form is $x^3+y^4$, we have that either
$g\requiv x^3+y^4$ ($E_6^+$) or $g\requiv x^3-y^4$ ($E_6^-$).
Therefore there exists an $\mathbb
R$-algebra automorphism $\phi$ of $\mathbb R[x,y]$ such that
$\phi(g)=(\phi(x))^3+(\phi(y))^4$ or such that
$\phi(g)=(\phi(x))^3-(\phi(y))^4$.
Since the coefficients of $x^3$ and $y^3$ in $g$ cannot both be zero, we can
ensure that the coefficient of
$x^3$ is non-zero by swapping the variables if necessary. Now, using Lemma
\ref{kjet} and Lemma~\ref{x^3}, $\jt(g,3)$ factorizes as
$c(g_1)^3$, $c\in\mathbb Q$ and $g_1=b_0x+b_1y\in\mathbb
Q[x,y]$. Again,
using Lemma \ref{kjet}, it follows that by applying
$x\mapsto\frac{x-b_1y}{b_0},\
y\mapsto y$ to $g$, $\phi$ is transformed such that $\phi_1(x)=c'x$,
$c'\in\mathbb
R$. Since $\phi$ is an automorphism, $\phi_1(y)=d_0x+d_1y$,
$d_0,d_1\in\mathbb R$,
with $d_1\neq 0$. Hence
\begin{equation*}
(\phi(y))^4=d_1^4y^4+\textnormal{terms of degree 4 and higher, not of the
form $\alpha y^4$, $\alpha\in\mathbb R$.}
\end{equation*}
If we can show that $(\phi(x))^3$ does not contain a term of the form
$\alpha y^4$, $\alpha\in\mathbb R$, then we can determine whether $g$
is of type $E_6^-$ or $E_6^+$ by considering the sign of the coefficient of
the monomial $y^4$. Now
\begin{eqnarray*}
\textnormal{jet}((\phi(x))^3,4)-\textnormal{jet}((\phi(x))^3,3)
&=&3(\phi_1(x)^2)(\phi_2(x)-\phi_1(x))\\
&=&3(c'x)^2(\phi_2(x)-\phi_1(x)),
\end{eqnarray*}
 which means that $(\phi(x))^3$ does not have terms of the form $\alpha y^4$,
 $\alpha\in\mathbb R$ .

\begin{algorithm}[h]
\caption{\label{alg:E_6}\label{E[6]} Algorithm for the case $E_6$}
\begin{algorithmic}[1]

\REQUIRE{$g\in \m^3\subset\mathbb Q[x,y]$ of complex singularity type $E_6$}
\ENSURE{the real singularity type of $g$, i.e.~$E_6^-$ or $E_6^+$}
\STATE $h:= \jt(g,3)$
\STATE $s:=$ coefficient of ${x^3}$ in $h$
\IF{$(s=0)$}
\STATE swap the variables $x$ and $y$
\ENDIF
\STATE factorize $h$ into linear factors over $\mathbb Q[x,y]$, with a factor
$g_1=b_0x+b_1y$
\STATE apply $x\mapsto \frac{x-b_1y}{b_0}$, $y\mapsto y$ to $g$
\STATE $d :=$ coefficient of $y^4$ in $g$
\IF{$(d>0)$}
\RETURN $E_6^+$
\ELSE
\RETURN $E_6^-$
\ENDIF

\end{algorithmic}
\end{algorithm}


 \begin{thebibliography}{99}
\bibitem{AVG1985} Arnold, V.I.; Gusein-Zade, S.M.; Varchenko, A.N.:
Singularities of Differential Maps. Vol.~I, Birkh\"auser (1985).
\bibitem{A1975} Arnold, V.I.:
\textit{Normal form of functions near degenerate critical points.},
Russian Mth. Surveys 29 ii (1975), 10-50.
\bibitem{DGPS}
Decker, W.; Greuel, G.-M.; Pfister, G.; Sch{\"o}nemann, H.:
\newblock {\sc Singular} {3-1-5} --- {A} computer algebra system for polynomial
computations.
\newblock {http://www.singular.uni-kl.de} (2012).
\bibitem{Kruger} Kr\"uger, K.: Klassifikation von
Hyperfl\"agensingularit\"aten, Diploma Thesis (1997).
\bibitem{GLS2007}Greuel, G.-M.; Lossen, C.; Shustin E.:
Introduction to Singularities and Deformations, Springer, Berlin (2007).
\bibitem{GP2008} Greuel G.-M.; Pfister G.;
A Singular introduction to Commutative Algebra, 2nd Ed., Springer,
Berlin (2008).
\bibitem{classify}
Kr\"uger, K.:
{\tt classify.lib}. {A} {\sc Singular} {3-1-5} library for classifying isolated
hypersurface singularities w.r.t. right equivalence, based on the determinator
of singularities by V.I. Arnold (2012).
\bibitem{realclassify}
Marais, M. and Steenpass, A.:
{\tt realclassify.lib}. {A} {\sc Singular} {3-1-5} library for classifying
isolated hypersurface singularities over the reals w.r.t. right equivalence,
based on the determinator of singularities by V.I. Arnold. This library is
based on classify.lib by Kai Kr\"uger, but handles the real case, while
classify.lib does the complex classification (2012).
\bibitem{primdec.lib} Pfister, G.; Decker, W.;  Schoenemann, H.; Laplagne, S.:
{\tt primdec.lib}. {A} {\sc Singular} {3-1-5} library for Primary Decomposition
and Radical of Ideals (2012).
\bibitem{Siersma} Siersma D.: Classification and deformation of Singularities,
disertation, University of Amsterdam (1974).
\bibitem{roots}
Tobis, A.:
{\tt rootsur.lib}. {A} {\sc Singular} {3-1-5} library for Counting number of
real roots of univariate polynomial (2012).
\bibitem{solve.lib} Wenk, M.: {\tt solve.lib}. Pohl, W.:
{A} {\sc Singular} {3-1-5} library for Complex Solving of Polynomial Systems
(2012).

\end{thebibliography}

\end{document}
