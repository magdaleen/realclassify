\documentclass[noend]{amsproc}

\linespread{1}
\renewcommand{\arraystretch}{1.3}

\usepackage{amsthm,amsmath,amsfonts,mathrsfs,amssymb}
\usepackage{algorithm}
\usepackage{algorithmic}
\usepackage[T1]{fontenc}   % for bold \Singular
\usepackage{multirow}

\newtheorem{theorem}{Theorem}
\newtheorem{defn}[theorem]{Definition}
\newtheorem{prop}[theorem]{Proposition}
\newtheorem{lemma}[theorem]{Lemma}

% ALGORITHM style
\renewcommand{\algorithmicrequire}{\textbf{Input:}}
\renewcommand{\algorithmicensure}{\textbf{Output:}}
\newcommand{\algorithmicbreak}{\textbf{break}}
\newcommand{\BREAK}{\STATE \algorithmicbreak}

\newcommand{\Singular}{\textsc{Singular}}
\newcommand{\realclassify}{\texttt{realclassify.lib}}
\newcommand{\classify}{\texttt{classify.lib}}

\DeclareMathOperator{\ord}{ord}
\DeclareMathOperator{\requiv}{\overset{r}{\sim}}
\DeclareMathOperator{\m}{\mathfrak{m}}
\DeclareMathOperator{\jet}{jet}
\DeclareMathOperator{\corank}{corank}
\DeclareMathOperator{\supp}{supp}
\DeclareMathOperator{\sign}{sign}
\DeclareMathOperator{\diag}{diag}
\DeclareMathOperator{\NF}{NF}
\DeclareMathOperator{\N}{\mathbb{N}}
\DeclareMathOperator{\Q}{\mathbb{Q}}
\DeclareMathOperator{\R}{\mathbb{R}}
\DeclareMathOperator{\C}{\mathbb{C}}
\DeclareMathOperator{\A}{\mathbb{A}}

\title[The classification of real singularities using \textsc{Singular}, %
Part I]%
{The classification of real singularities using \textsc{Singular}\\
Part I: Splitting Lemma and Simple Singularities}

\author{Magdaleen S. Marais}
\address{Magdaleen S. Marais\\
African Institute for Mathematical Sciences and Stellenbosch University\\
6 Melrose Rd\\
Muizenberg 7945, Cape Town\\
South Africa}
\email{magdaleen@aims.ac.za}

\author{Andreas Steenpa\ss}
\address{Andreas Steenpa\ss\\
Department of Mathematics\\
University of Kaiserslautern\\
Erwin-Schr\"odinger-Str.\\
67663 Kaiserslautern\\
Germany}
\email{steenpass@mathematik.uni-kl.de}

\thanks{ }
\subjclass[2000]{}
\keywords{}
\begin{document}

\begin{abstract}
We present algorithms to classify isolated hypersurface singularities over the
real numbers according to the classification by V.I.~Arnold (\cite{AVG1985}).
This first part covers the splitting lemma and the simple singularities; a
second part will be devoted to the unimodal singularities up to corank 2.
All algorithms are implemented in the \Singular{} library \realclassify{}
(\cite{realclassify}).
\end{abstract}

\maketitle


\section{Introduction}
The goal of this paper, together with its second part, is to present efficient
algorithms for the classification of isolated hypersurface singularities up to
modality~1 and corank 2 over the real numbers with respect to right
equivalence. We refer to the classification given by V.I.~Arnold in
\cite{AVG1985}. Hence we consider smooth real functions with a critical point
at the origin and critical value $0$, i.e.\@ functions in $\m^2$, where $\m$
denotes the ideal of function germs vanishing at the origin. Two function germs
$f, g \in \m^2 \subset \R[[x_1,\ldots,x_n]]$ are considered as right
equivalent, denoted by $f \requiv g$, if there exists an $\R$-algebra
automorphism $\phi$ of $\R[[x_1,\ldots, x_n]]$ such that $\phi(f) = g$.  We
have implemented all the algorithms presented here in the computer algebra
system \Singular.  The implementation is freely available as a \Singular{}
library called \realclassify{} which relies on \Singular's \classify{} to
determine, for a given polynomial, the type in Arnold's classification over the
complex numbers. The methods used in \classify{} will not be discussed in this
paper. For more information in this regard, \cite{Kruger} can be studied.

Using the Splitting Lemma (Theorem~\ref{SplittingLemma}), any function germ $f$
over the real numbers with an isolated singularity at the origin can be
written, after choosing a suitable coordinate system, as the sum of two
functions of which the variables are disjoint. One of the functions, called the
nondegenerate part of $f$, is a nondegenerate quadratic form and the other
function, called the residual part of $f$, is an element of $\m^3$. The number
of variables in the residual part is equal to the corank of $f$, denoted by
$\corank(f)$. In this paper we only consider germs with corank $0$, $1$
and~$2$.  A version of the Splitting Lemma for singularities over $\R$ and a
corresponding algorithm are discussed in section \ref{TheSplittingLemma}.

In \cite{AVG1985} Arnold classified the real singularities of modality $0$ and
$1$ up to stable equivalence into main types which split up into more subtypes
depending the sign of certain terms. Two functions are stably equivalent if
they are right equivalent after the direct addition of nondegenerte quadratic
forms. Hence after applying the Splitting Lemma,  we only need to consider the
residual part in order to complete the classification.   It can be easily seen
that the subtypes are complex equivalent to a complex singularity type of the
same name as its corresponding real main singularity type (see Table
\ref{normal forms}). In fact there is a bijection between the complex
singularity types of modality $0$ and $1$ and the main real singularity types.
Thus, if we can determine the complex singularity type of a function germ, we
only need to determine the correct subtype of the corresponding main real
singularity type.

We denote the $k$-jet of a function germ $f$ by $\jet(f,k)$. In section
\ref{ResultsRegardingTheFactorizationOfHomogeneousPolynomialsOverRAndQ} we
discuss the factorization of the lowest nonzero jets of two right-equivalent
functions. These results are used in sections
\ref{RealSingularitiesOfZeroModality} to \ref{ExceptionalSingularities}, where
the algorithms are given to classify germs in $\m^3$. Firstly, we show that the
$k$-jets of two right-equivalent functions, where $k$ is the order of the
functions, factorize in the same way over $\mathbb R$ (Lemma~\ref{kjet}). Since
the real numbers are floating numbers in \textsc{Singular}, it is not always
possible to factorize over the real numbers using \textsc{Singular}. In some
cases we prove that these lowest jets will always factorize in linear terms
over the rational numbers and thus in \textsc{Singular}, for example when the
lowest jet is of degree $k$ and has $k$ similar factors (Lemma \ref{x^3}). In
some other cases this unfortunately is not true. Since we can distinguish
between some singularities, by only considering the number of real roots of its
lowest nonzero jet, the similar factorization of the lowest nonzero jets is
still useful in these cases. Because the lowest nonzero jet of a function germ
is a homogeneous polynomial, we use dehomogenization and the library
``rootsur.lib" \cite{roots}, a library counting the real roots of a univariate
polynomial in \textsc{Singular}, to count the roots. We for instance use this
method in the classification of the $D_4$ (Algorithm~\ref{D[4]}) and $E_6$
(Algorithm \ref{E[6]}) cases.

Sometimes other methods are used in addition to the above methods, for example
the normal form of singularities of type $E_{14}$ is determined by
systematically computing higher jets of the $\mathbb R$-algebra automorphism
between the normal form of the function and the function itself considering the
terms of these functions, after the lowest jet is determined using the above
methods (Algorithm \ref{E[14]}).

In the cases $J_{10}$ and $J_{10+k}$, $k>0$ we used different normal forms than
Arnold. These forms are more suitable to work with over the real numbers and
specifically simplify the computations to classify these cases over the real
numbers. We include a list of the normal forms we used at the beginning of
section \ref{TheRealClassificationOfTheResidualPart}.

Lastly, we used the method of blowing up the origin (Algorithm
\ref{BlowingUp}), which simplifies singularities. In the case $Y_{r,s}$,
$r,s>4$ (Algorithm \ref{Y[r,s]}) enough information can be subtracted from the
resulting, simplified, singularity such that the case can be solved, in some
specific cases, unfortunately using approximations.


\section{Prerequisites}

\subsection{The Milnor Number}

We briefly recall the following well-know definition:

\begin{defn}
For $f \in \R[[x_1,\ldots,x_n]]$ and $p \in \A_{\R}^n$, the
\emph{Milnor number} of $f$ at $p$ is defined as
\[
\mu(f, p) := \dim_{\R}
\left( \R[[x_1-p_1, \ldots, x_n-p_n]] \bigg/
\left\langle \frac{\partial f}{\partial x_1}, \ldots,
\frac{\partial f}{\partial x_n} \right\rangle \right)
\in \N \cup \{\infty\} \,.
\]
If $p$ is the origin, we simply write $\mu(f)$ instead of $\mu(f, p)$.
\end{defn}

The Milnor number is known to be finite at isolated singularities
(cf.\@ Lemma~2.3 in \cite{GLS2007}) and to be invariant under right equivalence
(cf.\@ Lemma~2.10 ibid.). It is thus an important tool for the classification
of isolated singularities. We refer to \cite{GLS2007} for more properties of
this invariant.

We use Algorithm~\ref{alg_Milnor} to compute the Milnor number of a given
polynomial at the origin. In line 3, $\dim(\langle G \rangle)$ denotes the
dimension of $\langle G \rangle$ as an ideal, which is easy to read off from
the standard basis $G$. In line 6, $\NF_<$ is any weak normal form w.r.t.\@ the
monomial ordering~$<$. Note that the set $K$ is garantueed to be finite. We
refer to \cite{GP2008} for details on (local) monomial orderings, standard
bases and (weak) normal forms.

\begin{algorithm}[h]
\caption{\label{alg_Milnor} \textsc{Milnor}}
\begin{algorithmic}[1]

\REQUIRE{$f \in \Q[x_1,\ldots,x_n]$}
\ENSURE{$\mu(f)$}

\STATE $J := \left( \frac{\partial f}{\partial x_1}, \ldots,
\frac{\partial f}{\partial x_n} \right) \subset \Q[x_1,\ldots,x_n]$
\STATE compute a standard basis $G$ of $J$ w.r.t.\@ a local monomial ordering
$<$
\IF{$(\dim(\langle G \rangle) \neq 0)$}
\RETURN $\infty$
\ELSE
\STATE $K := \{\NF_<(m, G) \mid m \text{ a monomial in } \R[[x_1,\ldots,x_n]]
\text{ and } m \not\in J = \langle G \rangle \}$
\STATE $L :=$ a basis of the $\R$-vector space $\langle K \rangle_{\R}$
\RETURN $|L|$
\ENDIF

\end{algorithmic}
\end{algorithm}

\subsection{The Determinacy}

In general, the singularities we deal with in this paper are defined for power
series, but algorithmically, we want to work with polynomials. It is thus
important for our algorithmic approach that any power series is right
equivalent to a polynomial which can be obtained from it by leaving out terms
of sufficiently high order.

\begin{defn}
Let $f \in \R[[x_1,\ldots,x_n]]$ be a power series.

\begin{enumerate}
\item Let $f = \sum_{j=0}^\infty f_j$ be the decomposition of $f$ into
homogeneous parts $f_j$ of degree $j$.
For $k \in \N$, we define the \emph{$k$-jet} of $f$ as
\[
\jet(f,k) := \sum_{i=0}^k f_k \,.
\]
In other words, the $k$-jet of $f$ can be obtained from $f$ by leaving out all
terms of order higher that $k$.

\item $f$ is called \emph{$k$-determined} if
\[
\forall g \in \m^{k+1}: \quad f \requiv \jet(f,k)+g \,.
\]
\end{enumerate}
\end{defn}

The determinacy is, just as the Milnor number, both invariant under right
equivalence and finite for isolated singularities. We cite the following
statement (cf.\@ \cite{GLS2007}, Supplement to Theorem~2.23) due to its
importance for the algorithmic approach and refer to \cite{GLS2007} for further
results regarding the determinacy:

\begin{prop}
Let $f \in \m \subset \R[[x_1,\ldots,x_n]]$. If it holds that
\[
\m^{k+1} \subset \m^2 \left\langle \frac{\partial f}{\partial x_1}, \ldots,
\frac{\partial f}{\partial x_n} \right\rangle_{\R[[x_1,\ldots,x_n]]} \,,
\]
then $f$ is $k$-determined.
\end{prop}

As a direct consequence of this, any power series $f$ which has an isolated
singularity at the origin is $(\mu(f)+1)$-determined. But we can often compute
a much better upper bound for the determinacy by using the above statement as
in Algorithm~\ref{alg_Determinacy}. It is worth to note that
both the Milnor number and the determinacy of an arbitrary power series
$f \in \R[[x_1,\ldots,x_n]]$ do not change if we regard $f$ as an element of
$\C[[x_1,\ldots,x_n]]$. The same holds for the output of the corresponding
algorithms presented here.

\begin{algorithm}[h]
\caption{\label{alg_Determinacy} \textsc{Determinacy}}
\begin{algorithmic}[1]

\REQUIRE{$f \in \Q[x_1,\ldots,x_n]$ with an isolated singularity at the origin}
\ENSURE{an upper bound for the determinacy of $f$}

\STATE $k := \textsc{Milnor}(f)+1$
\STATE $J := \left( \frac{\partial f}{\partial x_1}, \ldots,
\frac{\partial f}{\partial x_n} \right) \subset \Q[x_1,\ldots,x_n]$
\STATE compute a standard basis $G$ of $(\m^2 J)$ w.r.t.\@ a local monomial
ordering $<$
\FOR{$(l = 1,\ldots,k-1)$}
\IF{$(\NF_<(\m^{l+1},G) = 0)$}
\STATE $k := l$
\BREAK
\ENDIF
\ENDFOR
\RETURN $k$

\end{algorithmic}
\end{algorithm}


\subsection{Results regarding the factorization of homogeneous polynomials
over $\mathbb R$ and $\mathbb Q$}%
\label{ResultsRegardingTheFactorizationOfHomogeneousPolynomialsOverRAndQ}

The results in this subsection are preliminary results that will be used in
sections \ref{RealSingularitiesOfZeroModality} to
\ref{ExceptionalSingularities}.

Let $f,g\in \m^2\subset\mathbb R[[x_1,\ldots,x_n]]$ such that
$f\requiv g$, i.e.\@ $\phi(f)=g$, where $\phi$ is an $\mathbb R$-algebra
automorphism of  $\mathbb R[[x_1,\ldots,x_n]]$. Since $\phi$ is an automorphism
it is defined by the images $\phi(x_1),\ldots,\phi(x_n)$. We define the $j$-jet
of $\phi$, denoted by $\phi_j$, to be the automorphism defined by
$\phi_j(x_1):=\jet(\phi(x_1),j),\ldots,\phi_j(x_n):=\jet(\phi(x_n),j)$. By
using the next result, given $f$ and $g$, we can determine $\phi_1$.

\begin{theorem}\label{kjet}
Let $f,g\in \m^2\subset\mathbb R[[x_1,\ldots,x_n]]$. Suppose $f\requiv g$
and $k$
is the lowest degree of $f$. If jet$(f,k)$ factorizes as
\[f_1^{s_1}(x_1,\ldots,x_n)\cdots f_t^{s_t}(x_1,\ldots,x_n)\]
over $\mathbb R$, then jet$(g,k)$ will factorize
as \[f_1^{s_1} (\phi_1(x_1),\ldots,\phi_1(x_n)),\cdots
f_t^{s_t}(\phi_1(x_1)),\ldots,(\phi_1(x_n)))\] over $\mathbb R$, where $\phi$
is an $\mathbb R$-algebra automorphism of $\mathbb R[[x_1,\ldots,x_n]]$
such that $\phi(f)=g$.
\end{theorem}
\begin{proof}
Since $f\requiv g$, there exists a ring automorphism $\phi\in\mathbb
R[[x_1,\ldots,x_n]]$ such that  $\phi(f)=g$. Now
assume that the monomials of lowest degree $k$ of $f$ factorize
as \[f_1^{s_1}(x_1,\ldots,x_n)\cdots f_t^{s_t}(x_1,\ldots,x_n)\] over
$\mathbb R$, i.e.\@ $f=f_1^{s_1}\cdots f_t^{s_t}+f'$, where the degree and
order of $f_1^{s_1},\ldots,f_t^{s_t}$ is $k$ and the order of $f'$ is greater
than $k$.  Now, for all $j\in\{1,\ldots,n\}$,
\[\phi(x_j)=\phi_1(x_j)+\textnormal{terms of degree higher than $1$}.\] Since
$\phi$ is a homomorphism, it follows that

\begin{eqnarray*}
\phi(f)&=&f_1^{s_1}\big(\phi(x_1),\ldots,\phi(x_n)\big)\cdots
f_t^{s_t}\big(\phi(x_1),\ldots,\phi(x_n)\big)+\phi(f')\\
&=&f_1^{s_1}\big(\phi_1(x_1),\ldots,\phi_1(x_n)\big)\cdots
f_t^{s_t}\big(\phi_1(x_1),\ldots,\phi_1(x_n)\big)\\
&&+\big(\phi-\phi_1\big)(f_1^{s_1}\cdots f_t^{s_t})+\phi(f').\\
\end{eqnarray*}
Since the $\deg\big(\phi_1(x_j))=1$ it follows that
\[\deg\big(f_1^{s_1}\big(\phi_1(x_1),\ldots,\phi_1(x_n)\big)\cdots
f_t^{s_t}\big(\phi_1(x_1),\ldots,\phi_1(x_n)\big)\big)=k.\] Because
\[
\textnormal{order}\big(\big(\phi-\phi_1\big)(f_1^{s_1}\cdots f_t^{s_t})\big)>k
\quad\textnormal{and}\quad\textnormal{order}\big(\phi(f')\big)>k
\]
it follows that
\begin{eqnarray*}
\textnormal{jet}(\phi(f),k)=\textnormal{jet}(g,k)
=f_1^{s_1}\big(\phi_1(x_1),\ldots,\phi_1(x_n)\big)\cdots
f_t^{s_t}\big(\phi_1(x_1),\ldots,\phi_1(x_n)\big).
\end{eqnarray*}
\end{proof}

Since \textsc{Singular} use real floating point numbers instead of real
numbers, rounding errors in computations in this setting can
occur. Therefore, in this paper, the above result would not be of much help
without the
following result.

\begin{lemma}\label{x^3}
If $f\in\mathbb Q[x,y]$ is homogeneous and factorizes as (i) $g_1^d$, (ii)
$g_1\cdot g_2^{d}$ or as  (iii) $g_1^2(g_1-g_2)(g_1+g_2)$, $d>1$, over
$\mathbb
R$, where $g_1,g_2$ are polynomials of degree $1$, then $f$ will factorize as
(i) $ag_1'^d$, (ii) $ag_1'\cdot g_2'^d$, (iii) $ag_1'^2(g_1'-g_2')(g_1'+g_2')$
respectively,
where $g_1', g_2'$ are polynomials of degree $1$ over $\mathbb Q$ and
$a\in\mathbb Q$.
\end{lemma}
\begin{proof}

(i) Let $f=(a_1x+a_2y)^d$, $a_1,a_2\in\mathbb R$. Without loss of generality,
suppose $a_1\neq 0$. Then $f=a_1^d(x+\frac{a_2}{a_1}y)^d$. Since $f=
a_1^dx^d+da_1^{d-1}a_2x^{d-1}y+\cdots+a_2^dy\in\mathbb Q[x,y]$, it follows
that $a_1^d\in\mathbb Q$. Hence $(x+\frac{a_2}{a_1}y)^d\in\mathbb Q[x,y]$
from which it follows that $(x+\frac{a_2}{a_1})^d\in\mathbb Q[x]$. Since
$\mathbb Q$ is a perfect field it follows that $\frac{a_2}{a_1}\in\mathbb
Q$. Thus $f=a{g'}_1^d$, where $a:=a_1^d\in\mathbb Q$ and
$g_1=x+\frac{a_2}{a_1}y\in\mathbb Q[x,y]$.

(ii) Let $f=(a_1x+a_2y)(a_3x+a_4y)^{d}$, $a_1,\ldots,a_4\in\mathbb
R$. Suppose $a_1,a_3\neq 0$.
For the cases $a_1,a_4\neq 0$, $a_2,a_3\neq 0$ and $a_2,a_4\neq 0$ the
proofs are similar.
Similar as above, it follows that $a_1a_3^{d}\in\mathbb Q$. Hence
$(x+\frac{a_2}{a_1}y)(x+\frac{a_4}{a_3}y)^d\in\mathbb Q[x,y]$ from which
it follows that $(x+\frac{a_2}{a_1})(x+\frac{a_4}{a_3})^d\in\mathbb
Q[x]$. Since $\mathbb Q$ is a perfect field, it follows that
$(x+\frac{a_2}{a_1})(x+\frac{a_4}{a_3})\in\mathbb Q[x]$ and
hence that $(x+\frac{a_4}{a_3})^{d-1}\in\mathbb Q[x]$. Therefore
also $(x+\frac{a_4}{a_3})\in\mathbb Q[x]$ which implies that
$(x+\frac{a_2}{a_1})\in\mathbb Q[x]$.
Thus $f=ag'_1g'^d_2$ with $a:=a_1a_3^d\in\mathbb Q$,
$g'_1:=(x+\frac{a_2}{a_1}y)\in\mathbb Q[x,y]$, and
$g'_2:=(x+\frac{a_2}{a_1}y)\in\mathbb Q[x,y]$.

(iii) If $f$ factorizes as $g_1^2(g_1-g_2)(g_1+g_2)$ over $\mathbb R$, it
follows similarly as above that $g_1\in\mathbb Q[x,y]$. Therefore
$(g_1-g_2)(g_1+g_2)=g_1^2-g_2^2\in\mathbb Q[x,y]$ and hence $g_2^2\in\mathbb
Q[x,y]$. Using (i) $g_2\in\mathbb Q[x,y]$ and the result follows.
\end{proof}


\section{The Splitting Lemma}\label{TheSplittingLemma}

The following well-known theorem, called the Splitting Lemma, allows us to
reduce the classification to germs of full corank or, algorithmically, to
a polynomial contained in $\m^3 \cap \R[x_1,\ldots,x_c]$ for a given input
polynomial of corank $c$. We present a version for singularities over the real
numbers, taking into account the signs of the squares.

\begin{theorem}\label{SplittingLemma}
If $f \in \m^2 \subset \R[[x_1,\ldots,x_n]]$ has an isolated singularity and if
its corank is $c$, then
\[
f \requiv g -\sum_{i=c+1}^{c+\lambda} x_i^2 +\sum_{i=c+\lambda+1}^n x_i^2
\]
with $g \in \m^3 \cap \R[[x_1,\ldots,x_c]]$. $g$ is called the residual part of
$f$ and $\lambda$ is called the inertia index of~$f$. Both $g$ and $\lambda$
are uniquely determined up to right equivalence.
\end{theorem}

The following proof is based upon the proofs of the Theorems 2.46 and 2.47 in
\cite{GLS2007}.

\begin{proof}
The corank of the Hessian matrix of $f$ at $0$ is $c$, so by the theory of
quadratic forms over $\R$ there is a transformation matrix $T$ such that
\[
T^t \cdot H(f)(\mathbf{0}) \cdot T = \diag(0,\ldots,0,-1,\ldots,-1,1,\ldots,1)
\,.
\]
Therefore the linear coordinate change
$(x_1,\ldots,x_n) \mapsto (x_1,\ldots,x_n) \cdot T^t$ transforms the 2-jet of
$f$ into
$\left(-\sum_{i=c+1}^{c+\lambda} x_i^2 +\sum_{i=c+\lambda+1}^n x_i^2\right)$
where $\lambda$ is the inertia index of $f$.
Applied to $f$, this transformation leads to
\begin{align*}
f^{(3)} (x_1,\ldots,x_n)
  :\!&= f((x_1,\ldots,x_n) \cdot T^t) \\
  &= g_3
  -\sum_{i=c+1}^{c+\lambda} x_i^2 +\sum_{i=c+\lambda+1}^n x_i^2
  +\sum_{i=c+1}^n x_i\cdot h_i^{(3)}
\end{align*}
with $g_3 \in \m^3 \cap \R[[x_1,\ldots,x_c]]$ and $h_i^{(3)} \in \m^2$. The
coordinate change $\varphi^{(3)}$ defined by
\[
\varphi^{(3)}(x_i) :=
\begin{cases}
x_i, &i \leq c, \\
x_i-\frac{1}{2}h_i^{(3)}, &i > c.
\end{cases}
\]
yields
\begin{align*}
f^{(4)} (x_1,\ldots,x_n)
  :\!&= f^{(3)}(\varphi^{(3)}(x_1,\ldots,x_n)) \\
  &= g_3 +g_4
  -\sum_{i=c+1}^{c+\lambda} x_i^2 +\sum_{i=c+\lambda+1}^n x_i^2
  +\sum_{i=c+1}^n x_i\cdot h_i^{(4)} \,.
\end{align*}
with $g_4 \in \m^4 \cap \R[[x_1,\ldots,x_c]]$ and $h_i^{(4)} \in \m^3$.
Continuing in the same manner, the last sum will be of arbitrarily high order.
It can be eventually left out because $f$ is finitely determined as an isolated
singularity.
\end{proof}

Since this proof is constructive, we can immediately derive algorithm
\ref{AlgorithmSplittingLemma} from it.

\begin{algorithm}[h]
\caption{\label{AlgorithmSplittingLemma} Algorithm for the Splitting Lemma}
\begin{algorithmic}[1]

\REQUIRE{$f \in \m^2 \subset \Q[x_1,\ldots,x_n]$ and $k \in \N$ such that $f$
is $k$-determined}

\ENSURE{the corank $c$ of $f$, the inertia index $\lambda$ of $f$ and
$g \in \m^3 \cap \Q[x_1,\ldots,x_c]$ such that
\[
f \requiv g -\sum_{i=c+1}^{c+\lambda} x_i^2 +\sum_{i=c+\lambda+1}^n x_i^2
\]
}

\STATE compute a transformation matrix $T \in \Q^{n \times n}$ such that
\[
T^t \cdot H(f)(\mathbf{0}) \cdot T = \diag(0,\ldots,0,-1,\ldots,-1,1,\ldots,1)
=:N
\]
\STATE $c :=$ nb. of zeroes on the diagonal of $N$
\STATE $\lambda :=$ nb. of entries equal to $-1$ on the diagonal of $N$
\STATE $f^{(3)} (x_1,\ldots,x_n) := f(T \cdot (x_1,\ldots,x_n)^t)$
\FOR{$(l = 3, \ldots, k)$}
\STATE write $f^{(l)}$ as
\[
f^{(l)} = \sum_{j=3}^l g_j
  -\sum_{i=c+1}^{c+\lambda} x_i^2 +\sum_{i=c+\lambda+1}^n x_i^2
  +\sum_{i=c+1}^n x_i\cdot h_i^{(l)}
\]
with $g_j \in \m^j \cap \Q[x_1,\ldots,x_c]$ and
$h_i^{(l)} \in \m^{l-1}$
\STATE $f^{(l+1)} := \varphi^{(l)}(f^{(l)})$ where $\varphi^{(l)}$ is defined
by
\[
\varphi^{(l)}(x_i) :=
\begin{cases}
x_i, &i \leq c, \\
x_i-\frac{1}{2}h_i^{(l)}, &i > c.
\end{cases}
\]
\ENDFOR

\STATE $g := \sum_{j=3}^k g_j$
\RETURN $c, \lambda, g$

\end{algorithmic}
\end{algorithm}


\section{The real classification of the residual part}%
\label{TheRealClassificationOfTheResidualPart}

In \cite{AVG1985} Arnold divided the real singularities of modality $0$ and
$1$, using stable equivalence, into main types which split up into more
subtypes by changing the sign in front of certain terms. Each of these subtypes
is complex equivalent to the complex singularity of the same name as its
corresponding real main singularity type. We include the complex
transformations for the corank two singularities in the table below. Since
there is a real main singularity type of modality $0$ or $1$ for each complex
singularity type of modality $0$ or $1$, respectively, there is a bijection
between the main real singularity types of modality $0$ and $1$ and the complex
singularity types of modality $0$ and $1$, respectively. It is not known yet
whether the complex forms of higher modal cases  split up into corresponding
real forms. The fact is that it is not known whether modality is preserved in
these cases. In the next table we list the normal forms that are used in this
article. From here onwards we will be working with stable equivalence. For all
nonquadratic forms $f$ it is thus only necessary, after applying the Splitting
Lemma to consider the residual part of function germs, i.e.\@ germs in $\m^3$.

\begin{table}[!hbp]
\centering
\caption{Real normal forms of singularities of modality $0$.}
\label{normal forms}
\begin{tabular}{|c|c|c|c|c|}
\hline
& Complex & Normal forms & \multirow{2}{*}{Equivalences} &
\multirow{2}{*}{Values of $k$} \\
& normal form & of real subtypes & & \\
\hline
\multirow{2}{*}{$A_k$} & \multirow{2}{*}{$x^{k+1}$} & $+x^{k+1}$ $(A_k^+)$ &
$A_k^+ \requiv A_k^-$ & \multirow{2}{*}{$k \geq 1$} \\
& & $-x^{k+1}$ $(A_k^-)$ & for~even $k$ & \\
\hline
\multirow{2}{*}{$D_k$} & \multirow{2}{*}{$x^2y+y^{k-1}$} &
$x^2y+y^{k-1}$ $(D_k^+)$ & $D_k^+ \requiv D_k^-$ &
\multirow{2}{*}{$k \geq 4$} \\
& & $x^2y-y^{k-1}$ $(D_k^-)$ & for even $k$ & \\
\hline
\multirow{2}{*}{$E_6$} & \multirow{2}{*}{$x^3+y^4$} & $x^3+y^4$ $(E_6^+)$ &
\multirow{2}{*}{-} & \multirow{2}{*}{-} \\
& & $x^3-y^4$ $(E_6^-)$ & & \\
\hline
$E_7$ & $x^3+xy^3$ & $x^3+xy^3$ & - & - \\
\hline
$E_8$ & $x^3+y^5$ & $x^3+y^5$ & - & - \\
\hline
\end{tabular}
\end{table}

It is known that the milnor number of a function germ $g$, which we denote by
$\mu(g)$, is the same regardless whether we work over the complex- or real
numbers. Therefore a singularity over the complex numbers is nondegenerate if
and only if the singularity is nondegenerate over the real numbers. The
restrictions to normal forms in the complex case are thus the same in the real
case. Furthermore we know that we can transform the normal forms in the real
case that differ from those we use in the complex case by a complex
transformation to a complex normal form. Since the milnor number is invariant
under complex transformations, we can then determine the restrictions. Let us
take the $X_9$ case as an example and consider the real normal form
$-x^4+ax^2y^2+y^4$. This normal form is complex equivalent to
$x^4+aix^2y^2+y^4$, which corresponds to the given corresponding complex normal
form. Since $a\in\mathbb R$, it follows that $(ia)^2=-a^2\neq4$ and therefore
this singularity is nondegenerate for all values of $a$.

Using the $1-1$ correspondence between the main real singularity types and
complex singularity types, and a \textsc{Singular} library ``classify.lib"
\cite{classify}, that classifies complex singularities, classifying a real germ
boils down to determining to which of the corresponding subtypes the germ is
equivalent.

Throughout the rest of the paper we write $f$ for the given input polynomial,
$g$ for its residual part which can be obtained by applying the Splitting
Lemma~\ref{AlgorithmSplittingLemma}, and $c$ for the corank of $f$.
We also assume that $f$, and thus $g$, is a polynomial over $\mathbb Q$.
With these notations, $g$ is a polynomial in $c$ variables.

\subsection{$\boldsymbol{A_1}$}
If $c = 0$, then it follows  that $f$ is of type
$A_1^+$ or of type $A_1^-$ depending on the inertia index $\lambda$ of $f$. If
the inertia index of $f$ is nonzero and less than the number of variables in
the base ring, then $f$ is both of type $A_1^+$ and $A_1^-$, depending how one
chooses to order the variables. If the inertia index is equal to the number of
variables in the base ring, $f$ is of type $A_1^-$. Lastly, if the inertia
index is $0$, then $f$ is of type $A_1^+$.

\subsection{$\boldsymbol{A_k, k > 1}$}
If $c=1$, then the singularity is of type
$A_k^+$ or of type $A_k^-$ for some $k>1$. Furthermore $g$ is a univariate
polynomial in this case, say $g\in\mathbb Q[x]$. Note that if $k$ is even, then
$A_k^+\requiv A_k^-$. The value of $k$ is given by the order of $g$ minus $1$.
This follows since $\pm x^{k+1}$ and $g$ are right-equivalent and thus have the
same order. The sign of the singularity type is determined by the sign of the
coefficient of $x^{k+1}$. This follows since it follows from Lemma \ref{kjet}
that $\jet(g,1)=\pm(\phi_1(x))^{k+1}=\pm(\alpha x)^{k+1}$, where $\phi(\pm
x^{k+1})=g$, $\alpha\in\mathbb R$, and the sign depends on the singularity
type.  Since $k+1$ is even and $\alpha\in\mathbb R$, $\phi$ does not change the
sign of the coefficient of $x^{k+1}$. We use algorithm~\ref{alg:A_k}, after
applying the Splitting Lemma in case $c=0$ or $c=1$.

\begin{algorithm}[h]
\caption{\label{alg:A_k} Algorithm for the case $A_k$}
\begin{algorithmic}[1]

\REQUIRE{$f\in \mathbb Q[x_1,\ldots,x_n]$ of complex singularity type $A_k$,
the output polynomial $g$ after applying
Algorithm \ref{AlgorithmSplittingLemma}, the corank $c$ of $f$ and the inertia
index $\lambda$ of $f$}

\ENSURE{the real singularity type of $f$, i.e.\@ $A_k^-$ or $A_k^+$,
$k\in\mathbb N$}

\IF{$c=0$}
\IF{$\lambda<n$}
\STATE type $:=A_1^+$
\ELSE
\STATE type $:=A_1^-$
\ENDIF
\ENDIF
\IF{$c=1$}
\STATE $k:= \ord(g)-1$
\IF{$k$ is even}
\STATE type $:=A_k^+$
\ELSE
\STATE $s:=$ coefficient of $x^{k+1}$
\IF{$s > 0$}
\STATE type $:=A_k^+$
\ELSE
\STATE type $:=A_k^-$
\ENDIF
\ENDIF
\ENDIF
\RETURN{type}

\end{algorithmic}
\end{algorithm}

The goal of the rest of the paper is to classify singularities of corank $2$.
In these cases $0\neq g\in\m^3$ is a polynomial in two variables, say
$g\in\mathbb Q[x,y]$. Using the \textsc{Singular} library {\tt classify.lib} we
determine the complex singularity type and thus the main real singularity type
of $g$, or equivalently $f$. The purpose of the remaining algorithms in the
paper is to classify the correct real subtype of $g$, or equivalently $f$. We
now consider each complex type, or equivalently every real main type,
seperately.

\subsection{$\boldsymbol{D_4}$}

The normal form of the complex singularity type $D_4$ is $x^2y+y^3$, which
splits into $x^2y+y^3$ ($D_4^+$) and $x^2y-y^3$ ($D_4^-$) in the real case.
The two cases can be distinguished by blowing-up; the details are carried out
in algorithm~\ref{alg:D_4}. Since the determinacy of
$D_4$ is $3$, it suffices to look at the $3$-jet. If we want to use
$x \mapsto x$, $y \mapsto 1$ as blowing-up map, we first have to make sure that
the coefficient of $x^3$ is non-zero. It is easy to check that this is achieved
by lines 2 to 13 of algorithm~\ref{alg:D_4}. Finally, the number of real roots
after blowing up the 3-jet is an invariant of the real subtype which is 1 in
the case $D_4^+$ and 3 for $D_4^-$.

\begin{algorithm}[h]
\caption{\label{alg:D_4}\label{D[4]} Algorithm for the case $D_4$}
\begin{algorithmic}[1]

\REQUIRE{$g\in \m^3\subset\mathbb Q[x,y]$ of complex singularity type $D_4$}
\ENSURE{the real singularity type of $g$, i.e.\@ $D_4^+$ or $D_4^-$}
\STATE $h := \jet(g,3)$
\STATE $s_1:=$ coefficient of ${x^3}$ in $h$
\STATE $s_2 :=$ coefficient of ${y^3}$ in $h$
\IF{$(s_1 = 0)$}
\IF{$(s_2 \neq 0)$}
\STATE swap the variables $x$ and $y$ in $h$
\ELSE
\STATE $t_1:=$ coefficient of ${x^2y}$ in $h$
\STATE $t_2:=$ coefficient of ${xy^2}$ in $h$
\IF{$(t_1+t_2 \neq 0)$}
\STATE apply $x\mapsto x$, $y\mapsto x+y$ to $h$
\ELSE
\STATE apply $x\mapsto x$, $y\mapsto 2x+y$ to $h$
\ENDIF
\ENDIF
\ENDIF
\STATE apply $x\mapsto x$, $y\mapsto 1$ to $h$
\STATE $n :=$ nb. of real roots of $h$
\IF{$(n<3)$}
\RETURN $D_4^+$
\ELSE
\RETURN $D_4^-$
\ENDIF

\end{algorithmic}
\end{algorithm}

Implementing the algorithms in this paper in \textsc{Singular}, we used the
library \texttt{rootsur.lib} (\cite{roots}) to count the number of real roots
of a univariate polynomial.

\subsection{$\boldsymbol{D_k, k > 4}$}

For the cases $D_k$ with $k > 4$, the complex normal form is $x^2y+y^{k-1}$. It
splits up into $x^2y+y^{k-1}$ ($D_k^+$) and $x^2y-y^{k-1}$ ($D_k^-$) for each
$k$ over the reals. We use the following two results from \cite{Siersma} to
distinguish between the two cases:

\begin{lemma}\label{kDeterminacyD[k]k>4}
A singularity of type $D_k^+$ or $D_k^-$ is $(k-1)$-determined.
\end{lemma}

\begin{lemma}\label{transformationD[k]}
Let $j\ge 4$. Then
\[
x^2y+a_0x^j+a_1x^{j-1}y+\cdots+a_jy^j\requiv x^2y+a_jx^j,
\quad a_0,\ldots,a_j\in\mathbb R,
\]
using the $\mathbb R$-algebra automorphisms
\begin{eqnarray*}
x&\mapsto&x+p_1,\textnormal{ where }
p_1=-\frac{1}{2}(a_1x^{j-2}+\cdots+a_{j-1}y^{j-2})\,,\\
y&\mapsto&y+p_2,\textnormal{ where } p_2=-a_0x^{j-2} \,.
\end{eqnarray*}
\end{lemma}

By Lemma \ref{kDeterminacyD[k]k>4} the determinacy of a singularity of  complex
type
$D_k$ is $k-1$. Therefore we only need to consider the
$(k-1)$-jet of $g$ in this case. Using Lemma~\ref{kjet} and Lemma~\ref{x^3}, we
transform $g$ into a polynomial of the form
\[x^2y+\textnormal{terms of degree higher than $3$}\]
by factorizing the $3$-jet of $g$ as $g_1^2g_2$, $g_1$ and $g_2$ of
degree $1$,
and then applying the automorphism defined by $g_1\mapsto x$, $g_2\mapsto y$ to
$g$. We
now systematically consider the terms of each degree $3<j<k$. By applying the
transformations in Lemma~\ref{transformationD[k]}, for each $j$, the only term
of total degree $j$ which possibly remains is $a_jy^j$. This term vanishes for
$j<k-1$ and it does not vanish for $j=k-1$, otherwise $g$ is not of complex
type $D_k$. Thus, after applying these transformations,
we can write $g$ as $g=x^2y+\alpha y^{k-1}$ with $a\neq0$. Clearly if
$\alpha>0$ then $x^2y+\alpha y^{k-1}\requiv x^2y+y^{k-1}$ and if $\alpha<0$
then $x^2y+\alpha y^{k-1}\requiv x^2y-y^{k-1}$.

\begin{algorithm}[h]
\caption{\label{alg:D_k} Algorithm for the case $D_k$, $k > 4$}
\begin{algorithmic}[1]

\REQUIRE{$g \in \m^3\subset\mathbb Q[x,y]$ of complex singularity type $D_k$,
$k\in\mathbb N$, $k>4$}

\ENSURE{the real singularity type of $g$, i.e.\@ $D_k^-$ or $D_k^+$}

\STATE $k:= \mu(g)$
\STATE $h:=\jet(g,k-1)$
\STATE factorize $\jet(h,3)$ as $h_1^2h_2$, where $h_1$ and $h_2$ are linear
\STATE apply $h_1\mapsto x$, $h_2\mapsto y$ to $h$
\FOR{$(j = 4, \ldots, k-1)$}
\IF{$(\jet(h,j)-x^2y\neq0)$}
\STATE write $\jet(h,j)-x^2y$ as
$a_0x^j+a_1x^{j-1}y+\cdots +a_jy^j,\quad a_0,\ldots a_j\in\mathbb Q$
\STATE apply $x\mapsto x-\frac{1}{2}(a_1x^{j-2}+\cdots
+a_{j-1}y^{j-2})$, $y\mapsto y-a_0x^{j-2}$ to $h$
\STATE $h:=\jet(h,k-1)$
\ENDIF
\ENDFOR
\STATE write $h$ as $h=x^2y+\alpha y^{k-1}$, $0\neq\alpha\in\mathbb Q$
\IF{$(\alpha>0)$}
\RETURN $D_k^+$
\ELSE
\RETURN $D_k^-$
\ENDIF

\end{algorithmic}
\end{algorithm}

\subsection{$\boldsymbol{E_6}$}

In this case, whose complex normal form is $x^3+y^4$, we have that either
$g\requiv x^3+y^4$ ($E_6^+$) or $g\requiv x^3-y^4$ ($E_6^-$).
Therefore there exists an $\mathbb
R$-algebra automorphism $\phi$ of $\mathbb R[x,y]$ such that
$\phi(g)=(\phi(x))^3+(\phi(y))^4$ or such that
$\phi(g)=(\phi(x))^3-(\phi(y))^4$.
Since the coefficients of $x^3$ and $y^3$ in $g$ cannot both be zero, we can
ensure that the coefficient of
$x^3$ is non-zero by swapping the variables if necessary. Now, using Lemma
\ref{kjet} and Lemma~\ref{x^3}, $\jet(g,3)$ factorizes as
$c(g_1)^3$, $c\in\mathbb Q$ and $g_1=b_0x+b_1y\in\mathbb
Q[x,y]$. Again,
using Lemma \ref{kjet}, it follows that by applying
$x\mapsto\frac{x-b_1y}{b_0},\
y\mapsto y$ to $g$, $\phi$ is transformed such that $\phi_1(x)=c'x$,
$c'\in\mathbb
R$. Since $\phi$ is an automorphism, $\phi_1(y)=d_0x+d_1y$,
$d_0,d_1\in\mathbb R$,
with $d_1\neq 0$. Hence
\begin{equation*}
(\phi(y))^4=d_1^4y^4+\textnormal{terms of degree 4 and higher, not of the
form $\alpha y^4$, $\alpha\in\mathbb R$.}
\end{equation*}
If we can show that $(\phi(x))^3$ does not contain a term of the form
$\alpha y^4$, $\alpha\in\mathbb R$, then we can determine whether $g$
is of type $E_6^-$ or $E_6^+$ by considering the sign of the coefficient of
the monomial $y^4$. Now
\begin{eqnarray*}
\textnormal{jet}((\phi(x))^3,4)-\textnormal{jet}((\phi(x))^3,3)
&=&3(\phi_1(x)^2)(\phi_2(x)-\phi_1(x))\\
&=&3(c'x)^2(\phi_2(x)-\phi_1(x)),
\end{eqnarray*}
 which means that $(\phi(x))^3$ does not have terms of the form $\alpha y^4$,
 $\alpha\in\mathbb R$ .

\begin{algorithm}[h]
\caption{\label{alg:E_6}\label{E[6]} Algorithm for the case $E_6$}
\begin{algorithmic}[1]

\REQUIRE{$g\in \m^3\subset\mathbb Q[x,y]$ of complex singularity type $E_6$}
\ENSURE{the real singularity type of $g$, i.e.\@ $E_6^-$ or $E_6^+$}
\STATE $h:= \jet(g,3)$
\STATE $s:=$ coefficient of ${x^3}$ in $h$
\IF{$(s=0)$}
\STATE swap the variables $x$ and $y$
\ENDIF
\STATE factorize $h$ into linear factors over $\mathbb Q[x,y]$, with a factor
$g_1=b_0x+b_1y$
\STATE apply $x\mapsto \frac{x-b_1y}{b_0}$, $y\mapsto y$ to $g$
\STATE $d :=$ coefficient of $y^4$ in $g$
\IF{$(d>0)$}
\RETURN $E_6^+$
\ELSE
\RETURN $E_6^-$
\ENDIF

\end{algorithmic}
\end{algorithm}


 \begin{thebibliography}{99}
\bibitem{AVG1985} Arnold, V.I.; Gusein-Zade, S.M.; Varchenko, A.N.:
Singularities of Differential Maps. Vol.~I, Birkh\"auser (1985).
\bibitem{A1975} Arnold, V.I.:
\textit{Normal form of functions near degenerate critical points.},
Russian Mth. Surveys 29 ii (1975), 10-50.
\bibitem{DGPS}
Decker, W.; Greuel, G.-M.; Pfister, G.; Sch{\"o}nemann, H.:
\newblock {\sc Singular} {3-1-5} --- {A} computer algebra system for polynomial
computations.
\newblock {http://www.singular.uni-kl.de} (2012).
\bibitem{Kruger} Kr\"uger, K.: Klassifikation von
Hyperfl\"agensingularit\"aten, Diploma Thesis (1997).
\bibitem{GLS2007}Greuel, G.-M.; Lossen, C.; Shustin E.:
Introduction to Singularities and Deformations, Springer, Berlin (2007).
\bibitem{GP2008} Greuel G.-M.; Pfister G.;
A Singular introduction to Commutative Algebra, 2nd Ed., Springer,
Berlin (2008).
\bibitem{classify}
Kr\"uger, K.:
{\tt classify.lib}. {A} {\sc Singular} {3-1-5} library for classifying isolated
hypersurface singularities w.r.t.\@ right equivalence, based on the
determinator of singularities by V.I. Arnold (2012).
\bibitem{realclassify}
Marais, M. and Steenpass, A.:
{\tt realclassify.lib}. {A} {\sc Singular} {3-1-5} library for classifying
isolated hypersurface singularities over the reals w.r.t.\@ right equivalence,
based on the determinator of singularities by V.I. Arnold. This library is
based on classify.lib by Kai Kr\"uger, but handles the real case, while
classify.lib does the complex classification (2012).
\bibitem{primdec.lib} Pfister, G.; Decker, W.;  Schoenemann, H.; Laplagne, S.:
{\tt primdec.lib}. {A} {\sc Singular} {3-1-5} library for Primary Decomposition
and Radical of Ideals (2012).
\bibitem{Siersma} Siersma D.: Classification and deformation of Singularities,
disertation, University of Amsterdam (1974).
\bibitem{roots}
Tobis, A.:
{\tt rootsur.lib}. {A} {\sc Singular} {3-1-5} library for Counting number of
real roots of univariate polynomial (2012).
\bibitem{solve.lib} Wenk, M.: {\tt solve.lib}. Pohl, W.:
{A} {\sc Singular} {3-1-5} library for Complex Solving of Polynomial Systems
(2012).

\end{thebibliography}

\end{document}
