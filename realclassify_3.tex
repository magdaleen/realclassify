\documentclass[noend]{amsproc}
\linespread{1}
\usepackage{amsthm,amsmath,amsfonts,mathrsfs,amssymb}
\usepackage{algorithm}
\usepackage{algorithmic}
\usepackage{multirow}

\newtheorem{theorem}{Theorem}
\newtheorem{defn}[theorem]{Definition}
\newtheorem{lemma}[theorem]{Lemma}
\theoremstyle{definition}
\newtheorem{remark}[theorem]{Remark}

% ALGORITHM style
%%%%%%%%%%%%%%%%%%%%%%%%%%%%%%%%%%%
\renewcommand{\algorithmicrequire}{\textbf{Input:}}
\renewcommand{\algorithmicensure}{\textbf{Output:}}

\newcommand{\Singular}{\textsc{Singular}}
\newcommand{\realclassify}{\texttt{realclassify.lib}}
\newcommand{\NF}[1]{\operatorname{NF}(#1)}

\DeclareMathOperator{\ord}{ord}
\DeclareMathOperator{\requiv}{\overset{r}{\sim}}
\DeclareMathOperator{\m}{\mathfrak{m}}
\DeclareMathOperator{\jt}{jet}
\DeclareMathOperator{\supp}{supp}
\DeclareMathOperator{\sign}{sign}
\DeclareMathOperator{\N}{\mathbb{N}}
\DeclareMathOperator{\Z}{\mathbb{Z}}
\DeclareMathOperator{\R}{\mathbb{R}}
\DeclareMathOperator{\C}{\mathbb{C}}
\DeclareMathOperator{\K}{\mathbb{K}}
\DeclareMathOperator{\T}{T}
\DeclareMathOperator{\Aut}{Aut}

\title{The classification of real singularities using \textsc{Singular}}

\author{Magdaleen S. Marais}
\address{Magdaleen S. Marais\\
African Institute for Mathematical Sciences and Stellenbosch University\\
6 Melrose Rd\\
Muizenberg 7945, Cape Town\\
South Africa}
\email{magdaleen@aims.ac.za}

\author{Andreas Steenpa\ss}
\address{Andreas Steenpa\ss\\
Department of Mathematics\\
University of Kaiserslautern\\
Erwin-Schr\"odinger-Str.\\
67663 Kaiserslautern\\
Germany}
\email{steenpass@mathematik.uni-kl.de}

\thanks{ }
\subjclass[2000]{}
\keywords{}
\begin{document}
\begin{abstract}
The algorithms implemented in the library ``realclassify.lib" in
\textsc{singular} are discussed in this paper. The purpose of this library is
to classify the~$0$ and $1$ modal isolated hypersurface singularities at $0$ of
corank $0$, $1$ and $2$ over the real numbers as computed by V.I.~Arnold in
\cite{AVG1985}.
\end{abstract}
\maketitle
\section{Introduction}
The goal of this paper is to present the algorithms that were implemented in a
library, ``realclassify.lib'' \cite{realclassify} in \textsc{singular}, which
classifies the $0$ and $1$ modal isolated hypersurface singularities at $0$ of
corank $0$, $1$ and $2$ over the real numbers $\mathbb R$, using
right-equivalence, as was computed by V.I.~Arnold in \cite{AVG1985}. Hence we
consider smooth real functions with critical point $0$ and critical value $0$,
i.e.~functions in $\m^2$, where $\m$ denotes the ideal of function germs
vanishing at the origin. Two function germs $f, g\in\mathbb R[[x_1,\ldots,
x_n]]$, $f,g\in \m^2$, where $\mathbb R[[x_1,\ldots,x_n]]$ is the power series
ring in coordinates $x_1,\ldots,x_n$ over $\mathbb R$, are considered as
right-equivalent, denoted by $f\requiv g$, if there exists an $\mathbb
R$-algebra automorphism $\phi$ of $\mathbb R[[x_1,\ldots, x_n]]$ such that
$\phi(f)=g$. Because the only possible non-approximated input power series over
the real numbers in computer systems, and thus in particularly in
\textsc{singular}, are polynomials with rational coefficients, some of the
algorithms we used apply only to such input polynomials.  There already is a
\textsc{Singular} library ``classify.lib'' \cite{classify}, which for a given
polynomial computes its type in Arnold's classification over the complex
numbers. We used this library as a basis to build ``realclassify.lib'' on. The
methods used in ``classify.lib'' will not be discussed in this paper. For more
information in this regard \cite{Kruger} can be studied.

Using the Splitting Lemma (Lemma~\ref{SplittingLemma}) any function germ $f$
over the real numbers with an isolated singularity at $0$, using a suitable
coordinate system, can be written as the sum of two functions of which the
variables do not coincide. One of the  functions, called the nondegenerate part
of $f$, is a nondegenerate quadratic form and the other function, called the
residual part of $f$, is an element of $\m^3$. The number of variables in the
residual part is equal to the corank, denoted by $c$, of $f$. In this paper we
will only consider germs with corank $0$, $1$ and $2$.

The algorithm that we used to implement the Splitting Lemma is discussed in
section \ref{TheSplittingLemma}.

In \cite{AVG1985} Arnold divided the real singularities of modality $0$ and $1$
up to stable equivalence into main types
which split up into more subtypes by changing the sign in front of certain
terms. Two functions are stably equivalent if they are right-equivalent after
the direct addition of nondegenerte quadratic forms. Hence, for all
nonquadratic forms $f$, after applying the Splitting Lemma we only need to
consider the residual part of $f$ to complete the classification of $f$.   It
can be easily seen that the subtypes are complex equivalent to a
complex singularity type of the same name as its corresponding real main
singularity type (see Table \ref{normal forms}). In fact there is a bijection
between the complex singularity
types of modality $0$ and $1$ and the main real singularity types. Thus, if we
can determine the complex singularity type of a function germ in these cases,
we only need to consider the subtypes of the
corresponding
main real singularity type to determine the real type of the function germ.

We denote the $k$-jet of a function germ $f$ by $\jt(f,k)$. In section
\ref{ResultsRegardingTheFactorizationOfHomogeneousPolynomialsOverRAndQ} we
discuss the factorization of the lowest nonzero jets of two right-equivalent
functions. These results are used in sections
\ref{RealSingularitiesOfZeroModality} to \ref{ExceptionalSingularities}, where
the algorithms are given to classify germs in $\m^3$. Firstly, we show that the
$k$-jets of two right-equivalent functions, where $k$ is the order of the
functions, factorize in the same way over $\mathbb R$ (Lemma~\ref{kjet}). Since
the real numbers are floating numbers in \textsc{Singular}, it is not always
possible to factorize over the real numbers using \textsc{Singular}. In some
cases we prove that these lowest jets will always factorize in linear terms
over the rational numbers and thus in \textsc{Singular}, for example when the
lowest jet is of degree $k$ and has $k$ similar factors (Lemma \ref{x^3}). In
some other cases this unfortunately is not true. Since we can distinguish
between some singularities, by only considering the number of real roots of its
lowest nonzero jet, the similar factorization of the lowest nonzero jets is
still useful in these cases. Because the lowest nonzero jet of a function germ
is a homogeneous polynomial, we use dehomogenization and the library
``rootsur.lib" \cite{roots}, a library counting the real roots of a univariate
polynomial in \textsc{Singular}, to count the roots. We for instance use this
method in the classification of the $D_4$ (Algorithm~\ref{D[4]}) and $E_6$
(Algorithm \ref{E[6]}) cases.

Sometimes other methods are used in addition to the above methods, for example
the normal form of singularities of type $E_{14}$ is determined by
systematically computing higher jets of the $\mathbb R$-algebra automorphism
between the normal form of the function and the function itself considering the
terms of these functions, after the lowest jet is determined using the above
methods (Algorithm \ref{E[14]}).

In the cases $J_{10}$ and $J_{10+k}$, $k>0$ we used different normal forms than
Arnold. These forms are more suitable to work with over the real numbers and
specifically simplify the computations to classify these cases over the real
numbers. We include a list of the normal forms we used at the beginning of
section \ref{TheRealClassificationOfTheResidualPart}.

Lastly, we used the method of blowing up the origin (Algorithm
\ref{BlowingUp}), which simplifies singularities. In the case $Y_{r,s}$,
$r,s>4$ (Algorithm \ref{Y[r,s]}) enough information can be subtracted from the
resulting, simplified, singularity such that the case can be solved, in some
specific cases, unfortunately using approximations.

\newpage


\section{The real classification of the residual part of $f$}%
\label{TheRealClassificationOfTheResidualPart}

In \cite{AVG1985} Arnold divided the real singularities of modality $0$ and
$1$, using stable equivalence, into main types which split up into more
subtypes by changing the sign in front of certain terms. Each of these subtypes
is complex equivalent to the complex singularity of the same name as its
corresponding real main singularity type. We include the complex
transformations for the corank two singularities in the table below. Since
there is a real main singularity type of modality $0$ or $1$ for each complex
singularity type of modality $0$ or $1$, respectively, there is a bijection
between the main real singularity types of modality $0$ and $1$ and the complex
singularity types of modality $0$ and $1$, respectively. It is not known yet
whether the complex forms of higher modal cases  split up into corresponding
real forms. The fact is that it is not known whether modality is preserved in
these cases. In the next table we list the normal forms that are used in this
article. From here onwards we will be working with stable equivalence. For all
nonquadratic forms $f$ it is thus only necessary, after applying the Splitting
Lemma to consider the residual part of function germs, i.e.~germs in $\m^3$.

\begin{table}[!hbp]
\label{tab:normal_forms}
\centering
\caption{Real Normal Forms of modality $0$ and $1$.}
\label{normal forms}
\begin{tabular}{|c|c|c|c|c|}
\hline
&Complex NF & NF of real subtypes& Transformation&Restrictions\\\hline
$A_k$&$x^{k+1}$&$x^{k+1}$&$x\mapsto x$, $y\mapsto y$&$k\ge 1$\\
&&$-x^{k+1}$&$x\mapsto {i}^{\frac{2}{k+1}}x$, $y\mapsto y$&$k\ge 1$\\
\hline
$D_k$&$x^2y+y^{k-1}$&$x^2y+y^{k-1}$&$x\mapsto x$, $y\mapsto y$&$k\ge 4$\\
&&$x^2y-y^{k-1}$&$x\mapsto i^{\frac{2k-3}{k-1}}x$,
$y\mapsto {i}^{\frac{2}{k-1}}y$&$k\ge 4$\\
\hline
$E_6$&$x^3+y^4$&$x^3+y^4$&$x\mapsto x$, $y\mapsto y$&-\\
&&$x^3-y^4$&$x\mapsto x$, $y\mapsto\sqrt{i} y$&-\\
\hline
$E_7$&$x^3+xy^3$&$x^3+xy^3$&$x\mapsto x$, $y\mapsto y$&-\\
\hline
$E_8$&$x^3+y^5$&$x^3+y^5$&$x\mapsto x$, $y\mapsto y$&-\\
\hline
$X_9$&$x^4+ax^2y^2+y^4$&$x^4+ax^2y^2+y^4$&$x\mapsto x$,
$y\mapsto y$&$a^2\neq 4$\\
&&$x^4+ax^2y^2-y^4$&$x\mapsto x$, $y\mapsto \sqrt{i}y$&-\\
&&$-x^4+ax^2y^2+y^4$&$x\mapsto\sqrt{i}x$, $y\mapsto y$&-\\
&&$-x^4+ax^2y^2-y^4$&$x\mapsto\sqrt{i}x$, $y\mapsto \sqrt{i}y$&$a^2\neq 4$\\
\hline
$J_{10}$&$x^3+ax^2y^2+y^6$&$x^3+ax^2y^2+y^6$&$x\mapsto x$,
$y\mapsto y$& $4a^3+27\neq0$\\
&&$x^3+ax^2y^2-y^6$&$x\mapsto x$, $y\mapsto \sqrt[3]{i}y$&$-4a^3+27\neq0$\\
\hline
$J_{10+k}$&$x^3+xy^4+ay^6$&$x^3+xy^4+ay^6$&$x\mapsto x$,
$y\mapsto y$&$a\neq 0$, $k>0$\\
&&$x^3-xy^4+ay^6$&$x\mapsto x$, $y\mapsto \sqrt{i}y$&$a\neq 0$, $k>0$\\
\hline
$X_{9+k}$&$x^4+x^2y^2+ay^{4+k}$&$x^4+x^2y^2+ay^{4+k}$&$x\mapsto x$,
$y\mapsto y$&$a\neq 0$, $k>0$\\
&&$x^4-x^2y^2+ay^{4+k}$&$x\mapsto x$, $y\mapsto iy$&$a\neq 0$, $k>0$\\
&&$-x^4+x^2y^2+ay^{4+k}$&$x\mapsto \sqrt{i}x$, $y\mapsto i^{\frac{3}{2}}y$
&$a\neq 0$, $k>0$\\
&&$-x^4-x^2y^2+ay^{4+k}$&$x\mapsto \sqrt{i}x$, $y\mapsto \sqrt{i}y$&$a\neq 0$,
$k>0$\\
\hline
$Y_{r,s}$&$x^2y^2+x^r+ay^s$&$x^2y^2+x^r+ay^s$&$x\mapsto x$, $y\mapsto y$
&$a\neq 0$, $r,s>4$\\
&&$x^2y^2-x^r+ay^s$&$x\mapsto i^{\frac{2}{r}}x$, $y\mapsto i^{\frac{2r-2}{r}}y$
&$a\neq 0$, $r,s>4$\\
&&$-x^2y^2+x^r+ay^s$&$x\mapsto x$, $y\mapsto iy$&$a\neq 0$, $r,s>4$\\
&&$-x^2y^2-x^r+ay^s$&$x\mapsto i^{\frac{2}{r}}x$, $y\mapsto i^{\frac{r-2}{r}}y$
&$a\neq 0$, $r,s>4$\\
\hline
$E_{12}$&$x^3+y^7+axy^5$&$x^3+y^7+axy^5$&$x\mapsto x$, $y\mapsto y$&-\\
\hline
$E_{13}$&$x^3+xy^5+ay^8$&$x^3+xy^5+ay^8$&$x\mapsto x$, $y\mapsto y$&-\\
\hline
$E_{14}$&$x^3+y^8+axy^6$&$x^3+y^8+axy^6$&$x\mapsto x$, $y\mapsto y$&-\\
&&$x^3-y^8+axy^6$&$x\mapsto x$, $y\mapsto \sqrt[4]iy$&-\\
\hline
$Z_{11}$&$x^3y+y^5+axy^4$&$x^3+y^5+axy^4$&$x\mapsto x$, $y\mapsto y$&-\\
\hline
$Z_{12}$&$x^3y+xy^4+ax^2y^3$&$x^3y+xy^4+ax^2y^3$&$x\mapsto x$, $y\mapsto y$&-\\
\hline
$Z_{13}$&$x^3y+y^6+axy^5$&$x^3y+y^6+axy^5$&$x\mapsto x$, $y\mapsto y$& -\\
&&$x^3y-y^6+axy^5$&$x\mapsto i^{\frac{11}{9}}x$, $y\mapsto \sqrt[3]i y$& -\\
\hline
$W_{12}$&$x^4+y^5+ax^2y^3$&$x^4+y^5+ax^2y^3$&$x\mapsto x$, $y\mapsto y$&-\\
&&$-x^4+y^5+ax^2y^3$&$x\mapsto\sqrt{i} x$, $y\mapsto y$&-\\
\hline
$W_{13}$&$x^4+xy^4+ay^6$&$x^4+xy^4+ay^6$&$x\mapsto x$, $y\mapsto y$&-\\
&&$-x^4+xy^4+ay^6$&$x\mapsto \sqrt{i}x$, $y\mapsto i^{\frac{7}{8}}y$&-\\
\hline
\end{tabular}
\end{table}

It is known that the milnor number of a function germ $g$, which we denote by
$\mu(g)$, is the same regardless whether we work over the complex- or real
numbers. Therefore a singularity over the complex numbers is nondegenerate if
and only if the singularity is nondegenerate over the real numbers. The
restrictions to normal forms in the complex case are thus the same in the real
case. Furthermore we know that we can transform the normal forms in the real
case that differ from those we use in the complex case by a complex
transformation to a complex normal form. Since the milnor number is invariant
under complex transformations, we can then determine the restrictions. Let us
take the $X_9$ case as an example and consider the real normal form
$-x^4+ax^2y^2+y^4$. This normal form is complex equivalent to
$x^4+aix^2y^2+y^4$, which corresponds to the given corresponding complex normal
form. Since $a\in\mathbb R$, it follows that $(ia)^2=-a^2\neq4$ and therefore
this singularity is nondegenerate for all values of $a$.

Using the $1-1$ correspondence between the main real singularity types and
complex singularity types, and a \textsc{Singular} library ``classify.lib"
\cite{classify}, that classifies complex singularities, classifying a real germ
boils down to determining to which of the corresponding subtypes the germ is
equivalent.


\subsection{Real parabolic $\mathbf 1$-modal singularities of corank $\mathbf 2$}
Given a vector $w=(w_1,w_2)$ of integers we define the weighted degree of
$x^\alpha y^\beta$ by \[w-\deg(x^\alpha y^\beta):=w_1\alpha+w_2\beta.\]

\subsubsection{$X_9:$}\label{X_9}
For the case $X_9$, the normal form given by Arnold in \cite{AVG1985} is
\[
\pm x^4 +a x^2 y^2 \pm y^4 \,, a \in \R \,,
\]
with $a^2 \neq 4$ if the signs of $x^4$ and $y^4$ are equal.

The structure of the real equivalence classes of this main type is rather
involved in comparison to most other cases, e.g.~we have that
$x^4 +3 x^2 y^2 +y^4$ can be transformed into $x^4+\frac{6}{5} x^2 y^2 +y^4$
via $x \mapsto \sqrt[4]{\frac{1}{5}}(x+y)$,
$y \mapsto \sqrt[4]{\frac{1}{5}}(x-y)$ and that $-x^4 +4x^2 y^2 -y^4$ can be
transformed into $x^4 -10 x^2 y^2 +y^4$ via
$x \mapsto \sqrt[4]{\frac{1}{2}}(x+y)$ and
$y \mapsto \sqrt[4]{\frac{1}{2}}(x-y)$. We will first examine this structure in
detail and then give an algorithm to determine the equivalence class of a given
singularity of real main type $X_9$.

If we were in the complex case, the normal form would be
\begin{equation}\label{eqn:X9_cnf}
x^4 +ax^2y^2 +y^4
\end{equation}
with $a \in \C$ and $a^2 \neq 4$. In order to investigate for which values
$a' \in \C$ this can be transformed, for a given $a$, into
$x^4 +a'x^2y^2 +y^4$, we may proceed as follows, e.g. using \Singular:
We first apply a generic coordinate transformation
\[
\varphi: \; x \mapsto rx+sy, \; y \mapsto tx+uy, \; r,s,t,u \in \C
\]
to (\ref{eqn:X9_cnf}). Note that it suffices to consider the 1-jet of $\varphi$
because $X_9$ is 4-determined as can be shown using the highest corner method
(cf. TODO). We can then set up the ideal of transformations which take $a$ to
$a'$. By eliminating $r$, $s$, $t$ and $u$ from this ideal, we get a polynomial
of degree 6 in both $a$ and $a'$ which does not involve any other variables.
Factorizing this polynomial gives the following solutions:
\begin{align*}
a'_1 &= a  & a'_3 &= \frac{2a+12}{a-2} & a'_5 &= \frac{-2a+12}{a+2} \\
a'_2 &= -a & a'_4 &= \frac{2a-12}{a+2} & a'_6 &= \frac{2a+12}{-a+2}.
\end{align*}
Hence the equivalence class of a generic complex singularity of main type $X_9$
can be represented by $x^4 +a'x^2y^2 +y^4$ for six different values of
$a' \in \C$. The non-generic cases are precisely those with
$a' \in \{-6, 0, 6\}$ where some of the above solutions coincide.

TODO: Draw a picture here?

If, as the next step, we restrict ourselves to singularities of main type $X_9$
given by polynomials with real coefficients, but still allow complex
transformations, the picture changes. In this case we have to distinguish
between the subtypes $X_9^{++}$, $X_9^{--}$, $X_9^{+-}$, and $X_9^{-+}$,
referring to the signs of $x^4$ and $y^4$ in the normal form, respectively. By
simply interchanging the variables $x$ and $y$, the last two subtypes can be
considered as one.
The question now is for which $i \in \{1, \ldots, 6\}$ there is a (possibly
complex) transformation that takes
$\pm x^4 +ax^2y^2 \pm y^4$ to $\pm x^4 +a'_i x^2y^2 \pm y^4$, and we may
ask this for each combination of signs.
It turns out that for transformations from $X_9^{++}$ to itself, from
$X_9^{--}$ to itself or from $X_9^{++}$ to $X_9^{--}$ and vice-versa, all the
solutions $a'_1, \ldots, a'_6$ hold valid. For those from $X_9^{+-}$ to
itself (and thus also from $X_9^{-+}$ to itself, from $X_9^{+-}$ to
$X_9^{-+}$
and vice-versa), only $a'_1$ and $a'_2$ remain. For any of the cases from
$X_9^{++}$ or $X_9^{++}$ to $X_9^{+-}$ or $X_9^{+-}$, there is a valid
transformation if and only if the parameter $a$ is in $\{-6, 0, 6\}$, and it
then becomes~$0$.

TODO: Draw a picture here?

If we allow only for real transformations, i.e. automophisms of $\R[[x,y]]$,
we get even more equivalence classes. Then, besides the identity map taking $a$
to $a'_1=a$ and identifying $X_9^{+-}$ with $X_9^{-+}$,
only the following equivalences hold:
\begin{align}
X_9^{++,a} \; &\requiv \; X_9^{++,a'_5} \; \text{ for } a > -2 \\
X_9^{--,a} \; &\requiv \; X_9^{--,a'_3} \; \text{ for } a < 2 \\
X_9^{++,a} \; &\requiv \; X_9^{--,a'_4} \; \text{ for } a < -2 \\
X_9^{--,a} \; &\requiv \; X_9^{++,a'_6} \; \text{ for } a > 2.
\end{align}

TODO: Explain how this can be computed?
TODO: Draw a picture here.

The main tool for determining the subtype of a given real singularity of main
type $X_9$ is blowing-up. Again, it suffices to consider the 4-jet of the
polynomial $f = kx^4 + lx^3y + mx^2y^2 + nxy^3 + oy^4 \in \R[x, y]$ defining
the given singularity. Before the blowing-up, we have to make sure that the
coefficient of $x^4$ is non-zero. If this is not
the case, this can be easily achieved by switching the variables if
$o \neq 0$, or otherwise by applying either $y \mapsto x+y$, $y \mapsto 2x+y$,
or $y \mapsto 3x+y$. These transformations yield $l+m+n$, $2l+4m+8n$, and
$3l+9m+27n$, respectively, as coefficients of $x^4$, which cannot all at once
be zero.

TODO: Say what WE take as normal forms.

Applying the blowing-up map $x \mapsto x \,, y \mapsto 1$ results in a
polynomial $f_{\text{blowup}} \in \R[x]$ of degree 4 which has either four, two
or no real roots. This number is invariant with respect to any transformation
applied to $f$ before the blowing-up. Blowing-up the normal forms shows the
following picture:

\begin{tabular}{l|c}
Subtype & Number of real roots after the blowing-up \\ \hline
$X_9^{++,a} \,, a < -2$ & 4 \\
$X_9^{+-}$ & 2 \\
$X_9^{++,a} \,, a > -2$ & 0 \\
$X_9^{--,a} \,, a < 2$ & 0
\end{tabular}

To distinguish the two cases where the number of real roots is zero, it
suffices to look at the sign of the (now for sure non-zero) coefficient of
$x^4$: It stays invariant under any real transformation whatsoever.

\realclassify{} also tries to determine the parameter $a$ as far as possible,
but in some cases it is not determined uniquely, and several possibilities are
given. The method we use proceeds as follows: Very similar to the complex case,
eliminating the coefficients of a generic coordinate transformation from the
ideal of transformations which take the given polynomial $f$ to its normal form
yields an univariate polynomial $p_f(a)$ of degree 6 in the parameter $a$. The
possible values of $a$ are among the real roots of this polynomial.

In order to
test these roots separately, we try to factorize $p_f$, but there is no
guaranty that it splits into linear factors and there are indeed cases where it
does not. We then employ different means to exclude as many
factors as possible. One method is to verify, for each factor, that it has at
least one real root in the interval which contains $a$ according to the
subtype. Finally, another check can be done by switching over to the
quotient ring defined by each factor. If the value of $a$ is indeed one of the
real roots of a certain factor of $p_f$, then the ideal of generic
transformations which take the given polynomial to its normal form has at least
one real root in this ring, which can in many cases be checked by using the
command \texttt{nrRootsDeterm()} from \Singular{}'s \texttt{rootsmr.lib}.

The complete algorithm to determine the subtype and the parameter of a given
real singularity of main type $X_9$ now reads as follows:


\subsubsection{$J_{10}$:}
For the main real singularity type $J_{10}$ we use the complex normal form
$x^3+axy^4+ y^6$, which splits up into $x^3+axy^4\pm y^6$ in the real case (see
Table \ref{normal forms}),  instead of the complex normal form
$x^3+ax^2y^2+xy^4$, which splits up into $x^3+ax^2y^2\pm xy^4$, used by Arnold
in \cite{AVG1985}. Using this different normal form makes calculations in
\textsc{Singular}, determining the real singularity type easier. Furthermore,
note that, in mapping $x\mapsto -x$, in the real normal forms we used, does not
change the sign of $y^6$, i.e. the term with the parameter $a$ as coefficient, while in Arnold's normal forms it changes the sign of
the term $xy^4$, which makes the normal forms we used in general more suitable
for computations in  the real case.

Now, having a polynomial $g$ over $\mathbb Q$ of complex type $J_{10}$, using
Lemma \ref{kjet} and Lemma \ref{x^3}, we transform $g$ to a polynomial of the
form \[b''x^3+\textnormal{ terms of higher degree,}\] where $b''\in\mathbb Q$.
Since $g$ is right-equivalent to one of the germs $x^3+axy^4\pm y^6$, for some
$a\in\mathbb R$, there exists an $\mathbb R$-algebra automorphism $\phi$ such
that $\phi(x^3+axy^4+y^6)=g$ or such that $\phi( x^3+axy^4-y^6)=g$. Clearly,
$\jt(\phi(x),1)=\sqrt[3]{b''}x$. Hence \[g=b''
x^3+c''x^2y^2+d''xy^4+e''y^6+\textnormal{ terms of higher $w$-degree,}\] where
$b'',c'',d'',e''\in\mathbb{Q}$ and $w=(\frac{1}{3},\frac{1}{6})$. Applying the
map defined by $x\mapsto \frac{1}{\sqrt[3]{b''}}x$ and
$y\mapsto\frac{1}{\sqrt[6]{b''}}y$ transforms $g$
to\[g=x^3+c'x^2y^2+d'xy^4+e'y^6+\textnormal{ terms of higher $w$-degree,}\]
where $c'=\frac{c''}{b''}\in\mathbb{Q}$, $d'=\frac{d''}{b''}\in\mathbb{Q}$ and
$e'=\frac{e''}{b''}\in\mathbb{Q}$. (Since it is only necessary to consider the
weighted 1-jet of $g$ from here onwards, as is clear from the rest of the
arguments, we may instead of the above transformation, transform $g$ by
multiplication with the scalar $\frac{1}{b''}$). We get rid of the term
$c'x^2y^2$ by applying te map defined by $x\mapsto x-\frac{c'}{3}y^2$ and
$y\mapsto y$, i.e.
\[g=x^3+dxy^4+ey^6+g_1,\]where $d,e\in\mathbb Q$ and $w-\deg(g_1)>1$. Note that
after the above transformations $\phi_1(x)=x$. Hence it follows that
$w-\deg(\phi(g_1))>1$. Therefore the singularity type of $g$ is determined by
the sign of $e$.

\begin{theorem}(Algorithm for the case $J_{10}$)
\end{theorem}
\noindent\textnormal{\bf Input:} $g\in \m^3\subset\mathbb Q[x,y]$ of complex
singularity type $J_{10}$.\newline
\textnormal{\bf Output:} the real singularity type of $g$, i.e.~$J_{10}^-$ or
$J_{10}^+$.
\begin{itemize}
\item $h:= \jt(g,3)$;
\item $s_1:=$ coefficient of $x^3$;
\item $s_2:=$ coefficient of $y^3$;
\item if ($s_1=0$)\newline
\phantom{}\quad\quad Swap the variables $x$ and $y$;
\item factorize $h$ over $\mathbb Q[x,y]$, with a factor $t_1x+t_2y$;
\item apply (to $g$) $x\mapsto\frac{x-t_2y}{t_1}$, $y\mapsto y$;
\item write $g$ as
$b''x^3+\textnormal{terms of higher degree},\quad b''\in\mathbb Q$;
\item if $(b'' <0)$\newline
\phantom{}\quad $x\mapsto -x$, $y\mapsto y$;
\item $g:= \frac{1}{\mid b''\mid} g$;
\item write $g$ as $x^3+c'x^2y^2+d'xy^4+e'y^6+$ terms of higher $w$-degree,
where $w=(\frac{1}{3},\frac{1}{6})$ and $c',d',e'\in\mathbb Q$;
\item Apply (to $g$) $x\mapsto x-\frac{c'}{3}y^2$, $y\mapsto y$;
\item write $g$ as $x^3+dxy^4+ey^6+$ terms of higher $w$-degree',\quad
$c,e\in\mathbb Q$.
\item if $(e>0)$\newline
\phantom{}\quad  type of singularity $:=J_{10}+$;\newline
\phantom{}else\newline
\phantom{}\quad type of singularity $:=J_{10}-$;
\item return type of singularity;
\end{itemize}

\subsection{Real hyperbolic $\mathbf 1$-modal singularities of corank $\mathbf 2$}
\subsubsection{$J_{10+k}$:}In the $J_{10+k}$ $k>0$ cases, similar to the $J_{10}$ case, we have used
different normal forms than Arnold in \cite{AVG1985}.
Here we have used the complex normal form $x^3+xy^4+a y^{6+k}$, instead of the
normal form $x^3+x^2y^2+ay^{6+k}$. Since the normal form $x^3+xy^4+a y^{6+k}$
does not split up in the real case, it is much more suitable than the normal
form $x^3+x^2y^2+ay^{6+k}$ that splits up in $x^3\pm x^2y^2+ay^{6+k}$ in the
real case. Using this normal form we thus only need the complex classification
in this case.

\subsubsection{$X_{9+k}$:}Considering the case when $g$ is of complex type $X_{9+k}$, we know that $g$ is
of the
form
\[
g=a_0x^4+a_1x^3y+a_2x^2y^2+a_3xy^3+a_4y^4+\textnormal{ terms of higher degree,}
\]
$a_0,\ldots,a_4\in\mathbb Q$.
Similar to the $X_9$-case , Section \ref{X_9}, it is possible to transform $g$ such that the coefficient of $x_4$ is non-zero. Hence we may assume that $a_0\neq 0$.

Now, $g\requiv h$, where $h=x^4+x^2y^2+ay^{4+k}$ if $g$ is of type
$X_{9+k}^{++}$, $h=-x^4-x^2y^2+ay^{4+k}$ if $g$ is of type $X_{9+k}^{--}$,
$h=-x^4+x^2y^2+ay^{4+k}$ if $g$ is of type $X_{9+k}^{-+}$ and
$h=x^4-x^2y^2+ay^{4+k}$ if $g$ is of type $X_{9+k}^{+-}$, for some
$0\neq a\in\mathbb R$.  It follows
from Lemma~\ref{kjet} that $\textnormal{jet}(g,4)$ factor as
$\pm\phi_1(x)^2(\phi_1(x)^2+\phi_1(y)^2)$ in the first two cases and as
$\pm\phi_1(x)^2(\phi_1(x)-\phi_1(y))(\phi_1(x)+\phi_1(y))$ in the second two
cases, where $\phi$ is the $\mathbb R$-algebra automorphism such that
$\phi(g)=h$. Since $a_0\neq 0$,  the dehomogenization $\jt(g,1)(x,1)$ has one
root in the first two
cases and
three roots in the last two cases.

Since the sign of the terms in the $4$-jet of $h$ in the first two cases
does not differ and the power of both $x$ and $y$ in all the terms is even,
we only need to consider the sign of the $x^4$-term to distinguish between
the $X_{9+k}^{++}$ and $X_{9+k}^{--}$ cases.

If $\jt(g,1)$ has three roots, we transform $g$ by using Lemma \ref{kjet}
and Lemma~\ref{x^3} such that $\phi_1(x)=\alpha x$, $\alpha\in\mathbb R$. We therefore may assume that
$\jt(g,1)$ is of the form
$\jt(g,1)=a_0x^4+a_1x^3y+a_2x^2y^2=\pm(\alpha x)^2((\alpha
x)^2\mp\phi_1(y)^2)=\pm(\alpha x)^4\mp (\alpha x)^2\phi_1(y)^2$,
$a_0,a_1,a_2\in\mathbb Q$. Since
$\phi$ is
an automorphism, $\phi_1(y)=\beta x+\gamma y$ with $\gamma \neq 0$,
and since the sign of $\gamma$ does not influence the sign of $(\alpha
x)^2\phi_1(y)^2$, and we know that the sign in front of $(\alpha x)^4$
and $(\alpha x)^2\phi_1(y)^2$ differ, we decide between the cases
$X_{9+k}^{+-}$ and $X_{9+k}^{-+}$ by considering the sign of the term $x^2y^2$.

\begin{algorithm}[h]
\caption{Algorithm for the case $X_{9+k}$\label{X[9+k]}}
\begin{algorithmic}[1]
\REQUIRE{ $g\in \m^3\subset\mathbb Q[x,y]$ of complex
singularity type $X_{9+k}$}
\ENSURE{the real singularity type of $g$, i.e.~$X_{9+k}^-$
or $X_{9+k}^+$.}
\STATE $g:=\jt(g,4)$
\STATE $s_1:=$ coefficient of ${x^4}$
\STATE $s_2:=$ coefficient of ${y^4}$
\IF{$s_2\neq0$ and $s_1=0$}
\STATE Swap the variables $x$ and $y$
\ENDIF
\IF{$s_2\neq0$ and $s_1=0$}
\STATE $t_1:=$ coefficient of ${x^3y}$
\STATE $t_2:=$ coefficient of ${x^2y^2}$
\STATE $t_3:=$ coefficient of ${xy^3}$
\IF{$t_1+t_2+t_3=0$}
\IF{$2t_1+4t_2+8t_3\neq0$}
\STATE Apply $x\mapsto x$, $y\mapsto 2y$ (to $g$)
\ELSE 
\STATE Apply $x\mapsto x$, $y\mapsto 3y$ (to $g$)
\ENDIF
\ENDIF
\STATE Apply $x\mapsto x$, $y\mapsto x+y$ (to $g$)
\ENDIF
\STATE Write $g$ as $g=a_0x^4+a_1x^3y+a_2x^2y^2+a_3xy^3+a_4y^4,\quad
a_0,\ldots,a_4\in\mathbb Q$
\STATE $g':=g(x,1)=a_0x^4+a_1x^3+a_2x^2+a_3x+a_4$
\STATE $n:=\#$ real roots of $g'$
\IF{$n=1$}
\IF{$a_0>0$}
\RETURN $X_{9+k}^{++}$
\ELSE
\RETURN $X_{9+k}^{--}$
\ENDIF
\ENDIF
\IF{$n=3$}
\STATE Factorize $g$ in linear factors over $\mathbb Q[x,y]$,
with $g_1:=t_1x+t_2y$ the factor with multiplicity $2$
\STATE Apply $x\mapsto\frac{x+t_2y}{t_1}$, $y\mapsto y$ (to $g$)
\STATE Write $g$ as $g=a_0x^4+a_1x^3y+a_2x^2y^2,\quad
a_0,a_1,a_2\in\mathbb Q$
\IF{$a_2>0$}
\RETURN $X_{9+k}^{-+}$
\ELSE
\RETURN $X_{9+k}^{+-}$
\ENDIF
\ENDIF

\end{algorithmic}
\end{algorithm}


\subsubsection{$Y_{r,s}$ and $\widetilde Y_{r,s}$} Next we consider the case when $g$ is of one of the following complex cases:
$Y_{r,s}$ and $\widetilde Y_r$, $r,s>4$. Note that in the complex case, normal
forms of type $\widetilde Y_r$ are right-equivalent to normal forms of type
$Y_{r,s}$. This is unfortunately not true in the real case and we hence have to
treat these cases separately.  Similar to the $D_4$ case, we distinguish
between the two cases, using Lemma \ref{kjet} and the library {\tt rootsur.lib},
noticing that the $4$-jet of polynomials of type $Y_{r,s}$ factorize into four
linear factors, while the $4$-jet of polynomials of type $\widetilde Y_r$ do
not have any linear factors.

For the case when $g$ is of type $\widetilde Y_r$, for some $r$ we have to
distinguish between the following normal forms in the real case
$\pm(x^2+y^2)^2+ax^r$ and determine the value of $r$. Using the fact that the
$4$-jet $(x^2+y^2)^2$ is always positive, substituting any real numbers for $x$
and $y$,  we distinguish between the two cases by considering the sign of the coefficient of $x^4$, after transforming the input polynomial to a polynomial of which the relating coefficient is nonzero. Such a transformation is explained in Section \ref{X_9}. It is
already known (see~\cite{AVG1985}) that $r=\frac{\mu-9}{2}+4$.

Considering the cases when $g$ is of type $Y_{r,s}$, for some $r$ and $s$,
$r,s>4$, we have to distinguish between the normal forms $\pm x^2y^2\pm
x^r+ay^s$ and determine the values of $r$ and $s$.  We determine the sign in
front of the monomial $x^2y^2$ in the same way as determining the sign in front
of the monomial $(x^2+y^2)^2$ in the case $\widetilde Y_r$.

For the determining of $r$ and $s$ we use the method of blowing up the origin.
By blowing up the origin once, polynomials right-equivalent to one of the
normal forms $\pm x^2y^2\pm x^r+ay^s$ transform to two different germs of which
one is right-equivalent to one of the germs $\pm x^2\pm x^{r-4}+ax^{s-4}y^s$
and one is right-equivalent to one of the germs  $\pm x^2\pm
x^{s-4}+ax^{r-4}y^r$, for some $0\neq a\in \mathbb R$, if $r$ and $s$ are
different, and to one germ which is right-equivalent to one of the germs $\pm
x^2\pm x^{r-4}+ax^{r-4}y^r$, for some $0\neq a\in\mathbb R$, if  $r=s$. Since
the milnor number of $f_1=\pm x^2\pm x^{s-4}+ax^{r-4}y^r$ is $s-5$ and the
milnor number of $f_2=\pm x^2\pm x^{r-4}+ax^{s-4}y^s$  is $r-5$, if $r,s\neq
4$, we determine $r$ and $s$ by calculating the milnor numbers of the resulting
germs. Now, if necessary swap $x$ and $y$ such that $r\le s$. If $r$ is uneven,
then $x^2y^2+x^{r}+ay^s\requiv x^2y^2- x^{r}+ay^s$ and
$-x^2y^2+x^{r}+ay^s\requiv -x^2y^2- x^{r}+ay^s$. When $r=6$ the sign of the
monomial $x^r$ in the normal form of $g$ is negative if the inertia index of
$f_1$, after applying the Splitting Lemma, is greater than zero, if the sign of
$y^2$ in $f_1$ is positive, and is greater than one if the sign of $y^2$ in
$f_1$ is negative. Otherwise the sign of $x^r$ in the normal form of $g$ is
positive. If $r$ is even and $r\neq 6$ we determine the sign in front of the
monomial $x^r$, similar to previous calculations, considering the sign of $x^{r-4}$
after transforming the residual part to a germ of which the coefficient of $x^{r-4}$ is non-zero.

We implemented Algorithm \ref{BlowingUp} that blows an input polynomial $g$ in
$\mathbb Q[x_1,x_2]$ of complex type $Y_{r,s}$ up at the origin and gives as
output a list $L$, containing the following entries. $L[1]$ and $L[2]$: the
values of $r$ and $s$, $r\le s$, respectively; $L[3]$: the inertia index of the
resulting germ, after blowing up $g$, with lowest milnor number, if this
resulting germ have a singularity on the exceptional divisor, i.e. $L[1]\neq
5$, and $0$ otherwise; $L[4]$: the sign of the monomial $x_1^r$ or $x_2^r$ in
the normal form of the resulting singularity with lowest milnor number; $L[5]$
the value $1$ if the resulting germs have singularities that occur at
irrational points on the exceptional divisor, and $0$ otherwise. If the value
of $L[5]$ is $1$, the values of $L[3]$ and $L[4]$ is determined using floating
real numbers and thus may be wrong.

In Algorithm \ref{BlowingUp}, we consider the different blowing up charts
seperately. Without loss of generality, let us consider the chart where $x_i$
is the local equation for the exceptional divisor. After determining the strict
transform $p$, we determine whether $p$ has singularities on the exceptional
divisor, and if this is the case, where on the exceptional divisor do the
singularities occur. We do this by determining a minimal prime decomposition
$P_1\cap\cdots\cap P_s$, of the radical of the ideal $P=\langle
\textnormal{jacob}(p),p,x_i\rangle$ over $\mathbb Q$, using the \textsc{Singular}
library {\tt primdec.lib} \cite{primdec.lib} (If $p$ has zero or one singularity
on the exceptional divisor, $s=1$, otherwise, if $p$ has two singularities on
the exceptional divisor, $s=1$ or $s=2$, depending whether all the $P_j$'s are
prime over $\mathbb R$).  If $P_j$ is not prime over $\mathbb R$ for any $j$,
then $p$ has singularities at irrational points on the exceptional divisor.
This can be determined by considering the degrees of a reduced standard basis,
using lexicographical ordering, for every $P_j$. If the degree of any basis
element of $P_j$, for some $j$, is of degree $2$ then $P_j$ is not prime over
$\mathbb R$.

As we will see later, if $p$ has no singularities on the exceptional divisor or the
singularities of $p$ occur at rational points on the exceptional divisor in one
chart, this is the case in both charts. Let us first consider this case.
In this case we firstly determine a list $L_1$ which contains the resulting
singularities at $0$, or $x_1$ in case $p$ has no singularities on the exceptional divisor, 
in both charts, after $g$ is blown up. Considering the chart with
exceptional divisor $x_i$, if $P=\langle 1\rangle$, then $p$ has no singularities on the
exceptional divisor and $x_1$ is added to the list $L_1$. Otherwise, we determine the
rational points on the exceptional divisor where singularities occur by
determining the zero set of $P$. We do this by determining the zero point
$(s_{1j},s_{2j})$ of each $P_j$, using a reduced basis determined by using
lexicographical ordering. Transforming $p$ by the transformation defined by
$x_1\mapsto x_1-s_{1j}$, $x_2\mapsto x_2-s_{2j}$ we map the singularity that
occur at  $(s_{1j},s_{2j})$ to $0$. We add these resulting singularities to
$L_1$. Repeating the process for both charts, $L_1$ contains all the resulting
singularities, possibly repeated, after blowing up $g$ at the origin, and $x_1$
for each resulting germ that does not have singularities on the exceptional divisor. We
simplify $L_1$ by deleting repeating singularities and ordering the entries
according to the milnor number of the entries. Having the list $L_1$, the
methods discussed above are used to determine the values of the list $L$.

If there occur a singularity of $p$ at an irrational point in some chart on the
exceptional divisor both charts contain two singularities of the strict
transform at irrational points on the exceptional divisor. Thus it is only
necessary to consider one chart. This can be seen as follows. Since $g$ is of
type $Y_{r,s}$, $g$ is of the form
\[g=(a_0x_1+a_1x_2)^2(b_0x_1+b_1x_2)^2+\textnormal{ higher terms in $x_1$ and
$x_2$, $a_0,a_1,b_0,b_1\in\mathbb R$.}\] Without loss of generality we consider
the chart where $x_1$ is the local equation for the exceptional divisor. Then
\[p=(a_0+a_1x_2)^2(b_0+b_1x_2)^2+\textnormal{ terms that are divisible by
$x_1$.}\] Taking $p$ and $x_1$ in $P$ into account the only possible zero
points of $P$ are the roots of the rational polynomial $p_1 =
(a_0+a_1x_2)^2(b_0+b_1x_2)^2$. If $a_1=0$ or $b_1=0$ it follows from the fact
that $\mathbb Q$ is a perfect field that $p_1$ only has a rational root. Hence
$a_1\neq 0$ and $b_1\neq 0$. Therefore   $(0,-\frac{a_0}{a_1})$ and
$(0,-\frac{b_0}{b_1})$ are the only possible zero points of $P$. Because one of
these points are irrational $a_0$ and $b_0$ are nonzero. Since $\mathbb Q$ is a
perfect field and $(a_0+a_1x_2)^2(b_0+b_1x_2)^2$ is  a rational polynomial it
follows that $(a_0+a_1x_2)(b_0+b_1x_2)$ is also a rational polynomial.
Therefore both roots $-\frac{a_0}{a_1}$ and $-\frac{b_0}{b_1}$ are irrational
if any one of the roots is irrational. Furthermore,
$(a_0+a_1t)(b_0+b_1t)\in\mathbb Q(t)$ is a minimal polynomial for
$-\frac{a_0}{a_1}$ and $-\frac{b_0}{b_1}$ over $\mathbb Q$. Hence $P$ has zero
points at $-\frac{a_0}{a_1}$ and $-\frac{b_0}{b_1}$ or no zero points at all.
Since every chart contain the whole exceptional divisor, except one point, each
chart will contain at least one singular point of $p$, and hence in this case
two singular points of $p$, on the exceptional divisor. Similarly, the
singularities of $p$ on the exceptional divisor in the chart with exceptional
divisor $x_2$, will occur at the irrational points $(-\frac{a_1}{a_0},0)$ and
$(-\frac{b_1}{b_0},0)$. Furthermore, as we will see in the next paragraph, in
such cases the milnor numbers of the two resulting singularities will be equal.
Hence such cases will only occur if the normal form of $g$ is of either one of
the forms $\pm x_1^2x_2^2\pm x_1^r+ax_2^r$.

In such a case, we first ensure that the singularities of $p$ occur at
irrational points $(0,e)$ and $(0,-e)$, for some $e\in\mathbb R$, on the
exceptional divisor. This is done by applying the rational transformation
$x_1\mapsto x_1$, $x_2\mapsto x_2-(\frac{a_0b_1+b_0a_1}{a_1b_1})x_1$ to $g$.
Now $g$ will be of the form
\[g=c(x_1+ex_2)^2(x_1-ex_2)^2+\textnormal{ higher terms in $x_1$ and $x_2$,
$c,e\in\mathbb R$}.\]
Determining $p$, again, we have
\[p=c(1+ex_2)^2(1-ex_2)^2+\textnormal{ terms that are divisible by $x_1$.}\]
 Working over the ring $\mathbb Q(t)\cong \mathbb Q(e)\cong \mathbb Q (-e)$,
 with minimum polynomial $(t-e)(t+e)$, we shift the singularities of $p$ to $0$
 by the transformations defined by $x_1\mapsto x_1$, $x_2\mapsto x_2-t$. Since
 $t$ can represent either $e$ or $-e$, the resulting polynomial represent both
 singularities working over $\mathbb Q(t)$. The milnor numbers for the
 resulting singularities, and thus L[1] and L[2], can be detemined working over
 the ring $\mathbb Q(t)$. Because both resulting singularities are represented by
 one polynomial over $\mathbb Q(t)$, the milnor number will be the same in both
 cases. Since $t$ represents $e$ or $-e$, $t$ does not have a sign. Therefore
 it is not enough working over $\mathbb Q(t)$ when determining the inertia
 index of the resulting singularities or the sign of the monomial $x_1^r$ or
 the monomial $x_2^r$ in  any one of the resulting normal forms. The resulting
 singularities over $\mathbb R$ is determined by replacing $t$ with $e$ and
 $-e$, respectively. Since it is only possible to work with floating real
 numbers in \textsc{Singular} approximation errors may occur. We determine the
 approximated values of $e$ and $-e$ using the library {\tt solve.lib}
 \cite{solve.lib} in \textsc{Singular}. Using the previously discussed methods
 the approximated values of $L[3]$ and $L[4]$ can now be determined.

\begin{algorithm}[h]
\caption{Algorithm for blowing up singularities of main type $Y_{r,s}$\label{BlowingUp}}
\begin{algorithmic}[1]

\REQUIRE{ $g\in\m^3\subset\mathbb Q[x_1,x_2]$ of complex
singularity type $Y_{r,s}$.}
\ENSURE{ a List $L$ with entries $L[1]$ and $L[2]$: the values
of $r$ and $s$, $r\le s$, respectively; $L[3]$: the inertia index of the
resulting germ, after blowing up $g$, with lowest milnor number if $L[1]\neq
5$, and $0$ otherwise; $L[4]$: the sign of the monomial $x_1^r$ or $x_2^r$ in
the normal form of the resulting singularity with lowest milnor number; $L[5]$
the value $1$ if the resulting germs have singularities on the exceptional divisor, i.e.~$L[1]\neq 5$, and occur at irrational points on the exceptional
divisor and $0$ otherwise. If the value of $L[5]$ is $1$, the values of $L[3]$
and $L[4]$ are determined using floating real numbers and thus may be wrong.}
\STATE $L_1:=\emptyset$
\STATE $L:=\emptyset$
\FOR{$i=1,2$}
\STATE $\phi_1:$ $x_i\mapsto x_i$, $x_j\mapsto x_ix_j$,
$j\neq i$
\STATE $p:=\phi_1(g)$
\WHILE{$p$ is divisible by $x_i$}
\STATE $p=p/x_i$
\ENDWHILE
\STATE Let $P_1,\ldots,P_s$ be the prime ideals in the prime
decomposition
of the radical of the ideal
$\langle\textnormal{jacob}(p),p,x_1\rangle$ over $\mathbb Q$ and let
$f_{1j}, f_{2j}$ be a reduced standard basis for $P_j$, using
lexicographical
ordering;
\IF{$\deg(f_{1j})=1$, i.e.~$P_j$ is prime, for all $j$}
\STATE $m=0$
\ELSE
\STATE $m=1$
\ENDIF
\IF{$m=0$}
\FOR{$(j=1; j\le s; j++)$}
\IF{$P_j=\langle 1\rangle$}
\STATE add $x_1$ to $L_1$
\ENDIF
\IF{$P_j\neq\langle 1\rangle$}
\FOR{$i=1,2$}
\STATE $\phi_2$: $x_1\mapsto x_1-s_1$,
$x_2\mapsto x_2-s_2$, where $s_1$ is the solution of
$x_1$ in $f_{2j}$ and $s_2$ is the
solution of $x_2$ in $f_{1j}$
\STATE Add $\phi_2(p)$ to $L_1$
\ENDFOR
\ENDIF
\ENDFOR
\ELSE
\STATE $\phi_3$: $x_1\mapsto x_1$,
$x_2\mapsto x_2 -\frac{a_0b_1+a_1b_0}{a_1b_1}x_1$, where $g=(a_0x_1+a_1x_2)^2(b_0x_1+b_1x_2)^2+$ higher terms
in $x_1$ and $x_2$,
$a_0,a_1,b_0,b_1\in\mathbb R$
\STATE $g :=\phi_3(g)=c(x_1-ex_2)(x_1+ex_2)+$ higher terms
in $x_1$ and $x_2$ $c,e\in\mathbb R$
\STATE $p:=\phi_1(g)$
\WHILE{$p$ is divisible by $x_i$}
\STATE $p=p/x_i$
\ENDWHILE
\STATE $\phi_4: x_1\mapsto x_1, x_2\mapsto x_2-t$
\STATE $h_1=\phi_4(p)\in\mathbb Q(t)[x_1,x_2]$
\STATE $\mu_1 = $ milnor number of $h_1$
\STATE Determine the real value $e$ and replace $t$ by $e$
in $h_1$,
i.e~now, $h_1\in\mathbb R[x_1,x_2]$
\STATE Apply the Splitting Lemma to $h_1$
\STATE $f_1:= $ the residual part of $h_1$
\STATE $\lambda:= $ the inertia index of $h_1$
\STATE $f'_1:=\jt(f_1,\mu_1-3)$
\STATE $s_1:=$ coefficient of $x_1^{\mu_1-3}$
\STATE $s_2:=$ coefficient of $x_2^{\mu_1-3}$
\IF{$s_1=0$ and $s_2\neq 0$}
\STATE Apply $x_1\mapsto x_2, x_2\mapsto x_1$
\ENDIF
\STATE sign = coefficient of $x_1^{\mu_1-3}$
\ENDIF
\ENDFOR
\IF{$m = 0$}
\STATE $l:=$ the size of $L_1$
\FOR{$i = 1, i\le l, i++$}
\IF{$L_1[i]=x_1$}
\STATE $\mu_i=4;$
\ELSE
\STATE $\mu_i=$ milnor number of $L_1[i]+4$;\newline
\ENDIF
\ENDFOR
\STATE Delete repeating entries of $L_1$ and order the entries of
$L_1$ from small to
big with regard to its milnor number; \newline
\IF{$(L_1[1]\neq x_1)$}
\STATE Apply the Splitting Lemma to $L_1[1]$
\STATE $f_1:=$ the residual part of $L_1[1]$
\STATE $\lambda:=$ inertia index of $L_1[1]$
\STATE $f_1':=\jt(f_1,\mu_1-3)$
\STATE $s_1:=$ coefficient of $x_1^{\mu_1-3}$
\STATE $s_2:=$ coefficient of $x_2^{\mu_1-3}$
\IF{$s_1=0$ and $s_2\neq 0$}
\STATE Apply $x_1\mapsto x_2$, $x_2\mapsto x_1$
\ENDIF
\STATE sign $=$ coefficient of $x_1^{\mu_1-3}$;\newline
\ELSE
\STATE $\lambda = 0$
\STATE sign $=0$
\STATE $L[1]=\mu_1$
\STATE $L[2]=\mu_2$
\STATE $L[3]=\lambda$
\STATE $L[4]=$ sign
\STATE $L[5]=m$
\ENDIF
\ENDIF
\RETURN $L$

\end{algorithmic}[1]
\end{algorithm}[h]

Here follows the algorithm we implemented to classify the singularities, using
the output list $L$ in Algorithm \ref{BlowingUp}, of type $\widetilde Y_r$ and
$Y_{r,s}$.

\begin{algorithm}[h]
\caption{Algorithm for the cases $\widetilde Y_r$ and $Y_{r,s}$
\label{Y[r,s]}}
\begin{algorithmic}[1]
\REQUIRE{ $f\in \m^3\subset\mathbb Q[x,y]$ of complex
singularity type $\widetilde Y_r$ or $Y_{r,s}$.}
ENSURE{ the real singularity type of $g$,
i.e.~$\widetilde Y_r^+$, $\widetilde Y_r^-$, $Y_{r,s}^{++}$, $Y_{r,s}^{+-}$,
$Y_{r,s}^{--}$, $Y_{r,s}^{-+}$.}
\STATE $\mu :=$ milnor number of $f$
\STATE $h :=  \jt(g,4)$
\STATE $s_1 := \textnormal{coefficient of }x^4$
\STATE $s_2 := \textnormal{coefficient of }y^4$
\IF{$s_1=0$ and $s_2\neq 0$}
\STATE Swap the variables $x$ and $y$ in $h$
\ENDIF
\IF{$s_1=0$ and $s_2=0$}
\STATE $t_1:=\textnormal{coefficient of }x^3y\ (\textnormal{in } h)$
\STATE $t_2:=\textnormal{coefficient of }x^2y^2\ (\textnormal{in } h)$
\STATE $t_3:=\textnormal{coefficient of }xy^3\ (\textnormal{in } h)$
\IF{$t_1+t_2+t_3=0$}
\IF{$2t_1+4t_2+8t_3\neq 0$}
\STATE Apply (to $h$) $x\mapsto x$, $y\mapsto y$
\ELSE
\STATE Apply (to $h$) $x\mapsto x$, $y\mapsto 3y$
\ENDIF
\STATE Apply (to $h$) $x\mapsto x$, $y\mapsto x+y$
\ENDIF
\ENDIF
\STATE Write $h$ as $a_0x^4+a_1x^3y+a_2x^2y^2+a_3xy^3+a_4y^4$,
$a_0,\ldots,a_4\in\mathbb Q, a_0\neq 0$
\STATE sign = coefficient of $x^4$ (in $h$)
\STATE Apply $x\mapsto x$, $y\mapsto 1$
\STATE Write $h$ as $a_0x^4+a_1x^3+a_2x^2+a_3x+a_4$
\STATE $r := \#$ real roots of $h$
\IF{$r=0$}
\STATE $r'=\frac{\mu-9}{2}+4$
\IF{sign$>0$}
\RETURN $\widetilde Y_{r'}^+$
\ELSE
\RETURN $\widetilde Y_{r'}^-$
\ENDIF
\ENDIF
\IF{$r\neq 0$}

\STATE Blowing up the origin of $g$ and let $L$ be a list with entries
$L[1]$ and $L[2]$: the values of $r$ and $s$, $r\le s$, respectively; $L[3]$:
the inertia index of the resulting singularity, after blowing up $g$, with
lowest milnor number if the resulting germs have singularities on the exceptional divisor, i.e~$L[1]\neq 5$, and $0$ otherwise; $L[4]$: the sign of the monomial
$x_1^r$ or $x_2^r$ in the normal form of the resulting singularity with lowest
milnor number; $L[5]$ the value $1$ if the resulting germs intersects the
exceptional divisor smoothly and occur at irrational points on the exceptional
divisor and $0$ otherwise.
\STATE $\mu_1:=L[1]$
\STATE $\mu_2:=L[2]$
\STATE $\lambda:=L[3]$
\STATE $\sign_1:= L[4]$
\STATE $m:=L[5]$

\IF{($\mu_1+1$) mod $2=1$}
\IF{sign$>0$}
\RETURN $Y_{\mu_1+1,\mu_2+1}^{++}$
\ELSE
\RETURN $Y_{\mu_1+1,\mu_2+1}^{+-}$
\ENDIF
\ELSE
\IF{$\mu_1=5$}
\IF{sign$<0$}
\STATE redefine $\sign_1$ as $\sign_1:=\lambda-1$
\ELSE
\STATE redefine $\sign_1$ as $\sign_1:=\lambda$
\ENDIF
\IF{$\sign_1>0$ and sign$>0$}
\RETURN $Y_{\mu_1+1,\mu_2+1}^{+-}$
\ENDIF
\IF{$\sign_1>0$ and sign$<0$}
\RETURN $Y_{\mu_1+1,\mu_2+1}^{--}$
\ENDIF
\IF{$\sign_1=0$ and sign$>0$}
\RETURN $Y_{\mu_1+1,\mu_2+1}^{++}$
\ENDIF
\IF{$\sign_1=0$ and sign$<0$}
\RETURN $Y_{\mu_1+1,\mu_2+1}^{-+}$
\ENDIF

\ELSE
\STATE $\sign_1=$ coefficient of $x^{\mu_1-3}$
\ENDIF
\IF{sign$>0$ and $\sign_1>0$}
\RETURN $Y_{\mu_1+1,\mu_2+1}^{++}$
\IF{sign$>0$ and $\sign_1<0$}
\RETURN $Y_{\mu_1+1,\mu_2+1}^{+-}$
\ENDIF
\IF{sign$<0$ and $\sign_1>0$}
\RETURN $Y_{\mu_1+1,\mu_2+1}^{--}$
\ENDIF
\IF{sign$<0$ and $\sign_1<0$}
\RETURN $Y_{\mu_1+1,\mu_2+1}^{-+}$
\ENDIF
\ENDIF
\ENDIF

\ENDIF
\end{algorithmic}[1]
\end{algorithm}[h]


\subsection{Real exceptional $\mathbf 1$-modal singularities of corank $\mathbf 2$}
\label{ExceptionalSingularities}
\subsubsection{Exceptional cases:}Lastly we treat the exceptional singularities. The cases $E_{12}$, $E_{13}$,
$Z_{11}$ and $Z_{12}$ are solved only using the complex classification of $g$.
Singularities of complextype $Z_{13}$ are solved, similar to case $D_{4}$,
using Lemma \ref{kjet} and Lemma \ref{x^3} to transform the $4$-jet of $g$, to
the form $x^3y$, after which the sign of the term $y^6$ is considered.

To classify the singularities of complextype $W_{12}$ and $W_{13}$ we only need
to consider the signs of the coefficients of the monomials of the form $x^4$ in
the $4$-jet of~$g$.

For the case $E_{14}$ we use Lemma \ref{kjet} and Lemma \ref{x^3} to write the
$3$-jet of $g$ in the form $sx^3$, $s\in\mathbb Q$. To simplify calculations we
divide $g$ by $s$ such that $\jt(g,3)=x^3$ and after the following
transformations multiply $g$ by $s$ again. We sistematically transform $g$ such
that the $4$-jet is $a_0x^3$ and then such that the $5$-jet is $a_0x^3$,
$a_0=\pm 1$.  We now, similarly to the case $D_4$, after multiplying $g$ with
$s$ again, consider the sign of the $y^8$-term to determine the real
singularity type. We only include the algorithm for the case $E_{14}$, since
the algorithms for the other exceptional cases are trivial.

\begin{algorithm}[h]
\caption{Algorithm for the case $X_{9+k}$\label{X[9+k]}}
\begin{algorithmic}[1]
\REQUIRE{ $g\in \m^3\subset\mathbb Q[x,y]$ of complex
singularity type $X_{9+k}$}
\ENSURE{the real singularity type of $g$, i.e.~$X_{9+k}^-$
or $X_{9+k}^+$.}
\STATE $g:=\jt(g,4)$
\STATE $s_1:=$ coefficient of ${x^4}$
\STATE $s_2:=$ coefficient of ${y^4}$
\IF{$s_2\neq0$ and $s_1=0$}
\STATE Swap the variables $x$ and $y$
\ENDIF
\IF{$s_2\neq0$ and $s_1=0$}
\STATE $t_1:=$ coefficient of ${x^3y}$
\STATE $t_2:=$ coefficient of ${x^2y^2}$
\STATE $t_3:=$ coefficient of ${xy^3}$
\IF{$t_1+t_2+t_3=0$}
\IF{$2t_1+4t_2+8t_3\neq0$}
\STATE Apply $x\mapsto x$, $y\mapsto 2y$ (to $g$)
\ELSE 
\STATE Apply $x\mapsto x$, $y\mapsto 3y$ (to $g$)
\ENDIF
\ENDIF
\STATE Apply $x\mapsto x$, $y\mapsto x+y$ (to $g$)
\ENDIF
\STATE Write $g$ as $g=a_0x^4+a_1x^3y+a_2x^2y^2+a_3xy^3+a_4y^4,\quad
a_0,\ldots,a_4\in\mathbb Q$
\STATE $g':=g(x,1)=a_0x^4+a_1x^3+a_2x^2+a_3x+a_4$
\STATE $n:=\#$ real roots of $g'$
\IF{$n=1$}
\IF{$a_0>0$}
\RETURN $X_{9+k}^{++}$
\ELSE
\RETURN $X_{9+k}^{--}$
\ENDIF
\ENDIF
\IF{$n=3$}
\STATE Factorize $g$ in linear factors over $\mathbb Q[x,y]$,
with $g_1:=t_1x+t_2y$ the factor with multiplicity $2$
\STATE Apply $x\mapsto\frac{x+t_2y}{t_1}$, $y\mapsto y$ (to $g$)
\STATE Write $g$ as $g=a_0x^4+a_1x^3y+a_2x^2y^2,\quad
a_0,a_1,a_2\in\mathbb Q$
\IF{$a_2>0$}
\RETURN $X_{9+k}^{-+}$
\ELSE
\RETURN $X_{9+k}^{+-}$
\ENDIF
\ENDIF

\end{algorithmic}
\end{algorithm}

\begin{algorithm}[h]
\caption{Algorithm for the case $E_{14}$}\label{E[14]}
\begin{algorithmic}[1]
\REQUIRE{$f\in \m^3$, $f\in\mathbb Q[x,y]$ of complex
singularity type $E_{14}$.}
\ENSURE{the real singularity type of $f$, i.e.~$E_{14}^+$ or
$E_{14}^-$.}
\STATE $h_1 :=\jt(g,3)$
\STATE $s :=$ coefficient of $h_1$
\IF{$s=0$}
\STATE swap variables $x$ and $y$ in $g$
\ENDIF
\STATE $g := \frac{g}{s}$
\STATE $h_2 :=\jt(g,3)$
\STATE Factors $:=$ the list of factors of $h_1$ over $\mathbb Q$
\STATE Write $g_1$ as $t_1x+t_2y$, $t_1,t_2\in\mathbb Q$, $t_1 := \pm-1$, where
$g_1$ is the first entry in the list $F$
\STATE Apply (to $g$): $x\mapsto \frac{x-t_2y}{t_1}$, $y\mapsto y$
\STATE Write $g$ as $a_0x^3+ $terms of higher degree,\quad$a_0\in\mathbb Q$
\STATE $g_3:=\frac{\jt(f,4)-a_0x^3}{3x^2}$
\STATE Apply (to $g$): $x\mapsto g_3$, $y\mapsto y$
\STATE $g_4:=\frac{\jt(f,5)-a_0x^3}{3x^2}$
\STATE Apply (to $g$): $x\mapsto x-g_4$, $y\mapsto y$
\STATE $g := g*s$
\STATE $w:=$ the coefficient of $y^8$ in $g$
\IF{$w>0$}
\RETURN $E_{14}^+$
\ELSE
\RETURN $E_{14}^-$
\ENDIF
\end{algorithmic}
\end{algorithm}

 \begin{thebibliography}{99}
\bibitem{AVG1985} Arnold, V.I.; Gusein-Zade, S.M.; Varchenko, A.N.:
Singularities of Differential Maps. Vol.~I, Birkh\"auser (1985).
\bibitem{A1975} Arnold, V.I.:
\textit{Normal form of functions near degenerate critical points.},
Russian Mth. Surveys 29 ii (1975), 10-50.
\bibitem{DGPS}
Decker, W.; Greuel, G.-M.; Pfister, G.; Sch{\"o}nemann, H.:
\newblock {\sc Singular} {3-1-5} --- {A} computer algebra system for polynomial
computations.
\newblock {http://www.singular.uni-kl.de} (2012).
\bibitem{Kruger} Kr\"uger, K.: Klassifikation von
Hyperfl\"agensingularit\"aten, Diploma Thesis (1997).
\bibitem{GLS2007}Greuel, G.-M.; Lossen, C.; Shustin E.:
Introduction to Singularities and Deformations, Springer, Berlin (2007).
\bibitem{GP2008} Greuel G.-M.; Pfister G.;
A Singular introduction to Commutative Algebra, 2nd Ed., Springer,
Berlin (2008).
\bibitem{classify}
Kr\"uger, K.:
{\tt classify.lib}. {A} {\sc Singular} {3-1-5} library for classifying isolated
hypersurface singularities w.r.t. right equivalence, based on the determinator
of singularities by V.I. Arnold (2012).
\bibitem{realclassify}
Marais, M. and Steenpass, A.:
{\tt realclassify.lib}. {A} {\sc Singular} {3-1-5} library for classifying
isolated hypersurface singularities over the reals w.r.t. right equivalence,
based on the determinator of singularities by V.I. Arnold. This library is
based on classify.lib by Kai Kr\"uger, but handles the real case, while
classify.lib does the complex classification (2012).
\bibitem{primdec.lib} Pfister, G.; Decker, W.;  Schoenemann, H.; Laplagne, S.:
{\tt primdec.lib}. {A} {\sc Singular} {3-1-5} library for Primary Decomposition
and Radical of Ideals (2012).
\bibitem{Siersma} Siersma D.: Classification and deformation of Singularities,
disertation, University of Amsterdam (1974).
\bibitem{roots}
Tobis, A.:
{\tt rootsur.lib}. {A} {\sc Singular} {3-1-5} library for Counting number of
real roots of univariate polynomial (2012).
\bibitem{solve.lib} Wenk, M.: {\tt solve.lib}. Pohl, W.:
{A} {\sc Singular} {3-1-5} library for Complex Solving of Polynomial Systems
(2012).

\end{thebibliography}

\end{document}
