\documentclass[noend]{amsproc}

\renewcommand{\arraystretch}{1.3}

\usepackage{amsthm,amsmath,amsfonts,mathrsfs,amssymb}
\usepackage{algorithm}
\usepackage{algorithmicx, algpseudocode}
\usepackage[T1]{fontenc}   % for bold \Singular
\usepackage{multirow}
\usepackage{natbib}
\usepackage{url}

\def\UrlFont{\fontfamily{lmtt}\selectfont}

\newtheorem{theorem}{Theorem}
\newtheorem{defn}[theorem]{Definition}
\newtheorem{prop}[theorem]{Proposition}
\newtheorem{lemma}[theorem]{Lemma}
\theoremstyle{definition}
\newtheorem{remark}[theorem]{Remark}

% ALGORITHM style
\renewcommand{\algorithmicrequire}{\textbf{Input:}}
\renewcommand{\algorithmicensure}{\textbf{Output:}}
\newcommand{\algorithmicbreak}{\textbf{break}}
\newcommand{\Break}{\State \algorithmicbreak}
\renewcommand{\algorithmicreturn}{\State \textbf{return}}

\newcommand{\Singular}{\textsc{Singular}}
\newcommand{\realclassify}{\texttt{realclassify.lib}}
\newcommand{\classify}{\texttt{classify.lib}}

\newcommand{\requiv}{\ensuremath{\mathrel{\overset{r}{\sim}}}}
\newcommand{\cequiv}{\ensuremath{\mathrel{\overset{c}{\sim}}}}
\newcommand{\sequiv}{\ensuremath{\mathrel{\overset{s}{\sim}}}}

\DeclareMathOperator{\ord}{ord}
\DeclareMathOperator{\m}{\mathfrak{m}}
\DeclareMathOperator{\jet}{jet}
\DeclareMathOperator{\corank}{corank}
\DeclareMathOperator{\supp}{supp}
\DeclareMathOperator{\sign}{sign}
\DeclareMathOperator{\diag}{diag}
\DeclareMathOperator{\NF}{NF}
\DeclareMathOperator{\N}{\mathbb{N}}
\DeclareMathOperator{\Q}{\mathbb{Q}}
\DeclareMathOperator{\R}{\mathbb{R}}
\DeclareMathOperator{\C}{\mathbb{C}}
\DeclareMathOperator{\K}{\mathbb{K}}
\DeclareMathOperator{\A}{\mathbb{A}}
\DeclareMathOperator{\Pj}{\mathbb{P}}
\DeclareMathOperator{\boldzero}{\mathbf{0}}

\hyphenation{equiva-lence}
\hyphenation{sin-gu-lar-ities}

\title[The Classification of Real Singularities Using \textsc{Singular}, %
Part III]%
{The Classification of Real Singularities Using \textsc{Singular}\\
Part III: Unimodal Singularities of Corank $2$}

\author{Magdaleen S. Marais}
\address{Magdaleen S. Marais\\
African Institute for Mathematical Sciences and Stellenbosch University\\
6 Melrose Rd\\
Muizenberg 7945, Cape Town\\
South Africa}
\email{magdaleen@aims.ac.za}

\author{Andreas Steenpa\ss}
\address{Andreas Steenpa\ss\\
Department of Mathematics\\
University of Kaiserslautern\\
Erwin-Schr\"odinger-Str.\\
67663 Kaiserslautern\\
Germany}
\email{steenpass@mathematik.uni-kl.de}

\thanks{This research was supported by the African Institute for Mathematical
Sciences and a grant awarded by Gert-Martin Greuel. We are thankful to both of
them.}

\keywords{%
hypersurface singularities, algorithmic classification, real geometry%
}

\begin{document}

\begin{abstract}
We present algorithms to classify isolated hypersurface singularities over the
real numbers according to the classification by V.I.~Arnold \citep{AVG1985}.
This first part covers the splitting lemma and the simple singularities; a
second and a third part will be devoted to the unimodal singularities up to
corank~2. All algorithms are implemented in the \Singular{} library
\realclassify{} \citep{realclassify}.
\end{abstract}

\maketitle


\section{Introduction}
\citet{AVG1985} present classification theorems for singularities over the
complex numbers up to modality~2 and for singularities over the real numbers up
to modality~1, including complete sets of normal forms. For the complex case,
they also give an algorithm how the type of a given singularity can be
computed, called the ``determinator of singularities''
\citep[cf.\@][ch.~16]{AVG1985}, but this question is left open for the real
case. The goal of this paper, together with its subsequent parts, is to present
algorithms for the classification of isolated hypersurface singularities up to
modality~1 and corank 2 over the real numbers with respect to right
equivalence.

We consider real functions with a critical point at the origin and critical
value~$0$, i.e.\@ functions in $\m^2$, where $\m$ denotes the ideal of function
germs vanishing at the origin. Two function germs $f, g \in \m^2 \subset
\R[[x_1,\ldots,x_n]]$ are considered as right equivalent, denoted by
$f \requiv g$, if there exists an $\R$-algebra automorphism $\phi$ of
$\R[[x_1,\ldots,x_n]]$ such that $\phi(f) = g$.

We have implemented all the algorithms presented here in the computer algebra
system \Singular{} \citep{DGPS}. The implementation is freely available as a
\Singular{} library called \realclassify{} which relies on \Singular's
\classify{} to determine, for a given polynomial, the type in Arnold's
classification over the complex numbers. The methods used in \classify{} will
not be discussed in this paper. For more information in this regard,
\citet{Kruger} can be studied.

In Section~\ref{sec:prerequisites}, we introduce basic notions and methods
which are frequently used for the algorithmic classification in the subsequent
sections. We first give an overview of the different notions of equivalence in
singularity theory and how they are related in
Subsection~\ref{subsec:equivalence}. Thereafter we recall some basic results on
the Milnor number and the determinacy in Subsections~\ref{subsec:milnor_number}
and~\ref{subsec:determinacy}, and we also recall how these invariants can be
computed. As a further prerequisite, we show that the homogeneous parts of
lowest degree of two right equivalent functions factorize in the same way over
$\R$ (Section~\ref{subsec:factorization}, Proposition~\ref{kjet}). We also show
that in some cases, this factorization can even be carried out over $\Q$ which
is important for the algorithmic aspect (Lemma~\ref{x^3}).

Using the Splitting Lemma (Theorem~\ref{thm:splitting_lemma}), any function
germ $f$ over the real numbers with an isolated singularity at the origin can
be written, after choosing a suitable coordinate system, as the sum of two
functions of which the variables are disjoint. One of the functions, called the
nondegenerate part of $f$, is a nondegenerate quadratic form and the other
function, called the residual part of $f$, is an element of $\m^3$. The number
of variables in the residual part is equal to the corank of $f$, denoted by
$\corank(f)$. In this paper we only consider germs with corank $0$, $1$
and~$2$. A version of the Splitting Lemma for singularities over $\R$ and a
corresponding algorithm are discussed in Section~\ref{sec:splitting_lemma}.

In \citet{AVG1985}, the real singularities of modality $0$ and $1$ are
classified up to stable equivalence into main types which split up into more
subtypes
depending on the sign of certain terms. Two functions are stably equivalent if
they are right equivalent after the direct addition of nondegenerate quadratic
forms. Hence after applying the Splitting Lemma, we only need to consider the
residual part in order to compute the correct subtype. It can be easily seen
that the subtypes are complex equivalent to a complex singularity type of the
same name as its corresponding real main singularity type (see
Table~\ref{tab:normal_forms}). In fact there is a bijection between the complex
types of modality $0$ and $1$ and the real main types. Thus, if we can
determine the complex type of a function germ, we only need to determine the
correct subtype of the corresponding real main type. The classification of the
residual part is given in Section~\ref{sec:residual_part}, together with
explicit algorithms for each singularity type.



\section{Acknowledgements}

We would like to thank Gerhard Pfister and Gert-Martin Greuel for insightful
conversations. We also would like to thank the two anonymous referees whose
suggestions greatly improved this article.

\section{Real exceptional $\mathbf 1$-modal singularities of corank $\mathbf 2$}
\label{ExceptionalSingularities}
In Lemma 7.3, p.27 of \citet{A1974}, it is shown that a semi-quasihomogeneous function $g\in\K[[x_1,\cdots,x_n]]$ with quasihomogeneous part $g_0$ of degree $d$ is right equivalent to a function of the form $g_0+\sum_{k=1}^s c_ke_k$, where $c_1,\ldots, c_s$ are constants and $e_1,\ldots,e_s$ is the set of all basis monomials in a fixed basis of the local algebra of the function $g_0$. This is done by systematically considering each diagonal with quasihomigeneous degree $d'>d$ on which at least one monomial of $g$ lie. Using the fact that the sum of the terms of degree $d'$ can be decomposed as
\[h:=\sum_{i=1}^n\frac{\partial g_0}{\partial x_i}v_i(x_i)+c_1e_1+\cdots+c_se_s,\]
it is shown that there is an automorphism $\phi_{d'}$, defined by $\phi_{d'}(x_i)=x_i-v_i(x_i)$, such that
\[\phi_{d'}(g)=g_0+[g_1+(c_1e_1+\cdots +c_se_s)-h]+R,\]
where the filtration of $R$ is greater than $d'$. By taking the finite determinacy of germs into account the goal is reached by this transformation. What happens in practice is that $\phi_{d'}$ replace all the terms of weighted degree $d'$ in $g$ by a linear combination of $e_1,\ldots e_s$. For each $d'$, $\phi_{d'}$ can, for simplicity, as is done in the given method below, be replaced in the following way by the composition of two automorphisms. First the terms of degree $d'$ in $g$ not in the local algebra is considered, which practically results in throwing these terms away. And then, secondly, the terms in the local algebra of degree $d'$ is considered, which practically results in replacement by a linear combination of the chosen basis. Determining $\phi_{d'}$ in this way turns out to be simpler in practise.

Since a chosen basis is not necessarily unique, the monomials in a representation of the right equivalence class of $g$ as $g_0+\sum_{k=1}^s c_ke_k$ is also not necessarily unique. In \cite{A1974}, Arnold has chosen a specific basis for the local algebra in each of the normal forms he determined by this method. In each case that turns out to be the monomials with parameters as coefficients in the normal forms. Since we work with Arnold's classification we need to choose the same basis. For the exceptional cases the local algebra has only one basis element. In \cite{realclassify2} it follows from Table 10 that the parameter in each of the exceptional normal forms has a unique value. 

We use the following method to determine the equivalence class of an input poynomial $g\in\mathfrak m^3\subset\R[[x,y]]$ of exceptional type:

\begin{enumerate}
\item Transform $g$ such that there are no terms underneath the diagonal created by the quasihomogeneous part of the corresponding complex normal form, which we will refer to as the diagonal in the next steps.
\item Determine which case $g$ is by considering the terms on the diagonal.
\item Throw all terms not in the local algebra of $g_0$, above the diagonal away.
\item Replace terms in the local algebra not in Arnold's chosen basis by a linear combination of elements in his chosen basis.
\item Scale the terms on the diagonal and read the value off of the parameter.
\end{enumerate}

The results in section 2.4 of \cite{realclassify1} is used in all the Algorithms for classifying the unimodal, corank 2, singularities. We use the notation introduced in \cite{realclassify2}.
 
\begin{algorithm}[ht]
\caption{Algorithm for the case $Z_{11}$}%
\label{alg:Z_11}
\begin{algorithmic}[1]

\Require{$g \in \m^3\subset\Q[x,y]$ of complex singularity type $Z_{11}$.}

\Ensure{A list with the following entries: [1] the real singularity type of $g$; [2] the normal form of $g$; [3] a minimal polynomial over $\Q$ for the parameter in the normal form; [4] if the parameter cannot be determined by its minimal polynomial alone, a list with entries: [1] a lower and/or [2] an upper bound of the parameter.}

\State $h:=\jet(g,4)$;
\State Factorize $h$ as $h_1h_2^3$;
\State Apply $h_1\mapsto x$ and $h_2\mapsto y$ to $g$;
\State Write $g$ as $x^3y+by^5+cxy^4+R,\quad a,b,c\in\Q$,\quad $R\in E^{(4,3)}_{17}$;
\State $p:=x^{15}-c^{15}b^{-11}$;
\If{$p$ has a degree $1$ factor over $\Q$}
\State MP := $x-cb^{-\frac{11}{15}}$;
\Else
\State $p_1:= x^5-c^{15}b^{-11}$;
\If{$p_1$ has a degree $1$ factor over $\Q$}
\State MP := $x^3-c^3b^{-\frac{11}{5}}$;
\Else
\State $p_2:= x^3-c^{15}b^{-11}$;
\If{$p_2$ has a degree $1$ factor over $\Q$}
\State MP := $x^5-c^5b^{-\frac{11}{3}}$;
\Else
\State MP := $p$;
\EndIf
\EndIf
\EndIf
\State type := $Z_{11}$;
\State NF := $x^3y+y^5+axy^4$;
\State List 1 := type, NF, MP, Emptyset;
\Return(List 1)

\end{algorithmic}
\end{algorithm}


\begin{algorithm}[ht]
\caption{Algorithm for the case $Z_{12}$}%
\label{alg:Z_12}
\begin{algorithmic}[1]

\Require{$g \in \m^3\subset\Q[x,y]$ of complex singularity type $Z_{12}$.}

\Ensure{A list with the following entries: [1] the real singularity type of $g$; [2] the normal form of $g$; [3] a minimal polynomial over $\Q$ for the parameter in the normal form; [4] if the parameter cannot be determined by its minimal polynomial alone, a list with entries: [1]  a lower and/or [2] an upper bound of the parameter.}

\State $h:=\jet(g,4)$;
\State Factorize $h$ as $h_1h_2^3$;
\State Apply $h_1\mapsto x$ and $h_2\mapsto y$ to $g$;
\State Write $g$ as $x^3y+bxy^4+cx^2y^3+dy^6+R,\quad a,b,c,d\in\Q$,\quad $R\in E^{(3,2)}_{13}$;
\State Replace $y^6$ with $-\frac{3}{b}x^2y^3$;
\State Write $g$ as $x^3y+bxy^4+cx^2y^3+R,\quad a,b,c\in\Q$,\quad $R\in E^{(3,2)}_{13}$;
\State $p:=x^{11}-c^{11}b^{-7}$;
\If{$p$ has a degree $1$ factor over $\Q$}
\State MP := $x-cb^{\frac{-7}{11}}$;
\Else
\State MP := $p$;
\EndIf
\State type := $Z_{12}$;
\State NF := $x^3y+xy^4+ax^2y^3$;
\State List 1 := type, NF, MP, Emptyset;
\Return(List 1)

\end{algorithmic}
\end{algorithm}

\begin{algorithm}[ht]
\caption{Algorithm for the case $Z_{13}$}%
\label{alg:Z_13}
\begin{algorithmic}[1]

\Require{$g \in \m^3\subset\Q[x,y]$ of complex singularity type $Z_{13}$.}

\Ensure{A list with the following entries: [1] the real singularity type of $g$; [2] the normal form of $g$; [3] a minimal polynomial over $\Q$ for the parameter in the normal form; [4] if the parameter cannot be determined by its minimal polynomial alone, a list with entries: [1] a lower and/or [2] an upper bound of the parameter.}

\State $h :=\jet(g,4)$;
\State Factorize $h$ as $h_1h_2^3$;
\State Apply $h_1\mapsto x$ and $h_2\mapsto y$ to $g$;
\State Write $g$ as $x^3y+by^6+cx^2y^3+dxy^5+R,\quad a,b,c,d\in\Q$,\quad $R\in E_{21}^{(5,3)}$;
\If{$b>0$}
\State NF := $x^3y+y^6+axy^5$;
\State type := $Z_{13}^+$;
\Else
\State NF := $x^3y-y^6+axy^5$;
\State type := $Z_{13}^-$;
\EndIf
\State $p := x^9-d^9|b|^{-7}$; 
\If{$p$ has a degree $1$ factor over $\Q$}
\State MP := $x-d|b|^\frac{-7}{9}$;
\Else
\State $p_1:=x^3-d^9|b|^{-7}$;
\If{ $p_1$ has a degree $1$ factor over $\Q$}
\State MP := $x^3-d^3|b|^{-\frac{7}{3}}$;
\Else
\State MP := $p$;
\EndIf
\EndIf
\State List 1 := type, NF, MP, Emptyset;
\Return(List 1)

\end{algorithmic}
\end{algorithm}

\begin{algorithm}[ht]
\caption{Algorithm for the case $W_{12}$}%
\label{alg:W_12}
\begin{algorithmic}[1]

\Require{$g \in \m^3\subset\Q[x,y]$ of complex singularity type $W_{12}$.}

\Ensure{A list with the following entries: [1] the real singularity type of $g$; [2] the normal form of $g$; [3] a minimal polynomial over $\Q$ for the parameter in the normal form; [4] if the parameter cannot be determined by its minimal polynomial alone, a list with entries: [1] a lower and/or [2] an upper bound of the parameter.}

\State $s := $ coefficient of the $x^4$ term in $g$;
\If{$s=0$}
\State Apply $x\mapsto y$, $y\mapsto x$ in $g$;
\EndIf
\State $h := \jet(g,4)$;
\State Factorize $h$ as $bh_1^4$, $b\in\Q$;
\State Apply $h_1\mapsto x$, $y\mapsto y$ to $g$;
\State Write $g$ as $bx^4+cy^5+dx^2y^3+exy^4+R$,\quad $b,c,d,e\in\Q$,\quad$R\in E_{23}^{(5,4)}$;
\If{$c>0$}
\State Apply $x\mapsto x$, $y\mapsto -y$ to $g$;
\EndIf
\State Write $g$ as $bx^4+cy^5+dx^2y^3+exy^4+R$,\quad $b,c,d,e\in\Q$,\quad$R\in E_{23}^{(5,4)}$;
\If{$b>0$}
\State NF := $x^4+y^5+ax^2y^3$;
\State type := $W_{12}^+$;
\Else
\State NF := $-x^4+y^5+ax^2y^3$;
\State type := $W_{12}^-$;
\EndIf
\State $p := x^{10}-d^{10}|b|^{-5}c^{-6}$;
\If{ $p$ has a degree $1$ factor over $\Q$}
\State MP := $x-d|b|^\frac{-1}{2}c^\frac{-3}{5}$;
\State List 1a := Emptyset;
\Else
\State $p_1:= x^2-d^{10}|b|^{-5}c^{-6}$;
\If{$p_1$ has a factor of degree $1$ over $\Q$}
\State MP := $x^5-d^5|b|^{-\frac{5}{2}}c^{-3}$;
\State List 1a := Emptyset;
\Else
\If{$d>0$}
\State List 1a : = $0$, Emptyset; 
\Else 
\State List 1a := Emptyset, $0$;
\EndIf
\State $p_2:= x^5-d^{10}|b|^{-5}c^{-6}$;
\If{$p_2$ has a degree $1$ factor over $\Q$}
\State MP := $x^2-d^2|b|^{-1}c^{-\frac{6}{5}}$;
\Else
\State MP := $p$;
\EndIf
\EndIf
\EndIf
\State List 1 := type, NF, MP, List 1a;
\Return(List 1)
\end{algorithmic}
\end{algorithm}

\begin{algorithm}[ht]
\caption{Algorithm for the case $W_{13}$}%
\label{alg:W_13}
\begin{algorithmic}[1]

\Require{$g \in \m^3\subset\Q[x,y]$ of complex singularity type $W_{13}$.}

\Ensure{A list with the following entries: [1] the real singularity type of $g$; [2] the normal form of $g$; [3] a minimal polynomial over $\Q$ for the parameter in the normal form; [4] if the parameter cannot be determined by its minimal polynomial alone, a list with entries: [1] a lower and/or [2] an upper bound of the parameter.}

\State $s := $ coefficient of the $x^4$ term in $g$;
\If{$s=0$}
\State Apply $x\mapsto y$, $y\mapsto x$ in $g$;
\EndIf
\State $h := \jet(g,4)$;
\State Factorize $h$ as $bh_1^4$, $b\in\Q$;
\State Apply $h_1\mapsto x$, $y\mapsto y$ to $g$;
\State Write $g$ as $bx^4+cx^3y^2+dx^2y^3+exy^4+ky^6+R$,\quad $b,c,d,e,k\in\Q$,\quad $R\in E_{19}^{(4,3)}$;
\State Replace $x^3y^2$ with $-{4b}^{-1}y^6$;
\State Write $g$ as $bx^4+cxy^4+dy^6+ey^3x^2+R$,\quad $b,c,d,e\in\Q$,\quad $R\in E_{19}^{(4,3)}$;
\State Apply $x\mapsto c^{-1}x$, $y\mapsto y$ to $g$;
\State Write $g$ as $bx^4+xy^4+dy^6+ey^3x^2+R$,\quad $b,d,e\in\Q$,\quad $R\in E_{19}^{(4,3)}$;
\If{$b<0$}
\State NF $:= x^4+xy^4+ay^6$;
\State type $:= W_{13}^+$;
\Else
\State NF $:= -x^4+xy^4+ay^6$;
\State type $:= W_{13}^-$;
\EndIf
\State $p := x^8-d^8|b|^3$;
\If{ $p$ factorize over $\Q$}
\State MP := $x-d|b|^\frac{3}{8}$;
\State List 1a := Emptyset;
\Else
\If{d>0}
\State List 1a : = $0$, Emptyset; 
\Else 
\State List 1a := Emptyset, $0$;
\EndIf
\State $p_1:= x^2-d^8|b|^3$;
\If{$p_1$ has a factor of degree $1$ over $\Q$}
\State MP := $x^4-d^4|b|^{\frac{3}{2}}$;
\Else
\State $p_2:= x^4-d^8|b|^3$;
\If{$p_2$ has a factor of degree $1$ over $\Q$}
\State MP := $x^2-d^2|b|^{\frac{3}{4}}$;
\Else
\State MP := $p$;
\EndIf
\EndIf
\EndIf
\State List 1 := type, NF, MP, List 1a;
\Return(List 1)
\end{algorithmic}
\end{algorithm}



\begin{algorithm}[ht]
\caption{Algorithm for the case $E_{12}$}%
\label{alg:E_12}
\begin{algorithmic}[1]

\Require{$g \in \m^3\subset\Q[x,y]$ of complex singularity type $E_{12}$.}

\Ensure{A list with the following entries: [1] the real singularity type of $g$; [2] the normal form of $g$; [3] a minimal polynomial over $\Q$ for the parameter in the normal form; [4] if the parameter cannot be determined by its minimum polynomial alone, a list with entries: [1] a lower and/or [2] an upper bound of the parameter.}

\State $s := $ coefficient of the $x^3$ term in $g$;
\If{$s=0$}
\State Apply $x\mapsto y$, $y\mapsto x$ in $g$;
\EndIf
\State $h := \jet(g,3)$;
\State Factorize $h$ as $bh_1^3$, $b\in\Q$;
\State Apply $h_1\mapsto x$, $y\mapsto y$ to $g$;
\State Write $g$ as $bx^3+R$,\quad $b\in\Q$,\quad $R\in E_4$;
\State $h_1:= b^{-1}g$;
\State $h_2:= \frac{\jet(h_1,4)-x^3}{3x^2}$;
\State Apply $x\mapsto x-h_2$, $y\mapsto y$ to $h_1$;
\State $h_3:=\frac{\jet(h_1,5)-x^3}{3x^2}$;
\State Apply $x\mapsto x-h_3$, $y\mapsto y$ to $h_1$;
\State $g:=b\cdot h_1$;
\State Write $g$ as $bx^3+cy^7+R$,\quad $b\in\Q$, \quad $R\in E_{22}^{(7,3)}$;
\State $t := $ coefficient of the $xy^5$ term in $g$;
\State NF $:= x^3+y^7+axy^5$;
\State type $:= E_{12}$;
\State $p:= x^{21}-b^{-7}c^{-15}t^{21}$;
\If{$p$ has a factor of degree $1$ over $\Q$}
\State MP $:= x-b^{-\frac{1}{3}}c^{-\frac{5}{7}}t$;
\Else
\State $p_1:= x^7-b^{-7}c^{-15}t^{21}$;
\If{$p_1$ has a factor of degree $1$ over $\Q$}
\State MP := $x^3-b^{-1}c^{-\frac{15}{7}}t^{3}$;
\Else
\State $p_2:= x^3-b^{-7}c^{-15}t^{21}$;
\If{$p_2$ has a factor of degree $1$ over $\Q$}
\State MP := $x^7-b^{-\frac{7}{3}}c^{-5}t^7$;
\Else
\State MP $:=p$;
\EndIf
\EndIf
\EndIf
\State List 1 := type, NF, MP, Emptyset;
\Return (List 1)
\end{algorithmic}
\end{algorithm}


\begin{algorithm}[ht]
\caption{Algorithm for the case $E_{13}$}%
\label{alg:E_13}
\begin{algorithmic}[1]

\Require{$g \in \m^3\subset\Q[x,y]$ of complex singularity type $E_{13}$.}

\Ensure{A list with the following entries: [1] the real singularity type of $g$; [2] the normal form of $g$; [3] a minimal polynomial over $\Q$ for the parameter in the normal form; [4] if the parameter cannot be determined by its minimal polynomial alone, a list with entries: [1] a lower and/or [2] an upper bound of the parameter.}

\State $s := $ coefficient of the $x^3$ term in $g$;
\If{$s=0$}
\State Apply $x\mapsto y$, $y\mapsto x$ in $g$;
\EndIf
\State $h := \jet(g,3)$;
\State Factorize $h$ as $bh_1^3$, $b\in\Q$;
\State Apply $h_1\mapsto x$, $y\mapsto y$ to $g$;
\State Write $g$ as $bx^3+R$,\quad $b\in\Q$,\quad $R\in E_4$;
\State $h_1:= b^{-1}g$;
\State $h_2:= \frac{\jet(h_1,4)-x^3}{3x^2}$;
\State Apply $x\mapsto x-h_2$, $y\mapsto y$ to $h_1$;
\State $h_3:=\frac{\jet(h_1,5)-x^3}{3x^2}$;
\State Apply $x\mapsto x-h_3$, $y\mapsto y$ to $h_1$;
\State $g:=b\cdot h_1$;
\State Write $g$ as $bx^3+cxy^5+R$,\quad $b\in\Q$, \quad $R\in E_{11}^{(5,2)}$;
\State Apply $x\mapsto c^{-1}x$, $y\mapsto y$ to $g$;
\If{$b<0$}
\State Apply $x\mapsto -x$, $y\mapsto -y$;
\EndIf
\State Write $g$ as $bx^3+xy^5+R$,\quad $b\in\Q$,\quad $R\in E_{11}^{(5,2)}$;
\State Replace $x^2y^3$ with $(-3b)^{-1}y^8$;
\State $t := $ coefficient of the $y^8$ term in $g$;
\State NF $:= x^3+xy^5+ay^8$;
\State type $:= E_{13}$;
\State $p:= x^{15}-b^{8}t^{15}$;
\If{$p$ factorize over $\Q$}
\State MP $:= x-b^{\frac{8}{15}}t$;
\Else
\State $p_1:= x^3-b^8t^{15}$;
\If{$p_1$ has a factor of degree $1$ over $\Q$}
\State MP := $x^5-b^{\frac{8}{3}}t^5$;
\Else
\State $p_2:= x^5-b^8t^{15}$;
\If{$p_2$ has a factor of degree $1$}
\State MP := $x^3-b^{\frac{8}{5}}t^3$;
\Else
\State MP $:=p$;
\EndIf
\EndIf
\EndIf
\State List 1 := type, NF, MP, Emptyset;
\Return (List 1)
\end{algorithmic}
\end{algorithm}

\begin{algorithm}[ht]
\caption{Algorithm for the case $E_{14}$}%
\label{alg:E_14}
\begin{algorithmic}[1]

\Require{$g \in \m^3\subset\Q[x,y]$ of complex singularity type $E_{14}$.}

\Ensure{A list with the following entries: [1] the real singularity type of $g$; [2] the normal form of $g$; [3] a minimal polynomial over $\Q$ for the parameter in the normal form; [4] if the parameter cannot be determined by its minimal polynomial alone, a list with entries: [1] a lower and/or [2] an upper bound of the parameter.}

\State $s := $ coefficient of the $x^3$ term in $g$;
\If{$s=0$}
\State Apply $x\mapsto y$, $y\mapsto x$ in $g$;
\EndIf
\State $h := \jet(g,3)$;
\State Factorize $h$ as $bh_1^3$, $b\in\Q$;
\State Apply $h_1\mapsto x$, $y\mapsto y$ to $g$;
\State Write $g$ as $bx^3+R$,\quad $b\in\Q$,\quad $R\in E_4$;
\State $h_1:= b^{-1}g$;
\State $h_2:= \frac{\jet(h_1,4)-x^3}{3x^2}$;
\State Apply $x\mapsto x-h_2$, $y\mapsto y$ to $h_1$;
\State $h_3:=\frac{\jet(h_1,5)-x^3}{3x^2}$;
\State Apply $x\mapsto x-h_3$, $y\mapsto y$ to $h_1$;
\State $g:=b\cdot h_1$;
\State Write $g$ as $bx^3+cy^8+R$,\quad $b\in\Q$, \quad $R\in E_{25}^{(8,3)}$;
\If{$b<0$}
\State Apply $x\mapsto -x$, $y\mapsto y$ to $g$;
\EndIf
\State Write $g$ as $bx^3+cy^8+R$,\quad $b\in\Q$, \quad $R\in E_{25}^{(8,3)}$;
\If{$c<0$}
\State NF $:= x^3+y^8+axy^6$;
\State type $:= E_{14}^-$;
\Else
\State NF $:= x^3+y^8+axy^6$;
\State type $:= E_{14}^+$;
\EndIf
\algstore{E_14}
\end{algorithmic}
\end{algorithm}
\begin{algorithm}[ht]
\begin{algorithmic}[1]
\algrestore{E_14}
\State $t:= $ coefficient of the $xy^6$ term in $g$;
\State $p:= x^{12}-b^{-4}|c|^{-9}t^{12}$;
\If{$p$ has a factor of degree $1$ over $\Q$}
\State MP := $x-b^{-\frac{1}{3}}|c|^{-\frac{3}{4}}t$;
\Else
\State $p_1:=x^4-b^{-4}|c|^{-9}t^{12}$
\If{$p_1$ has a factor of degree $1$ over $\Q$}
\State MP := $x^3-b^{-1}|c|^{-\frac{9}{4}}t^3$;
\Else
\If{$t>0$}
\State List 1a := 0, Emptyset;
\Else
\State List 1a := Emptyset, 0;
\EndIf
\State $p_3:= x^6-b^{-4}|c|^{-9}t^{12}$;
\If{$p_3$ has a factor of degree $1$ over $\Q$}
\State MP := $x^2-b^{-\frac{4}{6}}|c|^{-\frac{3}{2}}t^2$;
\Else
\State $p_4:=x^3-b^{-4}|c|^{-9}t^{12}$;
\If{$p_4$ has a factor of degree $1$ over $\Q$}
\State MP $:= x^4-b^{-\frac{4}{3}}|c|^{-\frac{9}{3}}t^4$;
\Else
\State $p_5:= x^2-b^{-4}|c|^{-9}t^{12}$;
\If{$p_5$ has a factor of degree $1$ over $\Q$}
\State MP := $x^6-b^{-2}|c|^{-\frac{9}{2}}t^6$;
\Else
\State MP := $p$;
\EndIf
\EndIf
\EndIf
\EndIf
\EndIf
\State List 1 := type, NF, MP, List 1a;
\Return (List 1)
\end{algorithmic}
\end{algorithm}

\section{Real hyperbolic singularities of corank 2}
The following result can be found in \citet{PdJ2000} for $\K=\C$. The proof for
the $\K=\R$ case is similar.

\begin{theorem}\label{faces}
Let $\K$ be either $\R$ or $\C$ and let $f\in\K[[x,y]]$ be convenient, let
$\Delta_1,\ldots,\Delta_r$ be the faces of the Newton polygon of $f$ and $d_i$
the slope of $\Delta_i$. Then $f=f_1\cdots f_r$, where $f_i\in\K[[x,y]]$ is
convenient such that the Newton Polygon of $f_i$ has only one face of slope
$d_i$, $i=1,\ldots,r$.
\end{theorem}

The following result is proved for $a\ge 4$, $b\ge 5$ and $\K=\C$ by
\citet{A1974}.

\begin{lemma}\label{principalpart}
Let $\K$ be either $\R$ or $\C$. For every nondegenerate power series
$f\in\K[[x,y]]$ with principal part $f_0=x^a+\lambda x^2y^2+y^b$, where
$0\neq\lambda\in\K$ and $a,b\in\N$, such that $f_0$ has two faces,
$f\overset{\K}\sim f_0$.
\end{lemma}
\begin{proof}
Using Theorem \ref{faces} and the form of $f_0=x^a+\lambda x^2y^2+y^b$, it
follows that
\begin{equation}\label{twofaces}
f=(c_1x^2+c_3y^{b-2}+xh_1+y^ch_3)(c_2y^2+c_4x^{a-2}+yh_2+x^dh_4),
\end{equation}
where $h_1,\ldots,h_4\in\K[[x,y]]$, $1=c_1c_4, \lambda=c_1c_2, 1=c_3c_2$,
$c>b-2$ and $d>a-2$. Since $f$ is finite determined, after repeatedly applying
$x\mapsto x-\frac{1}{2c_1}h_1$, $y\mapsto y-\frac{1}{2c_2}h_2$ followed by
$x\mapsto x-\frac{1}{c_4(a-2)}x^{a-d-2}h_4$,
$y\mapsto y-\frac{1}{c_3(b-2)}y^{b-c-2} h_3$, writing $f$ as
in~(\ref{twofaces}), adapting $h_1,\ldots,h_4$, $c$ and $d$ ($c_1,\ldots,c_4$
do not change) accordingly, after each application, we have that
\begin{equation}
f\overset{\K}\sim (c_1x^2+c_3y^{b-2})(c_2y^2+c_4x^{a-2})
=x^a+\lambda x^2y^2+y^b+2\lambda_1x^{c'}y^{d'}+E_{c'+d'},\label{firsttrans}
\end{equation}
where $\lambda_1=c_3c_4$, $a\le c'\in\N$, $b\le d'\in\N$. Note that, since
$w\dash\deg(h_1)\ge b-2$, where $w=(b-2,2)$, it follows after each application
that $c>b-2$ and similarly that $d>a-2$, i.e. $f_0$ stays unaffected. By now
repeatedly applying $x\mapsto x+\frac{1}{2\lambda_1}y^{d'-1}$,
$y\mapsto y-\frac{1}{2\lambda_1}x^{c'-1}$, writing $f$ as in
(\ref{firsttrans}), adapting $c'$ and $d'$ accordingly, after each application,
using the fact that $f$ is finite determined, it follows that
\[f\overset{\K}\sim \lambda x^2y^2+x^a+y^b.\qedhere\]
\end{proof}

The following method can be used to determine the equivalence class of an input polynomial $g\in\m^3\subset\R[[x,y]]$ of hyperbolic type:

\begin{enumerate}
\item Transform $g$ such that there are no terms underneath the newton polygon created by the normal form of its corresponding complex type, which we will refer to as the newton polygon in the next steps.
\item Determine which real case $g$ is, by considering the terms on the newton polygon.
\item Throw all the terms not on the newton polygon away.
\item Scale the terms on the newton polygon and reaf off the value of the parameter.
\item Check in the tables included \cite{realclassify2} which other normal forms are in the same equivalence class as the one determined.
\end{enumerate}

To apply the above method to the $X_{9+k}$ case, using a computer and thus working over $\Q$, we need the following prelimanary result.

\begin{lemma}\label{J10+kfactorization}
If $f\in\Q[x,y]$ is homogeneous and factorizes as $g_1^2(g_2)$ over $\R$, where $g_1$ is a polynomials of degree $1$ and $g_2$ is a polynomial of degree $2$ (which can possibly factorize over $\R$), then $f=ag_1'^2g_2'$, where $g_1'$ is a polynomial of degree $1$ over $\Q$, $g_2$ is a polynomial of degree 2 over $\Q$ and $a\in\Q$.
\end{lemma}

\begin{proof}
Let $f=(a_1x+a_2y)^2(a_3x^2+a_4xy+a_5y^2)$, $a_1,a_2,a_3,a_4,a_5\in\R$. Since the coefficient of the $x^4$ term in $f$ is a rational number it follows that 
\[(x+\frac{a_2}{a_1}y)^2(x^2+\frac{a_4}{a_3}xy+\frac{a_5}{a_3}y^2)\in\Q[x,y].\]
Since $\Q$ is a perfect field
$x+\frac{a_1}{a_2}y\in\Q[x,y]$. Therefore $f=ag_1'^2g_2'$, where
$g_1'=x+\frac{a_2}{a_1}y$, $g_2'=x^2+\frac{a_4}{a_3}xy+\frac{a_5}{a_3}y^2$ and $a=a_1^2a_3$.
\end{proof}

\begin{algorithm}[ht]
\caption{Algorithm for the case $X_{9+k}$}%
\label{alg:X_{9+k}}
\begin{algorithmic}[1]

\Require{$g \in \m^3\subset\Q[x,y]$ of complex singularity type $X_{9+k}$.}

\Ensure{A list with the following entries: [1] the real singularity type of $g$; [2] the normal form of $g$; [3] a minimal polynomial over $\Q$ for the parameter in the normal form; [4] if the parameter cannot be determined by its minimum polynomial alone, a list with entries: [1] a lower and/or [2] an upper bound of the parameter.}

\State Write $g$ as $t_0x^4+t_1x^3y+t_2x^2y^2+t_3xy^3+t_4y^4+R$, $t_0,\ldots,t_4\in\Q$, $R\in E_5$;
\If{($t_4\neq 0$ and $t_0=0$)}
\State Apply $x\mapsto y$, $y\mapsto x$ to $g$;
\EndIf
\If{($t_4=0$ and $t_0=0$)}
\If{$t_1+t_2+t_3=0$}
\If{$2t_1+4t_2+8t_3\neq 0$}
\State Apply $x\mapsto x$, $y\mapsto 2y$ to $g$;
\Else
\State Apply $x\mapsto x$, $y\mapsto 3y$ to $g$;
\EndIf
\State Apply $x\mapsto x$, $y\mapsto x+y$ to $g$;
\EndIf
\EndIf
\State Write $g$ as $a_0x^4+a_1x^3y+a_2x^2y^2+a_3xy^3+a_4y^4+R$, $a_0,\ldots,a_4\in\Q$, $R\in E_5$;
\State $h:=\jet(g,4)$;
\State factorize $h$ and let $g_1$ be the factor of $h$ of multiplicity $2$;
\State Apply $g_1\mapsto x$, $y\mapsto y$ to $g$;
\State Write $g$ as $a_0x^4+a_1x^3y+a_2x^2y^2+R$, $a_0,a_1,a_2\in\Q$, $R\in E_5$;
\State Apply $y\mapsto y-\frac{a_1}{2a_2}x$, $x\mapsto x$ to $g$;
\State $t_1 := $coefficient of the $x^2y^2$ term in $g$;
\For{$i=9;\ i\le 2(\mu-5);\ i++$}
\State $t := $ coefficient of the $xy^{\frac{i-(\mu-7)}{2}}$ term in $g$;
\State Apply $x\mapsto x-\frac{t}{2t_1}y^{\frac{i-(\mu-7)}{2}-2}$, $y\mapsto y$ to $g$;
\EndFor
\State Write $g$ as $b_0x^4+b_1x^2y^2+b_2y^{\mu-5}$, $b_0,b_1,b_2\in\Q$, $R\in E_7^{(1,2)}\cap E_{2\mu-9}^{(\mu-7,2)}$;
\If{$b_0>0$}
\If{$b_1>0$}
\State NF $:= x^4+x^2y^2+ay^{\mu-5}$;
\State type $:= X_{9+k}^{++}$;
\Else
\State NF $:= x^4-x^2y^2+ay^{\mu-5}$;
\State type $:= X_{9+k}^{+-}$;
\EndIf
\Else
\If{$b_1>0$}
\State NF $:= -x^4+x^2y^2+ay^{5\mu-9}$;
\State type $:= X_{9+k}^{-+}$;
\Else
\State NF $:= -x^4-x^2y^2+ay^{\mu-5}$;
\State type $:= X_{9+k}^{--}$;
\EndIf
\EndIf
\algstore{X_9+k}
\end{algorithmic}
\end{algorithm}
\begin{algorithm}[ht]
\begin{algorithmic}[1]
\algrestore{X_9+k}
\State $p:= x^4-b_2^4\left(\frac{b_0}{b_1^{2}}\right)^{\mu-5}$;
\If{$p$ has a factor of degree $1$ over $\Q$}
\State MP := $x-b_2\left(\frac{b_0}{b_1^{2}}\right)^{\frac{\mu-5}{4}}$;
\Else
\If{$t>0$}
\State List 1a := 0, Emptyset;
\Else
\State List 1a := Emptyset, 0;
\EndIf
\State $p_1:=x^2-b_2^4\left(\frac{b_0}{b_1^{2}}\right)^{\mu-5}$;
\If{$p_1$ has a factor of degree $1$ over $\Q$}
\State MP := $x^2-b_2^2\left(\frac{b_0}{b_1^{2}}\right)^{\frac{\mu-5}{2}}$;
\Else
\State MP := $p$;
\EndIf
\EndIf
\State List 1 := type, NF, MP, List 1a;
\Return (List 1)
\end{algorithmic}
\end{algorithm}

For the $J_{10+k}$ case we need the following additional Lemma.

\begin{lemma}
Suppose $f(x)=ax^3+bx+c\in\Q[x]$, $a,b,c\in\Q$, has a double root $q$. Then $q\in\Q$.
\end{lemma}
\begin{proof}
If $p$ is a double root of $f$, then $f(q)=aq^3+bq+c=0$ and $f'(q)=3aq^2+b=0$. Hence $f(q)=q(aq^2+b)+c=q(a(\frac{-b}{3a}+b)+c=0$ which implies that $q=-\frac{3c}{2b}$.
\end{proof}


\begin{algorithm}[ht]
\caption{Algorithm for the case $J_{10+k}$}%
\label{alg:J_{10+k}}
\begin{algorithmic}[1]

\Require{$g \in \m^3\subset\Q[x,y]$ of complex singularity type $J_{10+k}$.}

\Ensure{A list with the following entries: [1] the real singularity type of $g$; [2] the normal form of $g$; [3] a minimal polynomial for the parameter in the normal form; [4] if the parameter cannot be determined by its minimum polynomial alone, a list with entries: [1] a lower and/or [2] an upper bound of the parameter.}
 
\State $s := $ coefficient of $x^3$ in $g$;
\If{$s=0$}
\State Apply $x\mapsto y$, $y\mapsto x$ to $g$;
\EndIf
\State $h:=\jet(g,3)$;
\State factorize $h$ as $bg_1^3$, $b\in\Q$, $b>0$;
\State Write $g$ as $bx^3+cx^2y^2+dxy^4+ey^6+R$, $b,c,d\in\Q$, $R\in E_7$;
\State Apply $x\mapsto x+(3b)^{-1}cy^2$, $y\mapsto y$ to $g$;
\State Write $g$ as $bx^3+dxy^4+ey^6+R$, $b,d,e\in\Q$, $R\in E_7$;
\State $q:= $ double root of $f(x)=bx^3+dx+e$;
\State $t_1 := $coefficient of the $x^2y^2$ term in $g$; 
\For{$i=13;\  i\le 2(\mu-4);\ i++$}
\State $t:= $ coefficient of the $xy^{\frac{(i-(\mu-6))}{2}}$ term in $g$;
\State Apply $x\mapsto x-\frac{t}{2t_1}y^{\frac{(i-(\mu-6))}{2}-2}$, $y\mapsto y$ to $g$;
\EndFor
\State Write $g$ as $bx^3+cx^2y^2+ty^{\mu-4}+R$, $b,c,t\in\Q$, $R\in E_7^{(1,2)}\cap E_{2\mu-7}^{(\mu-6,2)}$;
\If{$c>0$}
\State NF := $x^3+x^2y^2+ay^{\mu-4}$;
\State type := $J_{10+k}^+$;
\Else
\State NF := $x^3-x^2y^2+ay^{\mu-4}$;
\State type := $J_{10+k}^-$;
\EndIf
\State $p:= x^6-\frac{b^2}{|c|^{3}}^{\mu-4}t^6$;
\If{$p$ has a factor of degree $1$ over $\Q$}
\State MP := $x-\frac{b^2}{|c|^{3}}^{\frac{\mu-4}{6}}t$;
\Else
\State $p_1 := x^2-\frac{b^2}{|c|^{3}}^{\mu-4}t^6$;
\If{$p_1$ has a factor of degree $1$ over $\Q$}
\State MP := $x^3-\frac{b^2}{|c|^{3}}^{\frac{\mu-4}{2}}t^3$;
\Else
\If{$t>0$}
\State List 1a := 0, Emptyset;
\Else
\State List 1a := Emptyset, 0;
\EndIf
\State $p_2 := x^3-\frac{b^2}{|c|^{3}}^{\mu-4}t^6$;
\If{$p_2$ has a factor of degree $1$ over $\Q$}
\State MP := $x^2-\frac{b^2}{|c|^{3}}^{\frac{\mu-4}{3}}t^2$;
\Else
\State MP := $p$;
\EndIf
\EndIf
\EndIf
\State List 1 := type, NF, MP, List 1a;
\Return (List 1)
\end{algorithmic}
\end{algorithm}
\newpage

\begin{thebibliography}{99}

\bibitem[{Arnold(1974)}]{A1974}
Arnold, V.I., 1974.
Normal forms of functions in neighbourhoods of degenerate critical points.
Russ. Math. Surv. 29(2), 10-50.

\bibitem[{Arnold et al.(1985)}]{AVG1985}
Arnold, V.I., Gusein-Zade, S.M., Varchenko, A.N., 1985.
Singularities of Differential Maps, Vol.~I.
Birkh\"auser, Boston.

\bibitem[{Decker et al.(2012)}]{DGPS}
Decker, W., Greuel, G.-M., Pfister, G., Sch{\"o}nemann, H., 2012.
\newblock {\sc Singular} {3-1-6} -- {A} computer algebra system for polynomial
computations. \\
\url{http://www.singular.uni-kl.de}

\bibitem[{de Jong and Pfister(2000)}]{PdJ2000}
de Jong, T., Pfister, G., 2000.
Local Analytic Geometry.
Vieweg, Braunschweig.

\bibitem[{Greuel et al.(2007)}]{GLS2007}
Greuel, G.-M., Lossen, C., Shustin, E., 2007.
Introduction to Singularities and Deformations.
Springer, Berlin.

\bibitem[{Greuel and Pfister(2008)}]{GP2008}
Greuel, G.-M., Pfister, G., 2008.
A Singular Introduction to Commutative Algebra, second ed.
Springer, Berlin.

\bibitem[{Kr\"uger(1997)}]{Kruger}
Kr\"uger, K., 1997.
Klassifikation von Hyperfl\"achensingularit\"aten.
Diploma Thesis, University of Kaiserslautern.
{\par\raggedright
\url{ftp://www.mathematik.uni-kl.de/pub/Math/Singular/doc/Papers/%
diplom_krueger.ps.gz}
\par}

\bibitem[{Kr\"uger(2012)}]{classify}
Kr\"uger, K., 2012.
{\tt classify.lib}. {A} {\sc Singular} {3-1-6} library for classifying isolated
hypersurface singularities w.r.t.\@ right equivalence, based on the
determinator of singularities by V.I. Arnold.

\bibitem[{Marais and Steenpa\ss(2012)}]{realclassify}
Marais, M., Steenpa\ss, A., 2012.
{\tt realclassify.lib}. {A} {\sc Singular} {3-1-6} library for classifying
isolated hypersurface singularities over the reals w.r.t.\@ right equivalence.

\bibitem[{Marais and Steenpa\ss(2012)}]{realclassify1}
Marais, M., Steenpa\ss, A., 2012. The classification of real singularities using \textsc{SINGULAR} Part I: Splitting Lemma and Simple Singularities.

\bibitem[{Marais and Steenpa\ss(2013)}]{realclassify2}
Marais, M., Steenpa\ss, A., 2012. The classification of real singularities using \textsc{SINGULAR} Part II: The Structure of the Equivalence Classes of the Unimodal Singularities.

\bibitem[{Siersma(1974)}]{Siersma}
Siersma, D., 1974.
Classification and Deformation of Singularities.
Dissertation, University of Amsterdam.
{\par\raggedright
\url{http://www.staff.science.uu.nl/~siers101/ArticleDownloads/%
DissertationSiersma.pdf}
\par}

\bibitem[{Tobis(2012)}]{roots}
Tobis, E., 2012.
{\tt rootsur.lib}. {A} {\sc Singular} {3-1-6} library for counting the number
of real roots of a univariate polynomial.

\end{thebibliography}

\end{docume