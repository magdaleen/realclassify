\documentclass[noend]{amsproc}

\renewcommand{\arraystretch}{1.3}

\usepackage{amsthm,amsmath,amsfonts,mathrsfs,amssymb,todonotes}
\usepackage{algorithm}
\usepackage{algorithmicx, algpseudocode}
\usepackage[T1]{fontenc}   % for bold \Singular
\usepackage{multirow}
\usepackage{natbib}
\usepackage{url}

\def\UrlFont{\fontfamily{lmtt}\selectfont}

\newtheorem{theorem}{Theorem}
\newtheorem{defn}[theorem]{Definition}
\newtheorem{prop}[theorem]{Proposition}
\newtheorem{lemma}[theorem]{Lemma}
\theoremstyle{definition}
\newtheorem{remark}[theorem]{Remark}

% ALGORITHM style
\renewcommand{\algorithmicrequire}{\textbf{Input:}}
\renewcommand{\algorithmicensure}{\textbf{Output:}}
\newcommand{\algorithmicbreak}{\textbf{break}}
\newcommand{\Break}{\State \algorithmicbreak}
\renewcommand{\algorithmicreturn}{\State \textbf{return}}

\newcommand{\Singular}{\textsc{Singular}}
\newcommand{\realclassify}{\texttt{realclassify.lib}}
\newcommand{\classify}{\texttt{classify.lib}}


\newcommand{\requiv}{\ensuremath{\mathrel{\overset{r}{\sim}}}}
\newcommand{\cequiv}{\ensuremath{\mathrel{\overset{c}{\sim}}}}
\newcommand{\sequiv}{\ensuremath{\mathrel{\overset{s}{\sim}}}}

\DeclareMathOperator{\size}{size}
\DeclareMathOperator{\MP}{MP}
\DeclareMathOperator{\type}{type}
\DeclareMathOperator{\val}{value}
\DeclareMathOperator{\ord}{ord}
\DeclareMathOperator{\m}{\mathfrak{m}}
\DeclareMathOperator{\jet}{jet}
\DeclareMathOperator{\corank}{corank}
\DeclareMathOperator{\supp}{supp}
\DeclareMathOperator{\sign}{sign}
\DeclareMathOperator{\diag}{diag}
\DeclareMathOperator{\NF}{NF}
\DeclareMathOperator{\N}{\mathbb{N}}
\DeclareMathOperator{\Q}{\mathbb{Q}}
\DeclareMathOperator{\R}{\mathbb{R}}
\DeclareMathOperator{\C}{\mathbb{C}}
\DeclareMathOperator{\K}{\mathbb{K}}
\DeclareMathOperator{\A}{\mathbb{A}}
\DeclareMathOperator{\Pj}{\mathbb{P}}
\DeclareMathOperator{\boldzero}{\mathbf{0}}

\hyphenation{equiva-lence}
\hyphenation{sin-gu-lar-ities}

\title[The Classification of Real Singularities Using \textsc{Singular}, %
Part III]%
{The Classification of Real Singularities Using \textsc{Singular}\\
Part III: Unimodal Singularities of Corank $2$}

\author{Magdaleen S. Marais}
\address{Magdaleen S. Marais\\
African Institute for Mathematical Sciences and Stellenbosch University\\
6 Melrose Rd\\
Muizenberg 7945, Cape Town\\
South Africa}
\email{magdaleen@aims.ac.za}

\author{Andreas Steenpa\ss}
\address{Andreas Steenpa\ss\\
Department of Mathematics\\
University of Kaiserslautern\\
Erwin-Schr\"odinger-Str.\\
67663 Kaiserslautern\\
Germany}
\email{steenpass@mathematik.uni-kl.de}

\thanks{This research was supported by the African Institute for Mathematical
Sciences and grants awarded by Gert-Martin Greuel and Wolfram Decker. We are thankful to all of
them.}

\keywords{%
hypersurface singularities, algorithmic classification, real geometry%
}

\begin{document}

\begin{abstract}
A non-article form draft of the third paper on the realclassification of singularities using  \textsc{Singular}.
\end{abstract}

\maketitle


\section{Introduction}
The algoritms constructed here gives for an arbitrary input polynomial $g\in\m^3\subset\Q[x,y]$ of, respectively, complex type \[J_{10}, J_{10+k}, X_{9+k}, Y_{r,s}, E_{12}, E_{13}, E_{14}, Z_{11}, Z_{12}, Z_{13}, W_{12}, W_{13}\] the following information:

\begin{itemize}
\item[1] The different real types to which $g$ is equivalent, with, respectively,
\item [1.1] the minimal polynomials $\MP$ of all the values of parameters, corresponding to each of the different normal forms of the type under consideration to which $g$ is equivalent and
\item[1.2] an interval for each parameter $a$ which determines which root of $\MP$ $a$ is.
\end{itemize}

Hence the algorithms determines exactly to which representatives of equivalence classes, given by the normal forms of Arnold, $g$ is equivalent. The representation of the parameters as roots of minimal polynomials enables the user to easily construct an extension field of $\Q$ in Singular in which the parameter exist. 
\section{Acknowledgements}


\section{Real exceptional $\mathbf 1$-modal singularities of corank $\mathbf 2$}
\label{ExceptionalSingularities}
In Lemma 7.3, p.27 of \citet{A1974}, it is shown that a semi-quasihomogeneous function $g\in\K[[x_1,\cdots,x_n]]$ with quasihomogeneous part $g_0$ of degree $d$ is right equivalent to a function of the form $g_0+\sum_{k=1}^s c_ke_k$, where $c_1,\ldots, c_s$ are constants and $e_1,\ldots,e_s$ is the set of all basis monomials in a fixed basis of the local algebra of the function $g_0$. This is done by systematically considering each diagonal with quasihomigeneous degree $d'>d$ on which at least one monomial of $g$ lie. Using the fact that the sum of the terms of degree $d'$ can be decomposed as
\[h:=\sum_{i=1}^n\frac{\partial g_0}{\partial x_i}v_i(x_i)+c_1e_1+\cdots+c_se_s,\]
it is shown that there is an automorphism $\phi_{d'}$, defined by $\phi_{d'}(x_i)=x_i-v_i(x_i)$, such that
\[\phi_{d'}(g)=g_0+[g_1+(c_1e_1+\cdots +c_se_s)-h]+R,\]
where the filtration of $R$ is greater than $d'$. By taking the finite determinacy of germs into account the goal is reached by this transformation. What happens in practice is that $\phi_{d'}$ replace all the terms of weighted degree $d'$ in $g$ by a linear combination of $e_1,\ldots e_s$. For each $d'$, $\phi_{d'}$ can, for simplicity, as is done in the given method below, be replaced in the following way by the composition of two automorphisms. First the terms of degree $d'$ in $g$ not in the local algebra are considered, which practically results in throwing these terms away. And then, secondly, the terms in the local algebra of degree $d'$ are considered, which practically results in replacement by a linear combination of the chosen basis. Determining $\phi_{d'}$ in this way turns out to be simpler in practise.

Since a chosen basis is not necessarily unique, the monomials in a representation of the right equivalence class of $g$ as $g_0+\sum_{k=1}^s c_ke_k$ is also not necessarily unique. In \cite{A1974}, Arnold has chosen a specific basis for the local algebra in each of the normal forms he determined by this method. In each case that turns out to be the monomials with parameters as coefficients in the normal forms. Since we work with Arnold's classification we need to choose the same basis. For the exceptional cases the local algebra has only one basis element. In \cite{realclassify2} it follows from Table 10 that the parameter in each of the exceptional normal forms has a unique value. 

We use the following method to determine the equivalence class of an input poynomial $g\in\mathfrak m^3\subset\R[[x,y]]$ of exceptional type:

\begin{enumerate}
\item Transform $g$ such that there are no terms underneath the diagonal created by the quasihomogeneous part of the corresponding complex normal form, which we will refer to as the diagonal in the next steps.
\item Determine which case $g$ is by considering the terms on the diagonal.
\item Throw all terms not in the local algebra of $g_0$, above the diagonal away.
\item Replace terms in the local algebra not in Arnold's chosen basis by a linear combination of elements in his chosen basis.
\item Scale the terms on the diagonal and read the value off of the parameter.
\end{enumerate}

The results in section 2.4 of \cite{realclassify1} is used in all the Algorithms for classifying the unimodal, corank 2, singularities. We use the notation introduced in \cite{realclassify2}.
 
\begin{algorithm}[ht]
\caption{Algorithm for the case $Z_{11}$}%
\label{alg:Z_11}
\begin{algorithmic}[1]

\Require{$g \in \m^3\subset\Q[x,y]$ of complex singularity type $Z_{11}$.}

\Ensure{A list with the following entries: [1] the real singularity type of $g$; [2] the normal form of $g$; [3] a minimal polynomial over $\Q$ for the parameter in the normal form; [4] a list with entries: [1] a lower and [2] an upper bound of the parameter.}

\State $h:=\jet(g,4)$;
\State Factorize $h$ as $h_1h_2^3$, $h_1,h_2$ of degree one;
\State Apply $h_1\mapsto x$ and $h_2\mapsto y$ to $g$;
\State Write $g$ as $x^3y+by^5+cxy^4+R,\quad a,b,c\in\Q$,\quad $R\in E^{(4,3)}_{17}$;
\State $p:=x^{15}-c^{15}b^{-11}$;
\State $\MP := $ the factor of $p$ over $\Q$ with a real root;
\State type := $Z_{11}$;
\State NF := $x^3y+y^5+axy^4$;
\State List 1a := $-\infty,\infty$;
\State List 1 := type, NF, MP, List 1a;
\Return(List 1)
\end{algorithmic}
\end{algorithm}

Line 6 in  Algorithm \ref{alg:Z_11} can be determined in \textsc{Singular} using {\tt rootsur.lib} or using Algorithm \ref{alg:Z_112}:
\begin{algorithm}[ht]
\caption{Algorithm determining the minimum polynomial in $Z_{11}$}%
\label{alg:Z_112}
\begin{algorithmic}[1]
\Require{ $p:=x^{15}-c^{15}b^{-11}$, $c,b\in\Q$}
\Ensure{ minimum polynomial over $\Q$ containing the real root of $p$}
\If{$p$ has a degree $1$ factor over $\Q$}
\State MP := $x-cb^{-\frac{11}{15}}$;
\Else
\State $p_1:= x^5-c^{15}b^{-11}$
\If{$p_1$ has a degree $1$ factor over $\Q$}
\State MP := $x^3-c^3b^{-\frac{11}{5}}$;
\Else
\State $p_2:= x^3-c^{15}b^{-11}$;
\If{$p_2$ has a degree $1$ factor over $\Q$}
\State MP := $x^5-c^5b^{-\frac{11}{3}}$;
\Else
\State MP := $p$;
\EndIf
\EndIf
\EndIf
\Return(MP)
\end{algorithmic}
\end{algorithm}

Similar methods can be used to determine the minimal polynomials for the parameters in the rest of the cases.


\begin{algorithm}[ht]
\caption{Algorithm for the case $Z_{12}$}%
\label{alg:Z_12}
\begin{algorithmic}[1]

\Require{$g \in \m^3\subset\Q[x,y]$ of complex singularity type $Z_{12}$.}

\Ensure{A list with the following entries: [1] the real singularity type of $g$; [2] the normal form of $g$; [3] a minimal polynomial over $\Q$ for the parameter in the normal form; [4] a list with entries: [1]  a lower and [2] an upper bound of the parameter.}

\State $h:=\jet(g,4)$;
\State Factorize $h$ as $h_1h_2^3$, $h_1,h_2$ of degree one;
\State Apply $h_1\mapsto x$ and $h_2\mapsto y$ to $g$;
\State Write $g$ as $x^3y+bxy^4+cx^2y^3+dy^6+R,\quad a,b,c,d\in\Q$,\quad $R\in E^{(3,2)}_{13}$;
\State Replace $y^6$ with $-\frac{3}{b}x^2y^3$;
\State Write $g$ as $x^3y+bxy^4+cx^2y^3+R,\quad a,b,c\in\Q$,\quad $R\in E^{(3,2)}_{13}$;
\State $p:=x^{11}-c^{11}b^{-7}$;
\State MP := factor of $p$ over $\Q$ with a real root;
\State type := $Z_{12}$;
\State NF := $x^3y+xy^4+ax^2y^3$;
\State List 1a: = $-\infty, \infty$;
\State List 1 := type, NF, MP, List 1a;
\Return(List 1)

\end{algorithmic}
\end{algorithm}

\begin{algorithm}[ht]
\caption{Algorithm for the case $Z_{13}$}%
\label{alg:Z_13}
\begin{algorithmic}[1]

\Require{$g \in \m^3\subset\Q[x,y]$ of complex singularity type $Z_{13}$.}

\Ensure{A list with the following entries: [1] the real singularity type of $g$; [2] the normal form of $g$; [3] a minimal polynomial over $\Q$ for the parameter in the normal form; [4] a list with entries: [1] a lower and [2] an upper bound of the parameter.}

\State $h :=\jet(g,4)$;
\State Factorize $h$ as $h_1h_2^3$, $h_1,h_2$ of degree one;
\State Apply $h_1\mapsto x$ and $h_2\mapsto y$ to $g$;
\State Write $g$ as $x^3y+by^6+cx^2y^3+dxy^5+R,\quad a,b,c,d\in\Q$,\quad $R\in E_{21}^{(5,3)}$;
\If{$b>0$}
\State NF := $x^3y+y^6+axy^5$;
\State type := $Z_{13}^+$;
\Else
\State NF := $x^3y-y^6+axy^5$;
\State type := $Z_{13}^-$;
\EndIf
\State $p := x^9-d^9|b|^{-7}$; 
\State $\MP: =$ factor of $p$ over $\Q$ with a real root; 
%\If{$p$ has a degree $1$ factor over $\Q$}
%\State MP := $x-d|b|^\frac{-7}{9}$;
%\Else
%\State $p_1:=x^3-d^9|b|^{-7}$;
%\If{ $p_1$ has a degree $1$ factor over $\Q$}
%\State MP := $x^3-d^3|b|^{-\frac{7}{3}}$;
%\Else
%\State MP := $p$;
%\EndIf
%\EndIf
\State List 1a := $-\infty, \infty$;
\State List 1 := type, NF, MP, List 1a;
\Return(List 1)

\end{algorithmic}
\end{algorithm}

\begin{algorithm}[ht]
\caption{Algorithm for the case $W_{12}$}%
\label{alg:W_12}
\begin{algorithmic}[1]

\Require{$g \in \m^3\subset\Q[x,y]$ of complex singularity type $W_{12}$.}

\Ensure{A list with the following entries: [1] the real singularity type of $g$; [2] the normal form of $g$; [3] a minimal polynomial over $\Q$ for the parameter in the normal form; [4] a list with entries: [1] a lower and  [2] an upper bound of the parameter.}

\State $s := $ coefficient of the $x^4$ term in $g$;
\If{$s=0$}
\State Apply $x\mapsto y$, $y\mapsto x$ in $g$;
\EndIf
\State $h := \jet(g,4)$;
\State Factorize $h$ as $bh_1^4$,$h_1$ of degree one and $b\in\Q$;
\State Apply $h_1\mapsto x$, $y\mapsto y$ to $g$;
\State Write $g$ as $bx^4+cy^5+dx^2y^3+exy^4+R$,\quad $b,c,d,e\in\Q$,\quad$R\in E_{23}^{(5,4)}$;
\If{$c>0$}
\State Apply $x\mapsto x$, $y\mapsto -y$ to $g$;
\EndIf
\State Write $g$ as $bx^4+cy^5+dx^2y^3+exy^4+R$,\quad $b,c,d,e\in\Q$,\quad$R\in E_{23}^{(5,4)}$;
\If{$b>0$}
\State NF := $x^4+y^5+ax^2y^3$;
\State type := $W_{12}^+$;
\Else
\State NF := $-x^4+y^5+ax^2y^3$;
\State type := $W_{12}^-$;
\EndIf
\State $p := x^{10}-d^{10}|b|^{-5}c^{-6}$;
%\If{ $p$ has a degree $1$ factor over $\Q$}
%\State MP := $x-d|b|^\frac{-1}{2}c^\frac{-3}{5}$;
%\State List 1a := Emptyset;
%\Else
%\State $p_1:= x^2-d^{10}|b|^{-5}c^{-6}$;
%\If{$p_1$ has a factor of degree $1$ over $\Q$}
%\State MP := $x^5-d^5|b|^{-\frac{5}{2}}c^{-3}$;
%\State List 1a := Emptyset;
%\Else
\If{$d>0$}
\State $\MP :=$ factor of $p$ over $\Q$ with a positive real root;
\State List 1a : = $0,\infty$; 
\Else 
\State $\MP :=$ factor of $p$ over $\Q$ with a negative real root;
\State List 1a := $-\infty,0$;
\EndIf
%\State $p_2:= x^5-d^{10}|b|^{-5}c^{-6}$;
%\If{$p_2$ has a degree $1$ factor over $\Q$}
%\State MP := $x^2-d^2|b|^{-1}c^{-\frac{6}{5}}$;
%\Else
%\State MP := $p$;
%\EndIf
%\EndIf
%\EndIf
\State List 1 := type, NF, MP, List 1a;
\Return(List 1)
\end{algorithmic}
\end{algorithm}

\begin{algorithm}[ht]
\caption{Algorithm for the case $W_{13}$}%
\label{alg:W_13}
\begin{algorithmic}[1]

\Require{$g \in \m^3\subset\Q[x,y]$ of complex singularity type $W_{13}$.}

\Ensure{A list with the following entries: [1] the real singularity type of $g$; [2] the normal form of $g$; [3] a minimal polynomial over $\Q$ for the parameter in the normal form; [4] a list with entries: [1] a lower and [2] an upper bound of the parameter.}

\State $s := $ coefficient of the $x^4$ term in $g$;
\If{$s=0$}
\State Apply $x\mapsto y$, $y\mapsto x$ in $g$;
\EndIf
\State $h := \jet(g,4)$;
\State Factorize $h$ as $bh_1^4$, $h_1$ of degree one and $b\in\Q$;
\State Apply $h_1\mapsto x$, $y\mapsto y$ to $g$;
\State Write $g$ as $bx^4+cx^3y^2+dx^2y^3+exy^4+ky^6+R$,\quad $b,c,d,e,k\in\Q$,\quad $R\in E_{19}^{(4,3)}$;
\State Replace $x^3y^2$ with $-{4b}^{-1}y^6$;
\State Write $g$ as $bx^4+cxy^4+dy^6+ey^3x^2+R$,\quad $b,c,d,e\in\Q$,\quad $R\in E_{19}^{(4,3)}$;
\State Apply $x\mapsto c^{-1}x$, $y\mapsto y$ to $g$;
\State Write $g$ as $bx^4+xy^4+dy^6+ey^3x^2+R$,\quad $b,d,e\in\Q$,\quad $R\in E_{19}^{(4,3)}$;
\If{$b<0$}
\State NF $:= x^4+xy^4+ay^6$;
\State type $:= W_{13}^+$;
\Else
\State NF $:= -x^4+xy^4+ay^6$;
\State type $:= W_{13}^-$;
\EndIf
\State $p := x^8-d^8|b|^3$;
%\If{ $p$ factorize over $\Q$}
%\State MP := $x-d|b|^\frac{3}{8}$;
%\State List 1a := Emptyset;
%\Else
\If{d>0}
\State $\MP := $ factor of $p$ over $\Q$ with positive real root;
\State List 1a : = $0, \infty$; 
\Else 
\State $\MP := $ factor of $p$ over $\Q$ with negative real root;
\State List 1a := $-\infty, 0$;
\EndIf
%\State $p_1:= x^2-d^8|b|^3$;
%\If{$p_1$ has a factor of degree $1$ over $\Q$}
%\State MP := $x^4-d^4|b|^{\frac{3}{2}}$;
%\Else
%\State $p_2:= x^4-d^8|b|^3$;
%\If{$p_2$ has a factor of degree $1$ over $\Q$}
%\State MP := $x^2-d^2|b|^{\frac{3}{4}}$;
%\Else
%\State MP := $p$;
%\EndIf
%\EndIf
%\EndIf
\State List 1 := type, NF, MP, List 1a;
\Return(List 1)
\end{algorithmic}
\end{algorithm}



\begin{algorithm}[ht]
\caption{Algorithm for the case $E_{12}$}%
\label{alg:E_12}
\begin{algorithmic}[1]

\Require{$g \in \m^3\subset\Q[x,y]$ of complex singularity type $E_{12}$.}

\Ensure{A list with the following entries: [1] the real singularity type of $g$; [2] the normal form of $g$; [3] a minimal polynomial over $\Q$ for the parameter in the normal form; [4] a list with entries: [1] a lower and [2] an upper bound of the parameter.}

\State $s := $ coefficient of the $x^3$ term in $g$;
\If{$s=0$}
\State Apply $x\mapsto y$, $y\mapsto x$ in $g$;
\EndIf
\State $h := \jet(g,3)$;
\State Factorize $h$ as $bh_1^3$, $h_1$ of degree one and $b\in\Q$;
\State Apply $h_1\mapsto x$, $y\mapsto y$ to $g$;
\State Write $g$ as $bx^3+R$,\quad $b\in\Q$,\quad $R\in E_4$;
\State $h_1:= b^{-1}g$;
\State $h_2:= \frac{\jet(h_1,4)-x^3}{3x^2}$;
\State Apply $x\mapsto x-h_2$, $y\mapsto y$ to $h_1$;
\State $h_3:=\frac{\jet(h_1,5)-x^3}{3x^2}$;
\State Apply $x\mapsto x-h_3$, $y\mapsto y$ to $h_1$;
\State $g:=b\cdot h_1$;
\State Write $g$ as $bx^3+cy^7+R$,\quad $b\in\Q$, \quad $R\in E_{22}^{(7,3)}$;
\State $t := $ coefficient of the $xy^5$ term in $g$;
\State NF $:= x^3+y^7+axy^5$;
\State type $:= E_{12}$;
\State $p:= x^{21}-b^{-7}c^{-15}t^{21}$;
%\If{$p$ has a factor of degree $1$ over $\Q$}
%\State MP $:= x-b^{-\frac{1}{3}}c^{-\frac{5}{7}}t$;
%\Else
%\State $p_1:= x^7-b^{-7}c^{-15}t^{21}$;
%\If{$p_1$ has a factor of degree $1$ over $\Q$}
%\State MP := $x^3-b^{-1}c^{-\frac{15}{7}}t^{3}$;
%\Else
%\State $p_2:= x^3-b^{-7}c^{-15}t^{21}$;
%\If{$p_2$ has a factor of degree $1$ over $\Q$}
%\State MP := $x^7-b^{-\frac{7}{3}}c^{-5}t^7$;
%\Else
%\State MP $:=p$;
%\EndIf
%\EndIf
%\EndIf
\State $\MP := $ factor of $p$ over $\Q$ with a real root;
\State List 1a := $-\infty, \infty$
\State List 1 := type, NF, MP, List 1a;
\Return (List 1)
\end{algorithmic}
\end{algorithm}


\begin{algorithm}[ht]
\caption{Algorithm for the case $E_{13}$}%
\label{alg:E_13}
\begin{algorithmic}[1]

\Require{$g \in \m^3\subset\Q[x,y]$ of complex singularity type $E_{13}$.}

\Ensure{A list with the following entries: [1] the real singularity type of $g$; [2] the normal form of $g$; [3] a minimal polynomial over $\Q$ for the parameter in the normal form; [4] a list with entries: [1] a lower and [2] an upper bound of the parameter.}

\State $s := $ coefficient of the $x^3$ term in $g$;
\If{$s=0$}
\State Apply $x\mapsto y$, $y\mapsto x$ in $g$;
\EndIf
\State $h := \jet(g,3)$;
\State Factorize $h$ as $bh_1^3$, $h_1$ of degree one and $b\in\Q$;
\State Apply $h_1\mapsto x$, $y\mapsto y$ to $g$;
\State Write $g$ as $bx^3+R$,\quad $b\in\Q$,\quad $R\in E_4$;
\State $h_1:= b^{-1}g$;
\State $h_2:= \frac{\jet(h_1,4)-x^3}{3x^2}$;
\State Apply $x\mapsto x-h_2$, $y\mapsto y$ to $h_1$;
\State $h_3:=\frac{\jet(h_1,5)-x^3}{3x^2}$;
\State Apply $x\mapsto x-h_3$, $y\mapsto y$ to $h_1$;
\State $g:=b\cdot h_1$;
\State Write $g$ as $bx^3+cxy^5+R$,\quad $b\in\Q$, \quad $R\in E_{11}^{(5,2)}$;
\State Apply $x\mapsto c^{-1}x$, $y\mapsto y$ to $g$;
\If{$b<0$}
\State Apply $x\mapsto -x$, $y\mapsto -y$;
\EndIf
\State Write $g$ as $bx^3+xy^5+R$,\quad $b\in\Q$,\quad $R\in E_{11}^{(5,2)}$;
\State Replace $x^2y^3$ with $(-3b)^{-1}y^8$;
\State $t := $ coefficient of the $y^8$ term in $g$;
\State NF $:= x^3+xy^5+ay^8$;
\State type $:= E_{13}$;
\State $p:= x^{15}-b^{8}t^{15}$;
%\If{$p$ factorize over $\Q$}
%\State MP $:= x-b^{\frac{8}{15}}t$;
%\Else
%\State $p_1:= x^3-b^8t^{15}$;
%\If{$p_1$ has a factor of degree $1$ over $\Q$}
%\State MP := $x^5-b^{\frac{8}{3}}t^5$;
%\Else
%\State $p_2:= x^5-b^8t^{15}$;
%\If{$p_2$ has a factor of degree $1$}
%\State MP := $x^3-b^{\frac{8}{5}}t^3$;
%\Else
%\State MP $:=p$;
%\EndIf
%\EndIf
%\EndIf
\State $\MP := $ factor of $p$ over $\Q$ with a real root;
\State List 1a := $-\infty, \infty$;
\State List 1 := type, NF, MP, List 1a;
\Return (List 1)
\end{algorithmic}
\end{algorithm}

\begin{algorithm}[ht]
\caption{Algorithm for the case $E_{14}$}%
\label{alg:E_14}
\begin{algorithmic}[1]

\Require{$g \in \m^3\subset\Q[x,y]$ of complex singularity type $E_{14}$.}

\Ensure{A list with the following entries: [1] the real singularity type of $g$; [2] the normal form of $g$; [3] a minimal polynomial over $\Q$ for the parameter in the normal form; [4] a list with entries: [1] a lower and [2] an upper bound of the parameter.}

\State $s := $ coefficient of the $x^3$ term in $g$;
\If{$s=0$}
\State Apply $x\mapsto y$, $y\mapsto x$ in $g$;
\EndIf
\State $h := \jet(g,3)$;
\State Factorize $h$ as $bh_1^3$, $h_1$ of degree one and $b\in\Q$;
\State Apply $h_1\mapsto x$, $y\mapsto y$ to $g$;
\State Write $g$ as $bx^3+R$,\quad $b\in\Q$,\quad $R\in E_4$;
\State $h_1:= b^{-1}g$;
\State $h_2:= \frac{\jet(h_1,4)-x^3}{3x^2}$;
\State Apply $x\mapsto x-h_2$, $y\mapsto y$ to $h_1$;
\State $h_3:=\frac{\jet(h_1,5)-x^3}{3x^2}$;
\State Apply $x\mapsto x-h_3$, $y\mapsto y$ to $h_1$;
\State $g:=b\cdot h_1$;
\State Write $g$ as $bx^3+cy^8+R$,\quad $b\in\Q$, \quad $R\in E_{25}^{(8,3)}$;
\If{$b<0$}
\State Apply $x\mapsto -x$, $y\mapsto y$ to $g$;
\EndIf
\State Write $g$ as $bx^3+cy^8+R$,\quad $b\in\Q$, \quad $R\in E_{25}^{(8,3)}$;
\If{$c<0$}
\State NF $:= x^3+y^8+axy^6$;
\State type $:= E_{14}^-$;
\Else
\State NF $:= x^3+y^8+axy^6$;
\State type $:= E_{14}^+$;
\EndIf

\State $t:= $ coefficient of the $xy^6$ term in $g$;
\State $p:= x^{12}-b^{-4}|c|^{-9}t^{12}$;
%\If{$p$ has a factor of degree $1$ over $\Q$}
%\State MP := $x-b^{-\frac{1}{3}}|c|^{-\frac{3}{4}}t$;
%\State List 1a := Emptyset;
%\Else
%\State $p_1:=x^4-b^{-4}|c|^{-9}t^{12}$
\%If{$p_1$ has a factor of degree $1$ over $\Q$}
\%State MP := $x^3-b^{-1}|c|^{-\frac{9}{4}}t^3$;
%\State List 1a := Emptyset;
%\Else
\If{$t>0$}
\State $\MP :=$ factor of $p$ over $\Q$ with positive real root;
\State List 1a := $0,\infty$;
\Else
\State $\MP := $ factor of $p$ over $\Q$ with negative real root;
\State List 1a := $-\infty, 0$;
\EndIf
%\State $p_3:= x^6-b^{-4}|c|^{-9}t^{12}$;
%\If{$p_3$ has a factor of degree $1$ over $\Q$}
%\State MP := $x^2-b^{-\frac{4}{6}}|c|^{-\frac{3}{2}}t^2$;
%\algstore{E_14}
%\end{algorithmic}
%\end{algorithm}
%\begin{algorithm}[ht]
%\begin{algorithmic}[1]
%\algrestore{E_14}
%\Else
%\State $p_4:=x^3-b^{-4}|c|^{-9}t^{12}$;
%\If{$p_4$ has a factor of degree $1$ over $\Q$}
%\State MP $:= x^4-b^{-\frac{4}{3}}|c|^{-\frac{9}{3}}t^4$;
%\Else
%\State $p_5:= x^2-b^{-4}|c|^{-9}t^{12}$;
%\If{$p_5$ has a factor of degree $1$ over $\Q$}
%\State MP := $x^6-b^{-2}|c|^{-\frac{9}{2}}t^6$;
%\Else
%\State MP := $p$;
%\EndIf
%\EndIf
%\EndIf
%\EndIf
%\EndIf
\State List 1 := type, NF, MP, List 1a;
\Return (List 1)
\end{algorithmic}
\end{algorithm}

\clearpage


\section{Real hyperbolic singularities of corank 2}
The following result can be found in \citet{PdJ2000} for $\K=\C$. The proof for
the $\K=\R$ case is similar.

\begin{theorem}\label{faces}
Let $\K$ be either $\R$ or $\C$ and let $f\in\K[[x,y]]$ be convenient, let
$\Delta_1,\ldots,\Delta_r$ be the faces of the Newton polygon of $f$ and $d_i$
the slope of $\Delta_i$. Then $f=f_1\cdots f_r$, where $f_i\in\K[[x,y]]$ is
convenient such that the Newton Polygon of $f_i$ has only one face of slope
$d_i$, $i=1,\ldots,r$.
\end{theorem}

The following result is proved for $a\ge 4$, $b\ge 5$ and $\K=\C$ by
\citet{A1974}.

\begin{lemma}\label{principalpart}
Let $\K$ be either $\R$ or $\C$. For every nondegenerate power series
$f\in\K[[x,y]]$ with principal part $f_0=x^a+\lambda x^2y^2+y^b$, where
$0\neq\lambda\in\K$ and $a,b\in\N$, such that $f_0$ has two faces,
$f\sim f_0$.
\end{lemma}
\begin{proof}
Using Theorem \ref{faces} and the form of $f_0=x^a+\lambda x^2y^2+y^b$, it
follows that
\begin{equation}\label{twofaces}
f=(c_1x^2+c_3y^{b-2}+xh_1+y^ch_3)(c_2y^2+c_4x^{a-2}+yh_2+x^dh_4),
\end{equation}
where $h_1,\ldots,h_4\in\K[[x,y]]$, $1=c_1c_4, \lambda=c_1c_2, 1=c_3c_2$,
$c>b-2$ and $d>a-2$. Since $f$ is finite determined, after repeatedly applying
$x\mapsto x-\frac{1}{2c_1}h_1$, $y\mapsto y-\frac{1}{2c_2}h_2$ followed by
$x\mapsto x-\frac{1}{c_4(a-2)}x^{a-d-2}h_4$,
$y\mapsto y-\frac{1}{c_3(b-2)}y^{b-c-2} h_3$, writing $f$ as
in~(\ref{twofaces}), adapting $h_1,\ldots,h_4$, $c$ and $d$ ($c_1,\ldots,c_4$
do not change) accordingly, after each application, we have that
\begin{equation}
f\overset{\K}\sim (c_1x^2+c_3y^{b-2})(c_2y^2+c_4x^{a-2})
=x^a+\lambda x^2y^2+y^b+2\lambda_1x^{c'}y^{d'}+E_{c'+d'},\label{firsttrans}
\end{equation}
where $\lambda_1=c_3c_4$, $a\le c'\in\N$, $b\le d'\in\N$. Note that, since
$w-\deg(h_1)\ge b-2$, where $w=(b-2,2)$, it follows after each application
that $c>b-2$ and similarly that $d>a-2$, i.e. $f_0$ stays unaffected. By now
repeatedly applying $x\mapsto x+\frac{1}{2\lambda_1}y^{d'-1}$,
$y\mapsto y-\frac{1}{2\lambda_1}x^{c'-1}$, writing $f$ as in
(\ref{firsttrans}), adapting $c'$ and $d'$ accordingly, after each application,
using the fact that $f$ is finite determined, it follows that
\[f\sim \lambda x^2y^2+x^a+y^b.\qedhere\]
\end{proof}

The following method can be used to determine the equivalence class of an input polynomial $g\in\m^3\subset\R[[x,y]]$ of hyperbolic type:

\begin{enumerate}
\item Transform $g$ such that there are no terms underneath the newton polygon created by the normal form of its corresponding complex type, which we will refer to as the newton polygon in the next steps.
\item Determine which real case $g$ is, by considering the terms on the newton polygon.
\item Throw all the terms not on the newton polygon away.
\item Scale the terms on the newton polygon and read off the value of the parameter.
\item Check in the tables included in \cite{realclassify2} which other normal forms are in the same equivalence class as the one determined.
\end{enumerate}

To apply the above method to the $X_{9+k}$ case, using a computer and thus working over $\Q$, we need the following prelimanary result.

\begin{lemma}\label{J10+kfactorization}
If $f\in\Q[x,y]$ is homogeneous and factorizes as $g_1^2(g_2)$ over $\R$, where $g_1$ is a polynomials of degree $1$ and $g_2$ is a polynomial of degree $2$ (which can possibly factorize over $\R$), then $f=ag_1'^2g_2'$, where $g_1'$ is a polynomial of degree $1$ over $\Q$, $g_2$ is a polynomial of degree 2 over $\Q$ and $a\in\Q$.
\end{lemma}

\begin{proof}
Let $f=(a_1x+a_2y)^2(a_3x^2+a_4xy+a_5y^2)$, $a_1,a_2,a_3,a_4,a_5\in\R$. Since the coefficient of the $x^4$ term in $f$ is a rational number it follows that 
\[(x+\frac{a_2}{a_1}y)^2(x^2+\frac{a_4}{a_3}xy+\frac{a_5}{a_3}y^2)\in\Q[x,y].\]
Since $\Q$ is a perfect field
$x+\frac{a_1}{a_2}y\in\Q[x,y]$. Therefore $f=ag_1'^2g_2'$, where
$g_1'=x+\frac{a_2}{a_1}y$, $g_2'=x^2+\frac{a_4}{a_3}xy+\frac{a_5}{a_3}y^2$ and $a=a_1^2a_3$.
\end{proof}

\begin{algorithm}[ht]
\caption{Algorithm for the case $X_{9+k}$}%
\label{alg:X_{9+k}}
\begin{algorithmic}[1]

\Require{$g \in \m^3\subset\Q[x,y]$ of complex singularity type $X_{9+k}$.}

\Ensure{A list with different lists as entries each containing the following entries: [1] a real singularity type of $g$; [2] a corresponding normal form of $g$; [3] a minimal polynomial for the parameter in the normal form in (2); [4] a list with entries: [1] a lower and [2] an upper bound of the parameter.}

\State Write $g$ as $t_0x^4+t_1x^3y+t_2x^2y^2+t_3xy^3+t_4y^4+R$, $t_0,\ldots,t_4\in\Q$, $R\in E_5$;
\If{($t_4\neq 0$ and $t_0=0$)}
\State Apply $x\mapsto y$, $y\mapsto x$ to $g$;
\EndIf
\If{($t_4=0$ and $t_0=0$)}
\If{$t_1+t_2+t_3=0$}
\If{$2t_1+4t_2+8t_3\neq 0$}
\State Apply $x\mapsto x$, $y\mapsto 2y$ to $g$;
\Else
\State Apply $x\mapsto x$, $y\mapsto 3y$ to $g$;
\EndIf
\State Apply $x\mapsto x$, $y\mapsto x+y$ to $g$;
\EndIf
\EndIf
\State Write $g$ as $a_0x^4+a_1x^3y+a_2x^2y^2+a_3xy^3+a_4y^4+R$, $a_0,\ldots,a_4\in\Q$, $R\in E_5$;
\State $h:=\jet(g,4)$;
\State factorize $h$ and let $g_1$ be the factor of $h$ of multiplicity $2$;
\State Apply $g_1\mapsto x$, $y\mapsto y$ to $g$;
\State Write $g$ as $a_0x^4+a_1x^3y+a_2x^2y^2+R$, $a_0,a_1,a_2\in\Q$, $R\in E_5$;
\State Apply $y\mapsto y-\frac{a_1}{2a_2}x$, $x\mapsto x$ to $g$;
\State $t_1 := $coefficient of the $x^2y^2$ term in $g$;
\For{$i=9;\ i\le 2(\mu-5);\ i++$}
\State $t := $ coefficient of the $xy^{\frac{i-(\mu-7)}{2}}$ term in $g$;
\State Apply $x\mapsto x-\frac{t}{2t_1}y^{\frac{i-(\mu-7)}{2}-2}$, $y\mapsto y$ to $g$;
\EndFor
\State Write $g$ as $b_0x^4+b_1x^2y^2+b_2y^{\mu-5}$, $b_0,b_1,b_2\in\Q$, $R\in E_5\cap E_{2\mu-9}^{(\mu-7,2)}$;

\If{$b_0>0$}
\If{$b_1>0$}
\State NF $:= x^4+x^2y^2+ay^{\mu-5}$;
\State type $:= X_{9+k}^{++}$;
\algstore{X_9+k}
\end{algorithmic}
\end{algorithm}
\begin{algorithm}[ht]
\begin{algorithmic}[1]
\algrestore{X_9+k}
\Else
\State NF $:= x^4-x^2y^2+ay^{\mu-5}$;
\State type $:= X_{9+k}^{+-}$;
\EndIf
\Else
\If{$b_1>0$}
\State NF $:= -x^4+x^2y^2+ay^{5\mu-9}$;
\State type $:= X_{9+k}^{-+}$;
\Else
\State NF $:= -x^4-x^2y^2+ay^{\mu-5}$;
\State type $:= X_{9+k}^{--}$;
\EndIf
\EndIf

\State $p:= x^4-b_2^4\left(\frac{b_0}{b_1^{2}}\right)^{\mu-5}$;
%\If{$p$ has a factor of degree $1$ over $\Q$}
%\State $MP_1 := x-b_2\left(\frac{b_0}{b_1^{2}}\right)^{\frac{\mu-5}{4}}$;
%\State List 1a := Emptyset;
%\If{$p$ has a factor of degree $1$ over $\Q$ and $\mu-5$ is uneven}
%\State $MP_2 := x+b_2\left(\frac{b_0}{b_1^{2}}\right)^{\frac{\mu-5}{4}}$;
%\State List 2a := Emptyset;
%\EndIf
%\Else
\If{$b_2>0$}
\State $\MP_1 := $ factor of $p$ over $\Q$ with positive real root;
\State List 1a := $0, \infty$;
\Else
\State $\MP_1 := $ factor of $p$ over $\Q$ with negative real root;
\State List 1a := $-\infty, 0$;
\EndIf
\If{$b_2>0$ and $\mu-5$ is uneven}
\State $\MP_2 :=$ factor of $p$ over $\Q$ with negative real root;
\State List 2a := $-\infty, 0$;
\Else
\If{$b_2<0$ and $\mu-5$ is uneven}
\State $\MP_2 := $ factor of $p$ over $\Q$ with positive real root;
\State List 2a := $0, \infty$;
\EndIf
\EndIf
%\State $p_1:=x^2-b_2^4\left(\frac{b_0}{b_1^{2}}\right)^{\mu-5}$;
%\If{$p_1$ has a factor of degree $1$ over $\Q$}
%\State $\MP_1, \MP_2 := x^2-b_2^2\left(\frac{b_0}{b_1^{2}}\right)^{\frac{\mu-5}{2}}$;
%\Else
%\State $\MP_1, \MP_2 := p$;
%\EndIf
%\EndIf
\State List 1 := type, NF, $MP_1$, List 1a;
\State List 2 := type, NF, $MP_2$, List 2a;
\If{$\mu-5$ is uneven}
\Return(List 1, List 2)
\Else
\Return (List 1)
\EndIf
\end{algorithmic}
\end{algorithm}
\clearpage

For the $J_{10+k}$ case we need the following additional Lemma.

\begin{lemma}
Suppose $f(x)=ax^3+bx+c\in\Q[x]$, $a,b,c\in\Q$, has a double root $q$. Then $q\in\Q$.
\end{lemma}
\begin{proof}
If $p$ is a double root of $f$, then $f(q)=aq^3+bq+c=0$ and $f'(q)=3aq^2+b=0$. Hence $f(q)=q(aq^2+b)+c=q(a(\frac{-b}{3a}+b)+c=0$ which implies that $q=-\frac{3c}{2b}$.
\end{proof}


\begin{algorithm}[ht]
\caption{Algorithm for the case $J_{10+k}$}%
\label{alg:J_{10+k}}
\begin{algorithmic}[1]

\Require{$g \in \m^3\subset\Q[x,y]$ of complex singularity type $J_{10+k}$.}

\Ensure{A list with different lists as entries each containing the following entries: [1] a real singularity type of $g$; [2] a corresponding normal form of $g$; [3] a minimal polynomial for the parameter in the normal form in (2); [4] a list with entries: [1] a lower and [2] an upper bound of the parameter.} 
%and [3] a aproximation for the parameter lying in the interval determined by (1) and (2).}
 
\State $s := $ coefficient of $x^3$ in $g$;
\If{$s=0$}
\State Apply $x\mapsto y$, $y\mapsto x$ to $g$;
\EndIf
\State $h:=\jet(g,3)$;
\State factorize $h$ as $bg_1^3$, $b\in\Q$, $b>0$;
\State Write $g$ as $bx^3+cx^2y^2+dxy^4+ey^6+R$, $b,c,d\in\Q$, $R\in E_7$;
\State Apply $x\mapsto x+(3b)^{-1}cy^2$, $y\mapsto y$ to $g$;
\State Write $g$ as $bx^3+dxy^4+ey^6+R$, $b,d,e\in\Q$, $R\in E_7$;
\State $q:= $ double root of $f(x)=bx^3+dx+e$;
\State Apply $x\mapsto x+qy$, $y\mapsto y$;
\State $t_1 := $coefficient of the $x^2y^2$ term in $g$; 
\For{$i=13;\  i\le 2(\mu-4);\ i++$}
\State $t:= $ coefficient of the $xy^{\frac{(i-(\mu-6))}{2}}$ term in $g$;
\State Apply $x\mapsto x-\frac{t}{2t_1}y^{\frac{(i-(\mu-6))}{2}-2}$, $y\mapsto y$ to $g$;
\EndFor
\State Write $g$ as $bx^3+cx^2y^2+ty^{\mu-4}+R$, $b,c,t\in\Q$, $R\in E_7^{(2,1)}\cap E_{2\mu-7}^{(\mu-6,2)}$;
\If{$c>0$}
\State NF := $x^3+x^2y^2+ay^{\mu-4}$;
\State type := $J_{10+k}^+$;
\Else
\State NF := $x^3-x^2y^2+ay^{\mu-4}$;
\State type := $J_{10+k}^-$;
\EndIf
\State $p:= x^6-\frac{b^2}{|c|^{3}}^{\mu-4}t^6$;
%\If{$p$ has a factor of degree $1$ over $\Q$}
%\State $\MP_1 := x-\frac{b^2}{|c|^{3}}^{\frac{\mu-4}{6}}t$;
%\State List 1a := Emptyset;

%\If{$p$ has a factor of degree $1$ over $\Q$ and $\mu-4$ is uneven}
%\State $MP_2 := x+\frac{b^2}{|c|^{3}}^{\frac{\mu-4}{6}}t$;
%\State List 2a := Emptyset;
%\EndIf
%\Else
%\State $p_1 := x^2-\frac{b^2}{|c|^{3}}^{\mu-4}t^6$;
%\If{$p_1$ has a factor of degree $1$ over $\Q$}
%\State $\MP_1 := x^3-\frac{b^2}{|c|^{3}}^{\frac{\mu-4}{2}}t^3$;
%\State List 1a := Emptyset;
%\If{$p_1$ has a factor of degree $1$ over $\Q$ and $\mu-4$ is uneven}
%\State $\MP_2 := x^3+\frac{b^2}{|c|^{3}}^{\frac{\mu-4}{2}}t^3$;
%\State List 2a := Emptyset;
%\EndIf
%\Else
\If{$t>0$}
\State $\MP_1 :=$ factor of $p$ over $\Q$ with a positive real root;
\State List 1a := $0, \infty$;
\Else
\State $\MP_1 := $ factor of $p$ over $\Q$ with a negative real root;
\State List 1a := $-\infty, 0$;
\EndIf
\algstore{J_{10+k}}
\end{algorithmic}
\end{algorithm}
\begin{algorithm}[ht]
\begin{algorithmic}[1]
\algrestore{J_{10+k}}
\If{$\mu-4$ is uneven}
\If{$t>0$}
\State $\MP_2 := $ factor of $p$ over $\Q$ with a negative real root;
\State List 2a := $-\infty, 0$;
\Else
\State $\MP_2 := $ factor of $p$ over $\Q$ with a postive real root;
\State List 2a := $0, \infty$;
\EndIf
\EndIf
%\State $p_2 := x^3-\frac{b^2}{|c|^{3}}^{\mu-4}t^6$;
%\If{$p_2$ has a factor of degree $1$ over $\Q$}
%\State $\MP_1, \MP_2 := x^2-\frac{b^2}{|c|^{3}}^{\frac{\mu-4}{3}}t^2$;
%\Else
%\State $\MP_1, \MP_2 := p$;
%\EndIf
%\EndIf
%\EndIf

\State List 1 := type, NF, $\MP_1$, List 1a;
\State List 2 := type, NF, $\MP_2$, List 2a;
\If{$\mu-4$ is even}
\Return (List 1)
\Else
\Return (List1, List2)
\EndIf
\end{algorithmic}
\end{algorithm}
\clearpage

Next we consider the case when $g\in\m^3$ is of one of the following complex cases:
$Y_{r,s}$ and $\widetilde Y_r$, $r,s>4$. Note that in the complex case, normal
forms of type $\widetilde Y_r$ are right-equivalent to normal forms of type
$Y_{r,r}$. This is unfortunately not true in the real case and we hence have to
treat these cases separately.  Similar to the $D_4$ case, we distinguish
between the two cases, using Proposition 8 in \cite{realclassify1} and the library {\tt rootsur.lib},
noticing that the $4$-jet of polynomials of type $Y_{r,s}$ factorize into four
linear factors, while the $4$-jet of polynomials of type $\widetilde Y_r$ do
not have any linear factors.

We first consider the case when $g$ is of real type $Y_{r,s}$, i.e.~the $4$-jet of $g(1,y)$ has real roots. Following the above method described for singularities of hyperbolic type, we run into trouble implementing the method. The reason is that in this case it is not easy to remove all the terms underneath the newton polygon created by the normal form of its corresponding complex type. The problem is caused by the fact that a polynomial in $\Q[x,y]$ that factorizes as $g_1^2g_2^2$ over $\R$, where $g_1$ and $g_2$ are polynomials of degree one, does not necessarily factorize into factors of degree zero and one over $\Q$. 

The next Lemma is needed for the case when the $4$-jet of $g$ does not factorize over $\Q$.

\begin{lemma}\label{IrrationalPoints}
Le $g$ be of real main singularity type $Y_{r,s}$. Then the $4$-jet of $g$ does not factorize into factors of degree zero and one over $\Q$, only if $g$ is of real main type $Y_{r,r}$.
\end{lemma}

\begin{proof}
Let $g$ be of type $Y_{r,s}$, for some $r$ and $s$,
$r,s>4$. For the determining of $r$ and $s$ we use the method of blowing up the origin.
By blowing up the origin once, polynomials right-equivalent to one of the
normal forms $\pm x^2y^2\pm x^r+ay^s$ transform to two different germs of which
one is right-equivalent to one of the germs $\pm x^2\pm x^{r-4}+ax^{s-4}y^s$
and one is right-equivalent to one of the germs  $\pm x^2\pm
x^{s-4}+ax^{r-4}y^r$, for some $0\neq a\in \mathbb R$, if $r$ and $s$ are
different, and to one germ which is right-equivalent to one of the germs $\pm
x^2\pm x^{r-4}+ax^{r-4}y^r$, for some $0\neq a\in\mathbb R$, if  $r=s$. Since
the milnor number of $f_1=\pm x^2\pm x^{s-4}+ax^{r-4}y^r$ is $s-5$ and the
milnor number of $f_2=\pm x^2\pm x^{r-4}+ax^{s-4}y^s$  is $r-5$, if $r,s\neq
4$, we determine $r$ and $s$ by calculating the milnor numbers of the resulting
germs. 


If there occur a singularity of the strict transform $p$ at an irrational point in some chart on the
exceptional divisor $P$ both charts contain two singularities of the strict
transform at irrational points on the exceptional divisor. This can be seen as follows. Since $g$ is of
type $Y_{r,s}$, $g$ is of the form
\[g=(a_0x+a_1y)^2(b_0x+b_1y)^2+\textnormal{ higher terms in $x$ and
$y$, $a_0,a_1,b_0,b_1\in\mathbb R$.}\] Without loss of generality we consider
the chart where $x$ is the local equation for the exceptional divisor. Then
\[p=(a_0+a_1y)^2(b_0+b_1y)^2+\textnormal{ terms that are divisible by
$x$.}\] Taking $p$ and $x$ in $P$ into account the only possible zero
points of $P$ are the roots of the rational polynomial $p_1 =
(a_0+a_1y)^2(b_0+b_1y)^2$. If $a_1=0$ or $b_1=0$ it follows from the fact
that $\mathbb Q$ is a perfect field that $p_1$ only has a rational root. Hence
$a_1\neq 0$ and $b_1\neq 0$. Therefore   $(0,-\frac{a_0}{a_1})$ and
$(0,-\frac{b_0}{b_1})$ are the only possible zero points of $P$. Because one of
these points are irrational $a_0$ and $b_0$ are nonzero. Since $\mathbb Q$ is a
perfect field and $(a_0+a_1y)^2(b_0+b_1y)^2$ is  a rational polynomial it
follows that $(a_0+a_1y)(b_0+b_1y)$ is also a rational polynomial.
Therefore both roots $-\frac{a_0}{a_1}$ and $-\frac{b_0}{b_1}$ are irrational
if any one of the roots is irrational. Furthermore,
$(a_0+a_1t)(b_0+b_1t)\in\mathbb Q(t)$ is a minimal polynomial for
$-\frac{a_0}{a_1}$ and $-\frac{b_0}{b_1}$ over $\mathbb Q$. Hence $P$ has zero
points at $-\frac{a_0}{a_1}$ and $-\frac{b_0}{b_1}$ or no zero points at all. Similarly, the
singularities of $p$ on the exceptional divisor in the chart with exceptional
divisor $x_2$, will occur at the irrational points $(-\frac{a_1}{a_0},0)$ and
$(-\frac{b_1}{b_0},0)$.

Since the $4$-jet of $g$, namely $p_1^h=(a_0x+a_1y)^2(b_0x+b_1y)^2$ factorize into factors of degree zero and one over $\Q$ if and only if $p_1=(a_0+a_1y)^2(b_0+b_1y)^2$ has rational roots, it is clear that the strict transform has singularities at irrational points if and only if $p_1^h$, does not factorize into factors of degree zero and one over $\Q$. 

In such a case, the singularities of $p$ can be moved such that they occur at
irrational points $(0,e)$ and $(0,-e)$, for some $e\in\mathbb R$, on the
exceptional divisor. This is done by applying the rational transformation
$x\mapsto x$, $y\mapsto y-(\frac{a_0b_1+b_0a_1}{2a_1b_1})x$ to $g$.
Now $g$ will be of the form
\[g=c(x+ey)^2(x-ey)^2+\textnormal{ higher terms in $x$ and $y$,
$c,e\in\mathbb R$}.\]
Determining $p$, again, we have
\[p=c(1+ey)^2(1-ey)^2+\textnormal{ terms that are divisible by $x$.}\]
 Working over the ring $\mathbb Q(t)\cong \mathbb Q(e)\cong \mathbb Q (-e)$,
 with minimum polynomial $(1-te)(1+te)$, we shift the singularities of $p$ to $0$
 by the transformations defined by $x\mapsto x$, $y\mapsto y-t$. Since
 $t$ can represent either $e$ or $-e$, the resulting polynomial represent both
 singularities working over $\mathbb Q(t)$. The milnor numbers for the
 resulting singularities can be detemined working over
 the ring $\mathbb Q(t)$. Because both resulting singularities are represented by
 one polynomial over $\mathbb Q(t)$, the milnor number will be the same in both
 cases.
\end{proof}

Algorithm \ref{alg:Y_{r,s}2} can be divided into two cases, where $g$ is the input polynomial:

\begin{itemize}
\item[1] lines 2 - 4 and 15 - 24:  The number of degree one factors of the $4$-jet of $g(1,y)$  over $\Q$ is $>0$.
\item[2] lines 6 - 24: The number of degree one factors of the $4$-jet of $g(1,y)$ over $\Q$ is zero.
\end{itemize}

\paragraph{Case 1:} The solution of this case is trivial.

\paragraph{Case 2:} lines 6 - 14:  In this case we first move the zero points of the $4$-jet of $g(1,y)$ to irrational points $t$, $-t$.  Since the $4$-jet of $g$ factorizes as $g_1^2g_2^2$ over the real numbers and $\Q$ is a perfect field $g_1g_2\in\Q[x,y]$. Since neither $g_1$ or $g_2$ is in $\Q[x,y]$, we can take $g_1g_2(1,y)$ as a minimal polynomial for $t$. Note that if $g$ can factorize into $4$ linear factors over $\R$ then $t\in\R$, and if not, $t=\pm ci$, $c\in\R$. 

\paragraph{ lines 15 - 24:} We are now able, similar to Case 1, using transformations in $\Q(t)[x,y]$ to transform $g$ to a polynomial of the form $bx^2y^2+dx^r+ey^s+R$, $b\in\Q$, $d,e\in\Q(t)$, $R\in E_{n_1+1}^{(n_1-2,2)}\cap E_{n_1+1}^{(2,n_1-2)}$.

Algorithm \ref{alg:Y_{r,s}} can be devided into the following parts, where $g$ is the input polynomial of complex type $Y_{r,s}$:

\begin{itemize}
\item[1]lines 2 - 53: The number of real roots of the $4$-jet of $g(1,y)$ is $>0$ and factorize in factors of degree one and zero over $\Q$. Hence $g$ is one of the types $Y_{r,s}^{\pm\pm}$, $r,s>4$.

\item[2]lines 56 - 101: The number of real roots of the $4$-jet of $g(1,y)$ is $>0$ and does not factorize into factors of degree one and zero over $\Q$. Hence, by Lemma \ref{IrrationalPoints}, $g$ is of one of the types $Y_{r,r}^{\pm\pm}$, $r>4$.

\item[3]lines 103 - 138: The number of real roots of the $4$- jet of $g(1,y)$ is zero. Hence $g$ is of one of the types $\widetilde Y_r^{\pm}$, $r>4$.
\end{itemize} 

\paragraph{Case 1:} This case is solved using the given method described for singularities of hyperbolic type.

\paragraph{Case 2:} line 56: Using Algoritm \ref{alg:Y_{r,s}2} we transform $g$, using transformations in $\Q(t)[x,y]$ to a polynomial of the form \[bx^2y^2+dx^r+ey^s+R, \quad b\in\Q,\quad d,e\in\Q(t),\quad R\in E_{n_1+1}^{(n_1-2,2)}\cap E_{n_1+1}^{(2,n_1-2)},\]such that $g(1,y)$ has zero points at $t, -t$..

lines 57 - 72: If $r$ is uneven then $g$ is either of both the types $Y_{r,r}^{+\pm}$ or of both the types $Y_{r,s}^{-\pm}$, depending on the sign of the $4$-jet of $g$, i.e.~the sign of $b$. And $a$ has two possible values of which the absolute value is the same. Since the values of $a$ differ only by sign, and $t$ represents two values which differ only by sign in $\R$, $a=\pm ct$ or $a=\pm c$, where $c\in \Q$. Hence $a^2\in\Q$. Therefore $x^2-b^r(de)^2$ is a polynomial with roots the two values of $a$.

lines 74 - 80: If $r$ is even and the sign of $d$ and $e$ differs, then $de<0$ and if the sign of $d$ and $e$ are the same $de>0$. Since $de$ has only one value when $r$ is even and $t$ represents two values, $de\in\Q$. In this case $g$ is either of the types $Y_{r,r}^{+\pm}$ or of types $Y_{r,r}^{-\pm}$, depending on the sign of the degree $4$-term of $g$. Since $de\in\Q$ a minimal polynomial for the value of $a$ for each corresponding normal form can be easily constructed.

line 81 - 89: If $r$ is even and $de>0$, then, similar as above $de\in\Q$. In this case the sign of the $x^r$ term in the normal form of $g$ and the parameter $a$ has the same sign. This sign, following the behavior of the normal form of $g$, is the opposite of the sign of the $4$-jet of $g$ if $g(1,y)$ has a real root and is the same as the sign of the $4$-jet of $g$ if $g(1,y)$ has no real roots. 

\paragraph{Case 3:} line 104: Let $g':=g$. We use Algorithm \ref{alg:Y_{r,s}2} to move the zero points of the $4$-jet of $g'$ to complex points $t$, $-t$, transforming $g'$ to a polynomial of the form $bx^2y^2+dx^r+ey^s+R$, $b\in\Q$, $d,e\in\Q(t)$, $R\in E_{n_1+1}^{(n_1-2,2)}\cap E_{n_1+1}^{(2,n_1-2)}$. Again $g_1g_2(1,y)$, where $g'$ factors as $g=g_1^2g_2^2$ over $\R$, is a minimal polynomial for $t$ and $-t$ over $\Q$. Note that in this case $t=ci$, $c\in\R$. 

line 106 - 137 : The type of $g$ can be easily determined by considering the sign of the $x^4$ term in $g$.  The only thing that is left is thus to determine the values for the parameter $a$. The sign of the values of the parameter $a$ can be determined, similar the the case when $de>0$ in Case~2, by considering whether $r=n_1$ is even and the number of real roots of $g(1,y)$. Also similar to case 2 $de=ct$, $c\in\Q$. Therefore $(de)^2\in\Q$. Following the discussion after Lemma 27 in \cite{realclassify2} we have that $b^{-\frac{1}{2}}de=\zeta ca^2$, where $\zeta = 1$ if $r=n_1$ is even and $\zeta^2=1$ if $r=n_1$ is uneven and $c=\left(\frac{1}{4}\right)^{r}$. Using this information a minimal polynomial for the values of $a$ can be easily constructed. 

\begin{algorithm}[ht]
\caption{Algorithm for the case $Y_{r,s}$ and $\widetilde Y_r$}%
\label{alg:Y_{r,s}}
\begin{algorithmic}[1]

\Require{$g \in \m^3\subset\Q[x,y]$ of complex singularity type $Y_{r,s}$ and $\widetilde Y_r$.}

\Ensure{A list with different lists as entries each containing the following entries: [1] a real singularity type of $g$; [2] a corresponding normal form of $g$; [3] a minimal polynomial for the parameter in the normal form in (2); [4] a list with entries: [1] a lower and [2] an upper bound of the parameter.} 
%and [3] a aproximation for the parameter lying in the interval determined by (1) and (2).}
 
\State $h:=\jet(g,4)$;
\If{ the number of real roots of $h(1,y)>0$}
\If{ the number of degree one factors of $h$ over $\Q$ is bigger than zero}
\State Apply Algorithm \ref{alg:Y_{r,s}2} to $g$;
\State Write $g$ as $bx^2y^2+dx^{n_1}+ey^{n_2}+R$, $R\in E^{(n_2-2,2)}_{n_2+1}\cap E^{(2,n_1-2)}_{n_1+1}$ ;
\State $\sigma_1 := \sign(b)$;
\If{$n_1$ is uneven}
\State $\NF_1 := \sigma_1x^2y^2+x^{n_1}+ay^{n_2}$;
\State $\type_1  := Y_{n_1,n_2}^{\sigma_1+}$;
\State $\NF_3 := \sigma_1x^2y^2-x^{n_1}+ay^{n_2}$;
\State $\type_3  := Y_{n_1,n_2}^{\sigma_1-}$;
\State $s_2, s_4 := 0$;
\Else
\State $\sigma_2 := \sign(d)$;
\State $\NF_1 := \sigma_1x^2y^2+\sigma_2x^{n_1}+ay^{n_2}$;
\State $\type_1  := Y_{n_1,n_2}^{\sigma_1\sigma_2}$;
\State $s_1, s_4 := \sigma_2$;
\EndIf
\If{$n_2$ is uneven}
\State $\NF_2 := \sigma_1x^2y^2+x^{n_2}+ay^{n_1}$;
\State $\type_2  := Y_{n_2,n_1}^{\sigma_1+}$;
\State $\NF_4 := \sigma_1x^2y^2-x^{n_2}+ay^{n_1}$;
\State $\type_4  := Y_{n_2,n_1}^{\sigma_1-}$;
\State $s_1, s_3 := 0$;
\Else
\State $\sigma_2 := \sign(e)$;
\State $\NF_2 := \sigma_1x^2y^2+\sigma_2x^{n_2}+ay^{n_1}$;
\State $\type_2  := Y_{n_2,n_1}^{\sigma_1\sigma_2}$;
\State $s_1, s_3 := \sigma_2$;
\EndIf
\State $P_1, P_3 := x^{2n_1}-|e^{2n_1}b^{-n_1}d^{2n_2}|$;
\State $P_2,P_4:= x^{2n_2}-|d^{2n_2}b^{-n_1}e^{2n_1}|$;
\For{$i=1,\ldots,4$}
\If{$s_i=-1$}
\State List $i$a := $-\infty, 0$;
\State $\MP_i := $ factor of $P_i$ over $\Q$ with a negative real root;
\EndIf
\If{$s_i=0$}
\State $\NF_{i+4} := \NF_i$;
\State $\type_{i+4} := \type_i$;
\State List ($i+4$)a := $0, \infty$;
\State List $i$a := $-\infty, 0$;
\State $\MP_{i+4} := $ factor of $P_i$ over $\Q$ with a positive real root;
\State $\MP_i := $ factor of $P_i$ over $\Q$ with a negative real root;
\EndIf
\If{$s_i = 1$}
\State List $i$a := $0, \infty$;
\State $\MP_i := $ factor of $P_i$ over $\Q$  with a positive real root;
\EndIf
\EndFor
\algstore{Y_{r,s}}
\end{algorithmic}
\end{algorithm}
\begin{algorithm}[ht]
\begin{algorithmic}[1]
\algrestore{Y_{r,s}}
\State List 1 := $\type_1,\NF_1, \MP_1$, List 1a; 
\State List 2 := $\type_2,\NF_2, \MP_2$, List 2a; 
\State List := List 1, List 2;
\If{$n_1$ is uneven}
\State List 3 := $\type_3,\NF_3, \MP_3$, List 3a; 
\State List := List, List 3;
\EndIf
\If{$n_2$ is uneven}
\State List 4 := $\type_4,\NF_4, \MP_4$, List 4a; 
\State List := List, List 4;
\EndIf
\Return (List)
\Else
%\State factorize $h$ as $bh_1h_2$, $h_1,h_2$ of degree $2$ and $b\in\Q$, over $\Q$; 
%\State $t_1:=$ coefficient of $xy$ in $h_1$;
%\State $t_2:=$ coefficient of $y^2$ in $h_1$;
%\State Apply $x\mapsto x$, $y\mapsto y-\frac{t_1}{t_2}x$ to $g$;
%\State $h:=\jet(g,4)$;
%\State factorize $h$ as $bh_1h_2$, $h_1,h_2$ of degree $2$ and $b\in\Q$, over $\Q$;
%\State Let $R:= \Q(t)$, where $h_1$ is the minimal polynomial of $t$;
%\State factorize $h$ as $bg_1^2g_2^2$, $b\in\Q$, over $\Q(t)$;
%\State Apply $g_1\mapsto x$, $g_2\mapsto y$ to $g$;

%\State $n_1:=4$;
%\While{ the coefficient of the term $x^{n_1}$ in $g$ is zero}
%\State $n_1:= n_1+1$; 
%\State $c_1:=$ the coefficient of $x^{n_1}y$;
%\State Apply $x\mapsto x$ , $y\mapsto y-c_1x^{n_1-2}$;
%\State $c_2:=$ the coefficient of $y^{n_1}x$;
%\State Apply $y\mapsto y$ , $x\mapsto x-c_1y^{n_1-2}$;
%\EndWhile
\State Apply Algorithm \ref{alg:Y_{r,s}2} to $g$;
\State setring $R$;
\State Write $g$ as $bx^2y^2+dx^{n_1}+ey^{n_1}+R$, $R\in E^{(n_1-2,2)}_{n_1+1}\cap E^{(2,n_1-2)}_{n_1+1}$ ;
\If{$n_1$ is uneven}
\State $\sigma := \sign(b)$;
\State $\NF_1,\NF_3 := \sigma x^2y^2+x^{n_1}+ay^{n_1}$;
\State $\NF_2,\NF_4 := \sigma x^2y^2-x^{n_1}+ay^{n_1}$;
\State $s_1, s_2 := 1$; 
\State $s_3, s_4 := -1$;
\State $P_1, P_2, P_3, P_4 := x^2-|b^{\frac{n_1}{2}}de|^2$;
%\Else
%\State $\NF_1, \NF_3 := -x^2y^2+x^{n_1}+ay^{n_1}$;
%\State $\NF_2, \NF_4 := -x^2y^2-x^{n_1}+ay^{n_1}$;
%\State $s_1, s-2 := 1$;
%\State $s_3, s_4 := -1$;
%\State $P _1,P_2,P_3,P_4 := x^2-|b^{\frac{n_1}{2}}de|^2$;
%\EndIf
\For{$i=1,\ldots,4$}
\If{$s_i = -1$}
\State List $i$a := $-\infty, 0$;
\State $\MP_i := $ factor of $P_i$ over $\Q$ with negative real root;
\Else
\State List $i$a := $0, \infty$;
\State $\MP_i := $ factor of $P_i$ over $\Q$ with positive real root;
\EndIf
\EndFor
\Else
\If{$de<0$}
\State $\sigma := \sign(b)$;
\State $\NF_1 := \sigma x^2y^2+x^{n_1}+ay^{n_1}$;
\State $\NF_2 := \sigma x^2y^2-x^{n_1}+ay^{n_1}$;
\State $\MP_1 := x+|b^{\frac{n_1}{2}}de|$;
\State $\MP_2 := x-|b^{\frac{n_1}{2}}de|$;
\State List 1a, List 2a := $-\infty, \infty$;
%\Else
%\State $\NF_1 := -x^2y^2+x^{n_1}+ay^{n_1}$;
%\State $\NF_2 := -x^2y^2-x^{n_1}+ay^{n_1}$;
%\State $\MP_1 := x+|b^{\frac{n_1}{2}}de|$;
%\State $\MP_2 := x-|b^{\frac{n_1}{2}}de|$;
%\State List 1a, List 2a := Emptyset;
%\EndIf
\Else
\State List 1a := $-\infty, \infty$;
\State $\sigma := \sign(b)$;
\If{number of real roots of $g(1,y)>0$}
\State $\NF_1 := \sigma x^2y^2-\sigma x^{n_1}+ay^{n_1}$;
\State $\MP_1 := x+\sigma |b^{\frac{n_1}{2}}de|$;
\Else
\State $\NF_1 := \sigma x^2y^2+\sigma x^{n_1}+ay^{n_1}$;
\State $\MP_1 := x-\sigma |b^{\frac{n_1}{2}}de|$;
\EndIf
%\If{$b<0$}
%\If{number of real roots of $g(1,y)>0$}
%\State $\NF_1 := -x^2y^2+x^{n_1}+ay^{n_1}$;
%\State $\MP_1 := x+|b^{\frac{n_1}{2}}de|$;
%\State List 1a := Emptyset;
%\Else
%\State $\NF_1 := -x^2y^2-x^{n_1}+ay^{n_1}$;
%\State $\MP_1 := x-|b^{\frac{n_1}{2}}de|$;
%\State List 1a := Emptyset;
%\EndIf
\EndIf
\EndIf
\EndIf
\State List 1 := $\type_1, \NF_1, \MP_1$, List 1a;
\algstore{Y_{r,s}}
\end{algorithmic}
\end{algorithm}
\begin{algorithm}[ht]
\begin{algorithmic}[1]
\algrestore{Y_{r,s}}
\If{$n_1$ is uneven}
\State List 1 := $\type_1, \NF_1, \MP_1$, List 1a;
\State List 2 := $\type_2, \NF_2, \MP_2$, List 2a;
\State List 3 := $\type_3, \NF_3, \MP_3$, List 3a;
\State List 4 := $\type_4, \NF_4, \MP_4$, List 4a;
\Return(List 1, List 2, List 3, List 4)
\Else
\If{$de<0$}
\Return (List 1, List 2)
\Else
\Return (List 1)
\EndIf
\EndIf
\Else
%\State factorize $h$ as $bh_1h_2$, $h_1,h_2$ of degree $2$ and $b\in\Q$, over $\Q$; 
%\State $t_1:=$ coefficient of $xy$ in $h_1$;
%\State $t_2:=$ coefficient of $y^2$ in $h_1$;
%\State Apply $x\mapsto x$, $y\mapsto y-\frac{t_1}{t_2}x$ to $g$;
%\State $h:=\jet(g,4)$;
%\State factorize $h$ as $bh_1h_2$, $h_1,h_2$ of degree $2$ and $b\in\Q$, over $\Q$;
%\State Let $R:= \Q(t)$, where $h_1$ is the minimal polynomial of $t$;
%\State factorize $h$ as $bg_1^2g_2^2$, $b\in\Q$, over $\Q(t)$;
\State $g':= g$;
%\State Apply $g_1\mapsto x$, $g_2\mapsto y$ to $g'$;
%\State $n_1:=4$;
%\While{ the coefficient of the term $x^{n_1}$ in $g'$ is zero}
%\State $n_1:= n_1+1$; 
%\State $c_1:=$ the coefficient of $x^{n_1}y$;
%\State Apply $x\mapsto x$ , $y\mapsto y-c_1x^{n_1-2}$;
%\State $c_2:=$ the coefficient of $y^{n_1}x$;
%\State Apply $y\mapsto y$ , $x\mapsto x-c_1y^{n_1-2}$;
%\EndWhile
\State Apply Algorithm \ref{alg:Y_{r,s}2} to $g'$;
\State setring $R$;
\State Write $g'$ as $bx^2y^2+dx^{n_1}+ey^{n_1}+R$, $R\in E^{(n_1-2,2)}_{n_1+1}\cap E^{(2,n_1-2)}_{n_1+1}$ ;
\State $\sigma := $ sign of the $x^4$ term of $g$;
\State $\NF_1:= \sigma(x^2+y^2)^2+ax^{n_1}$;
\State $\type_1 := \widetilde Y_{n_1}^\sigma$;
\If{$n-1$ is uneven}
%\If{coefficient of $x^4$ in $g$ is $>0$}
%\Else
%\State $\NF_1:=-(x^2+y^2)^2+ax^{n_1}$;
%\State $\type_1 := \widetilde Y_{n_1}^-$;
%\EndIf
\State $P_1:= x^8-((4de)^2d^{-1})^{2n_1}$;
\State $s:=0$;
\Else
\State $P_1 := x^4-((4de)^2d^{-1})^{n_1}$;
\If{the number of real roots of $g(1,y)$ $>0$}
\State $s := -\sigma$; 
\Else
\State $s := \sigma$; 
\EndIf
%\Else
%\If{the number of real roots of $g(1,y)$ $>0$}
%\State $\NF_1 := -(x^2+y^2)^2+ax^{n_1}$
%\State $\type_1 := \widetilde Y_{n_1}^-$;
%\State $P_1 := x^4-((4de)^2d^{-1})^{n_1}$;
%\State $s := 1$; 
%\Else
%\State $\NF_1 := (x^2+y^2)^2+ax^{n_1}$
%\State $\type_1 := \widetilde Y_{n_1}^+$;
%\State $P_1 := x^4-((4de)^2d^{-1})^{n_1}$;
%\State $s := -1$; 
%\EndIf
%\EndIf
\EndIf
\If{$s=-1$}
\State List 1a := $-\infty, 0$;
\State $\MP_1 := $ factor of $P_1$ over $\Q$ with a negative real root;
\EndIf
\If{$s:=1$}
\State List 1a := $0,\infty$;
\State $\MP_1 := $ factor of $P_1$ over $\Q$ with a positive real root;
\EndIf
\If{$s=0$}
\State $\NF_2 := \NF_1$;
\State $\type_2 := \type_1$;
\State $\MP_1 := $ factor of $P_1$ over $\Q$ with a positive real root;
\State $\MP_2 := $ factor of $P_1$ over $\Q$ with a negative real root;
\State List 1a := $0,\infty$;
\State List 2a := $-\infty, 0$;
\EndIf
\State List 1 := $\type_1, \NF_1, \MP_1,$ List 1a;
\If{$s\neq 0$}
\Return (List 1)
\Else
\State List 2 := $\type_2, \NF_2, \MP_2,$ List 2a;
\Return( List 1, List 2)
\EndIf
\EndIf
\end{algorithmic}
\end{algorithm}

\begin{algorithm}[ht]
\caption{Algorithm for removing terms underneath the newton polygon in the case $Y_{r,s}$ and $\widetilde Y_r$}%
\label{alg:Y_{r,s}2}
\begin{algorithmic}[1]
\Require{$g \in \m^3\subset\Q[x,y]$ of complex singularity type $Y_{r,s}$ and $\widetilde Y_r$.}
\Ensure{[1] $g'\in \m^3\subset\Q[x,y]$ such that $g\sim g'$ and there are no terms underneath the newton polygon of the normal form of $g$ appearing in $g'$; [2] a new basering. }

\State $h:=\jet(g,4)$;
\If{ the number of degree one factors of $h$ over $\Q$ is bigger than zero}
\State $R := \Q$
\State factorize $h$ as $bg_1^2g_2^2$, $b\in\Q$ over $\Q$;
\State Apply $g_1\mapsto x$, $g_2\mapsto y$ to $g$;
\Else
\State factorize $h$ as $bh_1h_2$, $h_1,h_2$ of degree $2$ and $b\in\Q$, over $\Q$; 
\State $t_1:=$ coefficient of $xy$ in $h_1$;
\State $t_2:=$ coefficient of $y^2$ in $h_1$;
\State Apply $x\mapsto x$, $y\mapsto y-\frac{t_1}{2t_2}x$ to $g$;
\State $h:=\jet(g,4)$;
\State factorize $h$ as $bh_1h_2$, $h_1,h_2$ of degree $2$ and $b\in\Q$, over $\Q$;
\State Let $R:= \Q(t)$, where $h_1(1,y)$ is the minimal polynomial of $t$;
\State factorize $h$ as $bg_1^2g_2^2$, $b\in\Q$, over $\Q(t)$;
\State Apply $g_1\mapsto x$, $g_2\mapsto y$ to $g$;
\EndIf
\State $n_1:=4$, $n_2:=4$;
\While{ the coefficient of the terms $x^{n_1}$ or $y^{n_2}$ in $g$ is zero}
\State $n_1:= n_1+1$; 
\State $c_1:=$ the coefficient of $x^{n_1}y$ in $g$;
\State Apply $x\mapsto x$ , $y\mapsto y-c_1x^{n_1-2}$ in $g$;
\State  $n_2:=n_2+1$;
\State $c_2:=$ the coefficient of $y^{n_2}$;
\State Apply $x\mapsto x-c_2y^{n_2-2}$, $y\mapsto y$ to $g$;
\EndWhile 
\Return($g$, $R$)
\end{algorithmic}
\end{algorithm}
 
\clearpage

\section{Real parabolic singularities of corank $2$}

Let $g\in\m^3$ be a polynomial of complex type $J_{10}$. Similar to the previous cases, taking Theorem 19 in \cite{realclassify2} into account, $g$ can easily be transformed to the form
\[x^3+bx^2y^2+dxy^4+ey^6\] Applying
\begin{equation}\label{Cancelling2ndTerm}
x\mapsto x-\frac{b}{3}y^2,\quad y\mapsto y.
\end{equation}
$g$ is transformed to a polynomial of the form
\[x^3+dxy^4+ey^6.\] Since there exists an
invertable transformation $\phi'$ such that
\begin{equation}\label{phi'}
\phi'(x^3+a'x^2y^2\pm |d'|xy^4)=x^3+a''x^2y^2\pm xy^4,\quad
a''=\frac{a'}{\sqrt{|d'|}},\ a',d'\in \R,
\end{equation}
it suffices to consider transformations $\phi$ such that $\phi(g)$ is of the
form
\begin{equation}\label{form}
x^3+a'x^2y^2\pm |d'|xy^4,\quad a',d'\in \R.
\end{equation}
Again taking Theorem 19 in \cite{realclassify2}  into account, we have that
\begin{equation}\label{eqc}
\phi(g)=x^3+3cx^2y^2+(3c^2+t^4d)xy^4+(c^3+t^4dc+et^6)y^6,\quad t,c\in\R.
\end{equation}
Since $c=t^2c'$, where $c'=\frac{c}{t^2}$, we rewrite (\ref{eqc}) as
\begin{equation}\label{eqc'}
\phi(g)=x^3+3t^2c'x^2y^2+t^4(3c'^2+d)xy^4+t^6(c'^3+dc'+e)y^6.
\end{equation}
Clearly, for a fixed value of $t\neq 0$, $c'$ is any real root of
$k(z)=z^3+dz+e$. Since $\phi'\circ\phi$ transform $f$ to the same polynomial,
regardless the value of $t$, we may assume that $t=1$ and $c=c'$.

We consider the following cases:
\begin{itemize}
\item[(AI)]$k(z)$ has one real root;
\item[(AII)]$k(z)$ has three real roots.
\end{itemize}

Note that
\begin{equation}\label{RelationOfRoots}
c_1+c_2+c_3=0\quad\textnormal{and}\quad c_1c_2+c_1c_3+c_2c_3=d,
\end{equation}
where $c_1$, $c_2$ and $c_3$ are the complex roots of $k(z)$.

(AI) We firstly consider the case when $k$ has one real root $c_1$ and two
roots that are complex conjugates $c_2$ and $c_3$, i.e.~there is only one
possible value for $c$, namely $c_1$. It follows from (\ref{RelationOfRoots})
that $c_2c_3=d+c_1^2$. Since the product of two complex conjugates are positive
$d+c_1^2>0$ which implies that $d+3c_1^2>0$. This means, considering
(\ref{eqc'}), that $g$ is of type $J_{10}^+$. Since there is only one possible
value for $c$, there is only one possible transformation that transforms $g$ to
(\ref{form}) and thus only one possible type $J_{10}^+$ and only one possible
value of $a''$, say $a$.  

Let $a=\frac{3c_1}{\sqrt{3c_1^2+d}}$, $a_2:=\frac{3c_2}{\sqrt{3c_2^2+d}}$ and $a_3=\frac{3c_1}{\sqrt{3c_1^2+d}}$. Note, since $a=\frac{3c_1}{\sqrt{3c_1^2+d}}$, $a$ is a solution of $p(x):=\frac{dx^2}{9-3x^2}-c_1^2$. Similarly, $a_2$ and $a_3$ are solutions of $p_1(x):=\frac{dx^2}{9-3x^2}-c_2^2$ and $p(x):=\frac{dx^2}{9-3x^2}-c_3^2$, respectively.  Using the fact that $c_1$, $c_2$ and $c_3$ are  the roots of $k(z)$, 
the polynomial
\[h(x):=(4d^3+27e^2)x^6+(-36d^3-243e^2)x^4+(81d^3+729e^2)x^2-729e^2\]
with $a$ and $-a$, as well as, $\pm a_2$ and $\pm a_3$ as roots, can be easily constructed. Now $a_2$ and $a_3$ is real if and only if $c_2=ci$ and $c_3=-ci$, $c\in\R$. In this case $c_1$ and thus $a$ and $e$ is $0$. 

Using the fact that $c_2c_3$ is positive and that $e=c_1c_2c_3$ therefore have the same sign as $c_1$, it follows, since $a=\frac{3c_1}{\sqrt{3c_1^2+d}}$, that $a$ has the same sign as $e$. Hence a minimal polynomial with an interval which determines the right root can be easily computed for $a$. 

(AII) We now consider the case where $k$ has three real roots $c_1<c_2<c_3$.
The three roots must be different, otherwise $k(z)$ and $k'(z)$ have a similar
root which implies that $g$ is nondegenerate. Hence $a''$ has three different values, namely
$a_j:=\frac{3c_j}{\sqrt{|d+3c_j^2|}}$, $j=1,2,3$, putting $c=c_j$ respectively.

Let $c_1<c_2<c_3$. Now, since $c_1+c_2+c_3=0$ not all three roots have the same
sign. Let the sign of $c_1$ be different from that of $c_2$ and $c_3$. In this
case note that $|c_1|=|c_2+c_3|=|c_2|+|c_3|$, i.e.~$|c_1|>|c_2|$ and
$|c_1|>|c_3|$. Then $3c_1^2+d=3c_1^2+c_1c_2+c_1c_3+c_2c_3>0$, since
$|c_1^2|>|c_1c_3|$, $|c_1^2|>|c_1c_2|$ and $c_1^2>0$, while $c_1c_2<0$,
$c_1c_3<0$ and $c_2c_3>0$. Hence, putting $c=c_1$, $g$ is transformed to
$J_{10}^+$.
Furthermore
$3c_j^2+d=3c_j^2+c_1(c_2+c_3)+c_2c_3=3c_j^2-(c_2+c_3)(c_2+c_3)+c_2c_3
=3c_j^2-(c_2^2+c_2c_3+c_3^2)$,
for $j=2,3$. Therefore, since $|c_2|<|c_3|$, putting $c=c_2$ result in
$J_{10}^-$ and putting $c=c_3$ result in $J_{10}^+$. The case where $c_2$ and $c_3$ have the same sign follows similar. 

In this case,  
\[h^-(x):=(-4d^3-27e^2)x^6+(-36d^3-243e^2)x^4+(-81d^3-729e^2)x^2-729e^2\]
is a polynomial with roots $\pm i a_1$, $\pm a_2$ and $\pm i a_3$. Here $h^{-}(x)$ is similarly constructed as in (AI), taking into account that $3c_2^2-d<0$ and $3c_j^2-d>0$, for $j=1,3$. Now $c_1c_3$ is negative, since $c_1$ and $c_3$ have opposite signs. Therefore $\frac{3c_2}{\sqrt{|3c^2+d|}}=a_2>0$ if and only if $c_1c_2c_3=e<0$. For the parameters which correspond with $J_{10}^+$
 \[h^+(x):=(4d^3+27e^2)x^6+(-36d^3-243e^2)x^4+(81d^3+729e^2)x^2-729e^2\]
is a polynomial with roots $\pm a_1$, $\pm i a_2$ and $\pm a_3$. Again, $h^+(x)$ is similarly constructed as $h^-(x)$.
Since $d$ is negative in this case it follows that $|c_1|<|c_3|$ if and only if $|a_3|<|a_1|$. If $c_2<0$ ($e>0$), we have that $|c_1+c_2|=|c_1|+|c_2|=|c_3|$ and hence that $|a_1|<|a_3|$. Similarly we have that $|a_3|<|a_1|$ if $e<0$. Therefore $a_1$ is the biggest negative real root and $a_3$ is the biggest positive real root of $h^+(x)$, respectively, if $e>0$, and $a_1$ is the smallest negative real root and $a_3$ is the smallest positive real root of $h^+(x)$, respectively, if $e<0$.  Note that, although the algorithm use {\tt solve.lib}, it determines an exact interval on which the root occurs. This is done with the help of {\tt rootsur.lib}.

\begin{algorithm}[ht]
\caption{Algorithm for the case $J_{10}$}%
\label{alg:J_{10}}
\begin{algorithmic}[1]

\Require{$g \in \m^3\subset\Q[x,y]$ of complex singularity type $J_{10}$.}

\Ensure{A list with different lists as entries each containing the following entries: [1] a real singularity type of $g$; [2] a corresponding normal form of $g$; [3] a minimal polynomial for the parameter in the normal form in (2); [4] a list with entries: [1] a lower and [2] an upper bound of the parameter.} 
%and [3] a aproximation for the parameter lying in the interval determined by (1) and (2).}
 
\State $g:=\jet(g,6);$
\State $s := $ coefficient of $x^3$ in $g$;
\If{$s=0$}
\State Apply $x\mapsto y$, $y\mapsto x$ to $g$;
\EndIf
\State $h:=\jet(g,3)$;
\State factorize $h$ as $bg_1^3$, $b\in\Q$, $b>0$;
\State Apply $x\mapsto g_1$, $y\mapsto y$ to $g$;
\State Write $g$ as $bx^3+ax^2y^2+dxy^4+ey^6$, $b,a,d,e\in\Q$;
\State $g:= \frac{1}{b}g$;
\State Write $g$ as $x^3+ax^2y^2+dxy^4+ey^6$, $a,d,e\in\Q$;
\State Apply $x\mapsto x-\frac{a}{3}y^2$, $y\mapsto y$ to $g$;
\State Write $g$ as $x^3+dxy^4+ey^6$, $d,e\in\Q$;
\State $k(x):= x^3+dx+e$;
\If{$k(x)$ has one real root}
\State $h(x):=(4d^3+27e^2)x^6+(-36d^3-243e^2)x^4+(81d^3+729e^2)x^2-729e^2;$
\If{$e>0$}
\State MP := factor of $h(x)$ over $\Q$ with a positive real root;
\State type := $J_{10}^+$;
\State NF := $x^3+ax^2y^2+xy^4$;
%\State $n := 8$;
%\State value := the real root of MP solved to precition of n digits;
%\State $a:=\val+\frac{1}{10}^n$;
%\State $b:=\val-\frac{1}{10}^n$;
%\While{MP has more than one real root on $(a,b)$}
%\State $n:=2n$;
%\State value := the real root of MP solved to precition of $n$ digits;
%\State $a:=\val+\frac{1}{10}^n$;
%\State $b:=\val-\frac{1}{10}^n$;
%\EndWhile
%\State List 1a := $a$,  $b$;
\State List 1a := $0,\infty$;
\Else
\If{$e>0$}
\State MP := factor of $h(x)$ over $\Q$ with a negative real root;
\State type := $J_{10}^+$;
\State NF := $x^3+ax^2y^2+xy^4$;
%\State $a:=\val+\frac{1}{10}^n$;
%\State $b:=\val-\frac{1}{10}^n$;
%\While{MP has more than one real root on $(a,b)$}
%\State $n:=2n$;
%\State $\val := $the real root of MP solved to precition of $n$ digits;
%\State $a:=\val+\frac{1}{10}^n$;
%\State $b:=\val-\frac{1}{10}^n$;
%\EndWhile
%\State List 1a := $a$, $b$, value;
\State List 1a := $-\infty, 0$;
\Else
\State $\MP := x$;
\State type := $J_{10}^+$;
\State NF := $x^3+ax^2y^2+xy^4$;
\State List 1a := $-\infty, \infty$;
\EndIf
\EndIf
\State List 1 := type, NF, MP, List 1a;
\Return (List 1)
\Else
\State $h^+(x):=(4d^3+27e^2)x^6+(-36d^3-243e^2)x^4+(81d^3+729e^2)x^2-729e^2;$
\State $h^-(x):=(-4d^3-27e^2)x^6+(-36d^3-243e^2)x^4+(-81d^3-729e^2)x^2-729e^2;$
\If{$e>0$}
\State $\MP_1 := $ factor of $h^-(z)$ with negative real root;
\State $\NF_1 := x^3+ax^2y^2-xy^4$;
\State $\type_1 := J_{10}^-$;
%\State $\val_1$ := the real root of $MP_1$ solved to precition of $n$ digits;
%\State $a:=\val_1+\frac{1}{10}^n$;
%\State $b:=\val_1-\frac{1}{10}^n$;
%\While{MP has more than one real root on $(a,b)$}
%\State $n:=2n$;
%\State $\val_1$ := the real root of $MP_1$ solved to precition of $n$ digits;
%\State $a:=\val_1+\frac{1}{10}^n$;
%\State $b:=\val_1-\frac{1}{10}^n$;
%\EndWhile
%\State List 1a := $a,b,\val_1$;
\State List 1a := $-\infty, 0$;
\State List 1 := type 1, NF, MP, List 1a;
\Else
\State $MP_1 := $ factor of $h^-(z)$ with positive real root;
\State $NF_1 := x^3+ax^2y^2-xy^4$;
\State $\type_1 := J_{10}^-$;
\algstore{J_10}
\end{algorithmic}
\end{algorithm}
\begin{algorithm}[ht]
\begin{algorithmic}[1]
\algrestore{J_10}
%\State $\val_1$ := the real root of $MP_1$ solved to precition of n digits;
%\State $a:=\val_1+\frac{1}{10}^n$;
%\State $b:=\val_1-\frac{1}{10}^n$;
%\While{MP has more than one real root on $(a,b)$}
%\State $n:=2n$;
%\State $\val_1$ := the real root of $MP_1$ solved to precition of $n$ digits;
%\State $a:=\val_1+\frac{1}{10}^n$;
%\State $b:=\val_1-\frac{1}{10}^n$;
%\EndWhile
%\State List 1a := $a,b,\val_1$;
\State List 1a := $0, \infty$;
\State List 1 := $\type_1, \NF, \MP$, List 1a;
\EndIf
\State $n:=8$
\State List L := list of real roots of $h^+(x)$ in increasing order solved with precition of $n$ digits;
\For{$i=1,\ldots,\size(L)$}
\State $a_i := L[i]-\frac{1}{10}^n$;
\State $b_i := L[i]+\frac{1}{10}^n$;
\EndFor
\While{$h^+(x)$ has more than one real root on any of the intervals $(a_i,b_i)$}
\State $n=2n$;
\State List L := list of real roots of $h^+(x)$ in increasing order solved with precition of $n$ digits;
\For{$i=1,\ldots,\size(L)$}
\State $a_i := L[i]-\frac{1}{10}^n$;
\State $b_i := L[i]+\frac{1}{10}^n$;
\EndFor
\EndWhile
\If{$e>0$}
\State $\MP_2:= $ factor of $h^+(x)$ over $\Q$ with a real root in the interval $(a_2,b_2)$;
\State $\NF_2:= x^3+ax^2y^2+xy^4$;
\State $\type_2 := J_{10}^+$;
\State List 2a := $a_2,b_2$;
\State $MP_3:= $ factor of $h^+(x)$ over $\Q$ with a real root in the interval $(a_4,b_4)$ ;
\State $NF_3:= x^3+ax^2y^2+xy^4$;
\State $\type_3 := J_{10}^+$;
\State List 3a := $a_4,b_4$;
\Else
\State $\MP_2:= $ factor of $h^+(x)$ over $\Q$ with a real root on the interval $(a_1,b_1)$;
\State $\NF_2:= x^3+ax^2y^2+xy^4$;
\State $\type_2 := J_{10}^+$;
\State List 2a := $a_1,b_1$;
\State $\MP_3:= $ factor of $h^+(x)$ over $\Q$ with a real root in $(a_3,b_3)$;
\State $\NF_3:= x^3+ax^2y^2+xy^4$;
\State $\type_3 := J_{10}^+$;
\State List 3a := $a_3,b_3$;
\EndIf
\State List 2 := $\type_2, \NF_2, \MP_2$, List 2a;
\State List 3 := $\type_3, \NF_3, \MP_3$, List 3a;
\Return( List1, List2, List3)
\EndIf
\end{algorithmic}
\end{algorithm}
%\clearpage


\begin{thebibliography}{99}

\bibitem[{Arnold(1974)}]{A1974}
Arnold, V.I., 1974.
Normal forms of functions in neighbourhoods of degenerate critical points.
Russ. Math. Surv. 29(2), 10-50.

\bibitem[{Arnold et al.(1985)}]{AVG1985}
Arnold, V.I., Gusein-Zade, S.M., Varchenko, A.N., 1985.
Singularities of Differential Maps, Vol.~I.
Birkh\"auser, Boston.

\bibitem[{Decker et al.(2012)}]{DGPS}
Decker, W., Greuel, G.-M., Pfister, G., Sch{\"o}nemann, H., 2012.
\newblock {\sc Singular} {3-1-6} -- {A} computer algebra system for polynomial
computations. \\
\url{http://www.singular.uni-kl.de}

\bibitem[{de Jong and Pfister(2000)}]{PdJ2000}
de Jong, T., Pfister, G., 2000.
Local Analytic Geometry.
Vieweg, Braunschweig.

\bibitem[{Greuel et al.(2007)}]{GLS2007}
Greuel, G.-M., Lossen, C., Shustin, E., 2007.
Introduction to Singularities and Deformations.
Springer, Berlin.

\bibitem[{Greuel and Pfister(2008)}]{GP2008}
Greuel, G.-M., Pfister, G., 2008.
A Singular Introduction to Commutative Algebra, second ed.
Springer, Berlin.

\bibitem[{Kr\"uger(1997)}]{Kruger}
Kr\"uger, K., 1997.
Klassifikation von Hyperfl\"achensingularit\"aten.
Diploma Thesis, University of Kaiserslautern.
{\par\raggedright
\url{ftp://www.mathematik.uni-kl.de/pub/Math/Singular/doc/Papers/%
diplom_krueger.ps.gz}
\par}

\bibitem[{Kr\"uger(2012)}]{classify}
Kr\"uger, K., 2012.
{\tt classify.lib}. {A} {\sc Singular} {3-1-6} library for classifying isolated
hypersurface singularities w.r.t.\@ right equivalence, based on the
determinator of singularities by V.I. Arnold.

\bibitem[{Marais and Steenpa\ss(2012)}]{realclassify}
Marais, M., Steenpa\ss, A., 2012.
{\tt realclassify.lib}. {A} {\sc Singular} {3-1-6} library for classifying
isolated hypersurface singularities over the reals w.r.t.\@ right equivalence.

\bibitem[{Marais and Steenpa\ss(2012)}]{realclassify1}
Marais, M., Steenpa\ss, A., 2012. The classification of real singularities using \textsc{SINGULAR} Part I: Splitting Lemma and Simple Singularities.

\bibitem[{Marais and Steenpa\ss(2013)}]{realclassify2}
Marais, M., Steenpa\ss, A., 2012. The classification of real singularities using \textsc{SINGULAR} Part II: The Structure of the Equivalence Classes of the Unimodal Singularities.

\bibitem[{Siersma(1974)}]{Siersma}
Siersma, D., 1974.
Classification and Deformation of Singularities.
Dissertation, University of Amsterdam.
{\par\raggedright
\url{http://www.staff.science.uu.nl/~siers101/ArticleDownloads/%
DissertationSiersma.pdf}
\par}

\bibitem[{Tobis(2012)}]{roots}
Tobis, E., 2012.
{\tt rootsur.lib}. {A} {\sc Singular} {3-1-6} library for counting the number
of real roots of a univariate polynomial.

\end{thebibliography}

\end{document}
