\documentclass[noend]{amsproc}
\linespread{1}
\usepackage{amsthm,amsmath,amsfonts,mathrsfs,amssymb}
\usepackage{algorithm}
\usepackage{algorithmic}
\usepackage{multirow}

\newtheorem{theorem}{Theorem}
\newtheorem{defn}[theorem]{Definition}
\newtheorem{lemma}[theorem]{Lemma}
\theoremstyle{definition}
\newtheorem{remark}[theorem]{Remark}

% ALGORITHM style
%%%%%%%%%%%%%%%%%%%%%%%%%%%%%%%%%%%
\renewcommand{\algorithmicrequire}{\textbf{Input:}}
\renewcommand{\algorithmicensure}{\textbf{Output:}}

\newcommand{\Singular}{\textsc{Singular}}
\newcommand{\realclassify}{\texttt{realclassify.lib}}
\newcommand{\NF}[1]{\operatorname{NF}(#1)}

\DeclareMathOperator{\ord}{ord}
\DeclareMathOperator{\requiv}{\overset{r}{\sim}}
\DeclareMathOperator{\m}{\mathfrak{m}}
\DeclareMathOperator{\jt}{jet}
\DeclareMathOperator{\supp}{supp}
\DeclareMathOperator{\sign}{sign}
\DeclareMathOperator{\N}{\mathbb{N}}
\DeclareMathOperator{\Z}{\mathbb{Z}}
\DeclareMathOperator{\R}{\mathbb{R}}
\DeclareMathOperator{\C}{\mathbb{C}}
\DeclareMathOperator{\K}{\mathbb{K}}
\DeclareMathOperator{\T}{T}
\DeclareMathOperator{\Aut}{Aut}

\title[The classification of real singularities using \textsc{Singular}, %
Part II]%
{The classification of real singularities using \textsc{Singular}\\
Part II: The Structure of the Equivalence Classes of the Unimodal %
Singularities}

\author{Magdaleen S. Marais}
\address{Magdaleen S. Marais\\
African Institute for Mathematical Sciences and Stellenbosch University\\
6 Melrose Rd\\
Muizenberg 7945, Cape Town\\
South Africa}
\email{magdaleen@aims.ac.za}

\author{Andreas Steenpa\ss}
\address{Andreas Steenpa\ss\\
Department of Mathematics\\
University of Kaiserslautern\\
Erwin-Schr\"odinger-Str.\\
67663 Kaiserslautern\\
Germany}
\email{steenpass@mathematik.uni-kl.de}

\thanks{ }
\subjclass[2000]{}
\keywords{}
\begin{document}
\begin{abstract}
The algorithms implemented in the library ``realclassify.lib" in
\textsc{singular} are discussed in this paper. The purpose of this library is
to classify the~$0$ and $1$ modal isolated hypersurface singularities at $0$ of
corank $0$, $1$ and $2$ over the real numbers as computed by V.I.~Arnold in
\cite{AVG1985}.
\end{abstract}
\maketitle


\section{Introduction}


\section{The Sets of Parameter Transformations
$\boldsymbol{P_1}$, $\boldsymbol{P_2}$, and $\boldsymbol{P_3}$}

For the unimodal singularities, we add the values of the parameter which occurs
in the normal form as given in Table~\ref{tab:normal_forms} in parentheses to
the name of the singularity (sub-)type if we want to refer specifically to the
corresponding equivalence class. For instance, we denote by $E_{14}(3)$ the
(complex or real) right-equivalence class of $x^3+y^8+3xy^6$.

For any specific equivalence class $C$, we denote by $\NF{C}$ its normal form
as shown in Table~\ref{tab:normal_forms}, i.e.\@ we write
$\NF{E_{14}(a)} = \NF{E_{14}^+(a)}$ for the polynomial $x^3+y^8+axy^6$ and
$\NF{E_{14}^-(a)}$ for $x^3-y^8+axy^6$.

\begin{defn}
Let $\K$ be either $\R$ or $\C$ and let $f$ and $g$ be two power series in
$\K[[x_1, \ldots, x_n]]$. We denote the set of $\K$-algebra automorphisms
of $\K[[x_1, \ldots, x_n]]$ which take $f$ to $g$ by $\T_{\K}(f,g)$, i.e.\@
\[
\T_{\K}(f,g)
:= \{\phi \in \Aut_{\K}(\K[[x_1, \ldots, x_n]]) \mid \phi(f) = g \} \,.
\]
\end{defn}

\begin{remark}
Let $\K$ be either $\R$ or $\C$. Let $\psi':\; \K[a] \rightarrow \K(b)$ be a
$\K$-algebra homomorphism. Then $\ker(\psi')$ is a prime ideal in $\K[a]$, thus
$\psi'$ can be extended to a $\K$-algebra homomorphism
$\psi:\; \K[a]_{\ker(\psi')} \rightarrow \K(b)$ defined by
$\psi(a) := \psi'(a)$.

Note that $\ker(\psi') = \langle 0 \rangle$ if and only if
$\psi'(a) \not\in \K \subset \K(b)$ (with the usual inclusion). In this case we
get a $\K$-algebra homomorphism $\psi:\; \K(a) \rightarrow \K(b)$.

Let $f \in \K(a)[[x_1,\ldots,x_n]]$ be a power series. We write $f(q)$ with
$q \in \K(b)$ as a shorthand notation for $\psi^*(f)$ where
\[
\psi^*:\;
(\K[a]_{\ker(\psi')})[[x_1,\ldots,x_n]] \rightarrow \K(b)[[x_1,\ldots,x_n]]
\]
is the canonical extension of the $\K$-algebra homomorphism
$\psi:\; \K[a]_{\ker(\psi')} \rightarrow \K(b)$
given by $\psi(a) := q$ as above. In order that this notation is well-defined,
we have to make sure that
$f \in (\K[a]_{\ker(\psi')})[[x_1,\ldots,x_n]]
\subseteq \K(a)[[x_1,\ldots,x_n]]$
(with the usual inclusion). For $\psi(a) = \psi'(a) = q \not\in \K$, this
condition always holds.

As a special case, we may also use this notation $b = a$ and thus get a
$\K$-algebra homomorphism $\psi:\; \K[a]_{\ker(\psi')} \rightarrow \K(a)$.
For $\psi'(a) = q \in \K$, we may even forget about $b$ and simply consider
$\psi:\; \K[a]_{\ker(\psi')} \rightarrow \K$. Note
that the notation $f(q)$ is compatible with the notations for equivalence
classes and normal forms introduced above in the sense that, e.g.,
$\NF{E_{14}(a)}(b) = \NF{E_{14}(b)}$.
\end{remark}

\begin{defn}
\phantom{X}\hfill
\begin{enumerate}
\item
Given power series $f,g \in \C(a)[[x_1,\ldots,x_n]]$, we define the
first set of parameter changes of $f$ and $g$ as
\begin{align*}
P_1(f, g)
:= \{ (u, v) \in \C^2 \mid
&f(u) \text{ and } g(v) \text{ are well-defined and } \\
&\T_{\C}(f(u), g(v)) \neq \varnothing \} \,.
\end{align*}

\item
Given power series $f,g \in \R(a)[[x_1,\ldots,x_n]]$, we define the
second set of parameter changes of $f$ and $g$ as
\begin{align*}
P_2(f, g)
:= \{ (u, v) \in \R^2 \mid
&f(u) \text{ and } g(v) \text{ are well-defined and } \\
&\T_{\C}(f(u), g(v)) \neq \varnothing \} \,.
\end{align*}

\item
Given power series $f,g \in \R(a)[[x_1,\ldots,x_n]]$, we define the
third set of parameter changes of $f$ and $g$ as
\begin{align*}
P_3(f, g)
:= \{ (u, v) \in \R^2 \mid
&f(u) \text{ and } g(v) \text{ are well-defined and } \\
&\T_{\R}(f(u), g(v)) \neq \varnothing \} \,.
\end{align*}

\item
For any polynomial $p(X) \in \C[X]$, we define the sets $C(p(X))$ and $R(p(X))$
as
\begin{align*}
C(p(X)) &:= \{ (a, ra) \in \C^2 \mid r \in \C, \; p(r) = 0 \} \,, \\
R(p(X)) &:= \{ (a, ra) \in \R^2 \mid r \in \C, \; p(r) = 0 \} \,.
\end{align*}
\end{enumerate}
\end{defn}

\begin{remark}
\phantom{X}\hfill
\begin{enumerate}
\item
Note that $P_3(f, g) \subseteq P_2(f, g) \subseteq P_1(f, g)$ for any two power
series $f,g \in \R(a)[[x_1,\ldots,x_n]]$.

\item
For any two unimodal singularity (sub-)types $T_1, T_2$, we simply write
$P_i(T_1,T_2)$ instead of $P_i(\NF{T_1(a)}, \NF{T_2(a)})$, e.g., we write
$P_1(E_{14}, E_{14})$ for $P_1(\NF{E_{14}(a)}, \NF{E_{14}(a)})$.
\end{enumerate}
\end{remark}


\section{Generic Coordinate Transformations}


\section{On the Computation of the Results}


\section{Results}

\begin{theorem}
The structure of the equivalence classes of the exceptional unimodal
singularities is as shown in Table~\ref{tab:exceptional_equivalences}.

\begin{table}[!htb]
\centering
\caption{$P_1$, $P_2$ and $P_3$ for the exceptional unimodal singularities}
\label{tab:exceptional_equivalences}
\begin{tabular}{|c|c||c|c|c|}
\hline
$T_1$ & $T_2$ & $P_1(T_1, T_2)$ & $P_2(T_1, T_2)$ & $P_3(T_1, T_2)$ \\
\hline\hline
$E_{12}$   & $E_{12}$   & $C(X^{21}-1)$ & $R(X-1)$      & $R(X-1)$ \\
\hline
$E_{13}$   & $E_{13}$   & $C(X^{15}-1)$ & $R(X-1)$      & $R(X-1)$ \\
\hline
$E_{14}^+$ & $E_{14}^+$ & $C(X^{12}-1)$ & $R(X^2-1)$    & $R(X-1)$ \\
\hline
$E_{14}^-$ & $E_{14}^-$ & $C(X^{12}-1)$ & $R(X^2-1)$    & $R(X-1)$ \\
\hline
$E_{14}^+$ & $E_{14}^-$ & $C(X^{12}+1)$ & $\varnothing$ & $\varnothing$ \\
\hline
$E_{14}^-$ & $E_{14}^+$ & $C(X^{12}+1)$ & $\varnothing$ & $\varnothing$ \\
\hline
$Z_{11}$   & $Z_{11}$   & $C(X^{15}-1)$ & $R(X-1)$      & $R(X-1)$ \\
\hline
$Z_{12}$   & $Z_{12}$   & $C(X^{11}-1)$ & $R(X-1)$      & $R(X-1)$ \\
\hline
$Z_{13}^+$ & $Z_{13}^+$ & $C(X^9-1)$    & $R(X-1)$      & $R(X-1)$ \\
\hline
$Z_{13}^-$ & $Z_{13}^-$ & $C(X^9-1)$    & $R(X-1)$      & $R(X-1)$ \\
\hline
$Z_{13}^+$ & $Z_{13}^-$ & $C(X^9+1)$    & $R(X+1)$      & $\varnothing$ \\
\hline
$Z_{13}^-$ & $Z_{13}^+$ & $C(X^9+1)$    & $R(X+1)$      & $\varnothing$ \\
\hline
$W_{12}^+$ & $W_{12}^+$ & $C(X^{10}-1)$ & $R(X^2-1)$    & $R(X-1)$ \\
\hline
$W_{12}^-$ & $W_{12}^-$ & $C(X^{10}-1)$ & $R(X^2-1)$    & $R(X-1)$ \\
\hline
$W_{12}^+$ & $W_{12}^-$ & $C(X^{10}+1)$ & $\varnothing$ & $\varnothing$ \\
\hline
$W_{12}^-$ & $W_{12}^+$ & $C(X^{10}+1)$ & $\varnothing$ & $\varnothing$ \\
\hline
$W_{13}^+$ & $W_{13}^+$ & $C(X^8-1)$    & $R(X^2-1)$    & $R(X-1)$ \\
\hline
$W_{13}^-$ & $W_{13}^-$ & $C(X^8-1)$    & $R(X^2-1)$    & $R(X-1)$ \\
\hline
$W_{13}^+$ & $W_{13}^-$ & $C(X^8+1)$    & $\varnothing$ & $\varnothing$ \\
\hline
$W_{13}^-$ & $W_{13}^+$ & $C(X^8+1)$    & $\varnothing$ & $\varnothing$ \\
\hline
\end{tabular}
\end{table}

\end{theorem}

\begin{theorem}
The structure of the equivalence classes of the $J_{10+k}$
singularities is as shown in Table~\ref{tab:J10+k_equivalences} where in each
case,
\begin{align*}
l &= \frac{6}{\gcd(6,k)}, \text{ and} \\
s &=
\begin{cases}
  1,  &\text{if } k \equiv 2 \pmod{4}, \\
  -1, &\text{else.}
\end{cases}
\end{align*}

\begin{table}[htb]
\centering
\caption{$P_1$, $P_2$ and $P_3$ for the $J_{10+k}$ singularities}
\label{tab:J10+k_equivalences}
\begin{tabular}{|c|c||c|c|c|}
\hline

$T_1$ & $T_2$ & $P_1(T_1, T_2)$ & $P_2(T_1, T_2)$ & $P_3(T_1, T_2)$ \\
\hline\hline

$J_{10+k}^+$ & $J_{10+k}^+$ & $C(X^l-1)$ &
$\begin{array}{ll}
  R(X^2-1), &\text{if } l \text{ is even} \\
  R(X-1), &\text{if } l \text{ is odd}
\end{array}$ &
$\begin{array}{ll}
  R(X^2-1), &\text{if } l \text{ is even} \\
  R(X-1), &\text{if } l \text{ is odd}
\end{array}$ \\
\hline

$J_{10+k}^-$ & $J_{10+k}^-$ & $C(X^l-1)$ &
$\begin{array}{ll}
  R(X^2-1), &\text{if } l \text{ is even} \\
  R(X-1), &\text{if } l \text{ is odd}
\end{array}$ &
$\begin{array}{ll}
  R(X^2-1), &\text{if } l \text{ is even} \\
  R(X-1), &\text{if } l \text{ is odd}
\end{array}$ \\
\hline

$J_{10+k}^+$ & $J_{10+k}^-$ & $C(X^l-s)$ &
$\begin{array}{ll}
  \varnothing, &\text{if } l \text{ is even} \\
  R(X-s), &\text{if } l \text{ is odd}
\end{array}$ &
$\varnothing$ \\
\hline

$J_{10+k}^-$ & $J_{10+k}^+$ & $C(X^l-s)$ &
$\begin{array}{ll}
  \varnothing, &\text{if } l \text{ is even} \\
  R(X-s), &\text{if } l \text{ is odd}
\end{array}$ &
$\varnothing$ \\
\hline
\end{tabular}
\end{table}

\end{theorem}

\begin{theorem}
The structure of the equivalence classes of the $X_{9+k}$
singularities is as shown in Table~\ref{tab:X9+k_equivalences} where in each
case,
\begin{align*}
l &= \frac{4}{\gcd(4,k)}, \text{ and} \\
s &=
\begin{cases}
  1,  &\text{if } k \equiv 4 \pmod{8}, \\
  -1, &\text{else.}
\end{cases}
\end{align*}

\begin{table}[htb]
\centering
\caption{$P_1$, $P_2$ and $P_3$ for the $X_{9+k}$ singularities}
\label{tab:X9+k_equivalences}
\begin{tabular}{|c|c||c|c|c|}
\hline

$T_1$ & $T_2$ & $P_1(T_1, T_2)$ & $P_2(T_1, T_2)$ & $P_3(T_1, T_2)$ \\
\hline\hline

$X_{9+k}^{++}$ & $X_{9+k}^{++}$ & $C(X^l-1)$ &
$\begin{array}{ll}
  R(X^2-1), &\text{if } l \text{ is even} \\
  R(X-1), &\text{if } l \text{ is odd}
\end{array}$ &
$\begin{array}{ll}
  R(X^2-1), &\text{if } k \text{ is odd} \\
  R(X-1), &\text{if } k \text{ is even}
\end{array}$ \\
\hline

$X_{9+k}^{++}$ & $X_{9+k}^{+-}$ & $C(X^l-1)$ &
$\begin{array}{ll}
  R(X^2-1), &\text{if } l \text{ is even} \\
  R(X-1), &\text{if } l \text{ is odd}
\end{array}$ &
$\varnothing$ \\
\hline

$X_{9+k}^{++}$ & $X_{9+k}^{-+}$ & $C(X^l-s)$ &
$\begin{array}{ll}
  R(X-1), &\text{if } k \equiv 4 \pmod{8} \\
  R(X+1), &\text{if } k \equiv 0 \pmod{8} \\
  \varnothing, &\text{else}
\end{array}$ &
$\varnothing$ \\
\hline

$X_{9+k}^{++}$ & $X_{9+k}^{--}$ & $C(X^l-s)$ &
$\begin{array}{ll}
  R(X-1), &\text{if } k \equiv 4 \pmod{8} \\
  R(X+1), &\text{if } k \equiv 0 \pmod{8} \\
  \varnothing, &\text{else}
\end{array}$ &
$\varnothing$ \\
\hline
\end{tabular}
\end{table}

\end{theorem}


 \begin{thebibliography}{99}
\bibitem{AVG1985} Arnold, V.I.; Gusein-Zade, S.M.; Varchenko, A.N.:
Singularities of Differential Maps. Vol.~I, Birkh\"auser (1985).
\bibitem{A1975} Arnold, V.I.:
\textit{Normal form of functions near degenerate critical points.},
Russian Mth. Surveys 29 ii (1975), 10-50.
\bibitem{DGPS}
Decker, W.; Greuel, G.-M.; Pfister, G.; Sch{\"o}nemann, H.:
\newblock {\sc Singular} {3-1-5} --- {A} computer algebra system for polynomial
computations.
\newblock {http://www.singular.uni-kl.de} (2012).
\bibitem{Kruger} Kr\"uger, K.: Klassifikation von
Hyperfl\"agensingularit\"aten, Diploma Thesis (1997).
\bibitem{GLS2007}Greuel, G.-M.; Lossen, C.; Shustin E.:
Introduction to Singularities and Deformations, Springer, Berlin (2007).
\bibitem{GP2008} Greuel G.-M.; Pfister G.;
A Singular introduction to Commutative Algebra, 2nd Ed., Springer,
Berlin (2008).
\bibitem{classify}
Kr\"uger, K.:
{\tt classify.lib}. {A} {\sc Singular} {3-1-5} library for classifying isolated
hypersurface singularities w.r.t. right equivalence, based on the determinator
of singularities by V.I. Arnold (2012).
\bibitem{realclassify}
Marais, M. and Steenpass, A.:
{\tt realclassify.lib}. {A} {\sc Singular} {3-1-5} library for classifying
isolated hypersurface singularities over the reals w.r.t. right equivalence,
based on the determinator of singularities by V.I. Arnold. This library is
based on classify.lib by Kai Kr\"uger, but handles the real case, while
classify.lib does the complex classification (2012).
\bibitem{primdec.lib} Pfister, G.; Decker, W.;  Schoenemann, H.; Laplagne, S.:
{\tt primdec.lib}. {A} {\sc Singular} {3-1-5} library for Primary Decomposition
and Radical of Ideals (2012).
\bibitem{Siersma} Siersma D.: Classification and deformation of Singularities,
disertation, University of Amsterdam (1974).
\bibitem{roots}
Tobis, A.:
{\tt rootsur.lib}. {A} {\sc Singular} {3-1-5} library for Counting number of
real roots of univariate polynomial (2012).
\bibitem{solve.lib} Wenk, M.: {\tt solve.lib}. Pohl, W.:
{A} {\sc Singular} {3-1-5} library for Complex Solving of Polynomial Systems
(2012).

\end{thebibliography}

\end{document}
