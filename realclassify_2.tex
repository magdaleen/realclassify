\documentclass[noend]{amsproc}

\renewcommand{\arraystretch}{1.3}
\linespread{1}

\usepackage{amsthm,amsmath,amsfonts,mathrsfs,amssymb}
\usepackage{algorithm}
\usepackage{algorithmic}
\usepackage{multirow}
\usepackage{rotating}

\newtheorem{theorem}{Theorem}
\newtheorem{defn}[theorem]{Definition}
\newtheorem{lemma}[theorem]{Lemma}
\theoremstyle{definition}
\newtheorem{remark}[theorem]{Remark}

% ALGORITHM style
%%%%%%%%%%%%%%%%%%%%%%%%%%%%%%%%%%%
\renewcommand{\algorithmicrequire}{\textbf{Input:}}
\renewcommand{\algorithmicensure}{\textbf{Output:}}

\newcommand{\Singular}{\textsc{Singular}}
\newcommand{\realclassify}{\texttt{realclassify.lib}}
\newcommand{\NF}[1]{\operatorname{NF}(#1)}
\newcommand{\tY}{\widetilde{Y}}

\DeclareMathOperator{\ord}{ord}
\DeclareMathOperator{\requiv}{\overset{r}{\sim}}
\DeclareMathOperator{\m}{\mathfrak{m}}
\DeclareMathOperator{\jt}{jet}
\DeclareMathOperator{\supp}{supp}
\DeclareMathOperator{\sign}{sign}
\DeclareMathOperator{\N}{\mathbb{N}}
\DeclareMathOperator{\Z}{\mathbb{Z}}
\DeclareMathOperator{\R}{\mathbb{R}}
\DeclareMathOperator{\C}{\mathbb{C}}
\DeclareMathOperator{\K}{\mathbb{K}}
\DeclareMathOperator{\T}{T}
\DeclareMathOperator{\Aut}{Aut}
\DeclareMathOperator{\Quot}{Quot}

\title[The classification of real singularities using \textsc{Singular}, %
Part II]%
{The classification of real singularities using \textsc{Singular}\\
Part II: The Structure of the Equivalence Classes of the Unimodal %
Singularities}

\author{Magdaleen S. Marais}
\address{Magdaleen S. Marais\\
African Institute for Mathematical Sciences and Stellenbosch University\\
6 Melrose Rd\\
Muizenberg 7945, Cape Town\\
South Africa}
\email{magdaleen@aims.ac.za}

\author{Andreas Steenpa\ss}
\address{Andreas Steenpa\ss\\
Department of Mathematics\\
University of Kaiserslautern\\
Erwin-Schr\"odinger-Str.\\
67663 Kaiserslautern\\
Germany}
\email{steenpass@mathematik.uni-kl.de}

\thanks{ }
\subjclass[2000]{}
\keywords{}
\begin{document}
\begin{abstract}
The algorithms implemented in the library ``realclassify.lib" in
\textsc{singular} are discussed in this paper. The purpose of this library is
to classify the~$0$ and $1$ modal isolated hypersurface singularities at $0$ of
corank $0$, $1$ and $2$ over the real numbers as computed by V.I.~Arnold in
\cite{AVG1985}.
\end{abstract}
\maketitle


\section{Introduction}

\begin{table}[htbp]
\centering
\caption{Normal forms of singularities of modality $1$ and corank $2$.}
\label{tab:normal_forms}
\begin{tabular}{|c|c|c|c|c|}
\hline

\multicolumn{1}{|c}{}
 & & Complex     & Normal forms     & \multirow{2}{*}{Restrictions} \\
\multicolumn{1}{|c}{}
 & & normal form & of real subtypes &                               \\
\hline\hline


\multirow{6}{*}{\begin{sideways}Parabolic\end{sideways}}

& \multirow{4}{*}{$X_9$} & \multirow{4}{*}{$x^4+ax^2y^2+y^4$}
  & $+x^4+ax^2y^2+y^4$ $(X_9^{++})$ & \multirow{2}{*}{$a^2\neq4$} \\\cline{4-4}
&&& $-x^4+ax^2y^2-y^4$ $(X_9^{--})$ &                             \\\cline{4-5}
&&& $+x^4+ax^2y^2-y^4$ $(X_9^{+-})$ & \multirow{2}{*}{$a^2\neq-4$}\\\cline{4-4}
&&& $-x^4+ax^2y^2+y^4$ $(X_9^{-+})$ &                             \\\cline{2-5} 

& \multirow{2}{*}{$J_{10}$} & \multirow{2}{*}{$x^3+ax^2y^2+xy^4$}
  & $x^3+ax^2y^2+xy^4$ $(J_{10}^+)$ & $a^2 \neq 4$ \\ \cline{4-5}
&&& $x^3+ax^2y^2-xy^4$ $(J_{10}^-)$ & -            \\ \hline


\multirow{12}{*}{\begin{sideways}Hyperbolic\end{sideways}}

& \multirow{2}{*}{$J_{10+k}$} & \multirow{2}{*}{$x^3+x^2y^2+ay^{6+k}$}
  & $x^3+x^2y^2+ay^{6+k}$ $(J_{10+k}^+)$
      & \multirow{2}{*}{$a \neq 0,\; k > 0$} \\ \cline{4-4}
&&& $x^3-x^2y^2+ay^{6+k}$ $(J_{10+k}^-)$ &   \\ \cline{2-5}

& \multirow{4}{*}{$X_{9+k}$} & \multirow{4}{*}{$x^4+x^2y^2+ay^{4+k}$}
  & $+x^4+x^2y^2+ay^{4+k}$ $(X_{9+k}^{++})$
      & \multirow{4}{*}{$a \neq 0,\; k > 0$}  \\ \cline{4-4}
&&& $-x^4-x^2y^2+ay^{4+k}$ $(X_{9+k}^{--})$ & \\ \cline{4-4}
&&& $+x^4-x^2y^2+ay^{4+k}$ $(X_{9+k}^{+-})$ & \\ \cline{4-4}
&&& $-x^4+x^2y^2+ay^{4+k}$ $(X_{9+k}^{-+})$ & \\ \cline{2-5}

& \multirow{4}{*}{$Y_{r,s}$} & \multirow{4}{*}{$x^2y^2+x^r+ay^s$}
  & $+x^2y^2+x^r+ay^s$ $(Y_{r,s}^{++})$
      & \multirow{4}{*}{$a \neq 0,\; r,s > 4$} \\ \cline{4-4}
&&& $-x^2y^2-x^r+ay^s$ $(Y_{r,s}^{--})$ &      \\ \cline{4-4}
&&& $+x^2y^2-x^r+ay^s$ $(Y_{r,s}^{+-})$ &      \\ \cline{4-4}
&&& $-x^2y^2+x^r+ay^s$ $(Y_{r,s}^{-+})$ &      \\ \cline{2-5}

& \multirow{2}{*}{$\tY_r$} & \multirow{2}{*}{$(x^2+y^2)^2+ax^r$}
  & $+(x^2+y^2)^2+ax^r$ $(\tY_r^+)$
      & \multirow{2}{*}{$a \neq 0,\; r > 4$} \\ \cline{4-4}
&&& $-(x^2+y^2)^2+ax^r$ $(\tY_r^-)$ &        \\ \hline


\multirow{12}{*}{\begin{sideways}Exceptional\end{sideways}}

& $E_{12}$ & $x^3+y^7+axy^5$ & $x^3+y^7+axy^5$ & - \\ \cline{2-5}

& $E_{13}$ & $x^3+xy^5+ay^8$ & $x^3+xy^5+ay^8$ & - \\ \cline{2-5}

& \multirow{2}{*}{$E_{14}$} & \multirow{2}{*}{$x^3+y^8+axy^6$}
  & $x^3+y^8+axy^6$ $(E_{14}^+)$ & \multirow{2}{*}{-} \\ \cline{4-4}
&&& $x^3-y^8+axy^6$ $(E_{14}^-)$ &                    \\ \cline{2-5}

& $Z_{11}$ & $x^3y+y^5+axy^4$ & $x^3y+y^5+axy^4$ & - \\ \cline{2-5}

& $Z_{12}$ & $x^3y+xy^4+ax^2y^3$ & $x^3y+xy^4+ax^2y^3$ & - \\ \cline{2-5}

& \multirow{2}{*}{$Z_{13}$} & \multirow{2}{*}{$x^3y+y^6+axy^5$}
  & $x^3y+y^6+axy^5$ $(Z_{13}^+)$ & \multirow{2}{*}{-} \\ \cline{4-4}
&&& $x^3y-y^6+axy^5$ $(Z_{13}^-)$ &                    \\ \cline{2-5}

& \multirow{2}{*}{$W_{12}$} & \multirow{2}{*}{$x^4+y^5+ax^2y^3$}
  & $+x^4+y^5+ax^2y^3$ $(W_{12}^+)$ & \multirow{2}{*}{-} \\ \cline{4-4}
&&& $-x^4+y^5+ax^2y^3$ $(W_{12}^-)$ &                    \\ \cline{2-5}

& \multirow{2}{*}{$W_{13}$} & \multirow{2}{*}{$x^4+xy^4+ay^6$}
  & $+x^4+xy^4+ay^6$ $(W_{13}^+)$ & \multirow{2}{*}{-} \\ \cline{4-4}
&&& $-x^4+xy^4+ay^6$ $(W_{13}^-)$ &                    \\ \hline

\end{tabular}
\end{table}


\section{The Sets of Parameter Transformations
$\boldsymbol{P_1}$, $\boldsymbol{P_2}$, and $\boldsymbol{P_3}$}

For the unimodal singularities, we add the values of the parameter which occurs
in the normal form as given in Table~\ref{tab:normal_forms} in parentheses to
the name of the singularity (sub-)type if we want to refer specifically to the
corresponding equivalence class. For instance, we denote by $E_{14}(3)$ the
(complex or real) right-equivalence class of $x^3+y^8+3xy^6$.

For any specific equivalence class $C$, we denote by $\NF{C}$ its normal form
as shown in Table~\ref{tab:normal_forms}, i.e.\@ we write
$\NF{E_{14}(a)} = \NF{E_{14}^+(a)}$ for the polynomial $x^3+y^8+axy^6$ and
$\NF{E_{14}^-(a)}$ for $x^3-y^8+axy^6$.

\begin{defn}
Let $\K$ be either $\R$ or $\C$ and let $f$ and $g$ be two power series in
$\K[[x_1, \ldots, x_n]]$. We denote the set of $\K$-algebra automorphisms
of $\K[[x_1, \ldots, x_n]]$ which take $f$ to $g$ by $\T_{\K}(f,g)$, i.e.\@
\[
\T_{\K}(f,g)
:= \{\phi \in \Aut_{\K}(\K[[x_1, \ldots, x_n]]) \mid \phi(f) = g \} \,.
\]
\end{defn}

\begin{remark}
Let $\K$ be either $\R$ or $\C$. As usual, we denote the quotient field
$\Quot(\K[a])$ by $\K(a)$. Let $f \in \K(a)[[x_1,\ldots,x_n]]$ be a power
series over this quotient field. Then $f$ can be written as
$f = \sum_{\nu \in \N^n} c_\nu \boldsymbol{x}^\nu$ with coeffients
$c_\nu = \frac{p_\nu}{q_\nu} \in \K(a)$ where $p_\nu, q_\nu \in \K[a]$ are
polynomials and $q_\nu \neq 0$ for $\nu \in \N^n$.

If we consider the polynomials $p_\nu, q_\nu$ as polynomial functions
$p_\nu, q_\nu: \; \K \rightarrow \K$, then we may also consider the
coefficients $c_\nu$ as functions
$c_\nu: \; \K \setminus V(q_\nu) \rightarrow \K$ where $V(q_\nu)$ is the set of
points where $q_\nu$ vanishes. Via this correspondence, we finally get power
series
$f(u) := \sum_{\nu \in \N^n} c_\nu(u) \boldsymbol{x}^\nu
\in \K[[x_1,\ldots,x_n]]$ for each value
$u \in \K \setminus \bigcup_{\nu \in \N^n} V(q_\nu)$.

Note that the notation $f(u)$ is compatible with the notations for equivalence
classes and normal forms introduced above in the sense that, e.g.,
$\NF{E_{14}(a)}(b) = \NF{E_{14}(b)}$.
\end{remark}

\begin{defn}
\phantom{X}\hfill
\begin{enumerate}
\item
Given power series $f,g \in \C(a)[[x_1,\ldots,x_n]]$, we define the
first set of parameter transformations of $f$ and $g$ as
\begin{align*}
P_1(f, g)
:= \{ (u, v) \in \C^2 \mid
&f(u) \text{ and } g(v) \text{ are well-defined and } \\
&\T_{\C}(f(u), g(v)) \neq \varnothing \} \,.
\end{align*}

\item
Given power series $f,g \in \R(a)[[x_1,\ldots,x_n]]$, we define the
second set of parameter transformations of $f$ and $g$ as
\begin{align*}
P_2(f, g)
:= \{ (u, v) \in \R^2 \mid
&f(u) \text{ and } g(v) \text{ are well-defined and } \\
&\T_{\C}(f(u), g(v)) \neq \varnothing \} \,.
\end{align*}

\item
Given power series $f,g \in \R(a)[[x_1,\ldots,x_n]]$, we define the
third set of parameter transformations of $f$ and $g$ as
\begin{align*}
P_3(f, g)
:= \{ (u, v) \in \R^2 \mid
&f(u) \text{ and } g(v) \text{ are well-defined and } \\
&\T_{\R}(f(u), g(v)) \neq \varnothing \} \,.
\end{align*}
\end{enumerate}
\end{defn}

\begin{remark}
\phantom{X}\hfill
\begin{enumerate}
\item
Note that $P_3(f, g) \subseteq P_2(f, g) \subseteq P_1(f, g)$ for any two power
series $f,g \in \R(a)[[x_1,\ldots,x_n]]$.

\item
For any two unimodal singularity (sub-)types $T_1, T_2$, we simply write
$P_i(T_1,T_2)$ instead of $P_i(\NF{T_1(a)}, \NF{T_2(a)})$, e.g., we write
$P_1(E_{14}, E_{14})$ for $P_1(\NF{E_{14}(a)}, \NF{E_{14}(a)})$.
\end{enumerate}
\end{remark}

It turns out that in some cases, the sets $P_1$, $P_2$ and $P_3$ are just
unions of sets of the form $(a, ra)_{a \in \K}$ for some $r \in \C$ and $\K$
either $\C$ or $\R$. For those cases we use the following auxiliary notation.

\begin{defn}
For any polynomial $p(X) \in \C[X]$, we define the sets $C(p(X))$ and $R(p(X))$
as
\begin{align*}
C(p(X)) &:= \{ (a, ra) \in \C^2 \mid a, r \in \C, \; p(r) = 0 \} \,, \\
R(p(X)) &:= \{ (a, ra) \in \R^2 \mid a, r \in \R, \; p(r) = 0 \} \,.
\end{align*}
\end{defn}

\begin{defn}
For $\Omega \subset \C$, let $(f_i: \Omega \rightarrow \C)_{i \in I}$ be a
family of complex-valued functions on $\Omega$. We define the joint graph of
$(f_i)_{i \in I}$ as
\[
\Gamma_\Omega((f_i)_{i \in I})
:= \{ (a, f_i(a)) \in \Omega \times \C \mid a\in \Omega,\; i \in I \}\,.
\]
\end{defn}


\section{Generic Coordinate Transformations}


\section{On the Computation of the Results}


\section{Results}


\begin{theorem}
The structure of the equivalence classes of the $X_9$ singularities is as shown
in Table~\ref{tab:X9_equivalences} where for $j = 1, \ldots, 6$ and
$\rho, \sigma \in \{1, i\}$, the function $f_j^{\rho, \sigma}$ is defined as
follows:
\begin{align*}
f_1^{\rho, \sigma}(a) &= +\rho \sigma \cdot a \,, &
f_3^{\rho, \sigma}(a) &= \frac{+2\sigma a+12\rho\sigma}{a-2\rho} \,, &
f_5^{\rho, \sigma}(a) &= \frac{-2\sigma a+12\rho\sigma}{a+2\rho} \,, \\
f_2^{\rho, \sigma}(a) &= -\rho \sigma \cdot a \,, &
f_4^{\rho, \sigma}(a) &= \frac{+2\sigma a-12\rho\sigma}{a+2\rho} \,, &
f_6^{\rho, \sigma}(a) &= \frac{-2\sigma a-12\rho\sigma}{a-2\rho} \,.
\end{align*}

Furthermore, we use the following notation:
\begin{align*}
\C'  &:= \C \setminus \{ -2, 2\} \,, &
\R'  &:= \R \setminus \{ -2, 2\} \,, &
\C'' &:= \C \setminus \{ -2i, 2i\} \,.
\end{align*}

\newcommand{\specialvrule}{\rule[-1.8ex]{0pt}{5ex}}
\begin{table}[htb]
\centering
\caption{$P_1$, $P_2$ and $P_3$ for the $X_9$ singularities}
\label{tab:X9_equivalences}
\begin{tabular}{|c|c||c|c|c|}
\hline

$T_1$ & $T_2$ & $P_1(T_1, T_2)$ & $P_2(T_1, T_2)$ & $P_3(T_1, T_2)$ \\
\hline\hline

$X_9^{++}$ & $X_9^{++}$ &
\multirow{6}{*}{$\Gamma_{\C'}\left(f_1^{1,1}, \ldots, f_6^{1,1}\right)$} &
\multirow{6}{*}{$\Gamma_{\R'}\left(f_1^{1,1}, \ldots, f_6^{1,1}\right)$} &
\begin{tabular}[x]{@{}l@{}}
    $\phantom{\cup}\; \Gamma_{\R'}\left(f_1^{1,1}\right)$\specialvrule \\
    $\cup\; \Gamma_{\R^{>-2}}\left(f_5^{1,1}\right)$\specialvrule
\end{tabular}
\\ \cline{1-2}\cline{5-5}

$X_9^{--}$ & $X_9^{--}$ &&&
\begin{tabular}[x]{@{}l@{}}
    $\phantom{\cup}\; \Gamma_{\R'}\left(f_1^{1,1}\right)$\specialvrule \\
    $\cup\; \Gamma_{\R^{<2}}\left(f_3^{1,1}\right)$\specialvrule
\end{tabular}
\\ \cline{1-2}\cline{5-5}

$X_9^{++}$ & $X_9^{--}$ &&&
$\Gamma_{\R^{<-2}}\left(f_4^{1,1}\right)$\specialvrule
\\ \cline{1-2}\cline{5-5}

$X_9^{--}$ & $X_9^{++}$ &&&
$\Gamma_{\R^{>2}}\left(f_6^{1,1}\right)$\specialvrule
\\ \hline


$X_9^{+-}$ & $X_9^{+-}$ &
\multirow{4}{*}{$\Gamma_{\C''}\left(f_1^{i,i}, \ldots, f_6^{i,i}\right)$} &
\multirow{4}{*}{$\Gamma_{\R}\left(f_1^{1,1}, f_2^{1,1}\right)$} &
\multirow{4}{*}{$\Gamma_{\R}\left(f_1^{1,1}\right)$} \\ \cline{1-2}

$X_9^{-+}$ & $X_9^{-+}$ &&& \\ \cline{1-2}

$X_9^{+-}$ & $X_9^{-+}$ &&& \\ \cline{1-2}

$X_9^{-+}$ & $X_9^{+-}$ &&& \\ \hline


$X_9^{++}$ & $X_9^{+-}$ &
\multirow{4}{*}{$\Gamma_{\C'}\left(f_1^{1,i}, \ldots, f_6^{1,i}\right)$} &
\multirow{4}{*}{$\{(-6,0), (0,0), (6,0)\}$} &
\multirow{4}{*}{$\varnothing$} \\ \cline{1-2}

$X_9^{++}$ & $X_9^{-+}$ &&& \\ \cline{1-2}

$X_9^{--}$ & $X_9^{+-}$ &&& \\ \cline{1-2}

$X_9^{--}$ & $X_9^{-+}$ &&& \\ \hline


$X_9^{+-}$ & $X_9^{++}$ &
\multirow{4}{*}{$\Gamma_{\C''}\left(f_1^{i,1}, \ldots, f_6^{i,1}\right)$} &
\multirow{4}{*}{$\{(0,-6), (0,0), (0,6)\}$} &
\multirow{4}{*}{$\varnothing$} \\ \cline{1-2}

$X_9^{-+}$ & $X_9^{++}$ &&& \\ \cline{1-2}

$X_9^{+-}$ & $X_9^{--}$ &&& \\ \cline{1-2}

$X_9^{-+}$ & $X_9^{--}$ &&& \\ \hline
\end{tabular}
\end{table}

\end{theorem}


\begin{theorem}
The structure of the equivalence classes of the $J_{10}$ singularities is as
shown in Table~\ref{tab:J10_equivalences} where for $j = 1, \ldots, 6$ and
$\rho, \sigma \in \{-1, +1\}$, the function $f_j^{\rho, \sigma}$ is defined as
follows:
\begin{align*}
f_1^{\rho, \sigma}(a) &= +\sqrt{\rho \sigma} \cdot a \,, \\
f_2^{\rho, \sigma}(a) &= -\sqrt{\rho \sigma} \cdot a \,, \\
f_3^{\rho, \sigma}(a)
&= + \sqrt{\frac{-\rho \sigma (a^2-\rho \cdot 4) (a^2-\rho \cdot 9)
    + a (a^2-\rho \cdot 3) \sqrt{a^2-\rho \cdot 4}}{2(a^2-\rho \cdot 4)}}\,, \\
f_4^{\rho, \sigma}(a)
&= - \sqrt{\frac{-\rho \sigma (a^2-\rho \cdot 4) (a^2-\rho \cdot 9)
    + a (a^2-\rho \cdot 3) \sqrt{a^2-\rho \cdot 4}}{2(a^2-\rho \cdot 4)}}\,, \\
f_5^{\rho, \sigma}(a)
&= + \sqrt{\frac{-\rho \sigma (a^2-\rho \cdot 4) (a^2-\rho \cdot 9)
    - a (a^2-\rho \cdot 3) \sqrt{a^2-\rho \cdot 4}}{2(a^2-\rho \cdot 4)}}\,, \\
f_6^{\rho, \sigma}(a)
&= - \sqrt{\frac{-\rho \sigma (a^2-\rho \cdot 4) (a^2-\rho \cdot 9)
    - a (a^2-\rho \cdot 3) \sqrt{a^2-\rho \cdot 4}}{2(a^2-\rho \cdot 4)}}\,.
\end{align*}

In each case, $\rho$ and $\sigma$ are given by
\begin{align*}
\rho &=
\begin{cases}
    +1, &\text{if } T_1 = J_{10}^+ \,, \\
    -1, &\text{if } T_1 = J_{10}^- \,, \\
\end{cases}
&\sigma &=
\begin{cases}
    +1, &\text{if } T_2 = J_{10}^+ \,, \\
    -1, &\text{if } T_2 = J_{10}^- \,. \\
\end{cases}
\end{align*}

Furthermore, we use the following notation:
\begin{align*}
\C'  &:= \C \setminus \{ -2, 2\}         \,, &
\R'  &:= \R \setminus \{ -2, 2\}         \,, \\
I_1 &:= \left] {\textstyle\frac{3}{\sqrt{2}}}, \infty \right[ \subset \R \,, &
I_2 &:= \left] 2, {\textstyle\frac{3}{\sqrt{2}}} \right[ \subset \R      \,, \\
I_3 &:= \left] {\textstyle-\frac{3}{\sqrt{2}}}, -2 \right[ \subset \R    \,, &
I_4 &:= \left] -\infty, {\textstyle-\frac{3}{\sqrt{2}}} \right[ \subset \R \,.
\end{align*}

\begin{table}[htb]
\centering
\caption{$P_1$, $P_2$ and $P_3$ for the $X_9$ singularities}
\label{tab:X9_equivalences}
\begin{tabular}{|c|c||c|c|c|}
\hline

$T_1$ & $T_2$ & $P_1(T_1, T_2)$ & $P_2(T_1, T_2)$ & $P_3(T_1, T_2)$ \\
\hline\hline

$J_{10}^+$ & $J_{10}^+$ &
$\Gamma_{\C'}\left(f_1^{\rho,\sigma}, \ldots, f_6^{\rho,\sigma}\right)$ &
\begin{tabular}[x]{@{}l@{}}
    $\phantom{\cup}\; \Gamma_{\R'}\left(f_1^{\rho,\sigma},
        f_2^{\rho,\sigma}\right)$ \\
    $\cup\; \Gamma_{\R^{>+2}}\left(f_3^{\rho,\sigma},
        f_4^{\rho,\sigma}\right)$ \\
    $\cup\; \Gamma_{\R^{<-2}}\left(f_5^{\rho,\sigma},
        f_6^{\rho,\sigma}\right)$ \\
    $\cup \left\{\left(0, \frac{-3}{\sqrt{2}}\right),
        \left(0, \frac{+3}{\sqrt{2}}\right)\right\}$ \\
\end{tabular} &
\begin{tabular}[x]{@{}l@{}}
    $\phantom{\cup}\; \Gamma_{\R'}\left(f_1^{\rho,\sigma}\right)$ \\
    $\cup\; \Gamma_{\R^{>+2}}\left(f_4^{\rho,\sigma}\right)$ \\
    $\cup\; \Gamma_{\R^{<-2}}\left(f_5^{\rho,\sigma}\right)$ \\
\end{tabular} \\
\hline

$J_{10}^-$ & $J_{10}^-$ &
$\Gamma_{\C}\left(f_1^{\rho,\sigma}, \ldots, f_6^{\rho,\sigma}\right)$ &
$\Gamma_{\R}\left(f_1^{\rho,\sigma}, f_2^{\rho,\sigma}\right)$ &
$\Gamma_{\R}\left(f_1^{\rho,\sigma}\right)$ \\
\hline

$J_{10}^+$ & $J_{10}^-$ &
$\Gamma_{\C'}\left(f_1^{\rho,\sigma}, \ldots, f_6^{\rho,\sigma}\right)$ &
\begin{tabular}[x]{@{}l@{}}
    $\phantom{\cup}\; \{(0, 0)\}$ \\
    $\cup\; \Gamma_{\R^{>+2}}\left(f_3^{\rho,\sigma},
        f_4^{\rho,\sigma}\right)$ \\
    $\cup\; \Gamma_{\R^{<-2}}\left(f_5^{\rho,\sigma},
        f_6^{\rho,\sigma}\right)$ \\
\end{tabular} &
\begin{tabular}[x]{@{}l@{}}
    $\phantom{\cup}\; \Gamma_{I_1} \left(f_3^{\rho,\sigma}\right)$ \\
    $\cup\; \Gamma_{I_2} \left(f_4^{\rho,\sigma}\right)$ \\
    $\cup\; \Gamma_{I_3} \left(f_5^{\rho,\sigma}\right)$ \\
    $\cup\; \Gamma_{I_4} \left(f_6^{\rho,\sigma}\right)$ \\
    $\cup \left\{\left(\frac{-3}{\sqrt{2}}, 0\right),
        \left(\frac{+3}{\sqrt{2}}, 0\right)\right\}$ \\
\end{tabular} \\
\hline

$J_{10}^-$ & $J_{10}^+$ &
$\Gamma_{\C}\left(f_1^{\rho,\sigma}, \ldots, f_6^{\rho,\sigma}\right)$ &
\begin{tabular}[x]{@{}l@{}}
    $\phantom{\cup}\; \{(0, 0)\}$ \\
    $\cup\; \Gamma_{\R}\left(f_3^{\rho,\sigma}, \ldots,
        f_6^{\rho,\sigma}\right)$ \\
\end{tabular} &
\begin{tabular}[x]{@{}l@{}}
    $\phantom{\cup}\; \Gamma_{\R^{>0}}\left(f_3^{\rho,\sigma},
        f_6^{\rho,\sigma}\right)$ \\
    $\cup\; \Gamma_{\R^{<0}}\left(f_4^{\rho,\sigma},
        f_5^{\rho,\sigma}\right)$ \\
    $\cup \left\{\left(0, \frac{-3}{\sqrt{2}}\right),
        \left(0, \frac{+3}{\sqrt{2}}\right)\right\}$ \\
\end{tabular} \\
\hline

\end{tabular}
\end{table}

\end{theorem}


\begin{theorem}
The structure of the equivalence classes of the $J_{10+k}$
singularities is as shown in Table~\ref{tab:J10+k_equivalences} where in each
case,
\begin{align*}
l &= \frac{6}{\gcd(6,k)}, \text{ and} \\
s &=
\begin{cases}
  1,  &\text{if } k \equiv 2 \pmod{4}, \\
  -1, &\text{else.}
\end{cases}
\end{align*}

\begin{table}[htb]
\centering
\caption{$P_1$, $P_2$ and $P_3$ for the $J_{10+k}$ singularities}
\label{tab:J10+k_equivalences}
\begin{tabular}{|c|c||c|c|c|}
\hline

$T_1$ & $T_2$ & $P_1(T_1, T_2)$ & $P_2(T_1, T_2)$ & $P_3(T_1, T_2)$ \\
\hline\hline

$J_{10+k}^+$ & $J_{10+k}^+$ & $C(X^l-1)$ &
$\begin{array}{ll}
  R(X^2-1), &\text{if } l \text{ is even} \\
  R(X-1), &\text{if } l \text{ is odd}
\end{array}$ &
$\begin{array}{ll}
  R(X^2-1), &\text{if } l \text{ is even} \\
  R(X-1), &\text{if } l \text{ is odd}
\end{array}$ \\
\hline

$J_{10+k}^-$ & $J_{10+k}^-$ & $C(X^l-1)$ &
$\begin{array}{ll}
  R(X^2-1), &\text{if } l \text{ is even} \\
  R(X-1), &\text{if } l \text{ is odd}
\end{array}$ &
$\begin{array}{ll}
  R(X^2-1), &\text{if } l \text{ is even} \\
  R(X-1), &\text{if } l \text{ is odd}
\end{array}$ \\
\hline

$J_{10+k}^+$ & $J_{10+k}^-$ & $C(X^l-s)$ &
$\begin{array}{ll}
  \varnothing, &\text{if } l \text{ is even} \\
  R(X-s), &\text{if } l \text{ is odd}
\end{array}$ &
$\varnothing$ \\
\hline

$J_{10+k}^-$ & $J_{10+k}^+$ & $C(X^l-s)$ &
$\begin{array}{ll}
  \varnothing, &\text{if } l \text{ is even} \\
  R(X-s), &\text{if } l \text{ is odd}
\end{array}$ &
$\varnothing$ \\
\hline
\end{tabular}
\end{table}

\end{theorem}

\begin{theorem}
The structure of the equivalence classes of the $X_{9+k}$
singularities is as shown in Table~\ref{tab:X9+k_equivalences} where in each
case,
\begin{align*}
l &= \frac{4}{\gcd(4,k)}, \text{ and} \\
s &=
\begin{cases}
  1,  &\text{if } k \equiv 4 \pmod{8}, \\
  -1, &\text{else.}
\end{cases}
\end{align*}

\begin{table}[htb]
\centering
\caption{$P_1$, $P_2$ and $P_3$ for the $X_{9+k}$ singularities}
\label{tab:X9+k_equivalences}
\begin{tabular}{|c|c||c|c|c|}
\hline

$T_1$ & $T_2$ & $P_1(T_1, T_2)$ & $P_2(T_1, T_2)$ & $P_3(T_1, T_2)$ \\
\hline\hline

$X_{9+k}^{++}$ & $X_{9+k}^{++}$ & $C(X^l-1)$ &
$\begin{array}{ll}
  R(X^2-1), &\text{if } l \text{ is even} \\
  R(X-1), &\text{if } l \text{ is odd}
\end{array}$ &
$\begin{array}{ll}
  R(X^2-1), &\text{if } k \text{ is odd} \\
  R(X-1), &\text{if } k \text{ is even}
\end{array}$ \\
\hline

$X_{9+k}^{++}$ & $X_{9+k}^{+-}$ & $C(X^l-1)$ &
$\begin{array}{ll}
  R(X^2-1), &\text{if } l \text{ is even} \\
  R(X-1), &\text{if } l \text{ is odd}
\end{array}$ &
$\varnothing$ \\
\hline

$X_{9+k}^{++}$ & $X_{9+k}^{-+}$ & $C(X^l-s)$ &
$\begin{array}{ll}
  R(X-1), &\text{if } k \equiv 4 \pmod{8} \\
  R(X+1), &\text{if } k \equiv 0 \pmod{8} \\
  \varnothing, &\text{else}
\end{array}$ &
$\varnothing$ \\
\hline

$X_{9+k}^{++}$ & $X_{9+k}^{--}$ & $C(X^l-s)$ &
$\begin{array}{ll}
  R(X-1), &\text{if } k \equiv 4 \pmod{8} \\
  R(X+1), &\text{if } k \equiv 0 \pmod{8} \\
  \varnothing, &\text{else}
\end{array}$ &
$\varnothing$ \\
\hline
\end{tabular}
\end{table}

\end{theorem}

\begin{theorem}
The structure of the equivalence classes of the $Y_{r,s}$ singularities is as
shown in Table~\ref{tab:Yrs_equivalences} where in each case
\begin{align*}
l &= \frac{r}{\gcd(r,s)} \cdot \gcd(2, r+1, s+1) \,, \\
s_1 &=
\begin{cases}
  1,  &\text{if } r \equiv 0 \pmod{4} \text{ or } s \equiv 0 \pmod{4}, \\
  -1, &\text{else,}
\end{cases} \\
s_2 &=
\begin{cases}
  1,  &\text{if } r \not\equiv 0 \pmod{2}
      \text{ or } \frac{s}{\gcd(r,s)} \equiv 0 \pmod{2}, \\
  -1, &\text{else.}
\end{cases}
\end{align*}

In the special case where $r = s$, additional equivalences occur. They are
listed in Table~\ref{tab:Yrr_equivalences}.

\begin{table}[htb]
\centering
\caption{$P_1$, $P_2$ and $P_3$ for the $Y_{r,s}$ singularities}
\label{tab:Yrs_equivalences}
\begin{tabular}{|c|c||c|c|c|}
\hline

$T_1$ & $T_2$ & $P_1(T_1, T_2)$ & $P_2(T_1, T_2)$ & $P_3(T_1, T_2)$ \\
\hline\hline

$Y_{r,s}^{++}$ & $Y_{r,s}^{++}$ &
\multirow{4}{*}{$C(X^l-1)$} &
\multirow{4}{*}{$R(X^l-1)$} &
\multirow{4}{*}{$R(X^{s+1}-1)$}
\\ \cline{1-2}

$Y_{r,s}^{-+}$ & $Y_{r,s}^{-+}$ &&&
\\ \cline{1-2}

$Y_{r,s}^{+-}$ & $Y_{r,s}^{+-}$ &&&
\\ \cline{1-2}

$Y_{r,s}^{--}$ & $Y_{r,s}^{--}$ &&&
\\ \hline


$Y_{r,s}^{++}$ & $Y_{r,s}^{-+}$ &
\multirow{4}{*}{$C(X^l-s_1)$} &
\multirow{4}{*}{$R(X^l-s_1)$} &
\multirow{4}{*}{$\varnothing$}
\\ \cline{1-2}

$Y_{r,s}^{-+}$ & $Y_{r,s}^{++}$ &&&
\\ \cline{1-2}

$Y_{r,s}^{+-}$ & $Y_{r,s}^{--}$ &&&
\\ \cline{1-2}

$Y_{r,s}^{--}$ & $Y_{r,s}^{+-}$ &&&
\\ \hline


$Y_{r,s}^{++}$ & $Y_{r,s}^{+-}$ &
\multirow{4}{*}{$C(X^l-s_2)$} &
\multirow{4}{*}{$R(X^l-s_2)$} &
\multirow{4}{*}{$\begin{cases}
  R(X^{s+1}-1), &\!\text{if } r \not\equiv 0 \pmod{2} \\
  \varnothing,  &\!\text{if } r \equiv 0 \pmod{2}
\end{cases}$}
\\ \cline{1-2}

$Y_{r,s}^{-+}$ & $Y_{r,s}^{--}$ &&&
\\ \cline{1-2}

$Y_{r,s}^{+-}$ & $Y_{r,s}^{++}$ &&&
\\ \cline{1-2}

$Y_{r,s}^{--}$ & $Y_{r,s}^{-+}$ &&&
\\ \hline


$Y_{r,s}^{++}$ & $Y_{r,s}^{--}$ &
\multirow{4}{*}{$C(X^l-s_1 s_2)$} &
\multirow{4}{*}{$R(X^l-s_1 s_2)$} &
\multirow{4}{*}{$\varnothing$}
\\ \cline{1-2}

$Y_{r,s}^{-+}$ & $Y_{r,s}^{+-}$ &&&
\\ \cline{1-2}

$Y_{r,s}^{+-}$ & $Y_{r,s}^{-+}$ &&&
\\ \cline{1-2}

$Y_{r,s}^{--}$ & $Y_{r,s}^{++}$ &&&
\\ \hline

\end{tabular}
\end{table}

\begin{table}[htb]
\centering
\caption{Additional equivalences for the $Y_{r,s}$ singularities in the
special case $r = s$}
\label{tab:Yrr_equivalences}
\begin{tabular}{|c|c||c|}
\hline

$T_1$ & $T_2$ & $P_3(T_1, T_2)$ \\
\hline\hline

$Y_{r,s}^{++}$ & $Y_{r,s}^{+-}$ &
\multirow{2}{*}{$\begin{cases}
  R(X+1),      &\!\text{if } r \equiv 0 \pmod{2} \text{ and } a < 0 \\
  \varnothing, &else
\end{cases}$}
\\ \cline{1-2}

$Y_{r,s}^{-+}$ & $Y_{r,s}^{--}$ &
\\ \hline

$Y_{r,s}^{+-}$ & $Y_{r,s}^{++}$ &
\multirow{2}{*}{$\begin{cases}
  R(X+1),      &\!\text{if } r \equiv 0 \pmod{2} \text{ and } a > 0 \\
  \varnothing, &else
\end{cases}$}
\\ \cline{1-2}

$Y_{r,s}^{--}$ & $Y_{r,s}^{-+}$ &
\\ \hline

\end{tabular}
\end{table}

\end{theorem}


\begin{theorem}
The structure of the equivalence classes of the $\tY_r$ singularities is as
shown in Table~\ref{tab:tYr_equivalences} where in each case
\begin{align*}
l &= \frac{4}{\gcd(4, r)}, \text{ and} \\
s &=
\begin{cases}
  1,  &\text{if } r \equiv 0 \pmod{8}, \\
  -1, &\text{else.}
\end{cases}
\end{align*}

\begin{table}[htb]
\centering
\caption{$P_1$, $P_2$ and $P_3$ for the $\tY_r$ singularities}
\label{tab:tYr_equivalences}
\begin{tabular}{|c|c||c|c|c|}
\hline

$T_1$ & $T_2$ & $P_1(T_1, T_2)$ & $P_2(T_1, T_2)$ & $P_3(T_1, T_2)$ \\
\hline\hline

$\tY_r^+$ & $\tY_r^+$ &
\multirow{2}{*}{$C(X^l-1)$} &
\multirow{2}{*}{$R(X^l-1)$} &
\multirow{2}{*}{$\begin{cases}
  R(X^2-1), &\text{if } r \equiv 1 \pmod{2} \\
  R(X-1),   &\text{if } r \equiv 0 \pmod{2}
\end{cases}$} \\
\cline{1-2}

$\tY_r^-$ & $\tY_r^-$ &
&
&
\\
\hline

$\tY_r^+$ & $\tY_r^-$ &
\multirow{2}{*}{$C(X^l-s)$} &
\multirow{2}{*}{$R(X^l-s)$} &
\multirow{2}{*}{$\varnothing$} \\
\cline{1-2}

$\tY_r^-$ & $\tY_r^+$ &
&
&
\\
\hline

\end{tabular}
\end{table}

\end{theorem}


\begin{theorem}
The structure of the equivalence classes of the exceptional unimodal
singularities is as shown in Table~\ref{tab:exceptional_equivalences}.

\begin{table}[!htb]
\centering
\caption{$P_1$, $P_2$ and $P_3$ for the exceptional unimodal singularities}
\label{tab:exceptional_equivalences}
\begin{tabular}{|c|c||c|c|c|}
\hline
$T_1$ & $T_2$ & $P_1(T_1, T_2)$ & $P_2(T_1, T_2)$ & $P_3(T_1, T_2)$ \\
\hline\hline
$E_{12}$   & $E_{12}$   & $C(X^{21}-1)$ & $R(X-1)$      & $R(X-1)$ \\
\hline
$E_{13}$   & $E_{13}$   & $C(X^{15}-1)$ & $R(X-1)$      & $R(X-1)$ \\
\hline
$E_{14}^+$ & $E_{14}^+$ & $C(X^{12}-1)$ & $R(X^2-1)$    & $R(X-1)$ \\
\hline
$E_{14}^-$ & $E_{14}^-$ & $C(X^{12}-1)$ & $R(X^2-1)$    & $R(X-1)$ \\
\hline
$E_{14}^+$ & $E_{14}^-$ & $C(X^{12}+1)$ & $\varnothing$ & $\varnothing$ \\
\hline
$E_{14}^-$ & $E_{14}^+$ & $C(X^{12}+1)$ & $\varnothing$ & $\varnothing$ \\
\hline
$Z_{11}$   & $Z_{11}$   & $C(X^{15}-1)$ & $R(X-1)$      & $R(X-1)$ \\
\hline
$Z_{12}$   & $Z_{12}$   & $C(X^{11}-1)$ & $R(X-1)$      & $R(X-1)$ \\
\hline
$Z_{13}^+$ & $Z_{13}^+$ & $C(X^9-1)$    & $R(X-1)$      & $R(X-1)$ \\
\hline
$Z_{13}^-$ & $Z_{13}^-$ & $C(X^9-1)$    & $R(X-1)$      & $R(X-1)$ \\
\hline
$Z_{13}^+$ & $Z_{13}^-$ & $C(X^9+1)$    & $R(X+1)$      & $\varnothing$ \\
\hline
$Z_{13}^-$ & $Z_{13}^+$ & $C(X^9+1)$    & $R(X+1)$      & $\varnothing$ \\
\hline
$W_{12}^+$ & $W_{12}^+$ & $C(X^{10}-1)$ & $R(X^2-1)$    & $R(X-1)$ \\
\hline
$W_{12}^-$ & $W_{12}^-$ & $C(X^{10}-1)$ & $R(X^2-1)$    & $R(X-1)$ \\
\hline
$W_{12}^+$ & $W_{12}^-$ & $C(X^{10}+1)$ & $\varnothing$ & $\varnothing$ \\
\hline
$W_{12}^-$ & $W_{12}^+$ & $C(X^{10}+1)$ & $\varnothing$ & $\varnothing$ \\
\hline
$W_{13}^+$ & $W_{13}^+$ & $C(X^8-1)$    & $R(X^2-1)$    & $R(X-1)$ \\
\hline
$W_{13}^-$ & $W_{13}^-$ & $C(X^8-1)$    & $R(X^2-1)$    & $R(X-1)$ \\
\hline
$W_{13}^+$ & $W_{13}^-$ & $C(X^8+1)$    & $\varnothing$ & $\varnothing$ \\
\hline
$W_{13}^-$ & $W_{13}^+$ & $C(X^8+1)$    & $\varnothing$ & $\varnothing$ \\
\hline
\end{tabular}
\end{table}

\end{theorem}


 \begin{thebibliography}{99}
\bibitem{AVG1985} Arnold, V.I.; Gusein-Zade, S.M.; Varchenko, A.N.:
Singularities of Differential Maps. Vol.~I, Birkh\"auser (1985).
\bibitem{A1975} Arnold, V.I.:
\textit{Normal form of functions near degenerate critical points.},
Russian Mth. Surveys 29 ii (1975), 10-50.
\bibitem{DGPS}
Decker, W.; Greuel, G.-M.; Pfister, G.; Sch{\"o}nemann, H.:
\newblock {\sc Singular} {3-1-5} --- {A} computer algebra system for polynomial
computations.
\newblock {http://www.singular.uni-kl.de} (2012).
\bibitem{Kruger} Kr\"uger, K.: Klassifikation von
Hyperfl\"agensingularit\"aten, Diploma Thesis (1997).
\bibitem{GLS2007}Greuel, G.-M.; Lossen, C.; Shustin E.:
Introduction to Singularities and Deformations, Springer, Berlin (2007).
\bibitem{GP2008} Greuel G.-M.; Pfister G.;
A Singular introduction to Commutative Algebra, 2nd Ed., Springer,
Berlin (2008).
\bibitem{classify}
Kr\"uger, K.:
{\tt classify.lib}. {A} {\sc Singular} {3-1-5} library for classifying isolated
hypersurface singularities w.r.t. right equivalence, based on the determinator
of singularities by V.I. Arnold (2012).
\bibitem{realclassify}
Marais, M. and Steenpass, A.:
{\tt realclassify.lib}. {A} {\sc Singular} {3-1-5} library for classifying
isolated hypersurface singularities over the reals w.r.t. right equivalence,
based on the determinator of singularities by V.I. Arnold. This library is
based on classify.lib by Kai Kr\"uger, but handles the real case, while
classify.lib does the complex classification (2012).
\bibitem{primdec.lib} Pfister, G.; Decker, W.;  Schoenemann, H.; Laplagne, S.:
{\tt primdec.lib}. {A} {\sc Singular} {3-1-5} library for Primary Decomposition
and Radical of Ideals (2012).
\bibitem{Siersma} Siersma D.: Classification and deformation of Singularities,
disertation, University of Amsterdam (1974).
\bibitem{roots}
Tobis, A.:
{\tt rootsur.lib}. {A} {\sc Singular} {3-1-5} library for Counting number of
real roots of univariate polynomial (2012).
\bibitem{solve.lib} Wenk, M.: {\tt solve.lib}. Pohl, W.:
{A} {\sc Singular} {3-1-5} library for Complex Solving of Polynomial Systems
(2012).

\end{thebibliography}

\end{document}
